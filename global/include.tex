
% Document setup ---------------------------------------------------------------

% PAGE SETUP -----
  \usepackage[doublespacing]{setspace} % Double line spacing

  \frenchspacing

  \usepackage[a4paper,bottom=25mm,top=25mm, right=25mm,left=35mm]{geometry} % Margins

  % \raggedbottom

  \usepackage{fancyhdr} % Headers and footers
  \pagestyle{fancyplain}
  \fancyhf{}
  \setlength\headheight{17pt}
  \renewcommand{\headrulewidth}{0pt}% No header rule
  \renewcommand{\footrulewidth}{0pt}% No footer rule
  \fancyhead[LE,RO]{\thepage}

  % TITLE PAGE ------------
  \usepackage{titling}
  \newcommand{\volumecover}[3]{
  \newgeometry{top=35mm,bottom=35mm,right=35mm,left=35mm}
  \begin{titlepage}
    \doublespacing
    \addtocounter{page}{#2}
    \centering
    \large Trinity College Dublin, the University of Dublin\par
    \vspace*{\fill}
    \large \uppercase{The Accompaniment of Plainchant in France, Belgium and Certain Other Catholic Regions: A Chronological Study of Theory and Practice from the French Revolution to the Second Vatican Council}\par
    \vspace*{\fill}
    \large {Thesis submitted for the degree of Doctor in Philosophy}\par
    \par
    \large {Cillian Long}\par
    \par
    \large {2022}\par
    \vspace*{\fill}
    \large {Complete in Two Volumes}\par
    \large {Volume #1}
  \end{titlepage}
  \setcounter{page}{#3}
  \restoregeometry
  }

  %Blank pages
\usepackage{afterpage}
\newcommand\insertemptypage{
  \null
  \thispagestyle{empty}
  \mbox{}
  \newpage
  }

  %Allow prepending to commands, particularly bibliograhical ones
   \usepackage{xpatch}

% COPY --------

\usepackage[german,french,british]{babel} % Correct hyphenation
\usepackage[none]{hyphenat} %Turn off hyphenation
   % \usepackage{luatexbase} fixes bug with microtype below
  \usepackage{luatexbase}
  \usepackage[activate={true,nocompatibility},final,tracking=true,factor=1100,stretch=10,shrink=10]{microtype} % Makes type look better
  % \widowpenalty=150 % Prevent widows
  % \clubpenalty=150 % Prevent orphans
  \usepackage{nowidow}
%   \clubpenalty=9996
% \widowpenalty=9999
% \brokenpenalty=4991
% \predisplaypenalty=10000
% \postdisplaypenalty=1549
% \displaywidowpenalty=1602

%Less hyphenation
\tolerance=1
\emergencystretch=\maxdimen
\hyphenpenalty=9000
\hbadness=90000

% \usepackage{titletoc}
% SECTION TITLES FORMAT ---

  \usepackage{etoolbox}
  % \patchcmd{\@makechapterhead}{\thechapter}{\numberstring{chapter}}{}{}
  % \patchcmd{\chaptermark}{\thechapter}{\numberstring{chapter}}{}{}
  % \makeatother
  \usepackage{fmtcount} % To convert numbers into textual equivalents
  \usepackage{titlesec} % To change the formatting of \chapter, \section etc
  \usepackage{xstring}%To get numbers and turn them into textual equivalents
  %Here we're changing the formatting of the chapter label and title to centered uppercase - display is required so they're on different lines. I have to move the label to the right by 1mm to make it match up with the title.
  \titleformat{\chapter}[display]{\centering}  {\hspace*{1mm}\MakeUppercase\chaptertitlename\
    %This changes the chapter number into text if it is supplied as a number. Appendices are lettered and so the function does not try to numerise them.
    \IfInteger{\thechapter}%If the chapter number is a number
          {\NUMBERstringnum{\thechapter}     } %Convert it into text equivalent - 3=three
         {\thechapter}%Otherwise the chapter is lettered (as an appendix) and this returns the letter
  }
  {12pt} %Space between chapter label and the title
  {\MakeUppercase} %Make title uppercase
  []
   %section titles
  \titleformat{\section}[hang]
  {\centering}{\thesection}{12pt}{} %Section titles
  \titleformat*{\subsection}{\itshape} %Subsection titles get changed to italics

  \usepackage{tocloft}%Modify how sections and chapters etc are represented on the table of contents


  %List of music examples - header format
    \renewcommand{\cftloftitlefont}{\hfill\normalfont} %Center from left, roman
    \renewcommand{\cftafterloftitle}{\hfill} %center from right

  %List of tables - header format
    \renewcommand{\cftlottitlefont}{\hfill\normalfont}
    \renewcommand{\cftafterlottitle}{\hfill}

  %Parts (volumes)
      %Roman font, and centre from left
    \renewcommand{\cftpartfont}{\hfill\normalfont}
      %Hack to centre from right
    \makeatletter
      \patchcmd{\l@part}{#1}{#1\hfill\hskip-\rightskip\mbox{}}{}{}
    \makeatother
      %Turn off display of pagination in TOC
     \cftpagenumbersoff{part}

% \renewcommand{\cftpartpagefont}{\normalfont}
    %% Do not allow \part{} to print an extra page, also centre the part title in the TOC
    % \renewcommand{\part}[1]{\addcontentsline{toc}{part}{\hspace{.439\linewidth} #1}}%439 is a hack to get volume title more or less centred
    \renewcommand{\part}[1]{\addcontentsline{toc}{part}{#1}}%439 is a hack to get volume title more or less centred

  %Chapter titles

  \renewcommand{\cftchapfont}{\normalfont}
  \renewcommand{\cftchappagefont}{\normalfont}%pagenumber in Roman
  \cftsetpnumwidth{\widthof{2pt}}
  \renewcommand{\cftchapleader}{\cftdotfill{\cftdotsep}}


  %Table of contents header
    %Centre it, times new roman, uppercase
  \renewcommand{\cfttoctitlefont}{\hfill\normalfont\MakeUppercase}
    %Have to add this for it to centre properly
  \renewcommand{\cftaftertoctitle}{\hfill}
  \AtBeginDocument{\renewcommand{\contentsname}{Table of Contents}}%Required when using babel

  \usepackage{etoc}
  \setcounter{secnumdepth}{2}
  \setcounter{tocdepth}{2}




% Abbreviations -----------------------------------------------------------------
% \usepackage[nomain,toc,acronym,section=chapter,nopostdot,nonumberlist,numberedsection]{glossaries}
% \setacronymstyle{short}
% \renewcommand{\glossarysection}[2][]{}
% Bibliography -----------------------------------------------------------------
\usepackage[backend=biber,
  bibencoding=utf8,
  isbn=false,
  doi=false,
  url=false,
  datecirca=true,
  useprefix=false,
  sorting=nyt,%Sort bibliography by name, then date, then title
  ]{biblatex-chicago}
 \addbibresource{bib/library.bib}


%`bibindex' is the macro called at each line in the bibliography. Check to see if a bibliographical entry has `covid' as one of its keywords (tags in Zotero) and if it does then prepend the relevant mark.
 \xapptobibmacro{bibindex}{\ifkeyword{covid}{\covid{}}{}}{}{}

 %Make bibliography smaller than body text, or same size at footnotes
 \renewcommand*{\bibfont}{\footnotesize}

 %Continuous footnote numbering
  \counterwithout{footnote}{chapter}
  %Footnote styling - small number and indented
  \usepackage[hang,flushmargin]{footmisc}

    %Remove language when printing citations
 \AtEveryCitekey{\clearlist{language}}
 \AtEveryBibitem{\clearlist{language}}
    %Remove Google Books unique IDs when printing citations
  \AtEveryCitekey{\clearfield{eprint}}
  \AtEveryBibitem{\clearfield{eprint}}

  %Change format of `circa'
  \renewcommand{\datecircadelim}{}
  \DefineBibliographyStrings{english}{circa = \emph{c}.}



% Appendices ------------------------------------------------------------------

\usepackage[titletoc]{appendix}
% Graphics ---------------------------------------------------------------------
\usepackage{graphicx}

\usepackage{censor} %To white-out signature

\usepackage{wrapfig}
%\graphicspath{{/home/cillian/Documents/phd/fragments/pieces/}}

% Musical examples -------------------------------------------------------------
\usepackage{lyluatex} % Lilypond compiler
%\usepackage[autocompile]{gregoriotex} % Gregorio compiler
\usepackage{float} % Allows use of H placement flag -- place item right here.

% % For epigraphs ----------------------------------------------------------------
% \usepackage{epigraph}



% Music fonts ------------------------------------------------------------------


% \usepackage[utf8x]{inputenx}
% \usepackage[utf8x]{inputenc}
\usepackage[T1]{fontenc}
\usepackage{fontspec}

\usepackage{lilyglyphs} % Inline music font
\usepackage{amssymb} %For \because symbol

%This Times New Roman (TNR) font has ligatures enabled by default
\setmainfont{Tex Gyre Termes}

\newfontfamily\greekfont[Script=Greek]{Linux Libertine O}

%This is the TNR font for the mathematics environment. Install instructions in the readme.
\usepackage[lite]{mtpro2}


% Quotes -----------------------------------------------------------------------
\usepackage{csquotes}

% \usepackage{etoolbox}
\AtBeginEnvironment{quote}{\singlespace\vspace{-2em}}%-2em is hacky way to bunch it up with preceding paragraph
\AtEndEnvironment{quote}{\vspace{-12pt}\endsinglespace} %formerly -\topsep -- to be kept under review -- spacing between bottom of quote and next element.
\patchcmd{\quote}{\rightmargin}{\leftmargin \parindent \rightmargin}{}{} %Sets the indentation of the quote - by the same amount as the paragraph indent for consistency's sake

\newcommand{\single}[2]{
  \begin{quote}
    \singlespacing
    #1\footnotemark
  \end{quote}
  \footnotetext{#2.}
}

%Typeset subscript prime symbol in mathematics mode
\makeatletter
\def\active@math@sprime{_\bgroup\sprim@s}
{\catcode`\`=\active \global\let`\active@math@sprime}
\def\sprim@s{%
  \prime\futurelet\@let@token\spr@m@s}
\def\spr@m@s{%
  \ifx`\@let@token
    \expandafter\spr@@@s
  \else
    \ifx_\@let@token
      \expandafter\expandafter\expandafter\spr@@@t
    \else
      \egroup
    \fi
  \fi}
\def\spr@@@s#1{\sprim@s}
\def\spr@@@t#1#2{#2\egroup}
\mathcode`\`="8000
\makeatother
%End of typesetting subscript prime symbol

\usepackage{tikz}
% \usetikzlibrary{tikzmark,arrows,shapes}


\usepackage{parallel}
\usepackage{calc}


%Define the column width
\newlength\columnWidth
%.5\parindent sets the gap between columns
\setlength{\columnWidth}{.5\linewidth+.5\parindent}
%Define a generic two-column environment
\newcommand{\dualcolumn}[2]{
  %Set up two columns of a certain width
  \begin{Parallel}[c]{\columnWidth}{\columnWidth}
    %Set single line spacing because we are citing a source
    \singlespacing%
    %Left column
    \ParallelLText{
      \leftskip=\parindent%
      \parindent=0pt%
      \sloppy%
      #1%
      %% Returning these values back to what they were in include.tex -- hacky!
      \tolerance=1%
      \emergencystretch=\maxdimen%
      \hyphenpenalty=9000%
      \hbadness=90000%
      %%
    }%
    %Right column
    \ParallelRText{%
      \rightskip=\parindent%
      \parindent=0pt%]
      \begin{itshape}%
        \sloppy%
        #2%
      \end{itshape}%
      %% Returning these values back to what they were in include.tex -- hacky!
      \tolerance=1%
      \emergencystretch=\maxdimen%
      \hyphenpenalty=9000%
      \hbadness=90000%
    }%
  \end{Parallel}%
}

\newcommand{\simplex}[3]{
  \dualcolumn{#1\footnotemark}{#3}
  \footnotetext{#2.}%
}


%Required to access variables starting with @
\makeatletter%
\newcommand{\duplex}[4]{%
  %Set up two-column environment
  \dualcolumn{%
    % Text for left-most column is stored in #1
    % The link's target is stored in the variable @L
      % This gets overwritten each time duplex environment is called
    #1\footnotemark\global\let\saved@Href@L\Hy@footnote@currentHref%
  }%
  {%
    %Similar to the above for the right-most column
    % Target is saved in @R
    #3\footnotemark\global\let\saved@Href@R\Hy@footnote@currentHref%
  }%
  %De-increment the number of the footnote
    %the second \footnotemark was necessary to have two column side-by-side, but the integer has now been incremented one too many times
  \addtocounter{footnote}{-1}%
  %Recall @L so the hyperref package knows where to send readers who click on the superscript link
  \let\Hy@footnote@currentHref\saved@Href@L%
  % #2 is whatever is going into the footnote for the left-most column
  \footnotetext{#2.}%
  %Increment the footnote counter
  \stepcounter{footnote}%
  %Recall the "R" variable
  \let\Hy@footnote@currentHref\saved@Href@R%
  %Set whatever is going into the right-most footnote
  \footnotetext{#4.}%
}%
%Return @ to its previous functionality
\makeatother%



% Captions ---------------------------------------------------------------------
\usepackage[font=footnotesize]{caption} % Reduce size of caption text
\usepackage[breaklinks,hidelinks,bookmarksnumbered]{hyperref} % Make references into hyperlinks: break over lines; hide links (dont draw bounding box), bookmarksnumbered=put in A into "Appendix A" (etc) in Adobe Reader bookmarks pane.
  \hypersetup{urlcolor=black,
  colorlinks=false,
  pdftitle={The Accompaniment of Plainchant in France, Belgium and Certain Other Catholic Regions: A Chronological Study of Theory and Practice from the French Revolution to the Second Vatican Council},
  pdfauthor={Cillian Long},
  pdfproducer={},
  pdfcreator={}
  }  % Hyperlink colour

%Make hyperlinks monospaced
\usepackage{url}
  \urlstyle{tt}

% \usepackage{subcaption} % Multiple captions in a row
\usepackage{cleveref} % Allows flexible reference tags

  \crefname{page}{p.}{pp.}
  \crefname{section}{§}{§§}


\renewcommand\thefigure{\arabic{figure}}

\usepackage{newfloat}
\DeclareFloatingEnvironment[
  fileext=loex,
    listname={List of Music Examples},
    name=Example,
    placement=tbhp,
    within=none
]{example}
\crefname{example}{ex.}{exx.}
\Crefname{example}{Ex.}{Exx.}
% \DeclareFloatingEnvironment[
%   fileext=loillus,
%     listname={List of Illustrations},
%     name=Illustration,
%     placement=tbhp,
%     within=none
% ]{illustration}
% \crefname{illustration}{illus.}{illus.}
% \Crefname{illustration}{Illus.}{Illus.}


% TeXCount ---------------------------------------------------------------------
%TC:macro \footnote [ignore]
%TC:macro \letter [ignore,ignore,ignore,ignore]
%TC:macro \fnletter [ignore,ignore,ignore,ignore]

%TC:macro \single [ignore,ignore]
%TC:macro \simplex [ignore,ignore,text]
%TC:macro \duplex [ignore,ignore,text,ignore]
%TC:macro \dualcolumn [ignore,text]
%TC:macro \cite [option:ignore,ignore]
%TC:macro \footcite [option:ignore,ignore]

\usepackage{lipsum}

%To enable landscape mode ------------------------------------------------------
\usepackage{rotating} % for 'sidewaystable' environment
\usepackage{pdflscape}

%%%%%%%%%%%%%%%%%%%%%%%%%%%%
%For a printable PDF, uncomment the following three lines which turns off the PDF /Rotate flag
% \makeatletter
% \renewcommand\PLS@Rotate[1]{}
% \makeatother
%%%%%%%%%%%%%%%%%%%%%%%%%%%%

%Make showframe work properly in landscape mode
\makeatletter
\newcommand*{\gmshow@textheight}{\textheight}
\newdimen\gmshow@@textheight
\g@addto@macro\landscape{%
  \gmshow@@textheight=\hsize
  \renewcommand*{\gmshow@textheight}{\gmshow@@textheight}%
}
\def\Gm@vrule{%
  \vrule width 0.2pt height\gmshow@textheight depth\z@
}%
\makeatother
\usepackage{longtable}
\usepackage{booktabs}
\usepackage{multirow}


\usepackage{textcomp}

\usepackage[normalem]{ulem}
