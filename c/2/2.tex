\chapter{France and Belgium prior to Solesmian influence}
\label{c:2}%
\section{Chant practice in Post-Revolutionary France}
\subsection{Liturgical performance traditions to the 1840s}
In consequence of the addition of feast days to the liturgical calendar from the sixteenth century, French and Belgian dioceses were faced with lacunæ in their chant repertories and therefore opted to source new, chant-like melodies from local composers.
\hlabel{ln:poisson}%
Léonard Poisson (\emph{c.}1685--1753) lamented that such melodies were ill-conceived, and that editors falsely permitted the major-minor harmony to infringe upon the modality of chant.
He historicised the latter and suggested that scholarship ought to return to the most ancient manuscripts to discover chant as it supposedly once was:

\simplex{Les plus anciennes pièces ſ\kern -0.5pt ont ordinairement les plus correctes pour l'expreſ\kern -0.5pt ſ\kern -0.8pt ıon et la livraiſ\kern -0.5pt on des paroles, \& qu'elles l'emportent de beaucoup ſ\kern -0.5pt ur la plûpart des nouvelles par la majeſ\kern -0.5pt té de leur chant, ſ\kern -0.5pt on goût \& sa régularité ; \& c'eſ\kern -0.5pt t ce qui me fait croire qu'on a eu tort de négliger les anciens.}
  {\cite[3]{PoissonTraitetheoriquepratique1750}}
{The most ancient pieces are ordinarily the most correct for the expression and connection of words, and thereby \linebreak{}surpass most of the new ones in the majesty of their chant, its taste and regularity; and this is what leads me to believe that we were wrong to neglect the ancients.}
\hlabel{ln:poisson_END}%
\noindent
As in Germany, the older-is-better notion was widely applied to French and Belgian church music, and was corroborated when Napoleon's conquests in Europe and the near East led to an influx of historical artefacts to France in particular.
Ironically, given the emperor's wish to modernise French society, those artefacts promoted an interest in French and European cultural heritage that awoke a sense of antiquarianism among music theorists and musicians who identified themselves with archaeological research.
Though the anticlerical sentiment of the French Revolution saw the closure of the \emph{maîtrises} (schools where boys and men had received their musical training), the former Revolutionary musician Bernard Sarrette (1765--1858) was permitted to establish a school of music in 1792 that in 1795 would grow into the celebrated Paris Conservatoire.\footcite[200]{ChouquetMaitrise1880}
The school's library gathered together pieces of music of historic or cultural importance into a kind of musical museum,\footcite[4]{EllisInterpretingMusicalEarly2005} and from 1844 similar collections provided the historian and archaeologist Adolphe Napoléon Didron (1806--67) with material for his \emph{Annales archéologiques}, a periodical dedicated to the curious mixture of music, architecture and sculpture.
Didron presented a thirteenth-century fauxbourdon harmonisation of `Patrem parit filia' and an organ accompaniment of `Regnantem sempiterna' as venerable artefacts.\footcite*[unpaginated supplements at pp.~248 and 318]{DidronRegnantemsempiterna1849}
During the ensuing decade, Didron's examples were re-published as important artefacts alongside others attributed to the organist of Saint-Étienne-du-Mont Gabriel Gauthier (1808--53) and director of the Rheims Conservatoire Louis-Simon Fanart (1807--83).\footcite[unpaginated supplement at p.~434]{VervoitteConsiderationschantecclesiastique1857}

In 1811, Alexandre-Étienne Choron (1771--1834) had likened chant to painting, sculpture and architecture worthy of authentic restoration and conservation for future generations,\footcite[9]{ChoronConsiderationsnecessiteretablir1811} establishing the École Choron in 1817 to promote the music of Palestrina and Bach as examples of appropriate styles of church music.
The school never recovered from the strictures placed on its funding after the July Revolution of 1830, and closed with Choron's death in 1834.
Nonetheless its influence is undisputed and it was revived two decades later as the École Niedermeyer.
As we have seen (\cref{sc:cecilianism}), the Cecilian movement promoted similar ideals in Bavaria and elsewhere, but there appears to be no direct link between the two movements and it is possible that they sprang up independently of one another.\footcite[25]{GarrattPalestrinaGermanRomantic2002}
Common to many European countries, however, was that the fantasy of the `old' became commingled with the rise of Ultramontanism, and many dioceses---which up to then could determine their own practices---began to unify their liturgies and music with those recommended by the Vatican.

Palestrina fever gripped the nineteenth-century popular imagination to such an extent that Victor Hugo called him the `father of harmony',\footcite[209]{HugoQuemusiquedate1840} and it was according to the unsound conflation of harmony with counterpoint and modality with major-minor progressions that many chant melodies were disfigured with chromatically altered pitches.
It was not until 1847---when the French theorist Jean-Louis-Félix Danjou (1812--66) discovered that Codex H.~159 at the Bibliothèque de l'École de médicine de Montpellier transmitted adiastematic neumes superimposed on a form of alphabetic notation---that scholars could begin to ascertain the pitch content of chant melodies from the tenth century.\footcite[This manuscript is known today as `the Dijon tonary' for it came from Saint-Bénigne de Dijon. See][578--9]{HileyWesternPlainchantHandbook1993}
That manuscript served as the primary impetus for the chant restoration movement which in turn culminated in the Vatican's turning away from Pustet's editions at the beginning of the twentieth century.
It was a ground-breaking discovery in its day, and Danjou also undertook what Fétis termed an archaeological excursion to Italy with Stéphen Morelot (1820--99) to uncover even more sources (`cette excursion archéologique').\footcite[195]{FetisBiographieuniversellemusiciens1867}
His survey of the available source material yielded results that were later taken up by Edmond de Coussemaker (1805--76) who sought to recapture the supposedly lost harmony of the Middle Ages.\footcite[121]{CoussemakerHistoireharmonieau1852}

\subsection{The rise of the \emph{orgue accompagnateur}}
\label{sc:orgue_accompagnateur}
Post-Revolutionary France was a time of widespread cultural, societal and musical change, and the anti-traditionalist movement had far-reaching effects on the church in general and church music in particular.
To some, even such instruments as the serpent were considered hallmarks of the \emph{ancien régime}, and attracted aesthetic criticism for disfiguring chant melodies with insipid ornaments and cadenzas.
In addition, the uneven sound produced by that instrument coupled with deteriorating standards of playing made it incapable of providing adequate support to choirs.\footcite[12]{HillsmanInstrumentalAccompanimentPlainChant1980}
Aesthetic considerations, as well as ideological ones, became defining characteristics of church music with the growing popularity of romanticism and liturgical mysticism.
In a bid to enrich the theology of the Parisian church of Saint-Étienne-du-Mont, its curate Nicholas-Théodore Olivier (1798--1854) sought a means to combine terrestrial and ethereal voices, as his biographer recounts:

\simplex{Le curé de Saint-Etienne ne croyait pouvoir donner trop de beauté au plain-champ [\textit{sic}], trop de perfection aux concerts spirituels, et demandait à la musique religieuse d'épuiser toutes ses ressources et toutes ses harmonies. Il aurait voulu qu'elle fût une image et un écho de celle du ciel.}
  {\cite[171]{BouclonHistoiremonseigneurOlivier1855}}
{
The curate of Saint-Etienne did not think too much beauty could be given to plainchant, or too much perfection to spiritual concerts, and called upon religious music to exhaust all its resources and harmonies. He would have liked it to be a reflection and an echo of that of the heavens.
}
\noindent
\hlabel{hl:lafage_reproduction_intro}%
In 1829, Olivier appointed as his \emph{maître de chapelle} Adrien de La Fage (1801--62), whose prompt excision of serpent accompaniment from the liturgy was to force French church music into a new age:

\simplex{Mon but principal en introduisant l'orgue dans le chœur était l'abolition de cet abominable et honteux usage connu seulement en France d'accompagner le chœur par le serpent, instrument grossier, si contraire aux voix, au goût et au bon sense, et dont la présence était le principal obstacle à tout progrès quelconque.}
  {\cite[141 n.~1]{LaFagereproductionlivresplainchant1853}}
{
My main goal in introducing the organ into the Choir was the abolition of that abominable and shameful practice known only in France of accompanying the choir by the serpent, that uncouth instrument, so contrary to voices, to taste and to common sense, and whose presence was the principal obstacle to any and all progress.
}
\noindent
La Fage was joined in his protests by Hector Berlioz (1803--69) who also took up a stance against the serpent.\footcite[163]{Girodmusiquereligieuse1855}
But not even the support of Olivier's archbishop made it an easy task to oust the instrument, for this was a polarising and progressive idea denigrated by conservative ecclesiastics as absurd, scandalous and even sacrilegious.
Moreover, serpentists held that their livelihoods were under threat, and Olivier was accused of `dethroning' the serpent, nearly making the organ into a kind of Marianne to progressives and a Robespierre to conservatives.
In the face of such opposition, however, Olivier and La Fage pressed on with their plans to procure a new organ for Saint-Étienne-du-Mont.
There already existed a small positive organ in the Choir built by the Dallery firm, but it was not loud enough to support singers and bass voices were said to drown it out.
A new instrument was required, therefore, to suit the needs of the church, and in response to the general lack of support,  Olivier decided to finance the project personally.\footcite[170--77]{BouclonHistoiremonseigneurOlivier1855}

It was to the English-born, protestant organ builder John Abbey that Olivier and La~Fage turned when they commissioned the new instrument.
That the work should have been contracted to an Englishman is not surprising because Post-Revolutionary periods of governmental sympathy to the church contributed to robust demand for new instruments, providing an ideal opportunity for English builders to cross the Channel to exercise their trade.
The French instrument maker Sébastien Érard had spent time in London during the Revolution, and had become acquainted with English developments in organ building; his piano-organ hybrid, the so-called `Piano Carré Organisé' (of which an example survives at the Cité de la Musique in Paris), dates from this period.
Balanced key action, horizontal bellows with compensating folds and composition pedals were some innovations that Abbey brought with him to Paris at Érard's invitation in 1826, but before long Abbey had struck out as an organ builder in his own right and elements of his instruments continue to survive in Parisian churches today.\footcite[223]{BicknellHistoryEnglishOrgan1999}

Abbey was probably already in talks with La Fage by the time Choron drafted a description of an ideal accompanying instrument consisting of Bourdon, Prestant, Dessus de flûte and Basson with an octave and a half of pedal pulldowns.
On publishing that description in 1830, Choron noted that such an instrument had just been built, but the lack of a discrete pedal stop furnished it with a weak bass, thus making it necessary to retain the support of string instruments such as cello and double bass.\footcite[260--61]{AlbrechtsbergerMethodesharmoniecomposition1830}
Incidentally, double basses are reported to have remained in use at some churches until at least the 1890s.\footcite[256 \S{}XII n.~6]{OchseOrganistsOrganPlaying2000}
Danjou gave an account of the dedication of the new instrument termed the \textit{orgue accompagnateur} when it was inaugurated in November 1829.
He described the interest that musicians showed in it while admitting that it had been conceived according to an unclear scheme (`établi sur des
données alors incertaines').\footcite[5--6]{Danjouaccompagnementplainchant1848}
A disposition of the instrument in this period has not yet come to light, but after his installation as curate of Saint-Roch on 7 February 1833, Olivier simply brought the organ with him and had it placed in the Chapelle de la vierge where it continues to function today.\footcite[202--203]{ChassantHistoireeveques1846}
Its present-day stoplist of six ranks---Montre~8$^\prime$, Bourdon~8$^\prime$, Prestant~4$^\prime$, Doublette~2$^\prime$ and Cymbale~II---with fourteen permanent pedal pulldowns bears at least some resemblance to Choron's stoplist of 1829.

Within two decades it became common, and perhaps even fashionable, for churches to procure a second, smaller instrument to accompany the choir (smaller, that is, compared with the grand orgue on the gallery).
Church authorities generally located it near the high altar in close proximity to celebrants and the choir.
The Parisian church of Mission-étrangères became the second into which such a smaller organ was introduced after La Fage was appointed \textit{maître de chapelle} there in 1831, and by 1835 the \textit{orgue accompagnateur} had made its way into the Choirs of Notre-Dame-de-Lorette, Saint-Eustache, Saint-Paul-Saint-Louis, Saint-Vincent-de-Paul and Saint-Merry.\footcite[6]{Danjouaccompagnementplainchant1848}
\hlabel{hl:letat_lavenir}%
The grand orgue in the west-end gallery continued to exercise its functions as a solo instrument just as before, but in some churches the singers were moved to the gallery where the organist exercised the dual functions of soloist and accompanist.\footcite[61]{Danjouetatavenirchant1843}
When the church of Saint-Paterne placed on the gallery an \emph{orgue accompagnateur} originally intended for the Choir, its builder Aristide Cavaillé-Coll (1811--99) regretted how the organ would not sound as it should:
\pagebreak{}

\duplex{Toutefois je tiens à vous dire que l'orgue de Saint-Paterne que j'avais vendu pour être placé dans le chœur où il devait produire un excellent effet comme orgue d'accompagnement a été placé contre mon gré sur une tribune où je savais d'avance qu'il ne produirait pas l'effet désiré. Si j'avais construit un orgue pour cette place j'aurais pu faire pour le même prix un instrument dont l'effet n'aurait rien laissé à désirer.}
  {Aristide Cavaillé-Coll to Mr Berland, curate at Beaugency on 9 September 1857, \bnf{} IFN-8451558 (3509), pp.~220--1}
{However, I must emphasize that the Saint-Paterne organ, which I had sold for installation in the Choir where it would have been most effective as an accompaniment organ, was placed contrary to my wishes on a gallery where I knew it would not produce the desired effect. If I had built an organ for this location I would have been able to build an instrument for the same price whose effect would not have been second-rate.}
  {\cite[Adapted from a partial translation in][145]{DouglassCavailleCollmusiciansdocumented1980}}
\noindent
The popularity of the \emph{orgue accompagnateur} led quickly to the introduction of larger multi-manual instruments to the Choir known (and still known) as \textit{orgues de chœur}.

Following the introduction of the \emph{orgue accompagnateur} at Saint-Étienne, organ design was influenced by the complementary factors of pitch and transposition.
In 1683, Nivers had described \textit{ton de la chapelle du roy} as the pitch of famous Parisian organs, this being about a tone lower than \textit{ton d'orchestre} and a semitone lower than \textit{ton de la chambre du roy}.
According to Nivers, \emph{ton de chapelle} was common for convent organs (`tel que sont ou doivent eſ\kern -0.5pt tre ordinairement les Orgues des Religieuses').\footcite[106]{NiversDissertationchantgregorien1683}
We can ascertain that during the 1850s taste for \textit{ton d'orchestre} (\emph{a}$^\prime$ = 434Hz) became established in Paris, yet the prevalent pitch of \emph{orgues accompagnateurs} in the 1830s and 40s was still \textit{ton de chapelle} (\emph{a}$^\prime$ between 370Hz and 392Hz).\footcite[97--8, 117, 330]{HaynesHistoryPerformingPitch2002}
Cavaillé-Coll's new instrument for Saint-Thomas-d'Aquin was initially tuned to \textit{ton de chapelle} but was subsequently sharpened to \textit{ton d'orchestre} at the request of `des artistes', and within five years of the installation of the \textit{orgues de chœur} at Saint-Roch in 1845 and at Sainte-Madeleine in 1846, Cavaillé-Coll had to retrofit these instruments with transposition mechanisms to allow accompaniment of masses `en musique' as opposed to those `en plain-chant'.\footcite[Cavaillé-Coll to M. l'abbé Pelletier, curate of Saint-Aignan d'Orléans, 28 June 1851, in][789]{DouglassCavailleCollmusiciansdocumented1980}
In some accompaniment books, the two genres of music were made distinct in adjacent music examples (identical but for their notations), the one in quadratic notation on a four-line staff and the other in modern notation on a five-line staff (\cref{mus:musiqueplainchant}).\footcite[26]{BruneauMethodesimplefacile1856}
The view that the principles governing the `true \emph{tonalité} of Gregorian chant' were different to those underpinning harmony in modern music became a common one, and even took up the entire first part of a textbook by the abbé B.~A. Bauwens.\footcite[p.~xiii]{Bauwensplainchantmisportee1861}
We shall return to the discussions concerning \emph{tonalité} later in this chapter.

\subsection{Organists and pianists as church musicians}
Alongside the École Choron, the Paris Conservatoire became a leading centre of church music training.
Charles-Simon Catel (1773--1830) had been appointed as teacher of harmony and counterpoint in 1795, Jean-Louis Adam (1758--1848) as teacher of piano and Luigi Cherubini (1760--1842) as director in 1822.
François Benoist (1795--1878), who had been a pupil of Adam's and Catel's from 1811, was appointed as the Conservatoire's first organ teacher in 1819 having won a \emph{premier prix de piano} that enabled him, as a \textit{pensionnaire du gouvernement français}, to pursue further study in Rome and Naples.
Benoist incorporated a method of chant accompaniment into his organ lessons as a stepping stone for improvisation that, until century's end, saw the addition of contrapuntal parts to a plainchant melody placed in the bass part.
\hlabel{hl:mine}%
Danjou recollected in 1848, however, that chant was also being placed in the top part during the 1820s (\cref{mus:danjou-conservatoire}), \footcite[11]{Danjouaccompagnementplainchant1848} but this was not formalised in the regulations of the \emph{concours} until 1851.
Benoist's process required the student to devise the opposite outer part of the texture before working out or improvising the inner parts.
This was not intended as a vocal accompaniment; rather, it provided a graduated exercise for organ students to arrive at an increasingly elaborate contrapuntal improvisation, namely the \emph{fugue d'école}.\footcite[82]{JuttenEvolutionenseignementimprovisation1999}

A harmony treatise of 1855 by Auguste-Mathieu Panseron (1795--1859), written for the training of pianists, contains Benoist's exemplification of that process.
Benoist called the initial procedure the simple accompaniment (`accompagnement simple', \cref{mus:benoist-simple}), and a more elaborate procedure---incorporating suspensions and dissonant passing notes---the accompaniment with dissonances (`avec dissonances', \cref{mus:benoist-dissonances}).\footcites[251]{PanseronTraiteharmoniepratique1855}[41]{Nisardvraisprincipesaccompagnement1860}
That this procedure, initially aimed at Benoist's organ students, should have made its way into a manual aimed at pianists is surprising.
The introduction of instruments such as the poïkilorgue and the harmonium into French churches together with the small number of organists being trained at the Conservatoire strongly suggest that amateur pianists constituted the primary cohort of keyboard players in French churches.
According to one observer, by the middle of the 1840s the dearth of trained organists meant that the instruments in many cathedrals, collegiate chapels and villages in Belgium too had been abandoned to pianists (`des orgues sont abandonnées à des pianists').\footcite[206]{Janssenvraisprincipeschant1845}
On the one hand, the Nancy musician Joseph Régnier claimed that the piano, being incapable of sustaining notes, produced an undesirable effect in an accompaniment of chant, and that pianists were incapable of producing diatonic harmonisations.\footcite[403]{Regnierorguesaconnaissance1850}
\hlabel{hl:woestyn}%
On the other hand, the author Eugène Woestyn (1813--88) was concerned with the lack of opportunities for pianists as composers, and thought they might be better served by careers as church musicians.
To that end, Woestyn brought out an introductory manual attempting to teach the basics of plainchant and the piano, but given that his publication, which was aimed at amateurs, contained nothing more than a glossary of terms and no music examples, its scope (and hence presumably its influence) was severely limited.\footcite[13]{Woestynlivrepianisteplainchant1852}

\hlabel{mn:benoistmine}%
One of Benoist's pupils Jacques-Claude-Adolphe Miné (1796--1854) published verset-like fauxbourdon harmonisations of chant in 1836 in which the bass part extends lower than \emph{C}, inviting the suspicion that they might have been composed for the piano.\footcites[32, 64--6]{MineMethodeorgue1836}
But one must take account of what Jean-Jacques Rousseau had described in 1768 as the five-octave `clavier à ravalement' whose compass extended down by a perfect fifth and up by a perfect fourth resulting in the range $\emph{F}\kern -1pt`$--\emph{f}$^{\prime\prime\prime}$.\footcite[pp. 405, unpaginated `Planche 1', fig.~1]{RousseauDictionnairemusique1768}
Manual keyboards \emph{à ravalement} were supposedly a rarity in 1785, however, when one encyclopedia described only `ravalement au clavier de pédale'.\footcite[77--8]{Encyclopediemethodiqueou1785}
Even further complexity was borne of some organs not always matching the compass described by Rousseau, extending below \emph{C} not by a perfect fifth but by a minor third to $\emph{A}`$ instead.
`Ravalement' was used most notably in compositions by Alexandre-Pierre-François Boëly (1785--1858) whose pedalboard \emph{à ravalement} of the last type permitted excursions to $\emph{A}`$ in the \emph{Messe du jour de Noël}.\footnote{See, for instance, the preface and `Rentrée de la procession' in \cite{BoelyMessejourNoel}; Further examples of Boëly's use of `ravalement' may be consulted in \cite[9]{BoelyPieceschoisiespour}.}%

The possibility that Miné depended on `ravalement' supports his claim that the harmonisations he composed were suitable for both the organ and the piano, at least as far as range was concerned.
But the textures of his accompaniments were arguably not idiomatic for either,\footcite[33]{ChristensenStoriesTonalityAge2019} the doublings quoted in \cref{mus:mine-fauxbourdon} being reminiscent of multiple stops sounding at different pitches on the pipe organ.
\hlabel{pg:mine_lauda-sion}%
Miné's publication \emph{Organiste accompagnateur} of 1845 was a pioneering attempt at presenting harmonisations of common chants, yet the passage quoted in \cref{mus:mine_lauda-sion} gives ample justification to Fétis's verdict that Miné's work was very defective and full of errors (`très défecteux et rempli d'erreurs').\footcites[7]{MineorganisteaccompagnateurRecueil1845}[148]{FetisBiographieuniversellemusiciens1867}

Several notable organists harmonised chant according to Benoist's procedure for the Conservatoire's \emph{concours} including César Franck (1822--90), whose harmonisation in 1842 was summarised by the jury thus: `bass fair, upper parts excellent'.
Following Franck's succession of Benoist as organ teacher at the Conservatoire in 1872, nothing about the procedure was altered beyond a simple name change from `choral' to `plain-chant'.\footcite[The chant that Franck harmonised has been preserved at the \emph{Archives nationales} and is reproduced in][pp.~149--50, p.~257 nn.~7--8]{OchseOrganistsOrganPlaying2000}
An example of the procedure as remembered by Franck's student Charles Tournemire (1870--1939) shows that the chant was transposed by the simple interval of a perfect fourth on migrating between top and bottom voices,\footnote{\covid{}\cite[105]{TournemirePrecisexecutionregistration1936}.} a trait also remembered by Louis Vierne (1870--1937) in a recollection of Franck's succession of Benoist to which we shall return below (\cref{sc:vierne_choral}).
%\nowidow[2]

The first decades of the nineteenth century had seen attempts by instrument makers and inventors to simplify keyboard instruments to allow relatively untrained players to play or transpose with relative ease.
A design for a `unifingered keyboard' (`clavier solidoighté') debuted at the Nantes exhibition during the 1860s and contained alternating white and split black notes to yield the same fingering for all major and minor scales and arpeggios.\footcite[4, 10]{DelcampMethodeelementairerelative1861}
Although there is no apparent evidence that the keyboard was incorporated into any instruments, it was described more than twenty years before Paul von Jankó took out a more famous patent for what was essentially the same system.
Another popular invention was the transposing keyboard, generally called the `clavier transpositeur' or `clavier mobile' on account of the player's being able to move the keyboard left or right by a number of semitones determined by the instrument maker.
The firm Roller \& Blanchet began incorporating transposing mechanisms into their pianos during the second quarter of the nineteenth century, and there is no doubt that the invention gained widespread popularity as not just a gimmick but as a useful practical tool.\footcite[177]{DanjouMecanismemusicaltranspositeur1845}
\hlabel{hl:clergeau}%
One of the first to market such an invention for the accompaniment of plainchant on the harmonium or piano (`pour orgue ou piano') was the abbé Clergeau, whose mechanism was reported to greatly simplify the task of the accompanist.
There was nothing difficult about moving the keyboard to suit the range of a singer, and the mechanism was claimed to make accompaniment of plainchant so easy that within a few days even a child could transpose a chant to any desired pitch.\footcite[3--5]{ClergeauMecanismemusicaltranspositeur1845}
The mechanism earned an enthusiastic approbation from the then \textit{maître de chapelle} of Notre Dame de Paris, Joseph Pollet, who recommended it for use by very mediocre organists (`des organistes très-médiocres'); these were probably pianists feeling their way around an organ.\footcite[15--16]{PolletRapportadresseau1845}
Further mechanisms were developed by Nisard, whose \textit{clavier grégorien} saw the light around 1850,\footcite[357]{FetisCorrespondance2006} and by François Guichené, who incorporated a transposing mechanism into a more elaborate accompaniment device to be examined in more detail below.

Clergeau's influence was considerable, and La Fage even dedicated a treatise on accompaniment to him, citing him as an ardent propagator of the organ in French churches and chapels.\footcite[3]{LaFageRoutinepouraccompagner1860}
Doubtless, the simplicity of his invention and canny marketing at amateur musicians paved the way for its widespread adoption in churches across the country.
\hlabel{hl:bare}%
Alexandre Bruneau in 1856 and Eugène Baré in 1884 recognised the mechanism's usefulness to amateur musicians,\footcites[5]{BruneauMethodesimplefacile1856}[20]{BareNouvellemethodesimple1884} while Émile Amiot and Philippe Morin make explicit reference to the device in their publication's title,\footnote{\covid{}\cite{AmiotMethodeelementaireaccompagnement1862}.} as did Eugène Henry who claimed to have devised a method of accompaniment suited to keyboards transposing or otherwise (`avec ou sans clavier transpositeur').\footcite[29--31]{HenryMethodepouraccompagner1869}
That the mechanism should be mentioned in such manuals at all validates Clergeau's claim that the device held widespread appeal.\footcite[3]{ClergeauMecanismemusicaltranspositeur1845}
As late as 1892, the music publishers E.\ Fromont published advertisements and even offered cash discounts (`escompte au comptant') on a range of transposing harmoniums costing between 210~F.\ and 3,900~F.\footnote{\cite{Calonneharmonieappliqueeau1892}, unpaginated back page.}

\section{Growing markets for instruments, manuals and accompaniment books}
\subsection{Instrumental automation}
The demand exerted by practically-minded amateurs gave rise to further specific developments in instrumental design.
Mechanisms to simplify the accompaniment of the Mass or Office gained widespread popularity because they were not much more expensive than a basic harmonium and were much cheaper than a pipe organ.
The accompaniment of Mass and Vespers by barrel-organs was already taking place in France by the end of the first decade of the nineteenth century, and it was suggested in 1821 that Germany's village churches might also benefit from their use.
Charles-Marie Widor (1844--1937) recalled hearing \emph{Adeste fideles} played on a barrel-organ outside his window in 1904, noting that the tune had been removed from its ordinary liturgical context and placed on the street.\footcite[59]{Widorrevisionplainchant1904}
But far from the typical characterisation of the street-side `Orgue de Barbarie', the barrel-organ actually struck a happy ideological balance for amateur practitioners and seasoned theorists alike.
Although it did not fit the solemnity of the church service, the instrument's use in such a setting pleased La Fage because polyphonic accompaniments of good quality could be performed by anybody:

\simplex{Les morceaux étant notés, sur le cylindre, à quatre parties, le chant est accompagné d'une manière toujours uniforme et en quelque sorte mechanique~; mais au moins elle peut faire supposer la présence d'un organiste instruit dans les élémens [\emph{sic}] de l'harmonie.}
  {\cite[198]{LaFageOrgueCabias1834}}
{With the pieces being pinned on the \linebreak{}cylinder in four parts, the chant is accompanied in a way that is always the same and, in a manner of speaking, mechanical, but at least the method of accompaniment assumes the tacit presence of an organist learned in the particulars of harmony.}
\noindent
The cylinders bearing the chant accompaniments could therefore be pre-notated by trained musicians whose presence was not required at the service.
As late as 1846, Cavaillé-Coll advertised a type of instrument with eleven cylinders that acted on five foundation stops---Bourdon, Prestant, Doublette, Nazard, Petite flûte and Clairon.
On his cylinders were pinned religious airs, offertories, cantiques, sorties and the plainchant Offices for the whole year (`les offices en plain-chant pour toute l'année') and cost 300~F.\ plus delivery.\footcite[626]{DouglassCavailleCollmusiciansdocumented1980}
The usefulness of such devices was limited by the availability of cylinders, whose proprietary nature made barrel-organs vulnerable to changes in the chant repertory that were to take place early in the next century (see \cref{sc:new_edn} below).

Several inventors began designing mechanisms with which parishes could retrofit their harmoniums.
They retained a degree of flexibility where repertory changes were concerned without sacrificing the desired simplicity of execution.
One such mechanism was designed by the Jesuit and chant editor Louis Lambillotte (1796--1855), about which no details are known, and another was designed by one Mr Porchet in 1888, who brought to light his \textit{Clavier lecteur-harmonisateur instantané de plain-chant, applicable aux instruments à clavier usuel, orgues, harmoniums, pianos}.\footcites[5]{NisardNoticevietravaux1863}[100]{Bulletinofficielpropriete1888}

Abbé Jean-Louis Cabias, a curate of Pontigny, invented the `orgue simplifié' or `orgue-Cabias' and devised a grid notation to represent the chromatic layout of its keyboard.
A set of adjacent rectangles denoted the keys of the instrument that obviated the need for an amateur to read musical notation (\cref{ex:cabias_notation}; my transcription is given in \cref{ex:cabias_notation_transcription}).\footcite[p.~198, musical supplement fig.~1]{LaFageOrgueCabias1834}
Instead, the player used the index finger of each hand to play a succession of keys indicated by the ascending series of numbers in the grid, repeated notes being indicated by a cross.
There are two conflicting descriptions of the sound this instrument made as reported by witnesses to it at the Industrial Exposition of 1834.
According to the official report, the mechanism produced 5/3 chords above the chant note in the bass part thereby making an endless succession of consecutive fifths unavoidable.\footcite[208--209]{FrancoeurRapportfaitpar1831}
According to La Fage, however, the mechanism in fact produced a unisonous accompaniment, a statement he would repeat some two decades hence.\footnote{\cites[198]{LaFageOrgueCabias1834}[148]{LaFageQuinzevisitesmusicales1856}.}
Later in the century, Baré criticised such automated mechanisms for their bland chord-against-note rhythmic style that was claimed to be imperfect (`l'étudiant en reconnaît bien vite la grande imperfection').\footcite[6]{BareNouvellemethodesimple1884}
Such a style will be examined in more detail later in this chapter.
\nowidow[2]

Around 1840, the abbé François Larroque invented a mechanism called `orgue milacor' which, according to La Fage, was the first such device to produce chords from a single key press,\footcite[145]{LaFageQuinzevisitesmusicales1856} but it too relied on a `new method of musical notation, introducing numbers and colours'\footcite[89]{HughesNilesNationalRegister1840} to allow an untrained player to accompany chant without deviating from the rules of harmony (`accompagner toute sorte de plain-chant sans s'écarter des règles de l'harmonie').\footcite[486]{deMerlineuxMemorialencyclopediqueprogressif1839}
Although the system won the `orgue milacor' a gold medal at the Toulouse Exposition des produits des beaux-arts,\footcite[115--6]{Expositionproduitsbeauxarts1840} its equal reliance on a proprietary system of notation must surely have limited its usefulness.

\hlabel{hl:guichene}%
During the 1850s, the abbé François Guichené (1808--77) incorporated a transposer in a new invention he called the `Symphonista',\footcites[207]{LessmannRezeptiongregorianischenChorals2016}[4--5]{DouglassCavailleCollmusiciansdocumented1980} a mechanism that signified the apogee of automated plainchant accompaniment.
The upper keyboard comprised a row of keys that produced 5/3 and 6/3 chords depending on which key was pressed, and a printed chart showed the player how to select the key appropriate to each successive note of the chant.
The inventor's understanding of ecclesiastical harmony was designed into the mechanism, and graphical lines indicated to the player the succession of keys that were to be avoided which would otherwise produce progressions of consecutive fifths or octaves.
6/3 chords were thus interspersed between 5/3 chords (the latter being indicated by a cross on the chart), and the flexibility of the mechanism allowed the player to accompany any chant edition without consecutive perfect consonances by simply playing single notes.\footcite[297--301; Elevation, cross-section and plan views of the `Symphonista' are printed on plate 177]{ArmengaudSystemeharmoniquedit1856}
The invention excited the curiosity of attendees at the 1855 Exposition universelle, among whom were Napoleon~III and Fétis, and the latter recorded his views on the `Symphonista' in the official report, as relayed by Nisard:
\nowidow[2]

\simplex{Le clerc de village, ou le chantre d'une petite église, qui ne connaît que le plain-chant, tel qu'il est dans les livres de chœur, peut, en posant le doigt sur la touche du clavier supérieur, dont le nom répond à la note du chant, faire entendre une harmonie complète et redoublée dans plusieurs octaves, et accompagner ainsi sa voix.

    \parindent=10pt
    Quelles que soient les suites des notes du chant, les successions des accords sont conformes aux règles d'une bonne harmonie.

    Si le chantre est musicien et organiste, il peut accompagner sur le clavier inférieur, comme on le fait avec un orgue ordinaire, et les harmonies réglées par le clavier supérieur ne se font plus entendre.}
      {\cites[6--7]{NisardNoticevietravaux1863}[See also][207]{LessmannRezeptiongregorianischenChorals2016}}
    {The village cleric, or the singer at a small church, who knows only plainchant, as it is presented in choir books, can, by placing a finger on the key of the upper keyboard, the name of which corresponds to the note of the chant, cause a full harmony to be heard doubled in several octaves, and to accompany his voice in this way.

    \parindent=10pt
    Whatever the succession of notes of the chant may be, the succession of chords conforms to the rules of good harmony.

    If the singer is a musician and \linebreak{}organist, he can accompany on the lower keyboard as one does on a regular organ, and the harmonies regulated by the upper keyboard will no longer be heard.}
\noindent
The favourable reception of the `Symphonista' at the Exhibition universelle earned Guichené a first-class silver medal and the \textit{Légion d'honneur}, and subsequently the firm of Houdart built and marketed the instrument with optional extras.
A `Symphonista' capable of playing the chant in the top or bottom part, and comprising two harmonic systems (one `à la Palestrina' and another `en harmonie moderne'), cost up to 1,500 F., and enabled amateurs to produce grammatically correct accompaniments without reference to any rule books.

\subsection{Methods, claims, and degrees of authoritativeness}
The closure of the \emph{maîtrises} during the French Revolution did not stifle the demand for teachers and methods of plainchant and its accompaniment.
Manuals were published to show amateur players how to accompany parish services without automated mechanisms or detailed reference to the rules.
The authors of such manuals may be classified into three groups.
\noclub[2]

The first group comprised musicians and scholars of international repute whose stances on plainchant harmony and rhythm carried national and international weight.
Jaak-Nikolaas Lemmens (1823--81), César Franck, Théodore Dubois (1837--1924), Alexandre Guilmant (1837--1911) and Charles-Marie Widor were among those conservatory musicians who composed accompaniments or wrote didactic texts on the matter, either to dispel myths or to promote appropriate principles.
We will return to the thoughts of such musicians during the course of this chapter and the next.

The second group comprised cathedral musicians composing accompaniments in a style approved by the diocese for adoption in seminaries and parish churches (although it was also common for dioceses to adopt the publications of individuals from the first group too).
César Franck's teacher Dieudonné Duguet (1794--1849), organist of Liège cathedral, published such a collection aimed at young organists and at those whose incomplete or superficial study of harmony, counterpoint and chant meant they could not construct accompaniments on their own.\footcites[173]{OchseOrganistsOrganPlaying2000}[3]{DuguetLivreorguecontenant}
Duguet's manual contains little descriptive prose; instead, the player needed only to learn those accompaniments notated on the page to rest content that the reputation of the harmoniser lent authority to their playing.\footcite[468]{Nouvelleslitteraires1848}
Franck himself, in collaboration with Lambillotte, later published a book of accompaniments with a wide scope, these being accompaniments of `chants communs' that were applicable to the rites of multiple dioceses.
Franck recognised it was `urgent' to offer such accompaniments so the organ would not become a hindrance in untrained hands.\footcites[p. iii]{LambillotteChantgregorien1857}[17]{SmithPlayingOrganWorks1997}
While it was common for the title pages of such publications to advertise their wide applicability with `à l'usage de tous les diocèses' or, with the rise of Ultramontanism, `selon le rite romain', it was impossible to provide coverage for every chant in use, particularly when new chant editions continued to be adopted hither and thither.
For this reason, more adaptive methods were called for.
\nowidow[2]

The third group consisted of those authors who responded to the demand for simplified rules.
Condensed manuals made it seem as though musical rules were simple to understand, master and execute.
Efforts to condense such a vast topic as chant harmonisation into a few pages had mixed success, however, because some approaches left too many aspects of the topic undiscussed.
For example, Jules de Calonne dedicates most of his four-page pamphlet to chord construction and limits his discussion of harmonic progressions to two ascending and descending scales, major and minor, harmonised according to the rule of the octave; failing, in other words, to demonstrate how those progressions might apply to chant accompaniment.\footcite[1--4]{CalonnePetitguideaccompagnateur1859}
\hlabel{ln:cg_rules}\hlabel{hl:cg_rules}%
The two-page, pocket-sized manual by the author `C.~G.' advertises its universal applicability to all chant editions (`toutes les éditions de chant') and sets out rules that were claimed to generate a chorale-like accompaniment.
The chant is placed in the top part and is always assigned to the fifth finger of the right hand (\cref{fig:cg}).\footcite[1--4]{G.Accompagnementplainchant1884}
Each possible plainchant note is provided in \emph{solfège} notation with two or three numbered chords.
Chains of chords are created by the accompanist by plotting each note of the chant in the right-most column and by playing the bass note and inner notes suggested in the left-most and centre columns.
Ideal harmonic progressions were to be created by starting on the chord numbered 1 and linking chords in such a way that no two with the same number followed consecutively.
For example, an accompaniment could be constructed using the sequence 1~\rightarrow{}~2~\rightarrow{}~3~\rightarrow{}~2~\rightarrow{} 1, creating a progression of chords with the chant note as root, third, fifth, third, and root respectively.
The sequence was always to terminate on a chord marked 1, in root position.
Chords were labelled additionally with letters: \emph{(a)} to be used in first, second, fourth, and sixth modes; \emph{(b)} never to be used in those modes; \textit{(c)} to conclude third and fourth modes; \textit{(d)} to be used often in seventh and eighth modes (particularly at terminal cadences), and \textit{(e)} to be used in seventh and eighth modes.
\hlabel{ln:cg_rules_END}%
\noclub[2]

The self-studying amateur was generally the target market for such didactic methods, and some authors (many amateurs themselves) took advantage of their audience's lack of knowledge to print unsound or, frequently, inaccurate principles.
Many manuals bear specious titles and frequently advertise bogus claims.
\hlabel{hl:calonne_abc}%
It was surely to piggyback on the success of Panseron's attractively-entitled \emph{solfège} manual \textit{L'A~B~C~musical, ou solfège} and Danjou's suggestion that practical rather than theoretical approaches were more appealing to amateurs that Jules de Calonne brought out another accompaniment manual entitled \textit{A.~B.~C.~de l'harmonie appliquée au plain-chant}.\footcites[218]{DanjouExamendiversesmethodes1845}[1]{Calonneharmonieappliqueeau1892}
  \hlabel{hl:kaltnecker}%
This format was reprised by Maurice Kaltnecker with \textit{L'A~B~C~du jeune accompagnateur} some fifty years later.\footnote{\cite{Kaltneckerjeuneaccompagnateur1937}.}
The alphabet implied that the methods were so elementary that even a child could grasp them, and authors attempted to trick prospective buyers with that purported simplicity.
One went as far as to insult the reader's intelligence should his method prove to be incomprehensible (which, as in all too many cases, it of course was):
\noclub[2]

\simplex{Le système en est d'une si grande simplicité qu'on le ferait comprendre en peu de temps, même à un enfant d'une intelligence ordinare~: une expérience suffisante autorise cette affirmation.}
  {\cite[1]{GodardTraiteelementaireharmonie1851}}
{The system is one of such great simplicity that you or even a child of ordinary intelligence could be enabled to understand it in no time: enough experience supports this affirmation.}
\noindent
Jules Carillion's manual supposedly taught chant and its accompaniment in five lessons, but the author absolved himself of the cursory explanations he provided by claiming his method was merely a stepping stone to the more technical methods of Amédée Gastoué (1873--1943) and François Brun (see \cref{sc:gastoue,sc:brun} below):

\simplex{Il m'a semblé que dans une sphère moins élevée, plus accessible peut-être à ceux qui n'ont pas déjà fait d'études spéciales, il y avait place pour une méthode plus modeste qui puisse servir de cours préparatoire, si l'on veut, aux manuels précités.}
  {\cite[5]{Carillionaccompagnementchantgregorien1916}}
{It seemed to me that on a less refined level, one more accessible perhaps to those who have not yet undertaken special studies, that there was room for a more accessible method that might serve as a kind of preparatory lesson to the above manuals.}

\pagebreak
\noindent
The abbé Bourguignon's method claimed to teach chant accompaniment in three months provided the student followed the special set of rules laid down by the author, but this can hardly have been possible given how little descriptive material his manual contains.\footcite[2, 11]{BourguignonMethodeelementaireharmonie1899}

It was common for authors to preface their methods of accompaniment with step-by-step guides on how to read the notes of the stave and how to construct chords, particularly in those manuals aimed at musicians without musical or academic backgrounds:

\simplex{Nous avons donc pensé qu'une méthode simple et facile d'accompagnement du plain-chant sur l'harmonium à clavier transpositeur, précédée des principes élémentaires de la musique et du plain-chant, pourrait être de quelque utilité à M.M.\ les Ecclésiastiques et Instituteurs de la campagne surtout, pour leur aider à former de jeunes organistes.}
  {\cite[6]{BareNouvellemethodesimple1884}}
{We thought therefore that a simple and easy method for plainchant accompaniment on the harmonium with a transposing keyboard, preceded by the basic principles of music and plainchant, could be of some use to ecclesiastics and country teachers especially, to help them in educating young organists.}
\noindent
Léon Dalmières used the Socratic method to present opposing views of the debate, and in music examples kept the player's hands in close proximity to one another for amateurs with elementary keyboard technique.\footcite[5, 50--52]{DalmieresPlainchantaccompagneau1856}
Didactic methods on chant accompaniment were usually aimed at a specific group, such as seminarians, young organists or the author's own students, and in some cases such authors relied on anecdotal evidence to back up the reliability of their methods.
J. B. Jaillet intended his method for young ecclesiastics and sought to make the modes and the modulation method (see \cref{sc:modulation} below) easy to understand:
\pagebreak{}

\simplex{Le désir de nous rendre utile, en aplantissant ces difficultés autant que possible, nous a porté, après avoir cherché pendant longtemps, à réduire en règles simples et peu nombreuses ce que nous pratiquons nous-même, quand nous accompagnons~; et nous devons dire que les résultats qui ont constamment couronné notre enseignement, ne nous permettent pas de douter de l'exactitude de ces règles, de leur utilité, et de leur étonnante fécondité.}
  {\cite[p. v]{JailletMethodenouvellepour1857}}
{Our desire to make ourselves useful, by smoothing out these difficulties as much as possible, has led us, after a long search, to summarise by a few simple rules what we practice ourselves when we accompany; and we must say that the results which continually perfected our teaching never led us to question the precision of these rules, their usefulness, and their astonishing fruitfulness.}

Since such publications were also available for purchase by the public at large, they sometimes gained popularity beyond their intended audience.
\hlabel{hl:eugenehenry}%
Eugène Henry had intended his manual to be used by his own students, but it gained a wider readership than he had expected and sold out its first two editions, obliging the author to change the method and to widen its scope.\footcite[unpaginated `Avertissement de l'auteur']{HenryMethodepouraccompagner1889}
Louis Müller also published a summary of his personal classes, but this, in contrast to Henry's method, cannot have been all that useful to public readers as it contained little descriptive material.\footcite[unpaginated preface  col.~1]{MullerPetittraiteharmonie1880}
J.~B.~Hingre had administrative reasons for publishing his method, as it meant he did not have to repeat himself to past pupils or to devise plans anew for future classes.\footcite[3]{HingreMethodeaccompagnementplainchant}

A trait common to many manuals was to make a concession to simplicity that ignored the modal question entirely, and instead to instruct the student in a simplistic harmonic formula.
Joseph Alémany in this way taught that chant ought to be accompanied using major and minor scales harmonised using the `règle d'octave',\footcite[19]{AlemanyMethodesimplefacile1862} whereas Bruneau presented a slightly more nuanced approach by claiming the third, fifth, sixth, seventh and eighth modes were to be accompanied in the harmonised scale of C major, and the first, second and fourth were to be accompanied in that of D minor.\footcite[p.~v]{BruneauMethodesimplefacile1856}
\hlabel{hl:dubois}%
Even Dubois relied on major-minor harmony in his manual, claiming that a conflation of modes with major and minor scales was justified by the needs of his manual's intended users:

\simplex{Notre travail n'est pas scientifique~; il s'addresse à ceux qui ne savent pas et ne peuvent pas apprendre~; il est simplement \textit{pratique}, et nous croyons qu'il peut rendre de véritables services.}
  {\cite[1]{DuboisAccompagnementpratiqueplainchant1884}}
{Our work is not scientific; it is aimed at those who are ignorant and who cannot learn. It is simply \emph{practical}, and we believe that it can be genuinely useful.}
\noindent
That concession to simplicity probably benefited those who already had a basic understanding of major-minor harmony without much of an idea about modality.
An organist of Coutances, the abbé Falaise, harmonised seventeen scales a player could use to accompany chant according to a major-minor understanding.
The author suggested, for example, that harmonised scales could be placed in the bass part of the following keys: C, D, E\kern 1pt\flat{}, E, F, G, A\kern 1pt\flat{}, A, B\kern 1pt\flat{}, B major; C, D, E\kern 1pt\flat{}, E, F, G, A minor.
Those could then be used to accompany the modes of plainchant, which were broken down into two, major and minor.\footcite[85--104]{FalaiseMethodetheoriquepratique1876}

Pre-harmonised scales formed the basis of countless elementary methods of plainchant accompaniment, and although Nisard would not resile from his views that plainchant was modal, he adopted a mnemonic approach for his practical method that sought to simplify matters.
Rather than providing a set of harmonised scales, however, Nisard required players to memorise a finite set of chords:

\simplex{Six lignes et demie d'accords et dix-sept exceptions, voilà donc tout le fonds de notre opuscule. A coup sûr, quiconque se sentirait incapable de meubler sa mémoire d'un si léger bagage, devrait renoncer à tout idée d'accompagner le plain-chant sur l'orgue.}
  {\cite[41]{Nisardaccompagnementplainchantorgue1860}}
{Six and a half lines of chords and seventeen exceptions, that is the basis of our pamphlet. Certainly, whoever might feel incapable of filling their memory with such a light load should give up any idea of accompanying plainchant on the organ.}
\noindent\hlabel{mn:nisardsimple}%
In another approach, La Fage's \textit{Routine pour accompagner le plain-chant} of 1860 differed from the reliance of Falaise and Nisard on scales and chords.
It presents pre-harmonised intervals and cadences as progressions of two or more chords in a supposedly exhaustive list, with reference to which players were supposed to concatenate a continuous accompaniment.\footcite[5--6]{LaFageRoutinepouraccompagner1860}
This method was highly accessible because it was neither limited by the chant of a specific edition nor by the memory of the player.
Rather, the burden of working out the part writing was borne by La Fage himself, thus allowing the reader to construct grammatically correct harmonisations without needing to learn any rules.

\subsection{New notational and annotative systems}
The nineteenth century was awash with systems to aid the visually impaired in reading verbal text, and some pioneers of such systems extended them to music too.
Louis Braille (1809--52), the blind inventor of the international tactile writing system that bears his name, and who also served as organist of the Parisian church of Saint-Nicholas-des-Champs, extended his system to the notation of music.
His embossed, cellular symbols, as codified in 1829, represented either a textual character or an Arabic numeral, depending on the context.
Similarly, those symbols were ascribed musical meanings and could represent pitches, rhythms and various supplementary indications.
Moreover, by converting quadratic notes into modern ones, Braille simplified the notation of chant so it too could be represented by his symbols.\footcite[18--20]{BrailleProcedepourecrire1829}

In a similar way, the complexity of the rules of plainchant harmonisation led other pedagogues to devise symbols or alternative forms of musical notation of their own, in attempts to make the rules accessible to less practiced players.
Authors who did so form a largely forgotten subculture in nineteenth-century France, and many are not known to posterity as organists, composers or music theorists.
Nonetheless, whereas each of them followed distinct courses of action, two general approaches can now be discerned.

The first approach attempted to replace musical notation by representing some concept, chord, harmony or rhythm by a glyph.
Arguably, however, some authors overloaded their glyphs with so much information that no advantages were gained over regular five-line notation.
The player had first to learn what the glyphs represented, which were generally based on a proprietary notation devised by authors according to fanciful and imaginative schemes, and had then to apply those pre-learned rules to accompaniment books printed by the same authors.
As we have seen in connection with the `orgue-Cabias', that approach was vulnerable to changes in the chant repertory, and methods that relied on proprietary notations often had short shelf lives.
\noclub[2]

\hlabel{hl:dedun_troisdun}%
Abbé Dedun's `three-in-one' system embodied that approach by attempting to combine melodic, harmonic and rhythmic information into a single glyph (\cref{ex:dedun_troisdun}).
The placement of the glyph on the staff gives the melodic note, the letter represents the \emph{solfège} equivalent of the root of the chord (the lozenge indicating G major or G minor, depending on its orientation) and rhythmic information is represented with stems and flags.
Further rules make the system even more complicated.
An adjunct letter \textit{r} (not to be confused with the same, larger letter signifying `ré', or `D') signifies that the bass note is not to be doubled anywhere by the right hand, but that the chant note is to be doubled instead.
A dot above a symbol signified a first inversion chord; a dot below, a second inversion.
A dot to the left of the symbol signified a seventh chord, while a black circle or square signified no chord at all.
In making the method so condensed, Dedun also made it inimical to easy assimilation.
The dizzying quantity of information conveyed by his unfamiliar glyphs gave the system few ready advantages over modern notation and probably alienated the amateur musicians at which it was aimed.\footcite[10]{Dedunsystemetroisou1889}
\noclub[2]

The second approach attempted to represent some chord or harmony by a single symbol, letter or number placed above or below the staff.
This annotative approach seemed ideal for music pedagogy because the annotations could be inscribed into any chant edition, while making a system's rules appear accessible to amateur players unfamiliar with chord construction, harmony and the rules of part-writing.
Three categories of annotative systems can now be observed, each defined according to what they required of the player.
The first category relied on the player to work out the required chord from the annotations, the second required players to recall a chord from memory when prompted by the annotation, while the third was a hybrid of the first two with the possibility of communicating more elaborate rhythmic frameworks.

The first category can be exemplified with the primitive system devised by Charles Duvois (\emph{c}.1830--\emph{c}.1892) in 1844 (\cref{ex:duvois_tantumergo}; my realisation is shown in \cref{mus:duvois_realised}).
\hlabel{mn:duvois}%
The Arabic numerals 1, 2 and 3 signify whether the chant note ought to be the root, the third or the fifth of a 5/3 chord,\footcite[27]{DuvoisMethodeelementaireaccompagnement1844} thus making it the antecedent of the method by `C.~G.' that we discussed above (see \cpageref{ln:cg_rules,ln:cg_rules_END}).\footnote{\cite[The method was also taken up in][72--80, though it should be noted that some of the indications provided by the author either mislabel or mistranscribe certain chords]{Courtoisaccompagnementplainchantprecedee1897}; See also that system devised by J.\ Dauphin as described below on \cpageref{hl_link:dauphin}.}
\hlabel{hl:rousseau}%
Duvois's system differs from that employed by one Franz Joseph Mayer, however, whose numerical annotations represented the notes themselves rather than chords (\cref{mus:mayer_numbers}).\footcite[26]{MayerLateinischeChoralgesangefur1867}
Arthur Rousseau's system (\cref{ex:rousseau_figured}) was quite similar to Duvois's, except that the numbers indicate the bass note by its simple interval below the chant, a trait that makes these annotations like an inverted form of figured bass.\footcite[63]{Rousseaupetitharmonistegregorien1889}
6/3 chords are also permitted, the resulting major or minor sixths below the chant being annotated with an Arabic 6, and the major or minor tenths or thirds below the chant being annotated with the Roman numeral X.

\hlabel{hl:auzet}%
An analogous system devised by the abbé V.\ Auzet also permitted 6/3 chords.
These were not represented by numbers but by uppercase or lowercase letters depending on whether the given chord was major or minor (\cref{ex:auzet_angelis}; my realisation is shown in \cref{mus:auzet_realised}).\footcite[64]{Auzetaccompagnementartistiqueplainchant1891}
5/3 chords are represented as \textit{F} or \textit{f} for \emph{fondamentale} if the bass note is the same pitch class as the melody note, \textit{T} or \textit{t} for \emph{tierce} if the bass is a third below, and \textit{Q} or \textit{q} for \emph{quinte} if the bass is a fifth below.
6/3 chords are represented by \textit{S} or \textit{s} for \emph{sixte} for basses a minor or major sixth below respectively; the player added the remaining triadic notes in all cases.

The second category can be exemplified by the system devised by Charles-Louis Hanon (1819--1900), composer of the notorious piano studies, who defined finite sets of chords to be used with what were believed to be common melodic formulæ.
Hanon parsed chant melodies into these formulæ with slur-like arcs (\cref{ex:hanon_lines}; my realisation is shown in \cref{mus:hanon_realised}).
Numbers within an arc signify the set from which chords were to be selected, and the player was to accompany each note of the chant with those chords until the end of the arc.
A transposition was to be effected by substituting the C-clef with F-clef and a flat signature, arcs below the staff prescribing sets of chords more convenient to the transposed register.\footcite[52]{HanonSystemenouveaupratique1860}
Hanon's publisher advertised the manual at least until the end of the century, indicating that demand for simplified systems pledging to teach chant accompaniment `at first sight' had by then not yet abated.\footnote{See \emph{La Croix} (Supplément) \textnumero{}~3877, 19 December 1895, p.~3.}

\hlabel{hl:allard}%
A comparable system to Hanon's was devised by B.\ Allard, whose method depends on finite sets of individually numbered chords.
A different set of chords is provided for each mode and at two pitch levels to provide for a dominant of `G' or `A'.
Each chant note is then accompanied with the numbered chord from that mode's set.
\Cref{ex:allard_table} shows the chords that Allard prescribed for use with chants in the third mode, such as with `Pange lingua' in \cref{ex:allard_pangelingua}.\footcite[36--7]{AllardTranspositionaccompagnementplain1880}

As the rhythmic debate became increasingly heated, authors could no longer disregard it, and were obliged to acknowledge the emerging rhythmic idealism as best they could.
Accompaniments devised according to systems belonging to the first and second categories produced `chord-against-note' accompaniments, where each chant note was accompanied by a single chord, thereby producing the `chorale texture'.
The third category combines a memorised set of rules with the player's active participation in working out chords according to annotations.
\nowidow[2]

\hlabel{hl:frere_sebastien}%
A letter-based system similar to that published by Auzet was adopted by Fr[ère] Sébastien according to a Solesmian rhythmic method (on which, see \cref{sc:solesmes_ictus}).
For the present it may be remarked that Sébastien's system incorporates the \dagger{} symbol to designate a 5/3 chord with a doubled third and the \times{} symbol to indicate when the bass part is to move in contrary motion with the chant.\footcite[12--20, 40]{SebastienAccompagnementchantgregorien1910}
\hlabel{hl:brune}%
Émile Brune's system involved even more active participation by the player and ought to be read with caution because it incorporates elements of the systems of Duvois and Rousseau: like the former, those numerals 1, 2 and 3 referred to chords in which the bass note was the same pitch class as the melody note, a third below or a fifth below, respectively; while like the latter, the numeral 6 indicated bass notes a major or minor sixth below the chant.
The indication 6/4 was an addition of Brune's and requires no further explication.
His system, although presented in modern notation in \cref{ex:brune_figured}, would ordinarily have been annotated on quadratic notation.
The greater ratio of chant notes to bass notes required Brune to annotate each of the former to instruct the player not to change the latter.\footcite[pp.~x, 115]{BruneNouvellemethodeelementaire1903}

\hlabel{hl:aumon_biret}%
Brune's system was apparently simplified by the abbés Aumon and Biret whose numerals were analogous to those used by Duvois, except where a numeral was underlined, which indicated a 6/3 chord.
In contrast to Brune's system, however, the notation reproduced in \cref{ex:aumonbiret_figured} shows that the numerals (and the chords that they represent) apply solely to the chant note at the quaver where the numeral is placed and must be sustained until a subsequent numeral institutes a change of chord.\footcite[125]{AumonMethodefacilecomplete1926}
The methods in the third category are most similar to inverted figured bass where the harmony endures until a change is prompted by the next set of numerals.
The player requires an understanding of the rhythmic framework implied by the annotations, and both the manuals of Brune and Aumon-Biret incorporate expansive descriptions of the topic.


\section{The rationale of \emph{tonalité ancienne}}
\subsection{The modulation method}
\label{sc:modulation}%
As discussed above (\cref{sc:haberl_circle}), some chant editions show sharping or flatting effected with no clear editorial motive, and necessitated subsequent editors to rely on further accidentals to avoid outlining the prohibited intervals of the augmented fourth and diminished fifth, or leaps of the same.
In the diocese of Paris, sharping the penultimate interval of a phrase in sequences ('Proses') was common practice,\footcite[39]{G.Nouvellemethodeplainchant1829} the chant treatise by one Léon Godard (1825--63) suggesting that cadences could be made more agreeable to modern ears by borrowing some features (sharps) from modern music (`faire quelques emprunts à la musique moderne').\footcite[18]{GodardTraiteelementaireharmonie1851}
Differences of opinion as to how chant harmony should be written sparked polemics in print and at conferences, while the degree to which polemic influenced practice varied among musicians and music theorists, professional and amateur alike.

Some accompaniment manuals introduced copious accidentals into their chants as a by-product of ignoring the modes entirely, preferring instead to describe chants as being `in' major or minor keys.
Georges Schmitt (1821--1900), organist of Saint-Sulpice from 1850 to 1863, instructed the harmonisers reading his manual to treat the first, second, third and fourth modes as minor, and the fifth, sixth, seventh and eighth modes as major, this notion being derived from the interval of a third above the final of a given mode.\footcite[23]{SchmittMethodeelementaireharmonisation1857}
\hlabel{hl:burotto}%
The same notion led Clément Burotto, sometime \emph{maître de chapelle} of Saints-Pierre-et-Paul in Marseille, onto thin ice when he proclaimed that the terminal cadence on `G' in a cited tetrardus chant was incorrect because the prevalence of the pitches `F', `A', and `B'\flat{} in the chant suggested an F major key-centre instead.\footcite[22]{Burottorestaurationplainchantson1869}

In 1792, Michael Haydn (1737--1806) had set a precedent for applying a classical approach to modulation.
In his \emph{Antiphonarium Romanum}, he used predominant \rightarrow{} dominant \rightarrow{} tonic progressions in various keys to establish structural hierarchies in the course of a chant harmonisation.\footnote{Michael Haydn, \emph{Antiphonarium Romanum} MH~533, \emph{A-Wn} Mus.~Hs.~18788; \cite[116]{WagenerBegleitunggregorianischenChorals1964}.}
Prior to 1860, such a modulatory approach became the de facto norm as harmonisers demarcated points of rest with perfect cadences comprised of sharped pitches with leading note \rightarrow{} tonic functions.\footcite[34]{JailletMethodenouvellepour1857}
Conflating the modes with major and minor scales suggests that Poisson's view (see \cpageref{ln:poisson,ln:poisson_END} above) was not yet widely accepted.
\noclub[2]

Several decades following Haydn's accompaniments were composed, Fétis advanced a theory of reposeful and non-reposeful scale degrees intended to justify the distribution of 5/3 and 6/3 chords in the `règle d'octave'.\footcites[9]{FetisMethodeelementaireabregee1824}[Described in][121]{SimmsChoronFetisTheory1975}
Through him, the idea that some chords could have a more intrinsic quotient of repose may have made its way into French plainchant accompaniments.
The rule of the octave was already prevalent in German thorough bass manuals, and was held by Franz Joseph Aloys Antony (1790--1837) to be just as applicable to Lutheran adaptations of plainchant as to any chorale (`wie überhaupt bei jedem Choräle').\footcite[57]{AntonyArchaeologischliturgischesLehrbuchgregorianischen1829}
\hlabel{hl:spencer}%
Germanic sources were the basis for an English treatise on the ecclesiastical modes, whose author Charles Child Spencer (1796 or '97--1869) parsed Lutheran chorales to demonstrate how he understood J.~S.~Bach to have modulated to different modes at the end of every phrase before modulating back for the terminal cadence.\footcite[37]{SpencerConciseExplanationChurch1846}
Schmitt, a former student of Antony's, stated that chants modulated frequently from one mode to another,\footcite[23]{SchmittMethodeelementaireharmonisation1857} and the notion became widespread enough for Francophone theorists to assume the mantle of systematising the method.

For composers such as Jacques-Louis Battmann (1818--86), accompaniments would habitually modulate to the keys whose tonic triads comprehend the concluding notes of phrases (\cref{mus:battmann_34}).\footcite[34, 44]{BattmannCoursharmonietheorique1855}
For other composers, however, modulation towards the phrase-end was not without some degree of sensitivity to the structure of the modes.
The tonal palette in Miné's \emph{Méthode d'orgue} has been criticised for its `obviously full-blooded functional harmonies',\footcite[33]{ChristensenStoriesTonalityAge2019} yet Miné and others sometimes attempted to modulate to the final and dominant of the mode rather than to tonic and dominant keys, marking the `dominante' of a second mode chant with F major rather than A major harmonies.\footcite[30]{MineMethodeorgue1836}
\hlabel{hl:fessy}%
Alexandre Fessy (like Miné, a former student of Benoist's) held that chant modulated to, and temporarily rested on, multiple different modes before modulating back to the principal mode of the chant, and that it was the function of the accompaniment to mark such points of repose:

\simplex{Pour accompagner le plain-chant convenablement il faut bien observer trois choses essentielles~: le ton dans lequel la pièce est écrite ou celui où elle module, les repos et la cadence finale, afin de bien déterminer le ton principal du morceau.}
  {\cite[10]{FessyManuelorguecontenant1845}}
{To accompany plainchant properly one must pay attention to three essential things: the key in which the piece is written or that to which it modulates, the rests and the terminal cadence, in order to determine the principal key of the piece.}

Limiting the accompaniment in that way did not go far enough for some theorists, however, and the Belgian N.~A.~Janssen claimed that those of `exemplary piety' would not tolerate accompaniments containing modern (that is, profane) features like sharping:

\simplex{Toutes reconnaîtront que l'église n'est pas un théâtre ou une salle de concert, que les sons graves du plain-chant ne sont pas des ariettes ou des romances, et que l'orgue, cet instrument sublime, n'est pas une guitare ou un piano.}
  {\cite[208]{Janssenvraisprincipeschant1845}}
{All will recognise that the church is not a theater or a concert hall, that the solemn sounds of plainchant are not arias or romances, and that the organ, this sublime instrument, is neither a guitar nor a piano.}
\hlabel{ln:gevaert_theatre}%
\noindent
Janssen therefore disavowed accompaniments that modified the chant with sharps, and his examples retained purely diatonic chants to preserve what he understood to be their modal characters.
In France, movement away from harmonisations containing accidentals other than B\kern 1pt\flat{} began gathering momentum when Danjou advocated for diatonic accompaniments over sharped ones, because the latter, he claimed, changed the character of diatonic chants too much.\footcite[408]{Danjouaccompagnementplainchant1847}
\nowidow[2]

\hlabel{hl:lafage_reproduction_unac}%
While we shall return to the diatonic viewpoint below, it may be noted that some hardline musicians (including, eventually, La Fage, as in \cref{ln:lafage_unaccomp}) went even further than abolishing the sharp by seeking to abolish accompaniment altogether, considering the practice to be equally as anachronistic and therefore equally as dispensable.\footcite[141]{LaFagereproductionlivresplainchant1853}
According to the mid-century view of François-Auguste Gevaert (1828--1908), however, unaccompanied chant was not a realistic proposition for parish churches because a villager was said to find such music bland and monotonous (`fade et monotone').\footcite[Cited in][187]{LessmannRezeptiongregorianischenChorals2016}
\hlabel{int:gev_first}%
Gevaert proposed instead the type of accompaniment reproduced in \cref{mus:gevaert_notation} that admitted, among other musical elements, cadential sharping.
Where the pitch to be sharped was in the chant, Gevaert wrote `\sharp{}~indispens' to remind the player (and presumably also the singer) that one must not ignore the accidental.\footcite[21, 44]{GevaertMethodepourenseignement1856}
\hlabel{ln:paris_congress}%
Whether to permit accompaniment at all was debated by one Paul Charreire (1820--98), the organist and \emph{maître de chapelle} of Limoges cathedral, who claimed that unaccompanied chant was nothing more than an old relic (`ce n'est plus qu'une vénérable relique'), requiring some sort of accompaniment to keep the congregation engaged.\footcite[38]{Congrespourrestauration1862}
But Charreire came no closer to settling the debate on what sort of harmony to use.
Among some music theorists, then, there arose the need for a theory of plainchant \emph{tonalité} that was practicable for harmonisations, that respected the imagined archaeological heritage of plainchant, and that steered a safe course between the Scylla of decadent harmony and the Charybdis of no accompaniment at all.
\pagebreak{}

\subsection{\emph{Tonalité}, diatonicism and cadential sharping}
Antiquarians of music theory refuted the appropriateness of a modulatory method of accompaniment, and instead sought a theory of plainchant \emph{tonalité} that could evoke a solemn aesthetic when applied to harmony.
In the words of Danjou:

\simplex{Dans la tonalité du plain-chant, il n'y a pas de modulations, pas de modes mineurs ou majeurs, pas d'attraction d'une note vers l'autre, et partant, l'intervention de toutes ces combinaisons de l'art moderne constitue non seulement un anachronisme complet entre la mélodie et l'accompagnement, mais encore une altération monstrueuse du caractère essentiel du chant religieux.}
  {\cites[409]{Danjouaccompagnementplainchant1847}[Reproduced in][col.~25]{DOrtigueDictionnaireplainchantmusique1854}[See also][200]{LessmannRezeptiongregorianischenChorals2016}}
{In the \emph{tonalité} of plainchant there are no modulations, no minor or major modes, no attraction from one note to another, and following on from this, the intrusion of all these influences of modern art amount not only to a complete anachronism between melody and accompaniment, but also to a monstrous warping of the essential character of religious chant.}
\noindent
Choron was one of the first theorists to postulate a notion that music history had bifurcated with Monteverdi and the `harmony of the dominant', leading Fétis to develop a theory of \emph{tonalité} that assumed Monteverdi's works represented a watershed in the history of music.\footcite[p.~xxxix]{ChoronDictionnairehistoriquemusiciens1810}
\label{sc:fetis_inconsistent}%
Fétis applied taxonomy to the history of music to define a series of evolutionary phases in the same way a natural historian would have done at the time.
Those phases included two distinct \emph{tonalités}: the \textit{tonalité ancienne} that had preceded Monteverdi, and the \textit{tonalité moderne} that succeeded him.
\emph{Tonalité ancienne} was defined according to the \emph{ordre unitonique} in which no attractive tendencies existed between pitches.\footcite[20]{ChristensenStoriesTonalityAge2019}
In \emph{tonalité moderne}, however, there existed three \emph{ordres}: the \textit{transitonique}, which allowed modulation to different keys using the dominant seventh; the \emph{pluritonique}, which allowed modulation through enharmonic respellings to bring unrelated keys closer together; and the \emph{omnitonique}, whose reliance on altered chords allowed multiple different tonics, making it the most futuristic.\footcite[127--32]{SimmsChoronFetisTheory1975}
According to Fétis, Monteverdi had instigated a visceral shock among the listening public of the seventeenth century by causing the \emph{ordre unitonique} to yield to the \emph{ordre transitonique} through the use of an unprepared dominant seventh in `Cruda Amarilli' from the fifth book of madrigals of 1605.\footcites[pp.~xliii, 250]{FetisTraitecomplettheorie1867}[14]{Dahlhaustonaliteharmoniqueetude1993}

\label{sc:fetis_sharping}%
While the \emph{ordre unitonique} meant an absence of key relationships and modulations, Fétis permitted \textit{ficta} at cadences,\footcite[14]{WagenerBegleitunggregorianischenChorals1964}
citing numerous examples in Palestrina, Pietro Aaron (\emph{c.}1480--\emph{c.}1545) and others whom he held to be the first harmonisers (`des harmonistes des premiers temps') to prove that cadential \textit{ficta} was admitted in Renaissance polyphony and \emph{falsobordone} practice, particularly at the terminal cadence.\footcites[153--4]{FetisTraitecomplettheorie1867}[107--109]{Fetisdemitondansplainchant1845}
\label{pg:fetis-sharping}%
At some point, Fétis's apprehension of cadential \emph{ficta} overlapped with some diocesan traditions---such as that Parisian practice of sharping the penultimate note of a sequence---but Danjou adopted a more lenient interpretation, leaving the application of cadential \emph{ficta} up to the `taste and experience' of the practitioner.

While Danjou's personal taste generally lent itself to cadential sharping, he proposed an alternative scheme: the diatonic method of `accompagnement naturel' consisted of 5/3 chords alone in which no \emph{ficta} whatever was permitted.
In several examples, Danjou used only those chords containing diatonic chant notes to harmonise the tetrardus cadence `A'~\rightarrow{}~`G' with F~major~\rightarrow{}~G~major instead of D~major~\rightarrow{}~G~major or minor harmonies.
The examples appear to have been carefully selected, however, so as not to contain `B'\kern 1pt\flat{} in the chant, and it is entirely possible that Danjou recognised the controversy raised by that pitch class and the stark contrasts it provoked in the harmony that were difficult to manage.\footcite[11--18]{Danjouaccompagnementplainchant1848}
Diatonic harmonisations by other composers had introduced such contrasts when `B'\kern 1pt\flat{} supplanted `B'\kern 1pt\natural{} and vice versa,\footcite[5]{MougelRequiem1862} perhaps producing too much of an auditory shock for Danjou to tackle in a few pages.
\noclub[2]

The diatonic debate was also raging among musicians in Belgium where the archbishop of Mechelen Engelbert Sterckx had decreed on 26 April 1842 that secular music was to be banned from his churches.\footnote{The decree is mentioned in Sterckx's approbation printed in the front matter to \cite{HenryNovumorganumRecueil1844}.}
Two years later Sterckx issued a further decree stipulating that churches incapable of buying an organ should look to the harmonium builder François Verhasselt (1813--1853).\footnote{\covid{}\cite[439--41]{HaineDictionnairefacteursinstruments1986}.}
In 1845, Janssen claimed to have uncovered the `true principles' of plainchant, and caused quite a stir when they too gained Sterckx's approval.\footcite[Danjou to Fétis, 6 April 1845, in][196]{FetisCorrespondance2006}
As we have seen, Janssen opted to retain diatonic chants, though he accompanied these with harmony that could contain sharps, seemingly to counterbalance the perceived monotony resulting from diatonic accompaniments.
But that approach was open to criticism on the grounds that it yoked together two opposing attitudes to \emph{tonalité}.
Janssen attempted to apply his theory to the psalm tones by permitting the fifth tone to retain its character while being accompanied by a progression of secondary dominants (\cref{mus:janssen_tone5}).\footcite[220]{Janssenvraisprincipeschant1845}
Yet, it is difficult to imagine this harmony as being anything other than monotonous during successive verses of a long psalm, and its appeal to modernity set it at odds with \emph{tonalité ancienne} while showing how widely the fissure ran between supposedly historical theories.
\nowidow[2]

The Belgian composer Edmond Duval (1809--73) produced an accompaniment book for the Mass and Office based on Janssen's `true principles' (\cref{mus:duval_madness}).
His accompaniments included contrapuntal filler between phrases of the chant, mimicking that practice common in Lutheran churches where interludes were placed between phrases of chorale melodies to allow the congregation an opportunity to draw breath.
Duval's interludes introduced chromatic notes in the top part---the same part that later resumed the diatonic chant melody---and contained shortened note values that accelerated the harmonic rhythm, disuniting interlude and accompaniment.
\hlabel{ln:duval_protestant}%
The preface acknowledges Christian Heinrich Rinck (1770--1846) as the inspiration for the interludes (`les \emph{Chorals} de Rinck m'ont fourni l'idée'), and their origins are likely to be found in that composer's solo organ composition \textit{Sechs Choräle} op.~78 that bridges two phrases of a chorale melody with contrapuntal padding.\footcite[unpaginated preface and p.~10]{Duvalorganistegregorienou1845}
Inducing contrast between interludes and accompaniments was apparently the name of the game for the Belgian composer Robert Julien Van Maldeghem too, whose bilingual preface in Dutch and French claims his preludes, interludes and postludes were newly composed (`entièrement de composition originale').
Their texture certainly made those compositions stand apart from the chorale-textured chant accompaniments, though Van Maldeghem exercised a heavier editorial hand than Janssen and Duval by transcribing the chants into modern notation with what he called corrections (`traduite avec corrections en notes modernes')---these consisted of sharped cadential notes so the harmony would traverse dominant \rightarrow{} tonic progressions.\footcite[1, 3]{VanMaldeghemMesseEngelenGregorianschen1841}

\hlabel{hl:hageman}%
In the face of such conflicting views, then, the reception of Janssen's principles was mixed.
On the one hand, the Dutch composer Herman Hageman singled out Janssen from among Homeyer, Sechter, Oberhoffer and Benz for adopting a diatonic method (`in wiens verdienstelijk werk de diatonische methode waardig is vertegenwoordigd').
The prevalence of applied dominant chords and a curiously inelegant `C'\kern 1pt\sharp{} (later becoming `C'\kern 1pt\natural{}) in one of Hageman's bass parts appear to corroborate the suggestion that he subscribed to Janssen's theories (\cref{mus:hageman_requiem}).\footcite[iii, 65]{HagemanVerzamelingvanGregoriaansche1859}
On the other hand, Morelot criticised the Janssen-Duval method in \textit{Revue de la musique religieuse, populaire et classique} (a journal edited by Danjou), and was particularly dismissive of Duval's accompaniments for their `auditory surprises' (`surprises d'oreille') and their borrowing from a Protestant tradition.\footcite[451--2]{MorelotRevuecritique1845}
Lemmens would later take up an anti-Protestant stance in his \emph{École d'orgue} which sought to counter-balance the coldness of Protestant worship (`de koudheid van dien eeredienst') with something more suitable to the Catholic liturgy.\footcites[p.~v]{LemmensJacques-NicolasEcoleorguebasee1869}[17]{ErensJaakLemmensstichter}
We shall return to Lemmens's views in \cref{sc:lemmens}.


\subsection{The shock of the old: Niedermeyer and \emph{tonalité ancienne}}
\label{hl:nied}%
From 1853, `accompagnement naturel' received fillip when the École Choron was re-opened as the École Niedermeyer to instruct \textit{maîtres de chapelle} and organists in what was claimed to be a historically informed approach to church music.\footcite[71]{EllisInterpretingMusicalEarly2005}
Louis-Abraham Niedermeyer (whom we first encountered at \cpageref{ln:niedermeyer_firstmention,ln:niedermeyer_firstmention_END} above) sought to promote such ideas among musicians at large, and began publishing the journal \emph{La Maîtrise} with Joseph d'Ortigue  as editor-in-chief.
\hlabel{int:dortigue}%
D'Ortigue had initially been opposed to the idea of accompaniment, but following his adoption of Niedermeyer's principles put forth a series of articles in which he repudiated the presence of the sharp in harmonisations of chant.\footcites[col.~497]{DOrtigueDiese1854}[54]{ChristensenStoriesTonalityAge2019}
The practice of totally eschewing sharps had begun to establish itself by the mid-1850s, and in his \emph{Dictionnaire} of 1853 d'Ortigue reproduced Danjou's article detailing the `accompagnement naturel'. By then even Duval had abandoned Janssen's `true principles' in favour of the diatonic method which some had described as the one `more intimately linked to plainchant' (`un style plus intimement lié au plain-chant').
Duval even provided examples of the diatonic style for La Fage's \emph{Cours complet de plain-chant} of 1856, in which the latter warned that the musical public `would have to examine the merits of these pieces' before widespread adoption was likely.\footcite[871--2]{LaFageCourscompletplainchant1856}

On the one hand, some French-speaking theorists believed that the appropriate style of chant harmonisation was to be derived from contrapuntal principles.
In Belgium, for instance, Janssen worked out the bass part using counterpoint before constructing the rest of his accompaniment:

\simplex{N'est-il pas d'abord évident qu'il faut être initié aux secrets du contrepoint pour trouver des basses applicables à la mélodie du chant~?}
  {\cite[207]{Janssenvraisprincipeschant1845}}
{Is it not firstly evident that one must be introduced to the secrets of counterpoint in order to find basses applicable to chant melodies?}

\pagebreak{}
\noindent
The Danjou conservatoire bass line (\cref{mus:danjou-conservatoire}) was probably worked out according to contrapuntal principles, and it is unsurprising, given his tutelage by Benoist, that Danjou also subscribed to the contrapuntal viewpoint, a particular emphasis being given to the consonance and dissonance of intervals:

\simplex{La science de l'accompagnement du plain-chant réside dans la connaissance des règles du contrepoint, c'est-a-dire dans l'art d'agencer entre eux divers intervalles harmonieux, d'où il résulte que la première notion qu'on doit posséder est celles des intervalles.}
  {\cite[5--6]{Danjouaccompagnementplainchant1848}}
{The science of plainchant accompaniment depends on knowing the rules of counterpoint, that is to say in the art of arranging diverse harmonic intervals, from which it follows that the first notion one needs to have is that of intervals.}
\noindent
On the other hand, Nisard disavowed contrapuntal accompaniment of chant and particularly those accompaniments worked out according to François Benoist's principles, saying that such an approach was appropriate for solo pieces on a grand orgue but not for vocal accompaniments.\footcite[41]{Nisardvraisprincipesaccompagnement1860}
Morelot was similarly critical of accompaniments in florid counterpoint and held that it imposed an external rhythmic scheme that deformed the rhythm of the melodies (`altérant inévitablement la constitution rhythmique du plain-chant').\footcite[69]{MorelotElementsharmonieappliquee1861}
\hlabel{hl:lafage_cours}%
Probably for that reason, La Fage was also critical of florid counterpoint, deeming it awkward and unfit for the accompaniment of voices (`Le contrepoint fleuri est incommode pour l'accompagnement des voix qui exécutent le plain-chant').\footcite[622]{LaFageCourscompletplainchant1856}
La Fage then suggested that chant should be accompanied harmonically rather than contrapuntally, a view to which he had subscribed since first introducing the \emph{orgue accompagnateur} at Saint-Étienne.
\hlabel{int:lafage}%
\label{ln:lafage_unaccomp}%
But in a notable \emph{volte-face}, La Fage later distanced himself from accompaniment and came to favour unaccompanied chanting instead.\footcite[141]{LaFagereproductionlivresplainchant1853}
The counterpoint-versus-harmony debate was generally an extension of the debate surrounding \emph{tonalité ancienne} and \emph{tonalité moderne}, but little consensus was reached on this topic before the early 1860s because neither approach had enough institutional or journalistic support to be widely adopted.

Niedermeyer recognised that the `accompagnement naturel' style followed traits already inherent in the modes, and collaborated with d'Ortigue on a \textit{Traité théorique et pratique de l'accompagnement du plain-chant} which was first published in 1857, reissued in 1859, and subsequently republished in a new edition in 1876 to add music examples by the organist-composer Eugène Gigout (1844--1925), Niedermeyer's son-in-law.\footcite[19]{GigoutPartiepratique1876}
Possibly in an attempt to distance their method from \emph{tonalité moderne}, Niedermeyer and d'Ortigue attempted to subvert the tonally oriented expectations of the listening public.
One cannot help seeing in their method's philosophy that the \emph{ordre unitonique} was deliberately asserting itself over the \emph{ordre transitonique} so as to reverse Fétis's notion that the \emph{ordre unitonique} had yielded to the \emph{ordre transitonique} in the seventeenth century.
To elicit a feeling of shock similar to that supposedly experienced by seventeenth-century listeners was evidently what Niedermeyer and d'Ortigue had in mind:

\duplex{Il est incontestable que la véritable harmonie du plain-chant doit être autre que celle de la musique, puisqu'elle découle d'une tonalité toute différente, \linebreak{}et il en résulte que certaines harmonies non-seulement justifiées, mais indispensables dans l'accompagnement du plain-chant, devront d'abord nous choquer et, comme dit très-bien M.\ de La Fage, paraître \textit{offensantes pour notre oreille}, parce qu'elles se trouveront en contradiction avec le sentiment des règles de l'harmonie moderne.}
  {\cite{NiedermeyerTraitetheoriquepratique1859}, 2è tirage:65}%
{Incontestably, the true harmonisation of plainsong must differ from that of modern music, for it has to do with a totally different system of tonality; \linebreak{}and it is inevitable that certain \linebreak{}harmonies which are not only justifiable, but indispensable in the accompaniment of plainsong will shock us at the outset, and, as Mr de La Fage has rightly said, will seem \emph{offensive to our ears} because they are contradictory to the sentiment of the rules of modern harmony.}
  {Adapted from \cite[38]{NiedermeyerGregorianAccompanimentTheoretical1905}}
\noindent
The irony unrecognised by Niedermeyer and d'Ortigue was that their `shock of the old' did not glean contrapuntal or harmonic rules from music pre-dating \emph{tonalite moderne}, but rather constituted a new method based on recent and contemporary discourse.
The six rules governing their method of accompaniment were as follows:
\pagebreak{}

\label{r:niedrules}%
\hlabel{ln:niedermeyer_rules}%
\dualcolumn{L'emploi exclusif, dans chaque mode, des sons de l'échelle.}{The exclusive use, in each mode, of notes of the scale.}
\ParallelPar
\dualcolumn{L'emploi fréquent dans chaque mode des accords déterminés par la finale et la dominante.}{The frequent use of triads of the final and dominant in every mode.}
\ParallelPar
\dualcolumn{L'emploi exclusif des formules harmoniques qui conviennent aux cadences de chaque mode.}{The exclusive use of harmonic formulæ proper to the cadences of each mode.}
\ParallelPar
\dualcolumn{Tout accord, autre que l'accord parfait et son premier dérivé, devra être exclu de l'accompagnement du plain-chant.}{Every chord other than consonant triads and their first inversions should be barred from plainchant accompaniment.}
\ParallelPar
\dualcolumn{Les lois qui régissent la mélodie du plain-chant doivent être obervées dans chacune des parties dont se compose son accompagnement.}{The laws that govern the plainchant melody must be observed in each of the accompanying parts.}
\ParallelPar
\hlabel{r:niedrule6}%
\duplex{Le plain-chant, étant essentiellement une mélodie, doit toujours être placé à la partie supérieure.}
  {\cite{NiedermeyerTraitetheoriquepratique1859}, 2è tirage:31--5}
{Plainchant, being essentially melody, should always be placed in the top part.}
  {Adapted from \cite[14--16]{NiedermeyerGregorianAccompanimentTheoretical1905}}
\hlabel{ln:niedermeyer_rules_END}%

\noindent
Niedermeyer was keen to demonstrate the practicality of his method and applied his rules to a harmonisation of the \emph{Missa de Angelis},\footcite[123]{NiedermeyerMessepourdoubles1858} though that alone did not preclude the intense debate that surrounded his principles during the 1860s.
In the year Niedermeyer and d'Ortigue first published their \emph{Traité}, César Franck cautioned the readers of his accompaniment book against any systematic exaggeration (`toute exagération systématique') that could put off the faithful with archaism, and opted himself for a compromise that comprehended some chromatic pitches.\footcites[pp. iii, 3]{LambillotteChantgregorien1857}
%\noclub[2]

\section{Evolving plainchant styles}
\subsection{French and Belgian plainchant congresses}
Post-Revolutionary anticlericalism had detached the French church from papal oversight, and French bishoprics procured chant editions according to their needs, opting for chant books whose melodies were assimilated from daily or weekly practice.
As Ultramontanism gained more sway, however, some dioceses adopted the descendants of the Medicean Gradual, while others convened provincial councils to consider their options.
Such councils took place in the dioceses of Reims (1849), Albi (June 1850), Bordeaux (July 1850), Aix and Toulouse (September 1850), Bourges at Clermont (October 1850) and at Auch (1851).\footcite[col. 3]{DOrtigueEpiscopatparle1860}
Some adopted the editions of neighbouring dioceses; others preferred chant editions by chant editors at the cutting edge of plainchant paleography, such as Lambillotte.
The narrow geographical scope of such councils meant, however, that their findings seldom exerted an influence beyond neighbouring dioceses.

Niedermeyer and d'Ortigue recognised the need for consensus on issues of plainchant style and elected to convene a plainchant congress of their own with a national focus, or at least a Parisian one.
Calls for subscribers circulated in issues of \textit{La Maîtrise} in 1859, and `séances préparatoires' were held on 25 May and 3 August 1860 at the Salle Érard to settle the agenda.
The appropriate style of church music was pegged for discussion, as were matters pertaining to plainchant performance and accompaniment.
Those séances resulted in the Parisian Congrès pour la restauration du plain-chant et de la musique religieuse of 1860, which took place from 27 November to 1 December 1860 and registered 97 attendees from amateur and professional backgrounds.
Among the professionals were Benoist, Cavillé-Coll, Morelot, d'Ortigue, Schmitt and Camille Saint-Saëns (1835--1921), and among the others were theorists and countless amateurs who have largely been forgotten in the decades since.
For the privilege of addressing the various sessions (or at the very least to have one's name included in the official report), speakers paid 10~F.\ in addition to their subscription fee, which might explain why representations from the same people were recorded in disparate debates while the views of others were not represented at all.
\nowidow[2]

The accompaniment of plainchant was discussed on the first two days of the congress, during which attendees parried various ideals to reach verdicts on texture, harmony, rhythm and the appropriate part in which to place the chant.
The last stirred no small amount of debate and various attendees argued for and against placing the chant in bass or treble parts depending on their ideological or organological inclinations.
Joseph Wackenthaler (1795--1869), organist at Notre-Dame de Strasbourg, had placed the chant in the top parts of chant accompaniments since at least 1854,\footcite[2]{Wackenthalerartaccompagnerplainchant1854} but Schmitt held that the trebles of his own Cliquot \emph{orgue de tribune} at Saint-Sulpice were not powerful enough for a tune-on-top accompaniment, and he was therefore compelled to play the chant on the Bombardes of the Pedal division.\footcite[533]{GrandjeanOrgelundOper2015}
Schmitt had first encountered Niedermeyer after moving to Paris in 1844, and was appointed as organ teacher at the École Niedermeyer in December 1856 following the death of the incumbent, Joseph Wackenthaler's son François-Xavier, during the previous October.\footcite[210]{OchseOrganistsOrganPlaying2000}
Schmitt's view---that chant, being like any other melody, ought to be placed in the top part---had been published in 1855, two years before Niedermeyer and d'Ortigue included the same rule in their \emph{Traité}.\footcite[55--6]{SchmittNouveaumanuelcomplet1855}
During the debate on the placement of the chant in an accompaniment, one F.\ Calla sought clarity for the term `accompagnement', the meaning of which suggesting vocal accompaniments in fauxbourdon (in which the chant was usually placed in the tenor part) as well as organ accompaniment of voices.
The congress's official report found that tuning ought to be the deciding factor, and also weighed in on the consonance of the accompaniment:
\pagebreak{}

\simplex{En ce qui concerne l'accompagnement du plain-chant, le Congrès est d'avis que l'on ne doit pas s'écarter d'une harmonie consonnante, en rapport avec la tonalité ecclésiastique, et que le chant soit, autant que possible, à la partie supérieure et dans un diapason qui réponde à la généralité des voix.}
  {\cite[227, 238, 242]{Vroyemusiquereligieusecongres1866}}
{As for plainchant accompaniment, the congress is of the opinion that one should not deviate from consonant harmony as it relates to the ecclesiastical \emph{tonalité}, and that the chant should be, as far as possible, placed in the top part and at a pitch appropriate for the general pitch of the voices.}

\hlabel{ln:repos_notation}%
In 1862, the editor and publisher Jean-Baptiste-Étienne Repos (1803--72) amalgamated two journals (\emph{Le Plain-chant} and \emph{La Paroisse}) into \textit{Revue de musique sacrée ancienne et moderne} whose scope matched that of \emph{La Maîtrise}.
Under the aegis of Repos's new journal, the Comité de rédaction et de patronage was formed to host a series of conferences during 1862 and 1863, including a meeting on 6 June 1862 at which prescripts for chant accompaniment were decided upon.
Schmitt, the official reporter, published its findings in the July issue of 1864, calling for a diatonic style based on major and minor scales.
So, instead of basing the harmony for protus chants on the notes of the protus scale, the \emph{Revue} found that protus chants were best accompanied in D minor instead (`de préférence en \emph{ré} mineur').\footcite[cols~280--81, ]{SchmittRapportconferencesouvertes1864}
The composer of the music examples (who might have been Schmitt himself) nevertheless followed the Niedermeyer and d'Ortigue method in practice at `E' \rightarrow{} `D' protus cadences: instead of using A major \rightarrow{} D~minor progressions, C major \rightarrow{} D~minor ones were used.
It matched that rule discussed above (see \cpageref{ln:niedermeyer_firstmention,ln:niedermeyer_firstmention_END}) that prohibited A minor \rightarrow{} D~minor progressions because they raised the embarrassing question of whether or not a `C' should be sharped.\footcites[Also discussed in][190--2]{LessmannRezeptiongregorianischenChorals2016}[Henri Potiron, whom we shall encounter below, offered an alternative view on a similar progression in the next century, this in an accompaniment manual that viewed chant accompaniment through a contrapuntal lens---disjunct motion was said to be preferable to conjunct motion. See][92]{PotironPetittraitecontrepoint1951}
The deuterus cadence `F' \rightarrow{} `E' (harmonised by Niedermeyer and d'Ortigue as in \cref{mus:niedermeyer_deuterus_natural})\footcite[42--4, 66]{NiedermeyerTraitetheoriquepratique1859} was harmonised quite differently, however, and included a `G'\kern 1pt\sharp{} in the terminal chord (\cref{mus:schmitt_deuterus}).\footcite[3, 5]{Schmittplainchantaccompagneselon1864}

\hlabel{ln:novello_notation}%
Beamed notation for chant transcriptions was a distinctive characteristic of Repos's \emph{Revue} but was by no means unique to it, having been pioneered by the music publisher and composer Vincent Novello (1781--1861) in England around 1834.
The beams indicated that notes belonged to the same neume, while the type of notehead (whether square or lozenge) was supposed by some theorists to represent the duration of the note.\footnote{\cite{WesleyConventMasstextnumero11834}, p.~2 n.\kern -1pt\rotatebox{-45}{$\raisebox{.52em}{$\div$}$}; A footnote claims the organ part was `entirely arranged' (or, we might assume, realised) by Novello from a thoroughbass part by Samuel Wesley (1766--1837), though we might suppose that Novello was also responsible for the notational style of the printed score.}
The same notational style continued to crop up in various publications throughout the nineteenth century, though not, as we shall see in \cref{ln:novello_legeay}, without certain changes.
\hlabel{ln:novello_notation_END}%

From 18 to 20 August 1863, a discussion of plainchant accompaniment also took place at Mechelen, but it appears that the admission of cadential sharping and modulation was not prejudicial to the diatonicism of an accompaniment:

\dualcolumn{L'accompagnement du plain-chant doit être diatonique, c'est-à-dire fondé sur l'échelle même du mode, en admettant toutefois les modulations mélodiques résultant du mélange des modes, de leur transposition et des tons relatifs au ton principal. Les altérations ne sont donc admises que comme exception, lorsqu'elles sont absolument nécessaires pour éviter les fausses relations.}{Plainchant accompaniment must be diatonic, that is to say based on the same modal scale, always open to melodic modulation resulting from the mixture of the modes, their transposition, and keys related the tonic key. Sharps are only admitted in exceptional cases when absolutely necessary to avoid false relations.}
\noindent
Further on, however, the report states that the `tonal feeling' (whatever that happened to be) must always dominate in an accompaniment (`Le sentiment tonal doit toujours dominer'), a statement surely conflicting with that desire for accompaniments to remain diatonic.
Sharping cadential intervals continued to be hotly debated, Duval and Lemmens arguing the purely diatonic side and Morelot the cadential \emph{ficta} one.
The former both conceded, however, that no better alternative to sharping had yet been found to harmonise terminal cadences approached from below, and so for them at least the matter remained open.\footcite[138--9, 141]{Vroyemusiquereligieusecongres1866}
\hlabel{ln:morelot_hexachord}%
Morelot, who had recently taken up a study of Greek music, attempted to justify sharping by describing an obscure process of hexachordal mutation where semitones were permitted to take the place of tones.\footcite[22]{MorelotElementsharmonieappliquee1861}
Fétis nonetheless approved of the method and claimed Morelot had entered along the only path to success (`Morelot est entré dans la seule voie où le succès est possible').\footcite[196]{FetisBiographieuniversellemusiciens1867}
Be that as it may, Morelot lamented over twenty years later that his method was in direct competition with that by Niedermeyer and d'Ortigue, which, he claimed, was not only based on ahistorical prescriptions but also enacted violence on the public's ears (`les résultats font violence à nos oreilles').\footcite[Stephen Morelot to Antonin Lhoumeau, 1 March 1878, printed in][3--6]{Lhoumeaualterationoudemiton1879}
\hlabel{ln:morelot_hexachord_END}%
\hlabel{int:kunc}%
The dichotomy between diatonicism and sharping was described during the 1880s by the theorist Aloys Kunc (1832--95):

\simplex{Deux écoles principales sont aujourd'hui en présence~: l'une ne demande pas d'autres accords que ceux qui sont formés des éléments mêmes de l'échelle diatonique~: l'autre pense qu'on peut introduire dans cette même échelle diatonique des dièses et former ainsi des demi-tons qu'elle ne comporte pas naturelement. Cette dernière école se partage encore entre deux systèmes~: les uns admettent les dièses dans les parties d'accompagnement et les proscrivent dans le chant~: les autres les admettent et dans le chant et dans les parties d'accompagnement.}
  {\cite[p.~32, p.~34 n~.1, pp.~41, 43]{RuellecongreseuropeenArezzo1884}}
{Two main schools are represented today: the one only admits chords formed from the very pitches of the diatonic scale; the other thinks sharps can be admitted to this same diatonic scale, and can thus form semitones that it would not naturally include. This latter school further divides into two systems: one accepts sharps in the accompaniment and forbids them in the chant; the other accepts sharps both in the chant and in the accompanying parts.}
\noindent
Although Kunc also noted that the chord-against-note style was `happily tending to disappear, it had not been completely eradicated by the 1890s when certain prominent musicians continued to produce chorale-textured accompaniments, as we shall see (\cref{ln:gounod_chordagainstnote}).

\subsection{Towards free rhythm}
In the period under consideration, performers of chant who did not apply the same durations to each note could subscribe to various mensural interpretations (of which Novello's, as mentioned above, was one example).
Their accompaniments could just as easily take on the chorale texture as those of equally rhythmed chanting.
Slower tempi were believed by some to enhance the solemnity of a feast,\footcite[137]{MannaertsGevaertStudyPlainchant2010} as was cadential sharping and the chorale texture.
By 1871, however, Gevaert considered the accompaniment in \cref{mus:gevaert_notation} to have been contrived to manufacture that solemnity:

\simplex{Ce mode d'accompagnement se prête à une harmonie assez riche et séduit par une vague teinte d'archaïsme~; au fond cependant il ne respecte pas la construction harmonique des mélodies grégoriennes. C'est le christianisme primitif habillé à la mode de la Renaissance.}
  {\cite[`Preface' p.~4]{GevaertVademecumorganistecontenant1871}}
{This method of accompaniment lends itself to rather a rich harmony and leads us on by a vague hint of archaism; basically though it does not follow the harmonic construction of Gregorian melodies. It is early Christianity cloaked in the fashion of the Renaissance.}
\noindent
For Gevaert, then, both the chorale texture and cadential sharping were to be considered old hat, and he set out rules which he believed produced accompaniments along more historical lines (see \cref{ln:gevaert_new} below).
\hlabel{hl:ovejero}%
While the organ teacher at Madrid's Escuela Nacional de Música Ignacio Ovejero (1828--1889) maintained the chord-against-note style in his accompaniments of 1876, Santiago Ruiz Torres has described them as being written using major-minor harmony (`el lenguaje adoptado resulta nuevamente tonal'), and that the alternation of tonic and dominant chords (such as those quoted in \cref{mus:ovejero}) contributed to the austerity of means sought by Ovejero.\footcite[993]{Torresfacetadesatendidaquehacer2013}
That composer certainly avoided overtly dissonant harmony, and echoed the pronouncements of Fétis when he claimed the `tonalidad' of chant was distinct from that of modern music:

\simplex{Le tonalidad antigua, ó sea la del canto llano es muy distinta de la que usamos en la música.}
  {\cite[4, 36]{OvejeroEscuelaorganistatratado1876}}
{The ancient tonality, that is, that of plainchant, is very different from the one we use in [modern] music.}

\noindent
Perhaps his restricted chord progressions were indeed a means of capturing a greater sense of austerity; but then again, the composer might also have wanted to keep his accompaniments simple to play.
The keyboard texture would have permitted an organist to play the three, largely conjunct upper parts with the right hand, and the simple, disjunct (but repetitive) bass part with the left.

Witt approached the chorale texture not from the stylistic perspective but from that of a practitioner, claiming that at fast tempi such textures verged on the unplayable at best and on the boring at worst:

\duplex{%
Denn wenn wir uns auch einen wahren Virtuosen denken, der wirklich seine Begleitung in der von dem richtigen Vortrage des Chorales bedingten Rapidität auszuführen im Stande ist, so entsteht doch dadurch, dass eine gauze Unzahl ähnlicher Akkorde und Akkordverbindungen vorkommt und vorkommen muss, unausbleiblich Monotonie.%
}%
  {\cite{WittOrganumcomitansad1881}, 3rd ed., p.~v}
{Now even if we imagine to ourselves a true virtuoso, who really is able to play his accompaniment with the rapidity required to give proper effect to the chant, yet unbearable monotony must be the result because innumerable chords and chord relations of the same kind are constantly occurring.}
  {Translation on similarly paginated English-language supplement and translated by H.\ S.\ Butterfield, p.~iii}%
\noindent%
% \hlabel{ln:franck_tiednotes}%
% While J. Henry reminded readers of his accompaniments to the importance of observing the notated ties,\footcite[1]{HenryNovumorganumRecueil1844} Franck intended notes common to consecutive chords to be held even without the ties being notated (`Il faut la tenir aussi longtemps'), advice that was likely provided for the benefit of amateur instrumentalists (`Ceci est par trop élémentaire pour que les organistes puissent l'ignorer'),\footcite[p. iv]{LambillotteChantgregorien1857}

For Nisard, moderate tempi were said to imbue a performance with a sense of majestic austerity (`majesteuse austerité'), and he proposed different tempi to suit different styles of chanting.\footcites[cols~75--6]{NisardAccompagnementplainchant1854}[257--8]{GrandjeanOrgelundOper2015}
Slower chanting required more movement in the accompaniment, largely in the bass part: the example quoted in \cref{mus:boulanger_lent} (written by the organist of Beauvais cathedral Joseph Boulenger) was therefore intended to counteract the relative stasis of the chant.
Quicker chanting required a less active accompaniment, so that quoted in \cref{mus:nisard_vif} was more free because chords were sustained for the duration of a neume or group of notes.
Although such florid and grouped styles of accompaniment bear some resemblance to the eighteenth-century approaches of \textit{Imitationen im Baß} and \textit{Gruppenbegleitung},\footcite[88--101]{SoehnerGeschichteBegleitunggregorianischen1931} Nisard made no reference to the earlier styles and appears to have arrived at his textures independently.
A notable distinction between the two eras concerns the harmony, which in the nineteenth century was decidedly more diatonic.

\hlabel{hl:nisard_dissonance}%
With that being said, however, Nisard's harmonic approach did not prohibit dissonance.
When each chant note was accompanied by its own chord (as in the chord-against-note style), the player could design an accompaniment that was mostly or even wholly consonant.
But to reduce the number of chords in an accompaniment, and particularly in an accompaniment of a florid melody, was to accept that certain melodic notes would need to be dissonant.
In Nisard's more sustained style, then, suspensions, passing notes and auxiliary notes were permitted,\footcite[40, 44--5]{Nisardvraisprincipesaccompagnement1860} but deciding whether a note was to be consonant or dissonant in, for example, the chant reproduced in \cref{mus:nisard_chant} was a delicate business, one that required a certain level of intuition for the placement of chords on the part of the composer.
On the top line of the passage shown in \cref{mus:nisard_counterpoint}, Nisard designed the accompaniment in such a way that chords coincided with squares and certain lozenges depending on the harmonic context; the chord-against-note style, as rendered below it, demonstrates how the sustained style could comprehend many fewer chords.

\hlabel{hl:populus}%
One Alphonse Populus (1831--1900) was similarly keen to move away from the chord-against-note style, and demonstrated at the Paris congress of 1860 a sustained style of his own.
The passage reproduced in \cref{mus:populus_congress} comprises more sustained chords and dissonances than would result from the consonant, chord-against-note style, though in contrast to Nisard's example Populus claimed his chant rhythm was to be performed freely.
Sustained notes were therefore to be held by the organist indefinitely until the singer had arrived at the point at which the harmony was to change.
Populus's use of chords was governed by the textual accent: multiple notes sung to the same syllable were to be considered `\emph{melodic} and not \emph{real}' (`\emph{mélodiques} et non \emph{réelles}'), the melisma at `mortuis' being one example.
Populus sometimes permitted chords to change mid-way through a syllable, though, suggesting that his practice was based on his own perception of the chant rather than on a set of rules.
Populus's harmonisations were pioneering in their day, and were described retrospectively as having been written `according to the rules of musical composition' (`selon les règles de composition musicale').
Their influence was felt most acutely in the two decades following the 1860 Paris congress, when Populus's method was adopted at the Parisian church of Saint-Jacques-du-Haut-Pas, producing what were described as `excellent results'.\footcite[p.~9 \S{}11 and n.~2]{RuellecongreseuropeenArezzo1884}

Charreire advanced a theory (perhaps derived from Fétis) that 5/3 chords had an inherent quality of repose and were therefore most appropriate for points of rest, whereas 6/3 and 6/4 chords propelled the movement of a harmonisation forward, thereby bringing about transitions more effectively.\footcite[41]{Congrespourrestauration1862}
Populus seemingly agreed, and decided that conjunct motion (particularly in the bass part) allowed the melody to flow, be more expressive and to conform to the spirit of chant (`rend la mélodie plus coulante, plus expressive et plus conforme à l'esprit du chant liturgique').
\hlabel{cc:populus_conjunct}%
A number of years after the congress, Populus expanded on the idea with three versions of the same chord-against-note accompaniment, distinguished by varying levels of disjunct and conjunct motion (\cref{mus:populus_three_bass}).
Readers were told that the succession of 5/3 chords in the first bass part produced a monotonous effect (`l'effet produit nous semble monotone'), anticipating Witt's reservations by some seven years; the second was said to be better; but the third was said to be ideal because 5/3, 6/3 and even some passing 6/4 chords resulted in a mostly conjunct bass part.\footcite[part I: pp. 10, 15--16; part II: p. 2]{PopulusEtudesorgue1863}
Critics of his style claimed it left too many elements to the whim of the player; Populus responded with the counter-claim that flexibility was not an undesirable trait.
If organists were permitted to synthesise pauses, intervals and other factors which Populus declined to define, then they could arrive at an accompaniment that best suited themselves.
\hlabel{cc:populus_conjunct_END}%

\subsection{The young Lemmens and Fétis's theory}
\label{sc:lemmens}%
As we shall see in the next chapter, under the influence of Solesmian paleographers Lemmens would eventually become a celebrated proponent of using fewer chords than chant notes.
Before adopting that approach, he propagated certain ideas on how to accompany passed to him by his teacher Fétis.
Following Lemmens's success in winning first prize in composition and organ playing, Fétis petitioned the Belgian interior minister to grant a travel bursary so Lemmens could pursue further study in Breslau.
There, he studied with the organist Adolf Heinrich Hesse (1809--1863) for about a year, following which Hesse concluded that he had nothing left to teach Lemmens (`Je n'ai plus rien à apprendre à M. Lemmens') and that the young organist could play the most difficult of Bach's music as well as he could (`il joue la musique la plus difficile de Bach aussi bien que je puis le faire').\footcite[267]{FetisBiographieuniversellemusiciens1867b}

Lemmens was appointed organ teacher at the Brussels conservatory in 1849 and briefly visited Paris in May of 1850, a trip organised by Cavaillé-Coll at Fétis's request.
Lemmens's playing style was pitted against that of Louis-James-Alfred Lefébure-Wély (1817--69) who had succeeded Georges Schmitt as titular organist of Saint-Sulpice in 1863.
One Sunday, Hesse witnessed Lefébure-Wély playing `in a serious and appropriate manner' one moment and arousing `tremendous amusement' the next.\footcite[55]{HesseEinigesueberOrgeln1853}
Lefébure-Wély's portrayal of floods and storms certainly made his performances appealing to parishoners, but they were evidently too boisterous for Cavaillé-Coll, who would have preferred Lemmens to have been appointed instead (`I had dreamed of seeing you in that position').\footcite[76--7]{DouglassCavailleCollmusiciansdocumented1980}
Clearly it was playing of the Hesse-Lemmens kind, and not that of Lefébure-Wély, that Richard Wagner (1813--83) had in mind when singling out the organ's capacity for discretion:
\pagebreak{}

\duplex{Für die einzig nothwendig erscheinende Begleitung hat das christliche Genie das würdige Instrument, welches in jeder unserer Kirchen seinen unbestrittenen Platz hat, erfunden; diess ist die Orgel, welche auf das Sinnreichste eine gross Mannigfaltigkeit tonlichen Austruckes vereinigt, weiner Natur nach aber virtuose Verzierung im Vortrag ausschliesst, und durch sinnliche Reize eine äusserlich störende Aufmerksamkeit nicht auf sich zu ziehen vermag.}
  {\cite[337]{WagnerEntwurfzurOrganisation1871}; Reprinted in \cite[345]{WagnerKirchenmusik1883}; Cited in \cite[p.~10]{HelfgottOrgelmesseUntersuchungorgelbegleiteten2009} where an incorrect reference directs the reader to a twelfth book in the \emph{Gesammelte Schriften}}
{For the only necessary accompaniment the genius of Christianity invented a becoming instrument, which holds its undisputed place in all our churches; this is the organ, which most ingeniously unites a great variety of tone-expression but of its very nature excludes all virtuosic flourishes, and cannot draw an outwardly disturbing notice to itself by sensuous charms.}
  {\cite[343]{WagnerPlanOrganisationGerman1966}; The original translation dates from 1898}

During Lemmens's first three months at the Brussels conservatory, he taught chant accompaniment using a system handed down by Fétis.
Consonant chords were disposed in a four-part chorale texture, any pitch not belonging to the mode being prohibited.
Lemmens reportedly published the system in \covid{}\emph{Journal d'orgue},\footcite[3]{Lemmenschantgregoriensa1886} and may have publicly demonstrated it during visits to the Parisian church of Saint-Vincent-de-Paul in 1851 and 1852.
Lemmens's influence seemingly led to the system's widespread adoption by organists over the next two-and-a-half decades (`un système pour l'accompagnement qu'il avait mis en lumière il y a vingt-cinq ans et qui est actuellement suivi par la plupart des organistes').\footcite[57]{Bourgault-Ducoudraynouveausystemepour1878}
But as we shall see (\cref{ln:lemmens_update}) Lemmens eventually changed his mind and considered Fétis's method less than adequate, removing any trace of the older system when the \emph{Journal d'orgue} was republished as the \emph{École d'orgue} in 1862.\footcite[4]{Lemmenschantgregoriensa1886}

During the 1850s and 1860s, several French organists travelled to Belgium to study counterpoint with Fétis and the organ with Lemmens, among whom were Guilmant, Charles-Marie Widor, Clément Loret (1833--1909; who later became the organ teacher at the École Niedermeyer) and Alphonse-Jean-Ernest Mailly (1833--1918).\footcite[51--2]{OchseOrganistsOrganPlaying2000}
Several narratives conflict on the pilgrimage of Guilmant and Widor to Brussels: one suggests Guilmant met Lemmens in Paris in 1860 when the latter invited the former to Brussels;\footcite[2]{AlexandreGuilmant18371911} a second suggests an invitation was extended to Guilmant following an organ recital given by Lemmens in Rouen;\footcite[p.~viii]{LeupoldOrganMusicAlexandre1999} a third suggests Cavaillé-Coll solicited Lemmens's invitation on behalf of both Guilmant and Widor, the latter entering the Brussels conservatory in 1863;\footcite[77]{DouglassCavailleCollmusiciansdocumented1980} and a fourth suggests Guilmant and Lemmens met in an organ builder's workshop, possibly that of Cavaillé-Coll or Merklin.\footcite[56--7]{LuedersAlexandreGuilmant18372002}
Whatever the chain of events that brought Guilmant and Widor to Brussels, their stays with Lemmens were equally as short as Lemmens's had been with Hesse.
Guilmant remained in Brussels for no longer than a couple of months (various histories disagree on just how long), whereas Widor remained for some time between four and twelve months.\footcites[3]{AlexandreGuilmant18371911}[179, 260 n.~9]{OchseOrganistsOrganPlaying2000}
We shall return to the accompaniments written by these younger organists in the next chapter.

Following Lemmens's marriage to the English-born soprano Helen Sherrington in 1857, he tendered his resignation to the Brussels conservatory and moved to London where he established a recital career, delivering some 282 concerts from the mid-1860s until 1878.\footcites[57]{FocquaertAspectsJacquesNicolasLemmens2013}[50--51]{FocquaertJacquesNicolasLemmensBelgische2014}
On Lemmens's departure from the Brussels conservatory, Mailly took over as organ teacher, holding a \emph{concours} in August 1877 that comprised an improvisation, a chant accompaniment with and without figured bass (`avec et sans basse chiffrée'), a prelude in the Gregorian \emph{tonalité} (`la tonalité grégorienne') and a series of modulations in different modes (`une suite de modulations dans differents tons').\footcites[294]{JouretConcoursconservatoireroyal1877}[Cited in][171]{OchseOrganistsOrganPlaying2000}
Mailly continued to attract organists from France to study in Belgium, including the Lyonais Paul Trillat (1853--1909) who was appointed organist at the Primatiale Saint-Jean-Baptiste, Lyon, in 1874.\footcite[9]{EmeryEnnemondTrillatmusicien1979}
\nowidow[2]

The reforms instituted by Adlung's circle during the eighteenth century (see \cpageref{ln:reform_restart_history,ln:reform_restart_history_END} above) led to an interruption in the heritage of plainchant accompaniment.
Earlier methods were actively driven out of musicians' praxis because a simpler chorale texture and diatonic harmony were considered more suitable for the church.
The effect of those reforms led nineteenth-century accompanists to reinvent the wheel somewhat, in total ignorance of earlier developments.
Their more stringent adherence to the popular modal theories of their century ensured that their accompaniments were more widely accepted.
By the 1870s, considerable energies were being invested in developing theories of free chant rhythm; in response, the sustained style of accompaniment (which had arguably reached its zenith already in the \emph{Gruppenbegleitung} style) became an obvious candidate for accompanying freely chanted melodies.
Although it was broadly recognised as the ideal method since chords could be placed on specific accents rather than on every note, it engendered just as much debate at the \emph{fin-de-siècle} as diatonicism had at the mid-century.
