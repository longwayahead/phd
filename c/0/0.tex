\chapter*{Introduction}
\addcontentsline{toc}{chapter}{Introduction}
Plainchant was never meant to be accompanied.
Yet the fact remains that during the nineteenth and twentieth centuries many Catholic musicians---and those of other denominations too---invested huge energies in composing chant accompaniments and discoursing on the subject.
Their colourful debates begat an avalanche of printed matter: the cache of manuals in which musicians outlined their methods runs to well in excess of a thousand published titles in some twenty languages, while that of composed accompaniments (uninhibited by the linguistic traditions in which they were written) runs to at least double that.
Some publications, whether in prose or in music, were also the subject of extensive revisions as musical mores changed from decade to decade and as one newly fashionable methodology replaced another.
Whereas, at one extreme, practically minded choir directors considered chant accompaniment a \emph{sine qua non} for choral support, at the other, historically minded purists believed chant accompaniment to be anachronistic.

The distinction between both extremes is not easily drawn, for during their ideological tug-of-war some advocates for accompaniment were converted to opposing the practice, and vice versa.
Although Adrien de La Fage was responsible for introducing the organ accompaniment of chant into French churches in 1829 (see \cpageref{int:lafage} below), from 1853 he distanced himself from the practice on the basis that it was anachronistic.
His new stance did not preclude the production of a pre-harmonised `routine' (whose second edition was published in 1860), however, intended to equip players with the means of creating their own accompaniments.
While this might be explained as a commercial ploy, abject contradiction is also a reality in the history of chant accompaniment.
Félix Clément opined, for instance, that accompaniment should be prohibited to better preserve the religious sentiment of the liturgy,\footcite[356]{ClementMethodecompleteplainchant1872} but later brought out a method of accompaniment in what he advertised as an appropriate style (see \cpageref{int:clement_accomp} below).
In 1856, the Belgian theorist François-Auguste Gevaert proposed his own method, remarking that parish congregations could not be expected to sing the chant repertory without accompaniment (see \cpageref{int:gev_first} below).
He later recanted that view following a study of Ancient Greek music, and proposed instead a new method which was reportedly constructed along more historical lines.
But two decades later, Gevaert refused to admit that accompaniment could be entertained, the one allowance he made being for the sake of choral support---in which case, accompaniments were to be in unison (see \cpageref{int:gevaert} below).

It was not unheard of, though it was decidedly less common, for opponents of chant accompaniment to be converted to favour the practice.
When the Swiss theorist Louis Niedermeyer demonstrated a diatonic theory of chant harmonisation during the 1850s, its purported basis in historical fact persuaded Joseph d'Ortigue to assist in fleshing out a new method of accompaniment whose effects continue to influence church music today, as we shall see (\cpageref{int:dortigue}).
Other theorists were more coy than d'Ortigue by permitting only certain tranches of the repertory to be accompanied.
Maurice Emmanuel held such a view in the case of the psalm tones (see \cpageref{int:emmanuel} below), whereas the German theorist Paul Schmetz considered syllabic chants to require a style of harmony distinct from that used for melismatic chants.
Schmetz's notion was picked up some decades later by the Italian composer Giulio Bas, though seemingly independently as we shall see (\cpageref{int:schmetz,int:schmetz_END,int:rich_syllabic,int:rich_syllabic_END}).

The matter of determining a composer's personal preferences at a given time is made challenging by the prevalence of shifting allegiances.
As we shall see (\cref{int:gigout_teppe,int:gigout_lhoumeau,int:gigout_megret}), the French organist-composer Eugène Gigout was a versatile harmoniser, and could design diatonic chant accompaniments to fit any rhythmic theory.
He and his contemporary Alexandre Guilmant were approached to write accompaniments to illustrate theories of chant rhythm in practice.
Guilmant held at least three discrete personal styles (\cref{ln:guilmant_style}), though eventually sided with Gevaert's view that accompaniment was in fact anachronistic.

Leo Söhner's and Heinz Wagener's histories of chant accompaniment have charted narrow courses through the eighteenth, nineteenth and early-twentieth centuries, their source material largely being limited to written or published works produced in German-speaking countries.
Their studies are therefore rather inward-looking and find common ground in a fascination with the compiling of inventories.\footcites{SoehnerGeschichteBegleitunggregorianischen1931}{WagenerBegleitunggregorianischenChorals1964}
That methodology is also common in the study of specific MSS,\footcites[56--7]{FellererCodXXVII841926}[96--98]{PrasslAnmerkungenzurOrgelbegleitung2012} and is prone to exaggerating the pervasiveness of techniques one might consider idiosyncrasies of particular religious houses.
Wagener's decision to conclude his study at the year 1866 and the brevity of Söhner's have meant that the influence of Cecilianism on accompaniment has not been considered prior to the present study.\footcite{SoehnerOrgelbegleitunggregorianischenGesang1936}

Arguably, nowhere was the production of accompaniments more important than at Solesmes, where (in the age prior to the widespread availability of recorded media) they constituted a readily available means of popularising the chant repertory and the principles of free rhythm which the Solesmian authorities were attempting to disseminate.
The stakes were high indeed in a competition against Cecilian editions that had long since gained the approval of the Holy See.
The use of accompaniments as propaganda was a facet of the chant restoration movement which has not been considered before, in the Anglophone literature or indeed elsewhere.
Some studies published to date have opted instead to view chant and its accompaniment through the lens of musical composition,\footcite{LessmannRezeptiongregorianischenChorals2016} whereas others take a more analytical view of specific accompaniments, neglecting to contextualise their findings in the wider context of evolving musical techniques (see, for instance, \cpageref{int:marier,int:marier_END}).
\nowidow[2]

This dissertation addresses the lacuna existing between studies in cultural history (those by Katherine Bergeron and Katharine Ellis being two examples)\footcites{EllisPoliticsPlainchantfindesiecle2013}{BergeronDecadentEnchantmentsRevival1998} and studies in music analysis by relating the available printed matter to archival source material.
The rich cache of correspondence between key figures at Solesmes sheds new light on tension arising between musical, theological and commercial ideologies in the Benedictine circle.
This dissertation does not limit itself to material in any one linguistic tradition and benefits from a survey of source material in some twenty languages.

While and examination of printed matter functions as the cornerstone of the present study, it must be noted that libraries and collections beset by fire, vandalism or war list items in their catalogues that have since been destroyed, thereby making it impossible to consult certain sources; and the impact of composers, theorists and practitioners who were indubitably engaged in relevant discourse will remain impossible to quantify.
Closures and restrictions brought about by the COVID-19 pandemic prevented the present author from visiting certain research collections to consult specific sources, and in spite of the best efforts of librarians and archivists it was impossible to avail of them in digitised formats owing to cost considerations, time constraints, or poor states of repair.
\hlabel{ln:covid_symbol}%
Where the importance of such a source nevertheless necessitates its inclusion, this is indicated by the symbol \covid{}.
\hlabel{ln:covid_symbol_END}%
Earlier extracts from chapters one and two were previously published elsewhere,\footcite{LongTheoryPracticePlainchant2020} but may be considered to have been superseded by the present redaction.
