\chapter{Acknowledgements}
I should like to thank Dr Andrew Johnstone, who, in 2017 at the organ console of Saint Bartholomew's church in Ballsbridge, suggested chant accompaniment as a worthy topic for study at doctoral level, and whose academic instincts, keen eye for detail and unfailing encouragement have been steadfast companions in the years since.

I owe much to the custodians and associated staff of archives and libraries who arranged for my physical access to their collections, and who acceded to my many requests for digital materials when international travel was ruled out by the COVID-19 pandemic. In particular, I would like to thank Père Patrick Hala OSB at Saint-Pierre de Solesmes, who also invited my contribution in 2020 to the \emph{Études grégoriennes}; Ross Everett and Fr Brian Kelly OSB at Quarr Abbey, Ryde; Sr Bernadette Byrne OSB and Sr Marie-Germain Fiévet OSB at Saint Cecilia's Abbey, Ryde; Frère Thomas Zanetti OSB at Saint-Wandrille de Fontenelle; Professor Joseph Verheyden at the Katholieke Universiteit, Leuven; and staff at the British Library, London, at the Lemmens Institute, Leuven, and at the Parisian reading rooms of the Bibliothèque nationale de France. I am also indebted to Dr Benedikt Leßmann, Dr João Vaz and the editorial board of the Société française de musicologie for agreeing to send me articles of theirs directly, and to Daria Drazkowiak and Áine Palmer for helping to source books and articles in collections to which they alone had ready access. I would also like to acknowledge the Trojan work of Maria Gannon, who cheerfully processed my avalanche of inter-library loan requests at the library of Trinity College Dublin.

I am grateful to Colin Mawby (d.2019), Fr Columba McCann OSB, and Peter Stevens for volunteering their ideas on the contemporary praxis of chant accompaniment in \linebreak{}interview. Although the transcripts of our thought-provoking conversations did not reach the final dissertation, the living tradition of chant accompaniment could not have been more plainly illuminated, casting much-needed light on benighted matters of theory.

I owe a special debt of gratitude to the linguists who assisted me in translating passages from some of the twenty or so languages traversed by my project. Without the expertise of Dr David Adams, Daria Drazkowiak, Dr Fiachra Long, Dr Wolfgang Marx, and Nina Suter the discussion that follows would have been much impoverished. Any blunders in translation are mine, and not theirs.

I thank Professor William P.\ Mahrt of Stanford University, California, and Dr Frank Lawrence of University College Dublin for their careful consideration of this dissertation and for a stimulating conversation at \emph{viva voce}.
I also thank Professor Gerard Gillen and Dr Simon Trezise for their comments, in December 2019, on an early draft of chapter two, and Dr Eamonn Bell, Dr Michael Lee and Dr David O'Shea for inviting me to present some of that research in January 2020.

I am most grateful to my aunt Maeve Long for enabling me to conduct my research on site at the University, and I acknowledge the Taylor Bequest for funding research trips to Leuven, London, Paris and Solesmes.
I thank the professional staff of the University for taking care of administrative matters on my behalf, including Gráinne Redican-O'Donnell of the Department of Music and Michelle Greally of the Academic Registry.

I owe my passion for music to Colin Nicholls who awakened and encouraged my curiosity in the pipe organ and music in general at an early age.

I thank Stephen Murphy for countless stimulating conversations on divers subjects, and I extend my thanks to Martin Bergin, Andrew Burrows, Shauna Caffrey, Rev.\ Kenneth Crawford, Daria Drazkowiak, Dr~Bláithín Duggan, David Grealy, Dr~Kerry Houston, Kim Morrissey, Lorraine Norton, Dr~David O'Shea and other friends and acquaintances who aided my work at one time or another by their support and good humour.

I am greatly indebted to my mother and father Dr Siobhán Dowling Long and \linebreak{}Dr Fiachra Long for their love and support, for the benefit of their long experience as academics, and for continually encouraging my curiosity about the world.
Repaying the debt I owe them is a well-nigh impossible task, but I hope that the dedication of this work to them might amount to the first instalment.
