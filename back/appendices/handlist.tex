\chapter{Handlist of accompaniment manuals}
\label{ap:handlist}
 \singlespacing

    \parindent=0pt
    \hangindent=0pt
  Francesco Severi. \emph{Salmi passaggiati per tutte le voci nella maniera che si cantano in Roma}. Rome:  Nicolò Borboni, 1615.

     \parindent=20pt
     \hangindent=20pt
     Adds a rudimentary figured bass part to psalm tones.\\

    \parindent=0pt
    \hangindent=0pt
  Guillaume-Gabriel Nivers. \emph{Dissertation sur le chant grégorien}. Paris, 1683.

     \parindent=20pt
     \hangindent=20pt
     Describes the pitches of notable Parisian organs.\\

    \parindent=0pt
    \hangindent=0pt
  Dom Bédos de Celles. \emph{L'art du facteur d'orgues}. Paris:  L. F. Delatour, 1766.

     \parindent=20pt
     \hangindent=20pt
     Contains advice on registration for accompanying singers.\\

    \parindent=0pt
    \hangindent=0pt
  Justin Heinrich Knecht. \emph{Vollständige Orgelschule für Anfänger und Geübtere}. Leipzig:  Breitkopf und Härtel, 1798.

     \parindent=20pt
     \hangindent=20pt
     See \cpageref{hl:knecht} above.\\

    \parindent=0pt
    \hangindent=0pt
  Georg Joseph Vogler. \emph{Choral-System}. Copenhagen, 1800.

     \parindent=20pt
     \hangindent=20pt
     A discussion of four-part harmonisation of chant is followed by an appendix of some 412 demonstrative examples.\\

    \parindent=0pt
    \hangindent=0pt
  François Fétis. \emph{Méthode élémentaire et abrégée d'harmonie et d'accompagnement}. Paris:  Ph. Petit, 1824.

     \parindent=20pt
     \hangindent=20pt
     Although this does not recommend a method of chant accompaniment specifically, its recommendations on accompaniment in general were influential on French church musicians.\\

    \parindent=0pt
    \hangindent=0pt
  \covid{}Adolphe Miné. \emph{Manuel simplifié de l'organiste, ou nouvelle méthode pour exécuter sur l'orgue tous les offices de l'année selon les rituels parisien et romain sans qu'il soit nécessaire de connaître la musique}. Paris:  Roret, \emph{c}.1835.

     \parindent=20pt
     \hangindent=20pt
     Proposes a new notational system using the alphabet to simplify the task of accompaniment, though the system was criticised by d'Ortigue who called it some of the most stunning charlatanism he had ever seen (`le fruit du charlatanisme le plus étonnant qu'on ait jamais vu', \emph{Dictionnaire} col. 91).\\

    \parindent=0pt
    \hangindent=0pt
  Adolphe Miné. \emph{Méthode d'orgue}. Paris:  A. Meissonnier, 1836.

     \parindent=20pt
     \hangindent=20pt
     Sets out a method of three-part chant harmonisation in two stages, the first being `choral' and the second `avec des prolongations'. It is similar to that technique described above (see \cpageref{hl:mine}).\\

    \parindent=0pt
    \hangindent=0pt
  \covid{}Théodore Nisard. \emph{Manuel des organistes de la campagne}. Paris, 1840. \\

    \parindent=0pt
    \hangindent=0pt
  \covid{}Mazingue. \emph{Harmonie du plain-chant}. Lille:  Lefort, 1841. \\

    \parindent=0pt
    \hangindent=0pt
  Sebastien Stehlin. \emph{Tonarten des Choralgesanges, nach alten Urkunden durch beigefügte Übersetzung in Fuguralnoten erklärt, und als eine Anleitung zum Selbstunterrichte nebst drei vollständigen Messen aus dem römischen Graduale zusammengestellt}. Vienna:  Peter Rohrmann, 1842.

     \parindent=20pt
     \hangindent=20pt
     See \cpageref{hl:stehlin_tonarten} above.\\\pagebreak{}

    \parindent=0pt
    \hangindent=0pt
  Félix Danjou. \emph{De l'état et de l'avenir du chant ecclésiastique en France}. Paris:  Parent-Desbarres, 1843.

     \parindent=20pt
     \hangindent=20pt
     See \cpageref{hl:letat_lavenir} above.\\

    \parindent=0pt
    \hangindent=0pt
  François Fétis. \emph{Méthode élémentaire de plain-chant à l'usage des séminaires, des chantres et organistes}. 1st ed. Paris:  Canaux, 1843;  2nd ed. Paris:  Canaux, 1846.

     \parindent=20pt
     \hangindent=20pt
     \S{}9 offers practical advice to organists about the \emph{tonalité} of plainchant and the `tons de l'orgue'.\\

    \parindent=0pt
    \hangindent=0pt
  Johann Nikolaus Neubig. \emph{Der gregorianische Gesang bei dem Amte der heiligen Messe und andern kirchlichen Feierlichkeiten mit beigefügter Orgelbegleitung zunächst für die Diözese Limburg bearbeitet}. Wiesbaden:  Ritter, 1844.

     \parindent=20pt
     \hangindent=20pt
     A volume of accompanied recitations in various textures. See \cpageref{hl:neubig} above.\\

    \parindent=0pt
    \hangindent=0pt
  Charles Duvois. \emph{Méthode élémentaire d'accompagnement du plain-chant à l'usage des séminaires et collèges}. Paris:  Leduc, 1844.

     \parindent=20pt
     \hangindent=20pt
     Provides a system of annotations placed above the chant that prompt the player to execute a particular chord.\\

    \parindent=0pt
    \hangindent=0pt
  N. Arnold Janssen. \emph{Les vrais principes du chant grégorien}. Paris:  P. J. Hanicq, 1845.

     \parindent=20pt
     \hangindent=20pt
     Discusses diatonicism and psalmody, and provides advice on the use of chant at Mass and during the Offices. Further advice is provided in the second appendix which is followed by a small number of music examples.\\

    \parindent=0pt
    \hangindent=0pt
  Alexandre Fessy. \emph{Manuel d'orgue contenant les principes de l'accompagnement du plainchant, du mélange des jeux de l'orgue et de la rubrique de l'office suivi de morceaux de différents caractères}. Paris:  E. Troupenas \& C\textsuperscript{ie}, 1845.

     \parindent=20pt
     \hangindent=20pt
     A method of three-part psalm-tone harmonisation is followed by a more elaborate `accompagnement composé de notes prolongées et suspendues', perhaps suggesting the influence of former organ teacher François Benoist. See \cpageref{hl:fessy} above.\\

    \parindent=0pt
    \hangindent=0pt
  Abbé Clergeau. \emph{Mécanisme musical transpositeur pour orgue ou piano : ses effets sur l'orgue ou sur le piano, ses conséquences dans le monde musical}. Sens:  Thomas-Malvin, 1845.

     \parindent=20pt
     \hangindent=20pt
     See \cpageref{hl:clergeau} above.\\

    \parindent=0pt
    \hangindent=0pt
  Charles Child Spencer. \emph{A Concise Explanation of the Church Modes}. 2nd ed. London:  Novello \& Co., 1846.

     \parindent=20pt
     \hangindent=20pt
     Among the first English texts tackling the accompaniment of chant which cites Germanic literature, bespeaking a Teutophone influence that might explain the author's description of the modulation method. Those descripions are augmented by an appendix of music examples drawn from the German chorale literature. See \cpageref{hl:spencer} above.\\

    \parindent=0pt
    \hangindent=0pt
  C. P. Projean. \emph{Méthode complète d'ophicléide pour l'accompagnement du plain-chant}. Lyon:  J. B. Pélagaud et C\textsuperscript{ie}, 1846.

     \parindent=20pt
     \hangindent=20pt
     Although this is not a textbook on the organ accompaniment of chant, the author was ophicleidist at the Lyon church of Saint-François-de-Salles and therefore intended his manual to benefit such instrumentalists. It is graduated in difficulty, first presenting intervallic exercises before introducing samples of chant at various transposition levels.\\

    \parindent=0pt
    \hangindent=0pt
  Johann Baptist Benz. \emph{Harmonia sacra: gregorianische Gesänge nach dem Bedürfnisse der Kirchen in der Speyerer Diöcese zusammengestellt und theils für eine theils für vier Stimmen mit Orgelbegleitung bearbeitet}. 1st ed. 1st vol. Speyer:  Eigenthum des Komponisten, 1850;  1st ed. 2nd vol. Speyer:  Eigenthum des Komponisten, 1851;  2nd ed. Speyer:  A. Bregenzer, 1864.

     \parindent=20pt
     \hangindent=20pt
     See \cpageref{hl:benz} above.\\

    \parindent=0pt
    \hangindent=0pt
  Johannes Wellens. \emph{Handleiding om het Gregoriaansch met gepaste harmonie te bezetten}. Cuijk:  J. Van Lindert, 1851.

     \parindent=20pt
     \hangindent=20pt
     Provides advice to musicians of different aptitudes on how to tackle chant accompaniment, though the discussion often meanders from one metaphor to another. Although the author makes references to a set of plates to elucidate his musical ideas, this was not included in the copy consulted.\\

    \parindent=0pt
    \hangindent=0pt
  \covid{}Jean-Baptiste Labelle. \emph{Répertoire de l'organiste, ou Recueil de chant grégorien à l'usage des églises du Canada}. Montréal:  J. Lovell, 1851. \\

    \parindent=0pt
    \hangindent=0pt
  Léon Godard. \emph{Traité élémentaire de l'harmonie appliquée au plain-chant}. Paris:  Guyot, 1851.

     \parindent=20pt
     \hangindent=20pt
     \S{}6 discusses five rules for chant harmonisation: consonant chords alone devised according to the mode of the chant, no perfect consonances in succession, as much contrary motion as possible, conclude a harmonisation with a perfect consonance, and only chords in 5/3 position to be used.\\

    \parindent=0pt
    \hangindent=0pt
  \covid{}Heinrich Oberhoffer. \emph{Der gregorianische Choral.\ Anleitung, denselben richtig zu singen und mit der Orgel zu begleiten, nebst einer kurzen Geschichte seiner Entstehung}. Trier:  Lintz, 1852. \\

    \parindent=0pt
    \hangindent=0pt
  Sebastien Stehlin. \emph{Die Naturgesetze im Tonreiche und das europäisch abendländische Tonsystem vom VII Jahrhundert bis auf unsere Zeit : für Freunde der Kunst, die das Harmoniereich und das Tonsystem inden primitiven Grundgesetzen zu betrachten wünschen}. Innsbruck:  Witting, 1852.

     \parindent=20pt
     \hangindent=20pt
     See \cpageref{hl:stehlin_hexachord} above.\\

    \parindent=0pt
    \hangindent=0pt
  Eugène Woestyn. \emph{Le livre de la pianiste et du plain-chant}. Paris:  Ploche, 1852.

     \parindent=20pt
     \hangindent=20pt
     See \cpageref{hl:woestyn} above.\\

    \parindent=0pt
    \hangindent=0pt
  Adrien de La Fage. \emph{De la reproduction des livres de plain-chant romain}. Paris:  Blanchet, 1853.

     \parindent=20pt
     \hangindent=20pt
     Sets out the author's rationale for introducing the organ accompaniment of chant in French churches in 1829, to replace the serpent. But the author also records his newly established preference for unaccompanied chanting. See \cpageref{hl:lafage_reproduction_intro} and \cpageref{hl:lafage_reproduction_unac} above.\\

    \parindent=0pt
    \hangindent=0pt
  Hilarión Eslava. \emph{Museo organico español}. Madrid:  Imp de D. José C. de la Peña, 1853.

     \parindent=20pt
     \hangindent=20pt
     See \cpageref{hl:eslava} above.\\

    \parindent=0pt
    \hangindent=0pt
  Joseph Wackenthaler. \emph{L'art d'accompagner le plain-chant romain : méthode claire et facile}. Paris:  Fleury, 1854.

     \parindent=20pt
     \hangindent=20pt
     The harmonisation of eight plainchant modes are discussed separately, with example chant harmonisations being preceded by a prelude in the same mode. The chant is placed in the top part except in fauxbourdon examples when it is placed in the tenor part.\\

    \parindent=0pt
    \hangindent=0pt
  Félix Clément. \emph{Méthode complète de plain-chant d'après les règles du chant grégorien et traditionnel, à l'usage des séminaires, des chantres, des écoles normales primaires et des maîtrises}. 1st ed. Paris:  Hachette, 1854;  \underline{2nd ed.} Paris:  Hachette, 1872.

     \parindent=20pt
     \hangindent=20pt
     Takes issue with the use of the organ by certain organists whose playing reportedly does not espouse the requisite sacred values. As a result, the author proposes that some chants be left unaccompanied (pp.~355--6).\\

    \parindent=0pt
    \hangindent=0pt
  \covid{}Jakob Schmitt. \emph{Méthode d'harmonie appliquée au plain-chant}. Paris:  Lutrin de la jeunesse, 1854. \\

    \parindent=0pt
    \hangindent=0pt
  Jacques-Louis Battmann. \emph{Cours d'harmonie théorique et pratique appliqué spécialement à l'étude de l'accompagnement du plain-chant}. Paris:  Fleury, 1855.

     \parindent=20pt
     \hangindent=20pt
     Outlines the modulation method using the dominant seventh at cadences and chords in open and closed positions.\\

    \parindent=0pt
    \hangindent=0pt
  Louis Girod. \emph{De la musique religieuse}. Namur:  F.-J. Douxfils, 1855.

     \parindent=20pt
     \hangindent=20pt
     A discussion of accompaniment is entertained in the second section which strays rather dubiously into aesthetic and philosophical territories. In spite of the author's meanders, he considers the accompaniment should be subordinate to the melody if the text is to be clearly discerned (p.~147).\\

    \parindent=0pt
    \hangindent=0pt
  Georges Schmitt. \emph{Nouveau manuel complet de l'organiste praticien}. Paris:  Roret, 1855.

     \parindent=20pt
     \hangindent=20pt
     Chapter 9 compares the tradition of accompanying chant in France to customs in Germany and England, though the discussion remains largely general in nature and stops short of recommending one approach or another.\\

    \parindent=0pt
    \hangindent=0pt
  Adrien de La Fage. \emph{Cours complet de plain-chant : Nouveau traité méthodique et raisonné du chant liturgique de l'Église latine, à l'usage de tous les diocèses}. Paris:  Gaume et C\textsuperscript{ie}, 1856.

     \parindent=20pt
     \hangindent=20pt
     Advises against constructing accompaniments of chant using counterpoint. See \cpageref{hl:lafage_cours} above.\\

    \parindent=0pt
    \hangindent=0pt
  \covid{}Joseph Franck. \emph{L'art d'accompagner le plain-chant de huit manières différentes}. Paris:  Repos, 1856.

     \parindent=20pt
     \hangindent=20pt
     A separately published appendix viewed by me details a ninth manner of accompaniment in which music examples are contrived to demonstrate Niedermeyer's principles.\\

    \parindent=0pt
    \hangindent=0pt
  Léon G. Dalmières. \emph{Le Plain-chant accompagné, au moyen des notions les plus simples réduites à cinq formules harmoniques}. Saint-Étienne, 1856.

     \parindent=20pt
     \hangindent=20pt
     Uses the Socratic method to present opposing views of a debate on chant accompaniment. The five areas covered by the publication comprise harmony, chords, chord progressions, praxis of accompaniment, and the application of certain fomulæ. The copy consulted did not contain the plates, however, even though space had clearly been allotted to them during the editorial \emph{mise-en-page}.\\

    \parindent=0pt
    \hangindent=0pt
  François-Auguste Gevaert. \emph{Méthode pour l'enseignement du plain-chant et la manière de l'accompagner}. 6th ed. Gand et Liège:  Gevaert, 1856.

     \parindent=20pt
     \hangindent=20pt
     Accompaniments are restricted to 5/3 and 6/3 chords and diatonic harmony, with sharping commonplace at cadences. A brief appendix containing examples of certain harmonised chants anticipates a summative discussion on the construction of preludes in the modes. See \cpageref{hl:lafage_reproduction_unac} above.\\

    \parindent=0pt
    \hangindent=0pt
  Alexandre Bruneau. \emph{Méthode simple et facile pour apprendre à accompagner le plain-chant  avec l'orgue à clavier transpositeur écrite en musique et en plain-chant}. Bourges, 1856.

     \parindent=20pt
     \hangindent=20pt
     Music examples are duplicated in quadratic and modern notations adjacent to one another, the author suggesting that their notational dissimilarities represent ontological differences between musics ancient and modern.\\

    \parindent=0pt
    \hangindent=0pt
  Herman Hageman. \emph{Verzameling van Gregoriaansche melodiën: in vierstemmig orgelaccompagnement, enz}. Nijmegen:  C. Pothast \& Langendam en Comp., 1856.

     \parindent=20pt
     \hangindent=20pt
     See \cpageref{hl:hageman} above.\\

    \parindent=0pt
    \hangindent=0pt
  Georges Schmitt. \emph{Méthode élémentaire d'harmonisation du plain-chant expressément composée pour les commençants sans maître}. Paris:  Régnier-Canaux, 1857.

     \parindent=20pt
     \hangindent=20pt
     Outlines the modulation method with harmonised chants and dominant sevenths. The part-writing is annotated with fingerings for the benefit of less able players, while the chants themselves are categorised by mode and placed either in top or bottom parts of the keyboard texture.\\

    \parindent=0pt
    \hangindent=0pt
  J. B. Jaillet. \emph{Méthode nouvelle pour apprendre facilement l'accompagnement du plain-chant}. Paris:  Régnier-Canaux, 1857.

     \parindent=20pt
     \hangindent=20pt
     The author annotates scale steps above each note of the chant to inform the major-minor harmonic progressions in use. On some occasions, a repeated note is annotated with a different scale step where the harmony is to effect a modulation.\\

    \parindent=0pt
    \hangindent=0pt
  Louis Niedermeyer \& Joseph D'Ortigue. \emph{Traité théorique et pratique de l'accompagnement du plain-chant}. 1st ed. Paris:  Repos, 1857;  2nd ed. Paris:  Heugel, 1876.

     \parindent=20pt
     \hangindent=20pt
     A manual of far-reaching influence which is discussed at greater length above in \cref{hl:nied}.\\

    \parindent=0pt
    \hangindent=0pt
  L. Bignon. \emph{Méthode pratique d'accompagnement du plain-chant}. Paris:  Blanchet, 1858.

     \parindent=20pt
     \hangindent=20pt
     The reader is seemingly supposed to absorb the method of accompaniment from a set of provided music examples, which incorporate cadential sharping in the chord-against-note style.\\

    \parindent=0pt
    \hangindent=0pt
  Adrien de La Fage. \emph{Routine pour accompagner le plain-chant, ou moyen prompt et facile d'harmoniser à première vue le plain-chant pris pour basse, sans avoir étudié l'harmonie et sans le secours d'un maître}. 1st ed. Paris:  Régnier-Canaux, 1858;  \underline{2nd ed.} Paris:  Régnier-Canaux, 1860.

     \parindent=20pt
     \hangindent=20pt
     A set of the most common intervals in the chant repertory are harmonised by the author which, when deployed, are meant to equip the player to concatenate his or her own accompaniment of any chant melody. \\

    \parindent=0pt
    \hangindent=0pt
  Sebastien Stehlin. \emph{Chorallehre nach den Grundgesetzen des mittelalterlichen Tonsystems}. Vienna:  k.k. Hof- und Staatsdruckerei, 1859.

     \parindent=20pt
     \hangindent=20pt
     A discussion of how to transcribe from quadratic notation into modern notation (see \S{}13) segues into a method of accompaniment which advocates for use of the dominant seventh and diminished chords. One of the provided examples was intended for a women's choir which the author dutifully arranges such that the sung chant is at pitch in the organ accompaniment; another is intended a men's choir, and is arranged such that the chant is an octave below the pitch of the melody in the organ part (pp.~55--8).\\

    \parindent=0pt
    \hangindent=0pt
  Jules de Calonne. \emph{Petit guide de l'accompagnateur du chant d'église}. Paris:  Noirel et Dewingle, 1859.

     \parindent=20pt
     \hangindent=20pt
     A four-page pamphlet that contains music examples duplicated in quadratic and modern notations. The author stops short of providing examples of harmonised chant melodies, however, regulating his exposition of the rules of chord construction to two scales harmonsied according to the rule of the octave, one major and one minor.\\

    \parindent=0pt
    \hangindent=0pt
  François Guichené. \emph{Vade mecum de l'organiste, ou Guide du clavier transpositeur pour l'accompagnement de tout le chant sacré}. Paris:  Repos, 1859.

     \parindent=20pt
     \hangindent=20pt
     Describes a mechanism by the use of which a user may automate the accompaniment of chant through following certain elementary rules. A single key press is said to produce a chord, so by playing one note after another in an approved sequence the player may create their own accompaniment without requiring any training in harmony. See \cpageref{hl:guichene} above.\\

    \parindent=0pt
    \hangindent=0pt
  Théodore Nisard. \emph{Les vrais principes de l'accompagnement du plain-chant sur l'orgue d'après les maîtres du XV\textsuperscript{e} et du XVI\textsuperscript{e} siècle}. Paris:  Repos, 1860.

     \parindent=20pt
     \hangindent=20pt
     The `true principles' in question are little more than the rules of florid counterpoint which the author attempts to apply to the accompaniment of chant. In that, arguably, the author is successful, since the rules in question permit certain chant notes to function as dissonances. The approach engendered no small amount of curiosity, criticism and controversy among the author's peers since the \emph{status quo} at the time of publication (at least in French and Belgian circles) was for each chant note to be harmonised consonantly. What cannot have been reassuring to some critics was the author's inclusion of cadential sharping in the music examples: this, at a time when diatonic theories were becoming \emph{à la mode}, could have been seen as a regressive step. Nonetheless, the principles proved highly influential in many quarters and inspired later musicians to reduce the number of chords in their own accompaniments. See \cpageref{hl:nisard_dissonance} above.\\

    \parindent=0pt
    \hangindent=0pt
  Théodore Nisard. \emph{L'accompagnement du plain-chant sur l'orgue enseigné en quelques lignes de musique et sans le secours d'aucune notion d'harmonie}. Paris:  Repos, 1860.

     \parindent=20pt
     \hangindent=20pt
     Although this manual was envisaged as the practical complement to the author's \emph{Les vrais principes}, its intended audience was more likely to have been less able musicians. The author describes a simplistic method that such musicians could use to arrive at their own accompaniment without needing to learn innumerable harmonic rules: ostensibly, the provided six and a half lines of chords with seventeen exceptions were all the harmonic resources a practitioner would require to discharge their responsibilites. \\

    \parindent=0pt
    \hangindent=0pt
  Charles-Louis Hanon. \emph{Système nouveau pratique et populaire pour apprendre à accompagner tout plain-chant à première vue en six leçons sans savoir la musique et sans professeur}. 4th ed. Boulogne-sur-Mer, \emph{c}.1860.

     \parindent=20pt
     \hangindent=20pt
     A novel system of annotations is set out using arcs above and below the chant to prompt the player to select chords from a numbered set. Those sets were devised by the author to suit supposedly common sequences of intervals that crop up in the chant repertory.\\

    \parindent=0pt
    \hangindent=0pt
  Stephen Morelot. \emph{Eléments d'harmonie appliquée à l'accompagnement du plain-chant d'après les traditions des anciennes écoles}. Paris:  Lethielleux, 1861.

     \parindent=20pt
     \hangindent=20pt
     The author resolves against the use of florid counterpoint to accompany chant and proposes that the chant be harmonsied consonantly instead.\\

    \parindent=0pt
    \hangindent=0pt
  \covid{}Emile Amiot \& Philippe Morin. \emph{Méthode élémentaire de l'accompagnement du plain-chant sur l'orgue transpositeur}. 2nd ed. Dijon:  Peutet-Pommes, 1861;  3rd ed. Paris:  Humbert, 1862. \\\pagebreak{}

    \parindent=0pt
    \hangindent=0pt
  Joseph Alémany. \emph{Méthode simple et facile pour apprendre soi-même à accompagner avec l'orgue le plain-chant et les cantiques}. Lyon:  J. B. Pélagaud et C\textsuperscript{ie}, 1862.

     \parindent=20pt
     \hangindent=20pt
     Conflates the modes with major and minor scales, though it should be noted that the copy consulted lacked the plates of music examples.\\

    \parindent=0pt
    \hangindent=0pt
  Adolphe Populus. \emph{Études sur l'orgue}. Paris:  Benoit ainé, 1863.

     \parindent=20pt
     \hangindent=20pt
     Some findings from the Paris congress of 1860 are discussed, including a music example of three bass lines with greater disjunct motion. The most conjunct was said to be the ideal. See \cpageref{hl:populus} above.\\

    \parindent=0pt
    \hangindent=0pt
  L. Petit. \emph{L'orgue pratique : gammes harmoniques majeurs et mineurs pour les huit tons du plain-chant dominantes la et sol }. Abbeville:  Vitoux, 1863.

     \parindent=20pt
     \hangindent=20pt
     This two-page pamphlet conflates the modes with major and minor scales, and places the chant in the bottom part whereas the upper parts are worked out according to the rule of the octave.\\

    \parindent=0pt
    \hangindent=0pt
  Charles Geispitz. \emph{Méthode complète pour l'application facile et immédiate de l'harmonie au plain-chant}. Soissons:  Hacard, 1863.

     \parindent=20pt
     \hangindent=20pt
     Music examples arranged in both \emph{solfège} and regular notation are used to appeal to an amateur audience. Fourteen plainchant modes are discussed individually, but not all are given rubrics: the reader is instructed to base harmonisations in the thirteenth and fourteenth modes on the rules provided for the fifth and sixth respectively. Tables of chords are provided instead of full music examples.\\

    \parindent=0pt
    \hangindent=0pt
  \covid{}Jean-Baptiste Labat. \emph{Etude sur l'harmonisation du chant des psaumes}. Montauban:  V. Bertuot, 1864. \\

    \parindent=0pt
    \hangindent=0pt
  Franz Xaver Haberl. \emph{Magister choralis}. 1st ed. Regensburg:  Friedrich Pustet, 1864;  4th ed. Regensburg:  Friedrich Pustet, 1877. Translated by Nicolas Donnelly;  \underline{9th ed.} Regensburg:  Friedrich Pustet, 1892. Translated by Nicolas Donnelly;  12th ed. Regensburg:  Friedrich Pustet, 1900.

     \parindent=20pt
     \hangindent=20pt
     Chapter 40, entitled `Upon Organ Accompaniment to Gregorian Chant', provides two sets of rules, general and specific, by which an accompaniment is to be constructed. The short, demonstrative music examples leave much to be desired since most do not contain inner parts. Although bass figures provide some indication of the required effect, the chordal textures they imply distance these examples from accompaniments by other Cecilian composers, such as Witt and Hanisch.
Perhaps that might explain why quotations from books of accompaniments by those figures is provided in an attempt to `illustrate' what Haberl had been describing, but they too must surely have been equally as impenetrable to the novice, who was seemingly expected to absorb their content without relevant guidance. The author reserves most of the discussion of organists and their manner of playing for Chapter 42, entitled `For Organists', in which the relationship between the organ and the accompaniment is briefly discussed; however, the material broached by author is arguably more of a summary of ideas and he makes few recommendations for performance practice.\\

    \parindent=0pt
    \hangindent=0pt
  F. Moncouteau. \emph{Méthode d'accompagnement du plain-chant}. Paris:  Adrien Le Clerc, 1864.

     \parindent=20pt
     \hangindent=20pt
     Chants placed in either the bottom or top parts of the texture are harmonised according to the modulation method.\\

    \parindent=0pt
    \hangindent=0pt
  \covid{}F. Auger. \emph{Méthode simple et facile pour accompagner en deux leçons le plain-chant}. Romorantin, 1864. \\

    \parindent=0pt
    \hangindent=0pt
  \covid{}Jean Baptiste Augustin Marie Joseph Déon. \emph{Méthode simplifiée pour l'accompagnement traditionnel du plain-chant sur l'orgue-harmonichordéon suivie d'un appendice sur les fonctions des registres}. Paris, 1864. \\

    \parindent=0pt
    \hangindent=0pt
  \covid{}Edmond Duval. \emph{Quelques considérations sur l'accompagnement diatonique du plain-chant par l'orgue}. Malines:  H. Dessain, 1864. \\

    \parindent=0pt
    \hangindent=0pt
  \covid{}Alexandre Bruneau. \emph{Nouvelle méthode simple et facile pour apprendre à accompagner le plain-chant sur tout orgue à clavier transpositeur}. Paris, 1865.

     \parindent=20pt
     \hangindent=20pt
     Likely to be a revision of that manual by Bruneau published in 1856.\\\pagebreak{}

    \parindent=0pt
    \hangindent=0pt
  Ludwig Schneider. \emph{Gregorianische Choralgesänge für die Hauptfeste des Kirchenjahres}. Frankfurt am Main:  G. Hamacher, 1866. Edited by Franz Joseph Mayer \& Erwin Schneider.

     \parindent=20pt
     \hangindent=20pt
     Sets forth eleven rules for diatonic harmonisation similar to those principles propagated in Francophone circles by Niedermeyer and d'Ortigue, though differences between the two methods exist and are discussed above (\cpageref{hl:schneider}). \\

    \parindent=0pt
    \hangindent=0pt
  Henry Poncet. \emph{Harmonie du plain-chant ou Méthode d'accompagnement pour la musique sacrée}. Aix:  Remondet-Aubin, 1868.

     \parindent=20pt
     \hangindent=20pt
     While the discussion generally concerns itself with the modulation method, the author also advises on maintaining variety in the accompaniment.\\

    \parindent=0pt
    \hangindent=0pt
  \covid{}Karl Emil von Schafhäutl. \emph{Der echte gregorianische Choral in seiner Entwikelung bis zur Kirchenmusik unserer Zeit}. Munich, 1869. \\

    \parindent=0pt
    \hangindent=0pt
  \covid{}J. N. Cayatte. \emph{Essay d'une introduction facile à l'accompagnement du plain-chant}. Billy-lès-Mangiennes, 1869. \\

    \parindent=0pt
    \hangindent=0pt
  Clément Burotto. \emph{La restauration du plain-chant et de son accompagnement}. Paris:  E. Gérard et C\textsuperscript{ie}, 1869.

     \parindent=20pt
     \hangindent=20pt
     The author notes that the pitches in the chant must surely suggest harmonising them in major and minor keys instead of resorting to modal harmonisations. See above on p.~\pageref{hl:burotto}.\\

    \parindent=0pt
    \hangindent=0pt
  Charles Dupart. \emph{Leçons pratiques et théoriques pour l'accompagnement du plain-chant sur l'orgue ou l'harmonium}. Lons-le-Saunier:  Gauthier Frères, 1869.

     \parindent=20pt
     \hangindent=20pt
     Contains about 220 harmonised intervals or short snippets that the pupil was supposed to repeat up to twenty times each to learn how to accompany chant. Since the manual does not contain much actual pedagogy (aside from how fourteen modes were arrived at and a brief exposition of the rules of part movement) it cannot have held much practical value for the committed student.\\

    \parindent=0pt
    \hangindent=0pt
  Eugène Henry. \emph{Méthode pour accompagner facilement et correctement le plain-chant, avec ou sans clavier transpositeur}. \underline{1st ed.} Rennes:  Bonnel, 1869;  3rd ed. Rennes:  Bonnel, 1878;  4th ed. Rennes:  Bonnel, 1889.

     \parindent=20pt
     \hangindent=20pt
     Although the transposing keyboard is referenced in the title, the author makes little ado of it. The manual was nonetheless popular enough to sell out its first two editions and gained a readership beyond that initially anticipated by the author. He therefore reportedly recast the material to suit his new audience. The modes are conflated with major and minor scales and the modulation method serves as the basis of harmonisation. See \cpageref{hl:eugenehenry} above.\\

    \parindent=0pt
    \hangindent=0pt
  Raymund Schlecht. \emph{Geschichte der Kirchenmusik}. Regensburg:  Verlag von Alfred Coppenrath, 1871.

     \parindent=20pt
     \hangindent=20pt
     Recommends the organ accompaniment of chant to cover up deficiencies in singing where they exist; and where they do not, the accompaniment is said to produce an overall better effect. A sustained style is to be preferred for longer chants where incessantly chordal accompaniments can cause fatigue; harmonisations should also be kept largely diatonic in nature (pp.~191--3).\\

    \parindent=0pt
    \hangindent=0pt
  \covid{}O. Naudet. \emph{Méthode très-élémentaire d'harmonium pour l'accompagnement du plain-chant à l'usage des commençants}. Vivey, 1871. \\

    \parindent=0pt
    \hangindent=0pt
  François-Auguste Gevaert. \emph{Vade-mecum de l'organiste contenant les chants les plus usuels de l'église catholique}. Gand et Liège:  Gevaert, 1871. In collaboration with Pierre-Jean Van Damme.

     \parindent=20pt
     \hangindent=20pt
     See above in \cref{ln:gevaert_new}.\\

    \parindent=0pt
    \hangindent=0pt
  \covid{}J. F. Meilhan. \emph{L'accompagnement correct et caractéristique du plain-chant romain d'après l'édition publiée à Rennes chez Vatar}. Nantes, 1872. \\

    \parindent=0pt
    \hangindent=0pt
  \covid{}Léon Roques. \emph{L'accompagnement du plain-chant mis à la portée de tout le monde}. Paris:  Hachette, \emph{c}.1872. \\\pagebreak{}

    \parindent=0pt
    \hangindent=0pt
  Félix Clément. \emph{Méthode d'orgue, d'harmonie et d'accompagnement comprenant toutes les connaissances nécessaires pour devenir un habile organiste}. 1st ed. Paris:  Hachette, 1873;  \underline{2nd ed.} Paris:  Hachette, 1894.

     \parindent=20pt
     \hangindent=20pt
     Seeks to rejuvenate the principles of the French Classical organ school and to oppose the diatonic principles of Niedermeyer. These, the author reckoned, where foreign to the Catholic liturgy since Niedermeyer was Protestant; but in spite of his theological purism, the author was evidently not so concerned with musical purism, for his part-writing exhibits certain prohibited intervals. See \cpageref{hl:felixclement} above.\\

    \parindent=0pt
    \hangindent=0pt
  Heinrich Oberhoffer. \emph{Die Schule des katholischen Organisten : Theoretisch-praktische Orgelschule}. 2nd ed. Trier:  Lintz, 1874.

     \parindent=20pt
     \hangindent=20pt
     The discussion of accompaniment in \S{}19 shows the author to be up-to-date with the latest developments in Germany, Belgium and France. An ideal system is then proposed with the following rules: the harmony is to be modal; 5/3 and 6/3 triads are to be used, though tetrads may be permitted over pedal notes or with an active bass part; while seventh chords are not outlawed, their use should be limited; syllabic accompaniments require a new chord or bass note for every one or two chant notes, while melsimatic accompaniments require fewer chords lest the accompaniment should become stiff and cumbersome (`steif und schwerfällig'); and tonic and dominant triads of the mode should be made most prominent. Some further specialised rules are then provided, including the requirement to change bass notes if the chant repeats often, to use suspensions where possible, and to avoid consecutive octaves and fifths (pp.~83--4).\\

    \parindent=0pt
    \hangindent=0pt
  \covid{}Ch. Roulleaux-Dugage. \emph{Petit traité pratique d'harmonisation du plain-chant}. Paris:  Jules Heinz, 1875. \\

    \parindent=0pt
    \hangindent=0pt
  Ignacio Ovejero. \emph{ Escuela del organista y tratado de canto llano}. Madrid:  Andrés Vidal, 1876.

     \parindent=20pt
     \hangindent=20pt
     The author adopts the chorale texture in his accompaniments. See the discussion above (\cpageref{hl:ovejero}).\\

    \parindent=0pt
    \hangindent=0pt
  \covid{}Léon Bernard. \emph{La théorie et la pratique du chant grégorien : ouvrage suivi des Principes élémentaires d'accompagnement diatonique}. Tournai:  Casterman, 1876. \\

    \parindent=0pt
    \hangindent=0pt
  Abbé Falaise. \emph{Méthode théorique et pratique de plain-chant suivie des principes de la musique et de dix-sept gammes d'harmonie pour l'accompagnement pratique et raisonné du chant en général}. 2nd ed. Paris:  Victor Sarlit, 1876.

     \parindent=20pt
     \hangindent=20pt
     The modes are conflated with major and minor scales, seventeen of which are pre-harmonised to benefit accompanists of chant.\\

    \parindent=0pt
    \hangindent=0pt
  \covid{}Charles Duluc. \emph{L'accompagnement du plain-chant mis à la portée de tout le monde}. 2nd ed. Paris:  Pérégally et Parvy, 1877. \\

    \parindent=0pt
    \hangindent=0pt
  \covid{}V. Ballu. \emph{Un mot sur le plain-chant, sa tonalité, son rythme et son accompagnement}. Paris:  Cartereau, 1878. \\

    \parindent=0pt
    \hangindent=0pt
  Antonin Lhoumeau. \emph{De l'altération ou du demi-ton accidentel dans la tonalité du plain-chant}. Niort:  L. Clouzot, 1879.

     \parindent=20pt
     \hangindent=20pt
     Proposes that sharped pitches be admitted in accompaniments after a theory of tetrachordal substitution inherited from the writings of the music historian Stéphen Morelot. See \cpageref{hl:lhoumeau_alteration} above.\\

    \parindent=0pt
    \hangindent=0pt
  \covid{}C. Hubert. \emph{L'art d'accompagner la musique et le plain-chant sur l'harmonium, le grand orgue et le piano}. Toulon, 1879. \\

    \parindent=0pt
    \hangindent=0pt
  \covid{}Ernest Grosjean. \emph{Théorie et pratique de l'accompagnement du plain-chant}. 4th ed. Verdun:  Meuse, 1879. \\\pagebreak{}

    \parindent=0pt
    \hangindent=0pt
  Dudley Buck. \emph{Illustrations with Choir Accompaniment with Hints in Registration}. New York:  G. Schirmer, 1880;  2nd ed. New York:  G. Schirmer, 1892.

     \parindent=20pt
     \hangindent=20pt
     Although the author's discussion, in Chapter Five, of `Accompaniment of the Chant' deals primarily with Anglican chant, some remarks are offered as to the accompaniment of Gregorian chant. In particular, the author suggests that the Gregorian repertory requires `no different treatment as to manner of accompaniment from the Anglican single or double chants', and the organist is permitted `to show his skills in varied harmonization of the unison melody'. The author neglects to elaborate on his claim that Gregorian chant belongs to `old Greek scales', and therefore requires a different harmonisation. Although the editions consulted were published in different years, each contained the same prose.\\

    \parindent=0pt
    \hangindent=0pt
  Louis Müller. \emph{Petit traité d'harmonie ou leçons élémentaires et pratiques pour accompagner le plain-chant}. Paris:  Colombier, 1880.

     \parindent=20pt
     \hangindent=20pt
     Music examples in minims are set adjacent to the same but in crotchets with the chant being placed in the bottom part of a four-part texture. Modulation according to a `circle of tonalities' is said to constitute a viable theory of chant harmonisation.\\

    \parindent=0pt
    \hangindent=0pt
  \covid{}Joseph Matly. \emph{Petit traité du plain-chant et de son accompagnement à l'usage des organistes}. Tréguier:  Le Flem., 1880. \\

    \parindent=0pt
    \hangindent=0pt
  B. Allard. \emph{Transposition et accompagnement du plain chant}. Paris:  L. Leconte \& C\textsuperscript{ie}, 1880.

     \parindent=20pt
     \hangindent=20pt
     Numbers annotated above quadratic chant prompt a player to choose the correspondly numbered chord from pre-harmonised sets. Two such sets are provided for each mode with dominants on `G' and `A', thereby allowing a player to offer a choice of transpositions. See \cpageref{hl:allard} above.\\

    \parindent=0pt
    \hangindent=0pt
  Michael Joannes Antonius Lans. \emph{De katholieke organist: onderricht in de begeleiding an den Gregoriaanschen zang en in het kerkelijk orgelspel met een aantal speeloefeningen}. Leiden:  J.W. van Leeuwen, 1881.

     \parindent=20pt
     \hangindent=20pt
     The third section is dedicated to the accompaniment of chant and follows a plan similar to Oberhoffer's.\\

    \parindent=0pt
    \hangindent=0pt
  \covid{}Alphonse Chabot. \emph{Méthode d'harmonium facile et raisonnée pour accompagner tout cantique à première vue}. 1st ed. Paris, 1881. \\

    \parindent=0pt
    \hangindent=0pt
  J. B. Bischoff. \emph{Méthode élémentaire d'orgue, d'harmonie et de plain-chant}. Rodez, 1881.

     \parindent=20pt
     \hangindent=20pt
     Attempts to distill Niedermeyer's principles into a more accessible format to suit those without much musical training. In four sections, the text is divided into lessons on the fundamentals of the harmonium, the organ, harmony in general and chant harmonisation in particular. The author includes examples of the last that are reminiscent of Niedermeyer's, but fingered for the benefit of students.\\

    \parindent=0pt
    \hangindent=0pt
  E. Radureau. \emph{Harmonisation du plain-chant}. 3rd ed. Moulins:  A. Ducroux \& Gourjon Dulac, 1882.

     \parindent=20pt
     \hangindent=20pt
     Harmonising in key of F major is recommended for chants that contain B\flat{}, while harmonising in the key of C major is recommended for chants that do not.\\

    \parindent=0pt
    \hangindent=0pt
  F. M. Jubin. \emph{Méthode d'harmonium sur un plan nouveau et traité d'harmonie appliqué à l'accompagnement du plain-chant et des cantiques}. Lyon:  Albert, 1882.

     \parindent=20pt
     \hangindent=20pt
     A set of exercises does not require the students to play the chant in their accompaniments, but to use sustained chords based on the harmony implied by the chant instead. Curiously, the author's example accompaniments are in the homorhythmic, chord-against-note style where the chant is placed in the top part of the texture.\\

    \parindent=0pt
    \hangindent=0pt
  C. G.. \emph{Accompagnement du plain-chant}. Paris:  Victor Sarlit, 1884.

     \parindent=20pt
     \hangindent=20pt
     See \cpageref{hl:cg_rules} above.\\

    \parindent=0pt
    \hangindent=0pt
  Théodore Dubois. \emph{Accompagnement pratique du plain-chant à la basse et à la partie supérieure à l'usage des personnes qui savent peu ou pas l'harmonie}. Paris:  Parvy, 1884.

     \parindent=20pt
     \hangindent=20pt
     This didactic method conflates fourteen plainchant modes with major and minor scales in the interest of simplicity. See \cpageref{hl:dubois} above.\\\pagebreak{}

    \parindent=0pt
    \hangindent=0pt
  Antonin Lhoumeau. \emph{De l'harmonisation des mélodies grégoriennes et du plain-chant en général}. Niort:  Thibaut-Aimé, 1884.

     \parindent=20pt
     \hangindent=20pt
     The capabilities of the modern organ are embraced, with example accompaniments bearing witness to manual changes, passages where the pedal part flits in and out, and trills. The appearance of the chant is sometimes fleeting as its poisition in the texture changes. See \cpageref{hl:lhoumeau_1884} above.\\

    \parindent=0pt
    \hangindent=0pt
  Paul Schmetz. \emph{Dom Pothier's Liber Gradualis (Tournayer Ausgabe), seine historische und praktische Bedeutung mit 7 Facsimiles einer vor dem Jahre 1379 geschriebenen Pergamenthandschrift}. Mainz:  Franz Kirchheim, 1884.

     \parindent=20pt
     \hangindent=20pt
     \S{}III attempts to apply Pothier's oratorical rhythm to a theory of chant harmonisation by adopting a five-line version of Desclée's quadratic notation, arranged in two staves, one bearing a treble clef and the chant melody and the other a bass clef and a rudimentary bass part. The latter is sometimes figured where chords other than 5/3 are required. In spite of the accompaniment following a chord-against-note style, the author evidently intended its rhythm to follow nuances deemed inherent in the chant part; these are imputed to the bass part by means of its neumatic layout matching that of the chant. See above on p.~\pageref{hl:schmetz_1884}.\\

    \parindent=0pt
    \hangindent=0pt
  \covid{}Eugène Weiss. \emph{Etude sur l'harmonisation du chant liturgique}. Paris:  Soc. anon. des Publications périodiques, 1884. \\

    \parindent=0pt
    \hangindent=0pt
  Eugène Baré. \emph{Nouvelle méthode simple et facile pour apprendre à accompagner le plain-chant avec le clavier transpositeur, contenant les principes élémentaires de la musique et du plain-chant, ainsi que des instructions sur le mécanisme et l'entretien des harmoniums}. Paris:  Delay, \emph{c}.1884.

     \parindent=20pt
     \hangindent=20pt
     The author acknowledges not only the transposing keyboard as a useful mechanism for accompanying chant but also the influence of Niedermeyer and d'Ortigue's diatonic method of chant harmony. See \cpageref{hl:bare} above.\\

    \parindent=0pt
    \hangindent=0pt
  Paul Schmetz. \emph{Die Harmonisierung des gregorianischen Choralgesanges : Ein Handbuch zur Erlernung der Choralbegleitung}. 1st ed. Dusseldorf:  L. Schwann, 1885;  \underline{2nd ed.} Dusseldorf:  L. Schwann, 1894.

     \parindent=20pt
     \hangindent=20pt
     Seven rules of a chant harmonisation are contrived to apply Pothier's rhythmic principles to the placement of chords. See \cpageref{hl:schmetz_1894} above.\\

    \parindent=0pt
    \hangindent=0pt
  \covid{}Eugène Henry. \emph{L'art d'accompagner le plain-chant à l'aigu par mouvement contraire}. 2nd ed. Châlons sur Marne:  Barbat, 1885. \\

    \parindent=0pt
    \hangindent=0pt
  C. Warwick Jordan. \emph{One Hundred and Fifty Harmonies for the Gregorian Tones With a Few Remarks as to their Accompaniment}. London:  Novello, Ewer and Co, \emph{c}.1885.

     \parindent=20pt
     \hangindent=20pt
     Recommends, among other techniques, the process of `free accompaniment' where the organist plays elaborative, contrapuntal passages to accompany recitations. Otherwise, the psalm tones have been structured in such a way as to conform to the principles of Anglican chanting.\\

    \parindent=0pt
    \hangindent=0pt
  William Stevenson Hoyte. \emph{Organ accompaniment of the choral service; practical suggestions to organists as to the selection and treatment of church music}. London:  Novello \& Co., \emph{c}.1885. Edited by John Frederick Bridge.

     \parindent=20pt
     \hangindent=20pt
     Appendix B discusses Gregorian accompaniment and advises that the intonation be given by the left hand and pedals in octaves. The  author argues that strident organ registrations should be used since chant is ordinarily sung in unsion, up to Principal on the Great organ with full Swell, manuals coupled to a 16' and 8' pedal registration\\

    \parindent=0pt
    \hangindent=0pt
  Jacques-Nicolas Lemmens. \emph{Du chant grégorien : sa mélodie, son rythme, son harmonisation}. Gand:  Duclos, 1886.

     \parindent=20pt
     \hangindent=20pt
     This posthumously published manual recommends a procedure of sustained accompaniment where multiple chant notes are accompanied by a single chord. See above in \cref{hl:lemmens_passingnotes}.\\

    \parindent=0pt
    \hangindent=0pt
  \covid{}John Wilberforce Doran \& Edward Dale Galloway. \emph{Intermodal Harmonies for the Gregorian Psalm Tones Preceded by an Explanatory Preface: Also, a Diatonic Harmony for the Responses at Mattins and Evensong According to the Sarum or Ancient English Use}. London:  Novello, Ewer and Co, \emph{c}.1886. \\\pagebreak{}

    \parindent=0pt
    \hangindent=0pt
  Arthur Rousseau. \emph{Le petit harmoniste grégorien, nouvelle édition contenant les principes de musique, de plain-chant et d'harmonium, l'harmonisation naturelle et artificielle du chant grégorien, sa transposition et l'accompagnement des cantiques populaires}. Prigonrieux-Laforce, 1886;  \underline{2nd ed.} Bourdeille, 1889.

     \parindent=20pt
     \hangindent=20pt
     Provides annotations corresponding to a basic set of rules of chord construction. The appropriate chord is to be supplied by the player at a given annotation. See \cpageref{hl:rousseau} above.\\

    \parindent=0pt
    \hangindent=0pt
  Frère Mélit-Joseph. \emph{Cours intuitif d'harmonie et d'accompagnement divisé en quatre parties : L'étude des accords et de leurs enchaînements; La modulation et l'improvisation; L'accompagnement de la mélodie; L'harmonisation du plain-chant}. 1st ed. Leipzig:  Breitkopf und Härtel, 1887;  \underline{3rd ed.} Leipzig:  Breitkopf und Härtel, 1908.

     \parindent=20pt
     \hangindent=20pt
     In part IV, on the accompaniment of chant, the author quotes Pierre-Jean Van Damme on the avoidance of dissonant chords (p.~124), and music examples are provided in the filled-and-void notational style (see above in \cref{hl:filled_and_void}). Features common to Van Damme's notation are stems used when two parts share the same note; triplets indicated in some inner parts; parenthesized notes; and the lengths of accompanying notes rendered as void notes (which are perhaps also dotted) to match the number of filled notes being accompanied. The use of obliques to indicate rests is one notable departure from Van Damme's notation; moreover, the use of thick-set breves for recitations is also a noteworthy feature.\\

    \parindent=0pt
    \hangindent=0pt
  \covid{}Fritz Volbach. \emph{Lehrbuch der Begleitung des gregorianischen Gesanges und des deutschen Chorals in den Kirchentonarten nach den Grundsätzen des polyphonen Satzes}. Berlin:  Heine, 1888. \\

    \parindent=0pt
    \hangindent=0pt
  Abbé Dedun. \emph{Le Système `trois d'un' (ou trois indications à l'aide d'un seul signe) pour accompagner facilement le plain-chant}. 2nd ed. Nancy:  R. Vagner, 1889.

     \parindent=20pt
     \hangindent=20pt
     Sets out a novel notational system ostensibly to simplify accompaniments of plainchant; in reality, however, the density of information conveyed by the author's unfamiliar symbols cannot have provided many advantages over modern notation. For a more detailed description of the notation, see above (\cpageref{hl:dedun_troisdun}).\\

    \parindent=0pt
    \hangindent=0pt
  \covid{}Alexandre Bruneau. \emph{Méthode pour harmonium en musique et plain-chant}. Paris:  Canaux, 1889. \\

    \parindent=0pt
    \hangindent=0pt
  Auguste Teppe. \emph{Premier problème grégorien : nature et fixation du rythme liturgique paroissial}. Châlons sur Marne:  F. Thouille, 1889.

     \parindent=20pt
     \hangindent=20pt
     Described by the author's biographer as `a bit abstract', the rhythmic theory outlined in this text is seemingly mensural in conception. To demonstrate its applicability to chant harmonisation, the author commissioned Eugène Gigout to compose two accompaniments in different styles on Teppe's proprietary transcription of the chant into modern notation. See \cpageref{hl:teppe_premier} above. A second book dealing the `second problème grégorien, or the process of harmonisation, was considerde by the author to be otiose. See \cref{hl:teppe_second}. The author had also planned to publish a third book, but that did not come to light either.\\

    \parindent=0pt
    \hangindent=0pt
  \covid{}Pierre Denis. \emph{Essais sur l'harmonisation du chant grégorien : suivis de plusieurs appendices}. Paris:  R. Haton, 1890. \\

    \parindent=0pt
    \hangindent=0pt
  \covid{}Heinrich Böckeler. \emph{Harmonielehre für Kirchen-Musik Aufgabenheft I Für 4 und 3 stimmigen Satz mit Grundakkorden}. Aachen:  Verl. des Gregorius-Hauses, \emph{c}.1890. \\

    \parindent=0pt
    \hangindent=0pt
  Edgar Tinel. \emph{Le chant grégorien : théorie sommaire de son exécution}. Malines:  H. Dessain, 1890.

     \parindent=20pt
     \hangindent=20pt
     Although chiefly a text on chant practice, it recommends that the organ accompaniment ought not to be too loud. See \cpageref{hl:tinel_organ} above.\\

    \parindent=0pt
    \hangindent=0pt
  Eustoquio de Uriarte. \emph{Tratado teórico-práctico de canto Gregoriano según la verdadera tradición}. Madrid:  Imprenta De Don Luis Aguado, 1890.

     \parindent=20pt
     \hangindent=20pt
     Chapter 9, `Del órgano y de los organistas', represents the introduction of Haberl's ideas on chant accompaniment to the Hispanosphere, though not without a word of caution since it was admitted that the line between sound doctrine and unsubstantiated claims was often blurred in Haberl's writings. See \cpageref{hl:uriarte} above.\\\pagebreak{}

    \parindent=0pt
    \hangindent=0pt
  V. Auzet. \emph{L'accompagnement artistique du plain-chant : méthode théorique et pratique}. Paris:  E. L'Huillier et C\textsuperscript{ie}, 1891.

     \parindent=20pt
     \hangindent=20pt
     Annotatations prompt a player to apply a set of rules governing chord construction. See \cpageref{hl:auzet} above.\\

    \parindent=0pt
    \hangindent=0pt
  Jules de Calonne. \emph{A. B. C. de l'harmonie appliquée au plain-chant}. Paris:  E. Fromant, 1892.

     \parindent=20pt
     \hangindent=20pt
     In this five-page pamphlet wherein music examples are duplicated in quadratic and normal notations and set adjacent to one another, the author's brief descriptions of intervals, chords and the rule of the octave cannot have proven that enlightening to any student, particularly since not a single music example is provided to illustrate his methodology. The manual was nonetheless furnished with an attractive title, which would surely have beguiled hapless amateurs. See \cpageref{hl:calonne_abc} above.\\

    \parindent=0pt
    \hangindent=0pt
  J. Brétêcher. \emph{Accompagnement du plain-chant et des cantiques populaires, grammaire musicale des principes et des règles élémentaires de l'harmonie}. 1st ed. Nantes:  Imp. Bourgeois, \emph{c}.1892;  \underline{5th ed.} Nantes:  Imp. Bourgeois, 1909.

     \parindent=20pt
     \hangindent=20pt
     Provides harmonised major and minor scales to be applied to the accompaniment of chant. Stylistically, the accompaniments contain largely 5/3 chords, and certain phrase endings are annotated in \emph{solfège}, indicating the keys to which the harmony must modulate.\\

    \parindent=0pt
    \hangindent=0pt
  F. Emery-Desbrousses. \emph{Études et biographes musicales suivies d'un aperçu sur les origines et l'harmonisation du plain-chant}. Paris:  Fischbacher, 1892. Edited by Henry Eymieu.

     \parindent=20pt
     \hangindent=20pt
     Relays Niedermeyer's principles before relaying that both Charles-Marie Widor and Théodore Dubois both follow them (p.~160). See \cpageref{hl:widordubois} above.\\

    \parindent=0pt
    \hangindent=0pt
  \covid{}A. Lourdault. \emph{Notions d'harmonisation du plain-chant}. Hainaut, 1892. \\

    \parindent=0pt
    \hangindent=0pt
  Antonin Lhoumeau. \emph{Rhythme, exécution et accompagnement du chant grégorien}. Tournai:  Desclée, 1892.

     \parindent=20pt
     \hangindent=20pt
     A textbook of far-reaching influence which is discussed above in \cref{hl:lhoumeau_1892}.\\

    \parindent=0pt
    \hangindent=0pt
  Louis Lepage. \emph{Traité de l'accompagnement du plain-chant}. 1st ed. Rennes:  Bossard-Bonnel, 1894;  2nd ed. Rennes:  Bossard-Bonnel, 1900.

     \parindent=20pt
     \hangindent=20pt
     The second edition includes a supplementary volume on `Notes Foreign to Chords'. The chain of events leading to that addition (as well as the circumstances leading to the use of a notational style for chant accompaniments pioneered at Solesmes) may be consulted above (\cpageref{hl:lepage}).\\

    \parindent=0pt
    \hangindent=0pt
  J. B. Berrouiller. \emph{ L'Accompagnateur du plain chant formé rapidement au moyen de gammes formules et marches harmoniques, et Psalmodies harmonisées}. Paris:  E. Gobert, 1895.

     \parindent=20pt
     \hangindent=20pt
     No descriptive prose is provided to explain how a set of harmonised scales in various modes relate to the accompaniment of plainchant. Instead, the author follows up these scales with harmonised tenors, mediations and terminations of psalm tones. Seemingly, the author purged any mention of transposition from his method, and added the following sentence to the title page: `Transposition is removed from this method, which facilitates accompaniment on the organ' (`Cette méthode supprime la transposition du clavier et facilite l'accompagnement sur les orgues').\\

    \parindent=0pt
    \hangindent=0pt
  \covid{}George Max. \emph{L'accompagnement du plain-chant}. \emph{c}.1895. \\

    \parindent=0pt
    \hangindent=0pt
  François-Auguste Gevaert. \emph{La mélopée antique dans l'Eglise latine}. Gand:  A. Hoste, 1895.

     \parindent=20pt
     \hangindent=20pt
     Although chiefly a textbook on music in antiquity, Gevaert concludes that the accompaniment of chant is to be dismissed, save in cases where vocal support is required. See above in \cref{hl:gevaert_1895}.\\

    \parindent=0pt
    \hangindent=0pt
  Louis Lootens. \emph{La théorie musicale du chant grégorien}. Paris:  Thorin et fils, 1895.

     \parindent=20pt
     \hangindent=20pt
     Recommends that the harmony of a plainchant accompaniment should be modernised as follows: first, that consecutive fifths and octaves be made permissible; second, that the dissonance in a dominant seventh chord need not resolve; and third, that false relations be permitted. The chant repertory is said to be replete with modulations, a tenet the author relies upon to justify the many cadences in his example accompaniments. Harmonisations by figures in the Haberl circle are criticised for not maintaining a common modal dominant.\\

    \parindent=0pt
    \hangindent=0pt
  \covid{}E. Meindre. \emph{Methode elementaire et complete pour l'accompagnement du plain-chant specialement destinée aux ecclesiastiques et aux eleves des seminaires et des maitrises}. Agen:  Imprimerie de Prosper Noubel, 1895. \\

    \parindent=0pt
    \hangindent=0pt
  \emph{The Elements of plainsong; compiled from a series of lectures delivered before the members of the Plainsong and Mediaeval Music Society}. 1st ed. London:  Bernard Quaritch, 1895. Edited by Henry Bremridge Briggs;  2nd ed. London:  The Plainsong and Mediaeval Music Society, 1909. Edited by Henry Bremridge Briggs.

     \parindent=20pt
     \hangindent=20pt
     The article on accompaniment by Walter Howard Frere (1863--1938) is subtly edited in the second edition, though remains general in nature and does not contain a single harmonised music example. With that being said, each edition discusses the two opposing viewpoints held by `the extreme purist', who denies the permissibility of accompaniment, and `the extreme vandal', who capitalises on modern harmonic resources in his or her accompaniments. \\

    \parindent=0pt
    \hangindent=0pt
  Josef Schildknecht. \emph{Orgelschule mit besonderer Rücksicht auf das Orgelspiel beim Kath.\ Gottesdienste}. Regensburg:  Verlag von Alfred Coppenrath, 1896.

     \parindent=20pt
     \hangindent=20pt
     Further editions appeared following the author's death, whose editors updated the principles of accompaniment according to the latest developments in harmony, rhythm and texture. See \cpageref{hl:schildknecht_orgelschule} above.\\

    \parindent=0pt
    \hangindent=0pt
  Célestin Leroy. \emph{Méthode pour accompagner le plain-chant et les cantiques}. Nantes:  Lanoë-Maseau, 1897.

     \parindent=20pt
     \hangindent=20pt
     Following a set of harmonised scales, the author provides example accompaniments in different modes and annotates certain cadences with \emph{solfège} scale steps, indicating modulations, and numerals, referring to a set of rules provided earlier in the textbook.\\

    \parindent=0pt
    \hangindent=0pt
  Léon Courtois. \emph{Méthode pratique d'accompagnement du plain-chant précédée d'un cours élémentaire d'harmonie}. Namur:  Wesmaël-Charlier, \emph{c}.1897.

     \parindent=20pt
     \hangindent=20pt
     Considering the author was a past pupil of the Lemmens Institute, it is unsurprising to note his use of the filled-and-void notational style. Similarities are evident in the examples to the harmonic approach adopted by pedagogues at that school, and it follows that certain example accompaniments are reproduced from accompaniment books of Van Damme and Oscar De Puydt. Certain accompaniments by Antonin Lhoumeau are also included (p.~99, n.~1 \& p.~141).\\

    \parindent=0pt
    \hangindent=0pt
  \covid{}Abbé Hardy. \emph{Petite méthode d'accompagnement du plain-chant par l'harmonie consonante}. Ardennes, 1898. \\

    \parindent=0pt
    \hangindent=0pt
  Abbé Bourguignon. \emph{Méthode élémentaire d'harmonie pour l'accompagnement du plain-chant et des cantiques}. \underline{2nd ed.} Paris:  H. Oudin, 1899;  3rd ed. Paris:  H. Oudin, 1907.

     \parindent=20pt
     \hangindent=20pt
     Provides a set of harmonised major and minor scales prior to discussing the accompaniment of chant itself. Notes common to successive chords are tied and cadential sharping is prevalent. An appendix describes accompanying `with melodic notes' (`avec notes mélodiques'), or, in other words, a procedure by which certain notes of the chant may be justified as dissonances.\\

    \parindent=0pt
    \hangindent=0pt
  Dobroslav Orel. \emph{Theoreticko-praktická rukověť chorálu římského pro bohoslovecké a učitelské ústavy pro kněží, ředitele kůru, varhaníky a přátele církevního zpěvu}. Hradci Králové:  Politické družstvo tiskové, 1899.

     \parindent=20pt
     \hangindent=20pt
     The first textbook detailing a method of chant accompaniment intended for a Bohemian audience. Examples by the Czech composer Františk Jirásek are included. See \cpageref{hl:orel} above.\\

    \parindent=0pt
    \hangindent=0pt
  \covid{}J. B. Piot. \emph{L'accompagnement du plain-chant : méthode élémentaire, raisonnée et pratique}. 1st vol. Lyon:  Vitte, 1900;  2nd vol. Lyon:  Vitte, 1902.

     \parindent=20pt
     \hangindent=20pt
     Written by the chaplain at the basilica of Fourvière and former \emph{maître de chapelle} at the minor seminary of Verrières, Loire.\\\pagebreak{}

    \parindent=0pt
    \hangindent=0pt
  Alfred Delaporte. \emph{Manuel théorique et pratique indispensable pour apprendre seul l'harmonie, la transposition, le contrepoint, la fugue, l'orchestration et le plain-chant grégorien avec les différentes manières de l'accompagner}. Paris:  Louis Gregh, 1901.

     \parindent=20pt
     \hangindent=20pt
     Provides harmonised scales and sets of modulations described as suitable for eight modes. Although these are harmonised in the chord-against-note style, the author acknowledges that such a style is unsuited to accompanying those chant melodies published by Joseph Pothier (p.~143). In an attempt to provide an appropriate style to accompany those chants, the author demonstrates a neumatic accompaniment whereby chords are made to coincide with the first notes of neumes---chords are also struck on notes preceding the first notes of neumes too. The author acknowledges the requirement to understand the tenets understanding Solesmian notation prior to devising an accompaniment based on them, so recommends Antoine Delpech's accompaniments in the \ldo{} as further reading (p.~147, n.~1).\\

    \parindent=0pt
    \hangindent=0pt
  \covid{}Edouard Dubourg. \emph{Méthode théorique et pratique d'accompagnement du plain-chant}. Evreux, 1901. \\

    \parindent=0pt
    \hangindent=0pt
  \hlabel{hl_link:dauphin}
    J. Dauphin. \emph{Traité pratique et raisonné d'harmonie}. Arras:  Procure de musique, 1901.

     \parindent=20pt
     \hangindent=20pt
     Provides rules for the application of consonant harmony, and in a section devoted to `accompagnement avec notes mélodiques' provides advice concerning the use of dissonance. In the latter, Joseph Pothier's \emph{Liber gradualis} and a dubious description of the \emph{ictus} inspire a method of accompaniment that establishes so-called principal notes requiring changes of chord. In syllabic chants, such principal notes coincide with the accented syllable, the initial syllable of a word, and the final note of a phrase. In neumatic chants, such principal notes coincide with the first note of a neume, the junction between two subdivided neumes, double or triple notes (such as \emph{bistropha} or \emph{tristropha}, cadential notes, anacruses, and isolated notes that correspond with an accented syllable. These rubrics are explicated by means of example accompaniments which the author has annotated with a confusing array of numerals and symbols, attempting to demonstrate when a given note has been set as the root, third or fifth of a chord. It is rather a confusing system, not least because the author relates those numerals to the root of the chord, whether it is sounding in the bass part or not; so, the chant note `A' might be annotated with the numeral 5 to show it is the fifth of, say, a D minor 5/3 chord, even though the bass part might in fact be an `F' (p.~100).\\

    \parindent=0pt
    \hangindent=0pt
  \covid{}Amintore Galli \& J. Tomadini. \emph{Del canto liturgico cristiano : sinopsi : con esempi e studi sull'accomp.\ dello stesso canto}. Milan:  Ricordi, 1902. \\

    \parindent=0pt
    \hangindent=0pt
  Robert Collette. \emph{L'harmonium diatonique : Nouvel instrument donnant au Plain-Chant l'accompagnement consonnant que réclame sa nature}. Liège:  École professionnelle Saint-Jean-Berchma, \emph{c}.1902.

     \parindent=20pt
     \hangindent=20pt
     Proposes a new keyboard layout consisting of the `white-note' pitches `C', `D', `E', `F', `G', `A', and `B', the `black-note' pitch `B'\kern 1pt\flat{}, and the red-note pitch `D' flatted by a syntonic comma (pp.~17--18). The last was intended for use in chords including B\kern 1pt\flat{}.\\

    \parindent=0pt
    \hangindent=0pt
  \covid{}Hte Garonne Curé de Pin. \emph{Manuel pratique de l'accompagnateur du plain-chant}. 1902. \\

    \parindent=0pt
    \hangindent=0pt
  Peter Piel. \emph{Harmonie-Lehre : Unter besondere Berücksichtigung der Anforderungen für das kirchliche Orgelspiel zunächst für Lehrer-Seminare}. 8th ed. Dusseldorf:  L. Schwann, \emph{c}.1903.

     \parindent=20pt
     \hangindent=20pt
     The author's death in 1904 did not dissuade subsequent editors from assuming the mantle to update this harmony treatise, and to bring the ideas on chant accompaniment more up to date. The updates to the chapter on chant accompaniment were presumably by Paul Mandersheid, even though Piel's name is featured solely on the cover page. When Mandersheid's edition was translated into Italian by Eduardo Dagnani, the material was adapted for its intended audience to include example accompaniments by Giulio Bas and Peter Wagner. These, together with an up-to-date bibliography of Italian books, were no doubt intended to benefit Italian students, seminarians or amateurs. See \cpageref{hl:piel_harmonielehre} above.  \\

    \parindent=0pt
    \hangindent=0pt
  Luigi Bottazzo \& Oreste Ravanello. \emph{L'organista di Chiesa}. Milan:  Casa editrice Musica Sacra, 1903.

     \parindent=20pt
     \hangindent=20pt
     \S{}3 provides harmonised modal scales followed by a set of pre-harmonised intervals, ostensibly to suit the accomaniment of any chant melody without recourse to any theoretical rules.\\\pagebreak{}

    \parindent=0pt
    \hangindent=0pt
  Emile Brune. \emph{Nouvelle méthode élémentaire de l'accompagnement du plain-chant grégorien}. 3rd ed. Rixheim:  F. Sutter, 1903;  4th ed. Paris:  Bonne Presse, 1929;  5th ed. Paris, 1932.

     \parindent=20pt
     \hangindent=20pt
     See \cpageref{hl:brune} above.\\

    \parindent=0pt
    \hangindent=0pt
  \covid{}Roberto Remondi. \emph{Regole pratiche, chiare e facili per imparare ad accompagnare il canto gregoriano a prima vista, seguito dall'esposizione di un metodo semplicissimo per trasportare con facilità le melodie gregoriane a seconda delle necessità vocali del coro.\ (Testo italiano e francese)}. Turin:  Marcello Capra, 1903.

     \parindent=20pt
     \hangindent=20pt
     Reviewed by Giulio Bas in \emph{Rassegna gregoriana} vol.~3, cols~154--6 who noted that Remondi did not tailor his method to any one chant edition.\\

    \parindent=0pt
    \hangindent=0pt
  Pierre Chassang. \emph{Manuel de l'Accompagnateur du Chant grégorien et des cantiques popularies}. Arras:  Procure de musique, 1904.

     \parindent=20pt
     \hangindent=20pt
     See \cpageref{hl:chassang} above.\\

    \parindent=0pt
    \hangindent=0pt
  Stanbrook Abbey. \emph{A Grammar of Plainsong in two Parts}. London:  Burns and Oates, 1905.

     \parindent=20pt
     \hangindent=20pt
     Although little more than a page is dedicated to chapter eight on the subject of `Accompaniment', it is among the first discussions of the practice in the English language to consider tonality, rhythm and style as they relate to the accompaniment of plainchant in the Catholic Church. The chapter enjoyed the tacit approbation of Heinrich Bewerunge who corresponded with Stanbrook's nuns prior to publication. See \cpageref{hl:stanbrook} above.\\

    \parindent=0pt
    \hangindent=0pt
  \covid{}Amédée Gastoué. \emph{Comment on peut s'inspirer des anciens pour l'accompagnement du chant romain}. 1905. \\

    \parindent=0pt
    \hangindent=0pt
  Franz Xaver Mathias. \emph{Die Choralbegleitung}. Regensburg:  Friedrich Pustet, 1905.

     \parindent=20pt
     \hangindent=20pt
     See above in \cref{hl:mathias}.\\

    \parindent=0pt
    \hangindent=0pt
  \covid{}William Gousseau. \emph{Essai d'accompagnement du chant grégorien}. Paris:  Alleton, \emph{c}.1905. \\

    \parindent=0pt
    \hangindent=0pt
  Charles Künster. \emph{Harmonisches System zur Begleitung der gregorianischen Choralmelodien}. St. Ottilien:  Missionsverlag, 1906.

     \parindent=20pt
     \hangindent=20pt
     The first half of this text attempts to describe the origin of harmony and how melodies can suggest one or another harmony. The second attempts to apply those findings to the accompaniment of chant, a chapter being dedicated to each of the modes: protus, deuterus, tritus and tetrardus modes. The author then discusses chord progressions cadences, concluding by delineating four criteria for appropriate chant harmony. It must apply to every piece in the repertory; it must follow established musical principles---presumably with respect to part movement, and so forth---and musical aesthetics (`den allgemeinen musikalischen Gesetzen und dem natürlichen musikalischen Gefühle'); it must conform to those rules provided by the author in the second half of the text; and it must be capable of both simple and more elaborate textures. Music examples are provided to exemplify the author's system of accompaniment, the chant in quadratic notation arranged above a four-part accompaniment where the transcribed chant is placed in the top part.\\

    \parindent=0pt
    \hangindent=0pt
  Dominicus Johner. \emph{Neue Schule des gregorianischen Choralgesanges}. 1st ed. Regensburg, New York and Cincinnati:  Pustet, 1906;  5th ed. Regensburg, New York and Cincinnati:  Pustet, 1921;  \underline{3rd ed.} Regensburg:  Friedrich Pustet, 1925. Translated by Hermann Erpf \& Max Ferrars;  6th ed. Regensburg:  Friedrich Pustet, 1929.

     \parindent=20pt
     \hangindent=20pt
     Proposes a diatonic method of accompaniment permitting the seventh chord without requiring it to be prepared or resolved. See \cpageref{hl:johner} above.\\

    \parindent=0pt
    \hangindent=0pt
  Max Springer. \emph{Die kunst der Choralbegleitung : Theoretisch-praktische Anleitung zum richtigen Singen und Begleiten des gregorianischen Chorals}. Regensburg:  Coppenrath, 1907;  New York:  Fischer, 1908.

     \parindent=20pt
     \hangindent=20pt
     See above in \cref{hl:springer}.\\

    \parindent=0pt
    \hangindent=0pt
  \covid{}Max Springer. \emph{Die liturgische Choralgesang in Hochamt und Vesper, dessen harmonisierung und Erklärung}. Regensburg, 1907. \\\pagebreak{}

    \parindent=0pt
    \hangindent=0pt
  Alfred Madeley Richardson. \emph{Modern organ accompaniment}. London:  Longmans Green, 1907.

     \parindent=20pt
     \hangindent=20pt
     Advocates for accompaniments in the chord-against-note style, the chantbeing placed in bottom, middle and top parts of the texture. Modulation at cadences is described rather briefly, as is the `beautiful effect' reportedly produced when only part of a chant is harmonised (pp.~179--82). The author reiterates Walter Howard Frere's view that the pedal division is to be used sparingly in the course of a chant accompaniment (p.~196).\\

    \parindent=0pt
    \hangindent=0pt
  \covid{}William Gousseau. \emph{Résumé du cours d'accompagnement du plain-chant}. Paris:  Alleton, 1907. \\

    \parindent=0pt
    \hangindent=0pt
  \covid{}Louis Raffy. \emph{Ecole d'orgue : l'accompagnement du plain-chant}. Saint-Leu-la-Forêt:  Procure de musique, 1908. \\

    \parindent=0pt
    \hangindent=0pt
  \covid{}Oreste Ravanello. \emph{Sul ritmo e sull'accompagnamento del canto gregoriano, studi ed osservazioni}. Padua:  Salmin, 1908. \\

    \parindent=0pt
    \hangindent=0pt
  F. Clement C. Egerton. \emph{A handbook of church music : a practical guide for all those having the charge of schools and choirs, and others who desire to restore plainsong to its proper place in the services of the church}. London:  R \& T Washbourne Ltd, 1909.

     \parindent=20pt
     \hangindent=20pt
     Chapter IX recommends a `simple accompaniment of the chant' but offers few tidbits of advice as to how that might be constructed, save for making the less-than-helpful assertion that `a good accompanist of plainsong is born rather than made'.\\

    \parindent=0pt
    \hangindent=0pt
  John Stainer. \emph{The Organ}. New York:  G. Schirmer, 1909. Edited by Harker F. Flaxington.

     \parindent=20pt
     \hangindent=20pt
     Recommends a `solid organ combination (of stops, most likely, though perhaps of texture too) for chant was ordinarily to be sung in unison. A knowledge of `the ancient Ecclesiastical modes' is said to be essential (p.~82), and it is expected that the organist should deploy word-painting in their accompanists as a matter of course.\\

    \parindent=0pt
    \hangindent=0pt
  Fr[ère] Sébastien. \emph{Accompagnement du chant grégorien}. Paris:  Lethielleux, 1910.

     \parindent=20pt
     \hangindent=20pt
     See \pageref{hl:frere_sebastien} above.\\

    \parindent=0pt
    \hangindent=0pt
  \covid{}Charles Danjou. \emph{Organiste en un mois : Cent vingt morceaux liturgiques et cantiques, précédés de formules précises et d'indications pour apprendre à accompagner}. 1st ed. Paris:  De Gigord, 1910;  2nd ed. Paris:  De Gigord, 1915. \\

    \parindent=0pt
    \hangindent=0pt
  Amédée Gastoué. \emph{Traité d'harmonisation du chant grégorien sur un plan nouveau}. Lyon:  Janin, 1910.

     \parindent=20pt
     \hangindent=20pt
     In two parts, this manual discusses consonances, so-called `notes mélodiques' (passing notes, échappées, anticipations, auxiliaries, appoggiaturas and pedals), and also recommends the use of consecutive fifths and octaves. It then delves into modality and counterpoint, making numerous citations from compositions by Charles Bordes, Émile Brune, Pierre Chassang, Antoine Delpech, William Gousseau, Alexandre Guilmant, Antonin Lhoumeau, Jean Parisot, and Peter Wagner, to say nothing of those examples specially composed by author himself.\\

    \parindent=0pt
    \hangindent=0pt
  \covid{}Abbé Thiverny. \emph{Accompagnement du plain-chant, du chant grégorien et des cantiques}. 1911. \\

    \parindent=0pt
    \hangindent=0pt
  \covid{}Paul Manderscheid. \emph{Der traditionelle Choral : sein Vortrag und seine Begleitung}. Dusseldorf:  L. Schwann, 1911. \\

    \parindent=0pt
    \hangindent=0pt
  Edwin Evans. \emph{The Modal Accompaniment of Plain Chant: A Practical Treatise}. London:  William Reeves, 1911.

     \parindent=20pt
     \hangindent=20pt
     A textbook in two halves, the first, theoretical, considers questions such as the appropriate modality of an accompaniment, whether to sustain chords, and whether to use sharps; the second, practical, largely consists of various harmonisations which place the chant in the tenor register, the same as that of accompaniments in two parts. But that is not applied everywhere as a rule, because the chant is made to flit sometimes between outer and inner parts of the texture; a list of endnotes makes an attempt at an exegesis of the method.\\\pagebreak{}

    \parindent=0pt
    \hangindent=0pt
  Henry William Richards. \emph{The Organ Accompaniment of the Church Services: A Practical Guide for the Student}. London:  J. Williams, 1911.

     \parindent=20pt
     \hangindent=20pt
     The author writes in chapter 13 that `in the main, the harmonies used [to accompany plainsong] should be composed on the models of Tallis or Palestrina'. The practice of lining-out the intonation in octaves on the organ is described. One example accompaniment is noteworthy for setting a psalm tone in different places in the texture (pp.~110--112). The beamed notation discussed above (p.~\pageref{ln:novello_notation}) is used for the notation of the example.\\

    \parindent=0pt
    \hangindent=0pt
  Peter Griesbacher. \emph{Kirchenmusikalische Stilistik und Formenlehre}. Regensburg, 1912.

     \parindent=20pt
     \hangindent=20pt
     The author's views on accompaniment may be consulted above in \cref{hl:griesbacher}.\\

    \parindent=0pt
    \hangindent=0pt
  \covid{}Abbé Duthu. \emph{L'Ave Maria de l'harmonie ou l'art d'arriver à l'accompagnement du chant religieux}. Paris:  Pinatel, 1912. \\

    \parindent=0pt
    \hangindent=0pt
  Henri Potiron. \emph{Méthode d'harmonie appliquée à l'accompagnement du chant grégorien (d'après l'édition vaticane)}. Paris:  Hérelle, 1912.

     \parindent=20pt
     \hangindent=20pt
     This stirred some controversy for railing against the theory of free rhythm promulgated by Solesmes, against which the author proposed a mensural theory of his own. Following the author's acceptance of Solesmian theories some ten years hence, he moved to suppress this publication by superseding it with others that codified the application of Solesmian rhythm in the accompaniment. For a description of the controversy and Potiron's eventual \emph{volte-face}, see above in \cref{hl:potiron_methode}.\\

    \parindent=0pt
    \hangindent=0pt
  François Brun. \emph{Traité de l'accompagnement du chant grégorien}. 2nd ed. Paris:  Schola Cantorum, 1912.

     \parindent=20pt
     \hangindent=20pt
     Attempts to summarise accompaniment of chant with three categories: those with the tune on top, those with sustained chords, and those in a so-called `accompagnement concertant' style, a kind of more elaborate accompaniment that might have been inspired by solo performance. To illustrate his method of chant accompaniment, the author reproduces some music examples which had previously been published by the Schola Cantorum around 1898, together with examples composed more recently by other French composers. See \cref{sc:brun}.\\

    \parindent=0pt
    \hangindent=0pt
  Ferdinand Gregor Molitor. \emph{Die diatonisch-rythmische Harmonisation der gregorianischen Choralmelodien}. Leipzig:  Breitkopf und Härtel, 1913.

     \parindent=20pt
     \hangindent=20pt
     See \cpageref{hl:molitor} above.\\

    \parindent=0pt
    \hangindent=0pt
  \covid{}Fr. Achille. \emph{L'enfant de chœur organiste en huit jours}. Paris:  Mignard, 1913. \\

    \parindent=0pt
    \hangindent=0pt
  \covid{}Curé de Courgis. \emph{Méthode Jeanne d'Arc pour le chant et l'harmonium}. 1913. \\

    \parindent=0pt
    \hangindent=0pt
  Maurice Emmanuel. \emph{Traité de l'accompagnement modal des psaumes}. Lyon:  Janin, 1913.

     \parindent=20pt
     \hangindent=20pt
     See \cpageref{hl:emmanuel} above.\\

    \parindent=0pt
    \hangindent=0pt
  \covid{}Théodore Dubois. \emph{L'accompagnement du plain-chant mis à la portée de tous}. Paris:  Au Ménestrel, 1914. \\

    \parindent=0pt
    \hangindent=0pt
  Jean Parisot. \emph{L'accompagnement modal du chant grégorien}. Paris:  Art catholique, 1914.

     \parindent=20pt
     \hangindent=20pt
     The first part describes such devices as passing notes, appoggiaturas, anticipations, anacruses, pedal points, sustained chords, the so-called `accompagnement concertant', and so forth; the second part provides music examples of perhaps a bar or two to demonstrate their use in practice.\\\pagebreak{}

    \parindent=0pt
    \hangindent=0pt
  Francis Burgess. \emph{The Teaching and Accompaniment of Plainsong}. London:  Novello \& Co., 1914.

     \parindent=20pt
     \hangindent=20pt
     The author recommends `modal accompaniment by limiting the materials of our harmonies to the notes of the diatonic scale with the flat seventh as an additional note' (p.~62), a practice absorbed seemingly unwittingly from Niedermeyer's practice. While the author is certainly cognizant of developments on the continent (he reproduces accompaniments by Peter Wagner, Franz Xaver Mathias, Michael Horn, Max Springer, Franz Nekes and Leo Manzetti), his blasé explanations of their praxis betrays a certain lack of familiarity with the \emph{modus operandi} of these composers. Speaking of the filled-and-void notation used by Horn, for instance, the author concludes that `apparently he dislikes the look of the quaver'; when in fact Horn had employed a different notation to make the distinction between sacred chant and secular harmony clear to the player. Horn's view on that subject is discussed above (\cpageref{hl:horn_notation}).\\

    \parindent=0pt
    \hangindent=0pt
  Jules Carillion. \emph{L'Accompagnement du chant grégorien en cinq leçons}. Paris:  Bonne Presse, 1916.

     \parindent=20pt
     \hangindent=20pt
     A pocket-sized manual containing `five lessons' comprising discussions on perfect chords, successions of chords, choosing chords, where to place them, and transposition. Each is followed by sets of exercises and solutions.\\

    \parindent=0pt
    \hangindent=0pt
  \covid{}Karl Cohen. \emph{Orgelbegleitung nebst Vor- u[nd] Nachspielen zu den Einheitsliedern der deutschen Diözesan-Gesangbücher}. Cologne:  Bachem, 1916. \\

    \parindent=0pt
    \hangindent=0pt
  Ernest Grosjean. \emph{Méthode pour l'accompagnement du chant grégorien}. Paris:  Biton, 1917.

     \parindent=20pt
     \hangindent=20pt
     Blitzes through the rules of chord construction prior to discussing how to set `des notes étrangères aux accords' as passing notes, auxiliaries, appoggiaturas, échappées, suspensions, and anticipations. Rules are then provided to set chords to neumes: disjunct neumes outline the chord required; but conjunct neumes require some notes to be consonant and others dissonant. Some fifteen pages of music examples follow, from which the reader is required to absorb the author's principles by osmosis, because little further descriptive matter is provided to explicate their \emph{modus operandi}.\\

    \parindent=0pt
    \hangindent=0pt
  Moritz Brosig. \emph{Handbuch der Harmonielehre und Modulation}. Leipzig:  F. E. C. Leuckart, 1918.

     \parindent=20pt
     \hangindent=20pt
     The author terminates most of his accompaniments on major chords, admitting sharps in the terminal chord when necessary. Cadential sharping is also employed, as are seventh chords, and the tenor part is often made to match the chant in similar or contrary motion (pp.~211--233).\\

    \parindent=0pt
    \hangindent=0pt
  Giulio Bas. \emph{Metodo per l'accompagnamento del canto gregoriano e per la composizione negli otto modi, con un' appendice sulla risposta nella fuga}. Turin:  Società Tipografico -- Editrice Nazionale, 1920;  Paris:  Desclée, 1923.

     \parindent=20pt
     \hangindent=20pt
     Collects the author's thoughts on accompaniment following two decades as unofficial composer of accompaniments for Solesmes. Of particular interest is the author's proclivity for harmonising the antecedent of phrases differently (or not at all) compared with the consequent. See above in \cref{hl:bas_apodose}.\\

    \parindent=0pt
    \hangindent=0pt
  \covid{}Abbé Coudray. \emph{Méthode préparatoire à l'accompagnement du plain-chant grégorien}. Saint-Brieuc:  Gaudu, 1921. \\

    \parindent=0pt
    \hangindent=0pt
  \covid{}Abbé Desmaris. \emph{Méthode théorique et pratique pour l'accompagnement du plain-chant}. Autun, 1921. \\

    \parindent=0pt
    \hangindent=0pt
  \covid{}Octave Rossion. \emph{Ecole d'accompagnement du chant grégorien}. Brussels:  Ledent, 1923. \\

    \parindent=0pt
    \hangindent=0pt
  \covid{}Abbé Fazembat. \emph{Le plain-chant et son accompagnement : Méthode à la portée des enfants eux-mêmes}. Bourdeaux, \emph{c}.1924. \\

    \parindent=0pt
    \hangindent=0pt
  Henri Tissot. \emph{Mélopées liturgiques et mélodies modernes: Comment on peut traiter leur harmonisation}. Besançon:  Imprimerie Bossanne, 1924.

     \parindent=20pt
     \hangindent=20pt
     Annotates music examples in a confusing manner, those numerals to the left of notes indicating fingerings, and thos beneath the staff indicating bass notes in simple intervals beneath the chant note---horizontal lines indicate that bass notes should remain static. A process of imitation is described whereby one accompanying part is to imitate the intervals previously traced out by the chant (p.~24). Examples by Jean Parisot and Maurice Emmanuel are provided, but no mention is made of the earlier style of `Imitationen im Baß', described above (see \cpageref{hl:imitationenimbass}).\\

    \parindent=0pt
    \hangindent=0pt
  F. Boulfard. \emph{Méthode d'accompagment du chant grégorien}. Paris:  Desclée, 1924.

     \parindent=20pt
     \hangindent=20pt
     Attempts to codify the process of accompaniment according to a theory of chant rhythm promulgated by Solesmes, which is not altogether surprising when considering that the author was a Benedictine oblate. That probably explains how the text came to be published by Desclée and also might explain why the Spanish Benedictine monk Maur Sablayrolles provided some introductory remarks. Predictably, numerous references are made in the body of the text to André Mocquereau's theory of the \emph{ictus} and to Giulio Bas's thoughts on modal harmony.\\

    \parindent=0pt
    \hangindent=0pt
  George Oldroyd \& Charles William Pearce. \emph{The Accompaniment of Plainchant: A Practical Guide for Students}. London:  J. Curwen \& Sons Ltd, 1924.

     \parindent=20pt
     \hangindent=20pt
     The authors permit 5/3 chords (except the diminished triad B/D/F), 6/3 chords and 6/4 chords as long as the harmony remains diatonic, and describes the coincidence whereby the chant, in a resonant acoustic, creates seemingly creates clusters. The authors are cognizant of continental textbooks, including those by Amédée Gastoué, and also allot space to Richard Runciman Terry's ideas concerning accompaniment, to whom the text is also dedicated.\\

    \parindent=0pt
    \hangindent=0pt
  \covid{}Henri Potiron. \emph{\emph{`L'accompagnement du chant grégorien: des rapports entre l'accent et la place des accords.'} Monographies grégoriennes. \emph{Vol.~5}}. Paris-Tournai:  Desclée, 1924. \\

    \parindent=0pt
    \hangindent=0pt
  Henri Potiron. \emph{Cours d'accompagnement du chant grégorien}. 1st ed. Paris:  Hérelle, 1925;  2nd ed. Paris:  Hérelle, 1927;  2nd ed. Tournai:  Desclée, 1933. Translated by Ruth C. Gabain.

     \parindent=20pt
     \hangindent=20pt
     The most notable difference between the first and second editions is the addition of more music examples. See \cref{hl:potiron_threetonalities,hl:potiron_cours}.\\

    \parindent=0pt
    \hangindent=0pt
  \covid{}Abbé P. Méroux. \emph{Nouvelle méthode pratique, simple et complète pour apprendre rapidement à accompagner le plain-chant grégorien suivant les lois de l'harmonie de la tonalité et du rythme}. Paris-Tournai:  Desclée, 1925. \\

    \parindent=0pt
    \hangindent=0pt
  Abbé Aumon \& Abbé Biret. \emph{Méthode facile et complète pour l'accompagnement du chant grégorien et des cantiques}. Vendée:  Petit séminaire de Chavagnes-en-Paillers, 1926.

     \parindent=20pt
     \hangindent=20pt
     See \cpageref{hl:aumon_biret} above.\\

    \parindent=0pt
    \hangindent=0pt
  Jules Jeannin. \emph{Sur l'importance de la tierce dans l'accompagnement grégorien}. Paris:  Hérelle, 1926.

     \parindent=20pt
     \hangindent=20pt
     Reacts against the fashion omitting the third from accompaniments and the use of dyads, presenting some well chosen music examples and arguments by François-Auguste Gevaert, Maurice Emmanuel and Amédée Gastoué in an atempt to justify the retention of triads. \\

    \parindent=0pt
    \hangindent=0pt
  Jean Hébert Desrocquettes \& Henri Potiron. \emph{\emph{`La théorie harmonique des trois groupes modaux et l'accord final des troisième et quatrième mode[s].} Monographies grégoriennes. \emph{Vol.~6}}. Paris-Tournai:  Desclée, 1926.

     \parindent=20pt
     \hangindent=20pt
     The three groups in question summarise what is rather a convoluted theory of chant harmony that is more fully explicated above (\cref{hl:three_groups}). While the cover page gives the title with the singular form of `mode', the first page gives the plural.\\

    \parindent=0pt
    \hangindent=0pt
  \covid{}Frere du Sacré-Coeur d'Arthabaska. \emph{Méthode facile et rapide pour accompagner le chant grégorien d'après les principes de Solesmes, suivie d'un appendice pour l'accompagnement des cantiques}. Tournai:  Desclée, 1927. \\\pagebreak{}

    \parindent=0pt
    \hangindent=0pt
  John Henry Arnold. \emph{Plainsong Accompaniment}. Oxford:  Oxford University Press, 1927.

     \parindent=20pt
     \hangindent=20pt
     \hlabel{cc:arnold_entry}%
     A sustained type of accompaniment reserving chord changes for specific notes owes much to developments in chant accompaniment on the continent which the author has apparently absorbed unwittingly. His use of inessential notes to reduce the number of chords and diatonic harmony owe much to French, Belgian and German theorists of the preceding four decades or so. But the author was not without renown himself, having written plainsong accompaniments for the hymnal \emph{Songs of Praise}, and he would later go on to revise those accompaniments in the \emph{English Hymnal} in 1933. His experimentation with accompanying the voices in their respective registers is a noteworthy departure from the textures of Alfred Madeley Richardson and Henry William Richards (p.~62). For further information, see \cite{HarperEnglishHymnalLiturgical2005}.\\

    \parindent=0pt
    \hangindent=0pt
  \covid{}Y. Chuberre. \emph{Petite voie agréable et facile de l'accompagnement du plain-chant}. 1928. \\

    \parindent=0pt
    \hangindent=0pt
  Jean Hébert Desrocquettes. \emph{\emph{`L'accompagnement rythmique d'après les principes de Solesmes.} Monographies grégoriennes. \emph{Vol.~8}}. Tournai:  Société de Saint Jean l'Évangéliste, 1928.

     \parindent=20pt
     \hangindent=20pt
     Attempts to apply André Mocquereau's theory of free rhythm to the choice of harmony and placement of chords. The author's dubious authority on harmonic matters lead to some erratic judgements which are to be considered with no small amount of caution. \\

    \parindent=0pt
    \hangindent=0pt
  \covid{}Henri Potiron. \emph{\emph{`La modalité grégorienne.} Monographies grégoriennes. \emph{Vol.~9}}. Paris-Tournai:  Desclée, 1928. \\

    \parindent=0pt
    \hangindent=0pt
  \covid{}B. Gatterdam. \emph{Kleine Schule der Choralbegleitung}. Regensburg, 1929. \\

    \parindent=0pt
    \hangindent=0pt
  \covid{}Edmond Chabot. \emph{Méthode d'accompagnement du chant grégorien d'après les pricipes rythmiques de l'Ecole de Solesmes}. Marseilles:  Publiroc, 1929. \\

    \parindent=0pt
    \hangindent=0pt
  Jean Hébert Desrocquettes \& Henri Potiron. \emph{Vingt-neuf pièces grégoriennes harmonisées, avec commentaires rythmiques, modaux et harmoniques}. Paris:  Hérelle, 1929.

     \parindent=20pt
     \hangindent=20pt
     Discussed above in \cref{hl:potiron_cours}.\\

    \parindent=0pt
    \hangindent=0pt
  \covid{}Hermann Halbig. \emph{Kleine gregorianische Formenleere}. Kassel:  Bärenreiter, 1930. \\

    \parindent=0pt
    \hangindent=0pt
  Leo Söhner. \emph{Die Geschichte der Begleitung des gregorianischen Chorals in Deutschland vornehmlich im 18.\ Jahrhundert}. Augsburg:  Filser, 1931.

     \parindent=20pt
     \hangindent=20pt
     This doctoral dissertation, supervised by Peter Wagner, traces the history of the organ accompaniment of chant from the sixteenth century to the eighteenth.\\

    \parindent=0pt
    \hangindent=0pt
  \covid{}Désiré Pirio. \emph{Harmonium et chant grégorien ou jeu de la simple note grégorienne, doigtée, transposée et rythmée}. Paris:  Desclée, 1931. \\

    \parindent=0pt
    \hangindent=0pt
  \covid{}Henri Potiron. \emph{Manuel pratique d'accompagnement des cantiques modernes et du chant grégorien selon les principes rythmiques et modaux de Solesmes}. Tournai:  Desclée, 1932. \\

    \parindent=0pt
    \hangindent=0pt
  \covid{}Albert M. Vogt. \emph{Orgelbegleitung zu den Gesängen im Gregorianischen Choral}. Abenheim:  Joh. Finger, 1932. \\

    \parindent=0pt
    \hangindent=0pt
  Achille Pierre Bragers. \emph{A Short Treatise on Gregorian Accompaniment According to the Principles of the Monks of Solesmes}. New York:  Fischer, 1934.

     \parindent=20pt
     \hangindent=20pt
     The first part provides the first exposition of the Desrocquettes--Potiron modal groups in the English language, and supplements that exposition with advice on chord placement and style. The second discusses eight modes in turn and provides harmonic formul\ae{} for cadences---the accompaniment of psalms is dealt with at the end. See above in \cref{hl:bragers}.\\\pagebreak{}

    \parindent=0pt
    \hangindent=0pt
  Leo Söhner. \emph{Kurze Anleitung zur Begleitung des gregorianischen Chorals}. Altötting:  Coppenrath, \emph{c}.1935.

     \parindent=20pt
     \hangindent=20pt
     Describes a method of rhythmical accompaniment that reserves either chord changes for notes of rhythmical importance demarcated by vertical \emph{episemata}. Several example accompaniments illustrate the author's descriptions, their harmony remaining diatonic. The very same content appeared in an edition of Schildknecht's \emph{Orgelschule} which the author was partly responsible for editing. See \cpageref{hl:schildknecht_orgelschule} above.\\

    \parindent=0pt
    \hangindent=0pt
  Leo Söhner. \emph{Die Orgelbegleitung zum gregorianischen Gesang}. Regensburg:  Friedrich Pustet, 1936.

     \parindent=20pt
     \hangindent=20pt
     Most likely a \emph{Habilitationsschrift} that picks up threads left down by the author in his doctoral dissertation of 1931. The text summarises the chief developments from the nineteenth century to the 1920s, but in so taciturn a fashion as to leave out much of the detail.\\

    \parindent=0pt
    \hangindent=0pt
  \covid{}Charles Tournemire. \emph{Précis d'exécution, de registration et d'improvisation à l'orgue}. Paris:  Éditions Max Eschig, 1936.

     \parindent=20pt
     \hangindent=20pt
     Recollects Franck's organ class at the Paris Conservatoire, wherein the `choral' idiom served as the basis of chant accompaniment, several music examples being provided. See \cpageref{hl:tournemire_franck} above.\\

    \parindent=0pt
    \hangindent=0pt
  Maurice Kaltnecker. \emph{L'A B C du jeune accompagnateur}. Nancy:  Société anonyme d'éditions, 1937.

     \parindent=20pt
     \hangindent=20pt
     Rather a brief pamphlet divided up into sections comprising perfect chords, harmonisation of scales placed in the top part, and thirty formul\ae{} which the author believed were common in the chant repertory. The pamphlet contains barely any descriptive prose, the included discursive notes being relegated to fleeting asides and footnotes. In spite of that arguably most debilitating drawback, the manual's title would probably have made it seem appealing to the unlearned audience at which it was aimed (see \cpageref{hl:kaltnecker} above).\\

    \parindent=0pt
    \hangindent=0pt
  Marcel Dupré. \emph{Manuel d'accompagnement du plain-chant grégorien}. Paris:  Leduc, 1937;  Paris:  Leduc, 1975. Translated by Josef Zimmermann.

     \parindent=20pt
     \hangindent=20pt
     Proposes a novel approach to harmonising an ascending scale by composing what are effectively three countersubjects in oblique and contrary motions. The author then divides up the scale into what he terms seven tetrachords and provides further harmonised countersubjects to match. From these, accompaniments of each modal scale are derived, each one being set in the top part of a four-part texture. The student is advised to learn each by heart (p.~16). Little else in terms of descriptive prose is provided, save on the matter of transposition, and so the student is left to absorb the practice from the included example accompaniments by themselves. It should be noted that the author uses the term `phrygien' to describe the protus mode and `dorien' to describe the deuterus which, while being perfectly in keeping with Maurice Emmanuel's conception of Greek scales, have the potential to confuse modern readers.\\

    \parindent=0pt
    \hangindent=0pt
  \covid{}Maria Frieda Loebenstein \& Corbinian Gindele. \emph{Der gregorianische Choral in Wesen und Ausführung}. Berlin:  Das Innere Leben, 1938. \\

    \parindent=0pt
    \hangindent=0pt
  Henri Potiron. \emph{Leçons pratiques d'accompagnement du chant grégorien}. 1st ed. Tournai:  Desclée, 1938;  \underline{2nd ed.} Paris-Tournai:  Desclée, 1952.

     \parindent=20pt
     \hangindent=20pt
     The second edition was intended to complement the author's 1951 manual \emph{Petit traité de contrepoint}, and contains an insert with `Notes complémentaires' that include references to the author's recently published \emph{Kyriale abrégé} (for more on those accompaniments, see above on page \pageref{hl:potiron_kyrialeabrege}). The text is divided into two parts comprising the principles of accompaniment (harmony, rhythm and modality) and a practical examination of selections from the chant repertory by mode, followed by a brief note on accompanied psalmody. The practical discussions bear some resemblance to the \emph{Vingt-neuf pièces grégoriennes} of 1929, particularly where sections of a chant melody are parsed to describe the underlying rationale for accompanying a given passage one way or another.\\

    \parindent=0pt
    \hangindent=0pt
  \covid{}Miguel Altisent. \emph{El acompa\~{n}amiento del Canto Gregoriano}. Barcelona, 1943. \\\pagebreak{}

    \parindent=0pt
    \hangindent=0pt
  Flor Peeters. \emph{Méthode pratique pour l'accompagnement du Chant Grégorien}. Malines:  H. Dessain, 1943;  Malines:  H. Dessain, 1949.

     \parindent=20pt
     \hangindent=20pt
     Codifies the method of accompaniment practiced by staff members of the Lemmens Institute during the 1940s in their preparation of the \emph{Nova Organi Harmonia}. Using the distinctive filled-and-void notational style, the author describes using few changes of chords and harmony that sometimes commences \emph{in media res} (see the first bar on p.~48). Although the harmony is diatonic, the author employs Roman numerals to describe chords as they relate to the final of a mode, and recommends minor chords in preference to major chords since the former are said to be `in conformity with the modal and archaic character and general spirit of plainchant' (p.~22 \S{}15). Should the use of major chords be unavoidable, then they are to be arranged as first-inversion chords. Both copies consulted are identical but for the later containing a translation of the text into English.\\

    \parindent=0pt
    \hangindent=0pt
  \covid{}Miguel Bernal Jim\'{e}nez. \emph{El acompa\~{n}amiento del canto gregoriano}. Morelia:  Escuela superior de música sagrada, 1944. \\

    \parindent=0pt
    \hangindent=0pt
  Francis Potier. \emph{L'art de l'accompagnement du chant grégorien}. Tournai:  Desclée, 1946.

     \parindent=20pt
     \hangindent=20pt
     Discusses the history of accompaniment based primarily on Francophone source material. The primacy of Solesmes's ideas is not questioned when classifying accompaniments by whether they fit Solemes's rhythmical theories or not. The author may not be excused from making certain value judgements at the expense of some texts that reportedly contain `very defective' chant rhythms which do not conform to Solesmes's principles (see \cpageref{hl:potier} above). The annotated bibliography included as an appendix is nonetheless worth perusing.\\

    \parindent=0pt
    \hangindent=0pt
  \covid{}L. Hazard. \emph{Précis d'accompagnement du plain-chant grégorien}. Nancy:  Société anonyme d'éditions, 1947. \\

    \parindent=0pt
    \hangindent=0pt
  Gregory Murray. \emph{The Accompaniment of Plainsong}. Society of St Gregory, 1947.

     \parindent=20pt
     \hangindent=20pt
     Among the first Anglophone texts to discuss the Desrocquettes--Potiron modal groups: the fact that the author had recently translated a textbook by Potiron should not be overlooked. But in a most conspicuous \emph{volte-face}, the author later became a vocal detractor of Solesmes's rhythmical theories when propagating a mensural theory of his own.\\

    \parindent=0pt
    \hangindent=0pt
  \covid{}Marcel Renoux. \emph{Harmonie moderne et harmonie grégorienne : Traité complet d'harmonie, application à l'accompagnement du grégorien}. Besançon:  Impr. Jacques et Demontrond, 1948. \\

    \parindent=0pt
    \hangindent=0pt
  Eugène Lapierre. \emph{Gregorian Chant Accompaniment}. 1st ed. Ohio:  Gregorian Institute of America, 1949.

     \parindent=20pt
     \hangindent=20pt
     Discusses Greek scales, intervals, and chord construction, modes, inessential notes, appoggiaturas, `retardations', passing notes, and pedal-point, accompanied psalmody, among other techniques. On the placement of chords in a phrase, the author maintains a strict adherence to André Mocquereau's rhythmic theory, and reproduces some example accompaniments superimposed with chrinomic squiggles to explicate the rather bizarre chor changes in thee `Dies irae' (see above in \cref{hl:dies_irae_mocquereau}). Each chapter is followed by a set of questions, to encourage greater engagement with the material. \\

    \parindent=0pt
    \hangindent=0pt
  Henri Potiron. \emph{Practical Instruction in Plainsong Accompaniment}. Tournai:  Desclée, 1949. Translated by Gregory Murray.

     \parindent=20pt
     \hangindent=20pt
     Provides rules for the accompaniment of chant depending on whether the chant occupies one hexachord or whether it mutates to another. The author then divides up the repertory by mode to discuss common intervallic and cadential patterns and the harmony he believes best suits them.\\

    \parindent=0pt
    \hangindent=0pt
  Henri Potiron. \emph{Petit traité de contrepoint et exercices d'ecriture preparatoires à l'accompagnement du chant gregorien}. Tournai:  Desclée, 1951.

     \parindent=20pt
     \hangindent=20pt
     Chapter five discusses chant accompaniment in terms of contrapuntal composition, conjunct motion being said to be most preferable (p.~92). While only a single chapter deals with the process of accompaniment, a footnote directs the reader to another text, \emph{Leçons pratiques d'accompagnement du chant grégorien}, where it is promised that the application of theory to practice is discussed in more detail (p.~95 n.~1).\\\pagebreak{}

    \parindent=0pt
    \hangindent=0pt
  \covid{}Wojciech Ignacy Lewkowicz. \emph{Harmonia gregoriańska czyli nauka akompaniamentu do melodii gregoriańskich}. Poznań:  Ksiegarnia Św. Wojciecha, 1959. \\

    \parindent=0pt
    \hangindent=0pt
  \covid{}Rembert Weakland. \emph{Modal Accompaniment}. Latrobe, Pa.:  Archabbey Press, 1959. \\

    \parindent=0pt
    \hangindent=0pt
  Celestino Eccher. \emph{Accompagnamento gregoriano : armonia, ritmo, modalità, stile.\ Teoria ed esempi.\ Appendice~: Gregoriano e polifonia}. Rome:  Desclée, 1960.

     \parindent=20pt
     \hangindent=20pt
     Dedicates a chapter each to accompanying melismata, the desirable number of parts in an accompaniment, modal harmony, recitation, and so forth. André Mocquereau's procedure of grouping chant notes into groups of twos and threes is said to influence chord changes (p.~52), and cadential chords are determined by the melodic interval at a cadences, rather than strictly by the mode, the cadence `F' \rightarrow{} `E' implying deuterus harmony, and so on. The accompaniment is therefore said to `modulate' to different modes (pp.~76--7).\\

    \parindent=0pt
    \hangindent=0pt
  \covid{}M. Cecile. \emph{Chant Accompaniment Simplified}. Collegeville:  The Liturgical Press, 1960. \\

    \parindent=0pt
    \hangindent=0pt
  \covid{}Henri Potiron. \emph{L'Accompagnement du chant grégorien suivant les types modaux}. Paris:  Schola Cantorum, 1961. \\

    \parindent=0pt
    \hangindent=0pt
  \covid{}Mary Theoda Wieck. \emph{A Theoretical Basis for Accompanying Gregorian Chant According to Medieval Principles}. 1963. \\

    \parindent=0pt
    \hangindent=0pt
  Michael Fleming. \emph{The Accompaniment of Plainsong}. Croydon:  RSCM, 1963.

     \parindent=20pt
     \hangindent=20pt
     Conveys brief comments on harmony, rhythm, texture and style, and the author's opinion that an accompaniment need not be constructed of too many chords (pp.~9, 11).\\

    \parindent=0pt
    \hangindent=0pt
  Heinz Wagener. \emph{Die Begleitung des gregorianischen Chorals im neunzehnten Jahrhundert}. Regensburg:  Gustav Bosse Verlag, 1964.

     \parindent=20pt
     \hangindent=20pt
     \renewcommand{\thefootnote}{$\dagger$}
Although this history purports to treat of the organ accompaniment of chant in the nineteenth century, its timeline relies chiefly on Germanic sources and ends around 1866. It was an odd choice indeed to terminate the history of chant accompaniment prior to the foundation of the \emph{Cäcilienverein} when that movement, as we discussed above in \cref{sc:cecilianism}, influenced the practice of Catholic Church music to a great extent. Owing to the author presenting inventories of accompaniment manuals and lengthy music examples in the course of his narrative, his reviewer Rudolf Ewerhart thought the subject of chant accompaniment not very entertaining (`der nicht sehr kurzweiligen Materie'), which was said to be made tedious by the author's sometimes poor sentence construction.\footnote{\cite{EwerhartReviewBegleitunggregorianischen1966}.}\\

    \parindent=0pt
    \hangindent=0pt
  \covid{}Henri Berthet. \emph{Méthode élémentaire d'accompagnement du chant grégorien (selon les principes rythmiques et modaux de Solesmes) et des cantiques, pour les séminaires, pour les paroisses}. 2nd ed. Saint-Étienne:  Stéfa, 1964. \\

    \parindent=0pt
    \hangindent=0pt
  \covid{}Anselmo Suca. \emph{Accompagnamento al canto gregoriano secondo il metodo di Solesmes}. Noci:  La Scala, 1993. \\

    \parindent=0pt
    \hangindent=0pt
  \covid{}Józef \L{}a\'{s}. \emph{Harmonizacja melodii modalnych}. Kraków:  WAM, 2002. Edited by Stanis\l{}aw Ziemia\'{n}ski. \\\pagebreak{}

    \parindent=0pt
    \hangindent=0pt
  A Benedictine Monk. \emph{The Beginner's Book of Chant: A Simple Guide for Parishes, Schools and Communities}. Farnborough:  Saint Michael's Abbey Press, 2003.

     \parindent=20pt
     \hangindent=20pt
     Several short excerpts of quadratic chant notation are provided with a rudimentary accompaniment notated in Roman numerals which are evidently and quite mistakenly assumed by the author(s) to make the intended accompaniment clear to the reader. The distinction between lower and uppercase numerals is not described, for instance, the former denoting minor chords and the latter major ones. These are then related to modern keys which are provided in square brackets adjacent to the music examples, such that `i' stands for an F\kern 1pt\sharp{} minor chord in one example and a D minor chord in another. Although most of the chords are tacitly in 5/3 position, a certain ambiguity arises when the player is prompted to construct 6/3 and seventh chords, since no descriptions of scale degrees, chord construction or part movement have been included to allow a player to decipher the intended harmony. It leaves the interpretation of the Roman numerals open to misinterpretation, a potentially disasterous misfortune when something akin to modern notation could just as easily have been employed. The neologism of chord symbols used as they relate to modern keys is not acknowledged in spite of the acceptance by the author(s) that `modern harmony ruins the modality' (p.~59).\\

    \parindent=0pt
    \hangindent=0pt
  \covid{}Miroslav Martinjak. \emph{Orguljska pratnja gregorijanskih napjeva}. Zagreb:  Glas Koncila, 2005. \\

    \parindent=0pt
    \hangindent=0pt
  \covid{}Fausto Caporali. \emph{L'accompagnamento del canto liturgico.\ Sussidio per l'improvvisazione organistica}. Padova:  Edizioni Armelin Musica, 2010.

     \parindent=20pt
     \hangindent=20pt
     This author also published an article on César Franck's accompaniments. See \cite{Caporalilibrodiaccompagnamenti2015}.\\

    \parindent=0pt
    \hangindent=0pt
  \covid{}Karlheinrich Hodes. \emph{Der gregorianische Choral : eine Einführung}. 5th ed. Mainz:  Ratgeber \& Sachbuch, 2012.

     \parindent=20pt
     \hangindent=20pt
     No more than about a page is dedicated to accompaniment.\\

    \parindent=0pt
    \hangindent=0pt
  J.~B. Hingre. \emph{Méthode d'accompagnement du plain-chant}. 2nd ed. Mirecourt:  Chassel, n.d.

     \parindent=20pt
     \hangindent=20pt
     Provides harmonised scales and certain intervals in the keys of C, G, F major and D and A minor that an accompanist was supposed to use to harmonise any chant melody. `The greatest difficulty for newcomers,' writes the author, `is to modulate properly'. (For more on the modulation method, see above in \cref{sc:modulation}.)\\
