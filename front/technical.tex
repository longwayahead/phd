\chapter{Technical notes}


\section*{Modes}
The eight modes of the Latin octoechos are here identified by their traditional ordinals (first, second, third, etc.) rather than by the pseudo-classical nomenclature of Glarean (Dorian, Hypodorian, Phrygian, etc.).

Modal characteristics that apply equally to the authentic and plagal ambits of a final (such as cadences) are described in terms of the tetrachord of finals (protus, deuterus, tritus and tetrardus).
Hence a characteristic shared by the first and second modes is dubbed `protus', and the so-called `Phrygian cadence' is here called a `deuterus cadence', except in cases where a cited authority uses a proprietary term in a specific context.

\section*{Musical nomenclature}
Pitch classes are identified by inverted commas, such as `C' or `D'.

Absolute pitches are identified in Helmholtz pitch notation, $\emph{F}\kern -1pt`$ being the `F' below Bottom `C', \emph{C} being Bottom `C', \emph{c} being Tenor `C', \emph{c}$^\prime$ being Middle `C', \emph{c}$^{\prime\prime}$ being the octave above that, and so forth.

References to specific notes within a polyphonic texture are made with respect to the number of parts from the bottom or top of the texture, as indicated. In a four-part texture, for example, the `second part from the top' identifies the alto part, whereas the `second part from the bottom' identifies the tenor.

Suspensions are identified thus: 7--6, 2--3, etc.

\section*{Transposition and scale steps}
The term `signature' is used instead of `key signature' to describe the number of sharps or flats left-most on the staff since their presence in a chant harmonisation usually serves to indicate the disposition of tones and semitones rather than major-minor keys.

Since in harmonised editions chants may be transposed away from their natural loci in the Guidonian gamut, it is sometimes necessary to identify scale steps numerically (\pitch{1}, \pitch{2}, \pitch{3}, etc.), \pitch{1} being the final of the mode.
Hence a cadence described as `protus \pitch{2} \rightarrow{} \pitch{1} with \negpitch{7}\kern 1pt\sharp{}' is equivalent to \lilypond[notime]{<< {\voiceOne \clef "alto" cis'1 d'} {\voiceTwo e' d'}>>}, whatever the transposition.



\section*{Time signatures}
The term `time signature' retains its conventional meaning. When such signatures are referenced in the narrative, numerator is separated from denominator by an oblique---6/8 therefore stands for \lilyTimeSignature{6}{8}, and so forth.

\section*{Clefs}
Clefs are identified either by type (G-clef, F-clef, C-clef) or by their placement on the staff (G2, F4, C3), staff lines being numbered from bottom to top---the latter are also equivalent to `treble', `bass' and `alto' clefs respectively.

\section*{Chords}
Chords are identified using numbers and obliques (such as 5/3 and 6/3) or by an ascending order of pitch classes (such as D/F/A).

\section*{Nomenclature for chant books and Mass parts}
The word `Gradual' with an uppercase letter `G' refers to the chant book, whereas the word `gradual' with a lowercase letter `g' refers to the portion of the Mass Proper chanted after the Epistle.
For the sake of consistency, other parts of the Mass or the Office are also rendered in lowercase.

\section*{Numerals and dates}
Numerical punctuation follows the British and American custom of improving the legibility of numbers above 999 by placing commas every third number, and by using periods for decimal points.

Dates are ordered D--M--Y, the month being spelled out in full.

To assist Anglophone readers in locating journal sources in other languages, the months of publication and so forth are translated into English.

\section*{Footnotes, endnotes and pagination}
The abbreviation `n.' or `nn.' reference footnotes or endnotes in source material, whereas the unabbreviated form `footnote' references footnotes in the present dissertation.

Square brackets surround the present author's corrections, amplifications and inferences, except when used in page references because certain Graduals use square brackets to distinguish supplementary pagination. Hence, the range pp.~[112--113] refers to p.~[112] and p.~[113].

\section*{Authorship}
It has not been possible to determine the authorship of every source, perhaps because the significance of a person's initials has been lost to history, because anonymous monastic \mbox{authors} represented the collective thoughts of their monasteries or orders, or because pseudonyms were believed to shield the identity of an author in an arena where public criticism of another's beliefs was common.

For the avoidance of doubt in cases of those monastics who take a religious name, that name is used here instead of their birth name.

\pagebreak{}

\section*{Localisation}
The sacred repertory discussed in this dissertation goes by different names in English, including `plainsong', `plainchant', `Gregorian chant' and simply `chant'.
For the sake of simplicity the last term is used, though it is considered synonymous with the others, except in certain cases where theorists use a given term in a specific context.

Certain French words are not readily translated into English, such as `le chant' (which can mean plainchant, a hymn, a melody, or another song), `la mesure' (which can mean bar or meter), and `le ton' (which can mean key or mode).
While the context usually suggests one translation is more likely than another, it is possible that a different nuance was intended by the original author.
The reader is therefore invited to consult each translation into English in conjunction with the original passage typeset adjacent to it.

The gender and number of the French noun `orgue' offers some challenges to Anglophone usage, because it is ordinarily masculine in the singular and feminine in the plural.
Whereas in Francophone usage the feminine plural term `des grandes orgues' refers to one, large instrument usually placed on an organ gallery, and the masculine singular form `le grand orgue' typically refers to a division of stops and the corresponding manual keyboard in such an instrument, the term `grand orgue' will here be used to describe a gallery organ.
In conformity with Francophone scholarship, the plural form will be treated as masculine.
The gender and number of `orgue' will be retained where a writer uses a given term in a specific context.

Place names used as metonyms for places of worship or their associated congregations are retained, such as `Solesmes' for the monastic foundation at Saint-Pierre de Solesmes.

The adjective `Solesmian', a neologism introduced into the Anglophone discourse in 1933,\footcite[p.~x]{PotironTreatiseAccompanimentGregorian1933} is used here to describe features or methods originating or in use at Solesmes.

\newpage{}
\section*{Handlist}
The handlist of accompaniment manuals (\cref{ap:handlist}) is compiled in chronological order.
It must not be considered exhaustive, in spite of augmenting by over one hundred volumes a list of accompaniment books compiled by Francis Potier in 1946.\footcite[pp.~68--98]{Potierartaccompagnementchant1946}

Certain volumes are followed by brief notes detailing salient characteristics.
Should multiple editions be listed for a single volume and should the content of a note relate to a specific edition, then the ordinal number of the relevant edition will be underlined; otherwise, the note refers to all listed editions.

\section*{PDF copies}
The PDF version of this dissertation renders as intended in Adobe Acrobat Reader DC v2021.007.20091, running on Windows 10.
