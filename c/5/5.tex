\chapter{Modernism}
\section{Broadening modality}
\subsection{At the apogee of chromaticism}
\label{hl:springer}\label{sc:springer_chromatic}\label{cc:chromatic}%
The diatonicism preferred by Mathias and Bas was set quite apart from the type of harmony preferred by Cecilian composers who admitted cadential sharping.
We have already discussed Nekes's reluctance to diverge from the latter tradition, and how the Wiltbergers followed his lead.
But in some quarters, the use of sharping was believed to exist at a single point on a spectrum that spanned between Gevaert's hexachordal accompaniment and unbridled chromatic harmony.
It was towards the chromatic end of the spectrum that some composers were drawn, first in their permitting sharping for stylistic effect and then by their adopting a similar kind of chromaticism to that used by Gorączkiewicz and others.
Prior to discussing a movement to promote the unrestricted admittance of chromatic notes to chant accompaniments, we shall first consider how some composers of the 1900s and 1910s came to prefer more sharps than their Cecilian predecessors had admitted.

As we have seen (\cpageref{ln:springer_omit}), Springer adopted sharping in his accompaniments after having previously expressed a preference for diatonicism.
His \emph{volte-face} is notable for being in the opposite direction to some French and Belgian composers, who had abandoned their preference for chromaticism in favour of diatonicism.
Prior to taking on his revised stance, Springer had campaigned for diatonicism in his manual \emph{Die Kunst der Choralbegleitung}, a book that won some celebrity in America in 1908 when the Benedictine monks of Conception MO published its translation into English.\footnote{\covid{}\cite{SpringerKunstChoralbegleitungTheoretischpraktische1907}.}
But Springer later put down his preference for diatonicism to `youthful over-zeal',\footnote{\cite[31]{HugleChromaticsUseAbuse1917}; Hügle translates the passage from a \covid{}1910 issue of \emph{Gregorianische Rundschau}.} and assumed a more tolerant attitude to the use of sharps.
It earned his style the epithet `relaxed diatonicism' (`gelockerter Diatonik'), a phrase coined by Söhner presumably because Springer only admitted sharps when they did not effect modulations.\footcite[52]{SoehnerOrgelbegleitunggregorianischenGesang1936}

Springer was nonetheless cognisant that some musicians simply preferred diatonicism, and suggested therefore that any sharps printed in his accompaniments could safely be ignored by the player.\footcite[p.~iii]{SpringerOrganumcomitansad1910}
This was the inverse of the compromise that Mathias---presumably with the intention of selling his accompaniment on the French and Belgian markets---had made several years earlier (see \cpageref{ln:mathias_omit_chromatic} above).
Similarities to Mathias did not stop there either, for Springer also seems to have adopted a method similar to Mathias's nine stages, whereby the accompaniment was designed to follow certain characteristics in the chant (\cref{mus:springer_221}).\footcite[pp.~x, 57--8, 221]{SpringerArtAccompanyingPlain1908}
Perhaps Springer used the half-diminished chord quoted in \cref{mus:springer_halfdim_16} because the chant note it accompanied was annotated by a caret.
Why other notes annotated similarly do not receive equally dissonant chords is not altogether clear, though this might have something to do with the hierarchy of accents in Springer's method.
A particularly prevalent accented note perhaps required harmony of even greater vividness.
Springer furthermore flouted traditional contrapuntal rules by permitting parallel bare fifths, a sonority he deployed without reserve.
He even lists several instances of their use in the preface to his accompanied Gradual, no doubt to head off any accusations by grammarians that his accompaniments were benighted by amateurish blunders.\footcite[pp.~v, 16]{SpringerOrganumcomitansad1910}

Certain Anglophone critics remained out of touch with Springer's new stance, and continued to cite his earlier judgement that `chromatics can have no place in Gregorian accompaniment, although it has charm for some'.\footnote{See~\covid{}\emph{Musica divina}, August--September 1913, p.~191; Translated in \cite[7]{UseChromaticsAccompaniment1917}.}
That particular translation appeared in the magazine \emph{The Catholic Choirmaster} in 1917, almost a decade after Springer had turned away from diatonicism.
It fell to Dom Gregory Hügle, prior of the same Conception Abbey that had translated Springer's manual in the first place, to clarify matters,\footnote{Mark Everist is apparently mistaken to record Hügle as the prior of a certain Conception Abbey in Minnesota. See \cite[140]{EveristMozartGhostsHaunting2012}.} taking to the same magazine to alert Anglophone readers that Springer now favoured a moderate, discreet use of sharps.\footcite[30]{HugleChromaticsUseAbuse1917}

The prospect of admitting sharps in contexts other than cadences did not escape the notice of composers at the Regensburg Kirchenmusikschule, where Josef Renner the younger (1868--1934) freely admitted sharps in a manner his predecessors had not.
Renner had studied composition with Rheinberger in Munich,\footcite[260]{KillyDictionaryGermanBiography2005} and in 1893 succeeded Hanisch as the organist of Regensburg cathedral.
He was then appointed as a lecturer in organ playing at the Regensburg Kirchenmusikschule in 1896.\footnote{\cite[326]{VierhausDeutschebiographischeEnzyklopaedie2007}; Renner is captured in a seated position at the organ console of Regensburg cathedral in 1905 in the photograph we discussed above on page \pageref{ln:renner_photo}.}
The chromatic bass part quoted in \cref{mus:renner_deo_5} dates from 1914, succeeding Springer's accompaniments by some years but adopting a similar approach to the use of sharps.
The bass part climbs chromatically from `G' to `A' (traversing a diminished chord), the alto part at `gratias' bringing `C'\kern 1pt\sharp{} into close proximity with `C'\kern 1pt\natural{} two notes later.
Do these sharps effect modulations to A minor and D minor respectively, however brief such modulations may be?
It is not altogether clear what Renner's thoughts on the matter were: perhaps his aims were more aesthetic in nature, and his admission of sharping could quite simply have been a modern affectation; perhaps, also, his use of sharps was simply a matter of personal preference, a factor that is necessarily difficult to quantify.

Renner was nonetheless keen to compose accompaniments that were easy to play.
To that end, he claimed to arrange the parts in one comfortable hand position (`in einer, und zwar der bequemsten Lage').
The claim holds up in the accompanied psalm tones where only cadences were accompanied---the organ rested for recitations.
But it does not hold up when considering the accompaniment quoted in \cref{mus:renner_deo_5} which, especially if played without pedals, calls for frequent changes of hand position and the transferring of inner parts from one hand to the other.\footnote{\cite[unpaginated `Vorwort', p.~5]{RennerOrganumcomitansad1914}; While no date of publication was marked on Renner's pamphlet, it was published no later than 1914 when it was reviewed in \cite[383]{BooksReceived1914}.}

\label{hl:griesbacher}%
One of Renner's colleagues made an even more daring foray beyond Cecilian sharping practice by admitting to an individual chant accompaniment all the notes of the chromatic octave.
Peter Griesbacher (1864--1919) was appointed as a lecturer in counterpoint at Regensburg's Kirchenmusikschule in 1911,\footcite[55]{MusikalischeRundschau1911} and soon thereafter sought to re-establish chromaticism as the appropriate harmonic language of chant accompaniment.
Not only did Griesbacher view diatonicism as an inartistic principle (`ein völlig unkünstlerischer Grundsatz'), but he also dismissed it as a modern invention without a basis in history, a claim that was surely levelled at the theorists professing to apply long-lost musical methods to harmony of the modern age.
As far as Griesbacher was concerned, chromaticism was far better at capturing the modern Zeitgeist than any made-up diatonic theory; and it was also more stylistically appropriate than diatonicism because it permitted more conjunct motion in the accompanying parts, all while the chant itself remained diatonic (\cref{mus:griesbacher_credo_202}).
He argued that conjunct motion was a feature of the chant repertory which should therefore be matched in the accompaniment, going on to posit that diatonicism had been forcing composers to use disjunct motion which, in his opinion, resulted in ugly accompaniments that did not suit the repertory at all.\footnote{\covid{}\cite{GriesbacherQuatuormodicantandi1911}; Reproduced in \cite[202]{GriesbacherChoralundChroma1912}.}

Griesbacher followed up his admittedly incendiary view with a textbook on aestheticism in church music where the question of chromaticism was broached once again.
He provided a similarly chromatic harmonisation that was said to capture his ideal style (\cref{mus:griesbacher_ideal_88}), one derived from sonorities popularised by Richard Wagner:

\simplex{Meine Devise heisst: Choral und Wagner! Choral und volle Freiheit der Harmonie! Choralbegleitung ohne jede Einschränkung der künstlerischen Idee! Mag sie manchem Ohre heute noch herbe klingen, die Zeit wird kommen, wo sie Leben gewinnt.}
  {\cite[88--9]{GriesbacherKirchenmusikalischestilistikund1912}}
{My motto is chant and Wagner! Chant and full freedom of harmony! Chant accompaniment without any restriction of the artistic idea! It may still sound bitter to some ears today, but its time will come.}
\noindent
We might recognise in Griesbacher's motto a certain elaboration of Nekes's view that the accompaniment had `rights' just as much as the chant did.
It seems to have gained the support of one Anton Möhler (1866--1939) who hoped an accompanied Vatican Gradual would be produced by Griesbacher in his Wagnerian idiom.\footcite[32]{MohlerUberChoralbegleitung1912}
Möhler even proposed that chromatic accompaniments were required so that modern ears did not become fatigued, this in a textbook on Catholic Church musical aesthetics.\footcite[144--6]{MoehlerAesthetikkatholischenKirchenmusik1915}

Griesbacher's failure to produce a chromatic, accompanied Gradual was probably due to the backlash against his system arising from various quarters.\footnote{Griesbacher applied chromaticism to the psalms for Vespers. See \covid{}\cite{GriesbacherPsalteriumvespertinumTonos1913}.}
The critical reception to Nekes's use of sharping in 1906 had not boded well for Griesbacher, and sure enough, Möhler's approval was drowned in a cacophony of opposition to such chromatic accompaniments.
The Swiss musician Joseph Frei (1872--1945), despite being partial to cadential sharping provided it did not distract the listeners from the chant,\footcite[251]{FreiChoralundChroma1912} rose up against Griesbacher in the periodical \covid{}\emph{Chorwächter} to accuse the composer of bungling his way through harmonisations.
Griesbacher took to Regensburg's \emph{Musica sacra} to defend his own track record as a composer.
The matter was batted between the two polemicists without either side ceding much ground to the other.

\pagebreak{}
Franz Josef Breitenbach (1853--1934), another Swiss musician, weighed in on the matter by accusing Griesbacher of wishing to create a `very subjectively coloured tone picture' (`ein ganz subjektiv gefärbtes Tonbild').\footcite[249]{BreitenbachChoralundChroma1912}
Mitterer voiced his opinion in the \emph{CVK} by claiming his own ideal lay somewhere between the two extremes of diatonicism and chromaticism.\footnote{\cvk{4031}.}
In other words, the weight of consensus rested with cadential sharping rather than with Griesbacher's chromaticism.
Without the support of his peers, his style all but petered out, leading Heinz Wagener to deem his attempt at resurrecting chromaticism a failure.\footcite[100]{WagenerBegleitunggregorianischenChorals1964}

Chant aesthetics were nonetheless developing to a point at which certain harmony treatises by deceased Cecilian composers (such as Peter Piel) were being revised to suit evolutions in taste.
\hlabel{hl:piel_harmonielehre}%
Bearing in mind that Piel had died in 1904, the 1910 edition of his \emph{Harmonie-Lehre} incorporated a revised and expanded chapter on accompaniment that was presumably the work of the book's editor Paul Mandersheid.
Manderscheid's Italian translator Eduardo Dagnani made some additions of his own to suit the Italian market, such as the addition of example accompaniments by Bas and Peter Wagner, as well as the provision of an up-to-date bibliography to benefit Italian students.\footcites[256--60]{PielTrattatodicomposizione1911}[Manderscheid's editorship of the 1910 edition is not mentioned in conjunction with its entry in][p.~111 under T544]{PeroneHarmonyTheoryBibliography1997}
The driving force behind that translation was Giovanni Tebaldini (1864--1952), Bas's former teacher, who had been critical of Bas for adopting the Solesmian approach to accompaniment when a Cecilian one was, he claimed, more practical.\footcite[98]{MilaneseGiovanniTebaldiniaccompagnamento2017}
Perhaps the rationale for Tebaldini's criticism stemmed from the nineteenth-century view of Solesmian scholarship: that its value was more theoretical than practical.
Such had been the view propagated by the Vatican prior to the twentieth-century decrees establishing Solesmes as the seedbed of Catholic church music.
But considering Nekes's reluctance to adopt the Church's new stance, it is quite possible that other Cecilians were equally as reluctant.
\nowidow[2]

Yet another edition of Piel's treatise was published in 1910 this time in Polish, but little about it seems to have been updated from Piel's own eighth edition of \emph{c}.1903.
In fact, the copy of that edition consulted by the present author omitted the discussion of accompaniment, perhaps because Piel's methods were considered out of date.
That is not to say, however, that chant was edited out of the Polish edition: where chant was used to demonstrate harmonic progressions, the editor retained it, including those psalm tones Piel had parsed using Roman numerals to demonstrate major-minor progressions at cadences.\footnote{Compare, for instance, Piel's proposed psalm tone harmonisations using Roman numerals in \cite[55]{PielHarmonieLehreUnterbesonderer1903} with the same discussion in \cite[55]{PielWykladnaukiharmonii1910}.}

\hlabel{hl:schildknecht_orgelschule}%
Schildknecht's \emph{Orgelschule} also underwent revisions, reaching its twentieth edition around 1935, some three-and-a-half decades following the author's death.
Among the editors by then was Söhner, who was no doubt responsible for updating the chapters on accompaniment.
The inference is supported by the appearance of identical prose in a separate publication attributed to Söhner's sole authorship in which he describes a more recent Solesmian method of accompaniment that was not devised until two decades following Schildknecht's death, to be discussed below.\footcites[140--60]{SchildknechtJosefSchildknecht1935}{SoehnerKurzeAnleitungzur1935}

\subsection{Broadening concepts of diatonicism and modality}
Just as Griesbacher had sought to extend chant harmony to comprehend chromaticism, so certain other theorists sought to extend diatonicism to comprehend more dissonance.
Nineteenth-century theories of diatonic chant harmonisation rendered exclusively as consonant chords seemed no longer fit for purpose, and although the French ear in particular remained prejudiced against the use of sharps, it did not take exception to the dissonant-laden accompaniments that Lepage and others had been popularising since the end of the nineteenth century.
Perhaps the normalisation of dissonant sonorities might be explained by wider developments in musical modernism; whatever the reason, extensions to chords were to be permitted in chant accompaniments provided that the chords themselves remained diatonic.
Like rhythm, then, diatonicism itself became free as composers assembled the notes of plainchant into chords of the seventh and ninth.\footcite[359, 363]{Lessmannanachronismemusicalaccompagnement2019}

The tendency to permit more dissonance in an accompaniment arose contemporaneously with the desire to limit the frequency of chord changes.
By permitting more dissonance, composers could justify greater quantities of chant notes above a single bass note.
In contrast to those Cecilian idealogues who held that chant and accompaniment should be granted equal status, Benedictine practitioners in particular recognised that the accompaniment ought to be relegated to the background.
%In 1899, Kienle had permitted the interludes alone to stray from diatonic harmony,\footcite[36--7]{KienleChoralschuleHandbuchzur1899}
\hlabel{hl:johner}%
From as early as 1906, the Beuron-based Benedictine monk Dominicus Johner (1874--1955) set forth a `strictly diatonic' (`streng diatonisch zu verfahren') scheme that not only permitted bare fifths but also seventh chords without any preparation,\footnote{\cite[207]{JohnerNeueSchulegregorianischen1906}.\label{fn:johner_firstedn}} anticipating by four years the publication of Debussy's `La cathédrale engloutie'.
The passage describing seventh chords was revised in 1921 to address not only how they could be approached but also how they could be quitted:

\duplex{Manchmal wird man den Septimenakkord vorbereiten können. Häufig wird er unvorbereitet eintreten müssen. Die Septime kann regelmäßig aufgelöst werden oder liegen bleiben, oder nach oben oder sprungweise nach unten gehen.}
  {\cite[p.~135*]{JohnerNeueSchulegregorianischen1921}; Compare to p.~208 in the edition cited in \cref{fn:johner_firstedn}}
{Occasionally it will be feasible to \mbox{furnish} a preparation before the seventh-chord. It will frequently obtrude itself without preparation. The seventh can always be resolved or left unresolved: it may be led upwards or by step downwards.}
  {\cite[290--91]{JohnerNewSchoolGregorian1925}; This translation was made from the fifth Teutophone edition}
\noindent
The seventh chord was therefore considered to be a sonority in its own right, a consideration that extended to all its diatonic dispositions, between which Johner drew no distinction.
\hlabel{hl:molitor}%
He probably discussed the idea with his \emph{confrère}  Ferdinand Gregor Molitor (1867--1926) who proffered advice of his own on the matter in 1913.
Along with a discussion of seventh chords, Molitor provided more avant-garde principles that flouted conventional rules.
For instance, there was a framework for using bare fifths depending on an interpretation of chant rhythm, and proposals for different textures to suit different kinds of chants.
A freely composed, artistic kind of accompaniment was said to be best for accompanying melismatic passages sung by soloists (\cref{mus:molitor_melisma_99}).
The chant itself was not to be replicated by the accompaniment, whose texture often increased in density the closer it approached a cadence.
The latter technique was a difficult one, however, and was said to be beyond all but the most experienced players.
Consequently, Molitor offered little advice about how it might be applied, leaving that kind of detail to the imagination of his readers and the skill of prospective players.

That is not to say that Molitor shied away from providing any instructions at all.
In fact, he suggested that accompaniments were to be played in three parts (save, perhaps, at cadences) and on manuals alone.
The chant, as noted, was to be omitted entirely, a measure obviously fit for relegating accompaniments to the background:

\simplex{In diesem Falle trete sie vollständig in den Hintergrund und beschränke sich darauf, der frei hingleitenden Melodie einen weichen harmonischen Untergrund zu bieten, der noch lediglich den Zweck hat, die Melodie um so deutlicher hervortreten zu lassen.}
  {\cite[92--101]{MolitordiatonischrhythmischeHarmonisationgregorianischen1913}}
{In this case it moves into the surroundings and is limited to providing the freely moving melody with a soft harmonic background, whose only purpose is to make the melody stand out all the more clearly.}
\noindent
The desirability of omitting the chant and increasing the level of dissonance is evident also from Bewerunge's 1916 statement that `the organ ought not to play the melody' and from his conclusion that frequent changes of harmony were best avoided.\footcite[252]{McCarthyHeinrichBewerunge18622015}

A more sustained style of accompaniment in which greater numbers of chant notes were accompanied with fewer changes of harmony presented quite a peculiar problem to composers.
How were dissonances to be handled?
\label{sc:brun}%
The problem was initially addressed by Brun, whose parsing of the Schola Cantorum accompaniments (see \cpageref{ln:schola_accomps} above) led him to identify three categories of accompaniment: one that reproduced the chant in the top part throughout, another that omitted the chant and reserved chords for `des \emph{notes réelles}', and a third that comprehended a more elaborate texture for which Brun coined the term `accompagnement concertant'.\footcites[19--20]{BrunTraiteaccompagnementchant1912}[Discussed in][56]{DelSordomonemicanellaccompagnamento2000}

While it is true that Brun overlooked many other approaches to accompaniment, his third category was particularly significant because it comprehended the more dissonant texture that was gaining popularity among those seeking to detach chant from accompaniment.
Marc de Ranse, the \emph{maître de chapelle} of the Parisian church of Saint-Charles de Monceau, composed an accompaniment in the `concertant' idiom for the Ascension-tide chant `Viri Galil\oe{}i' (\cref{mus:deranse_ascension}), a footnote directing a `discreet and light' registration to be used.
A heavy registration would hardly have suited the accompaniment's texture, which ascends gradually through the keyboard's register.
The process of adding notes to create ascending complexes of decorative, diatonic dissonance results in a gentle, amorphous aura that might be described as an auditory equivalent of incense.\footcites[14]{DeRanseSupplementIntroitJeudi1909}[Re-printed in][54--5]{BrunTraiteaccompagnementchant1912}

The definition of the `concertant' idiom might not limit itself to mild decorative dissonance, for Heinrich Wismeyer (1898--1984) arguably produced a similar effect during the 1930s by maintaining pedal notes, ascending chords and restricting the accompaniment to an octave or two above sung pitch, depending on whether the chant was sung in the monks' octave or in the nuns' (\cref{mus:wismeyer_sustained}).\footnote{\covid{}\cite{WismeyerOrgelbegleitungChoralgesangen1933}; Reproduced in \cite[118]{Potierartaccompagnementchant1946}.}
The celebrated Belgian organist Flor Peeters (1903--1986) will be discussed below, but for the moment it should be noted that, in 1946, he demonstrated a similar texture to Wismeyer's where the accompaniment also remained resolutely above sung pitch (\cref{mus:peeters_angelis_73}).\footcite[73]{PeetersPracticalMethodPlainChant1949}
Peeters gave no details about the antecedents of his texture, however, when he instantiated it as just another available method.

\hlabel{hl:emmanuel}%
Whereas Brun discussed the Schola Cantorum's methods as they were at the end of the nineteenth century, a former Schola Cantorum teacher Maurice Emmanuel (1862--1938) brought out a manual in 1913 that took a self-consciously historicist view on the subject.
Emmanuel had held the post of teacher of music theory from 1907 to 1912, and succeeded his mentor Bourgault-Ducoudray as the teacher of music history at the Paris Conservatoire in 1909.\footcite[265]{Corbierrelationsmusicalesfrancohelleniques2018}
It was under Bourgault-Ducoudray that Emmanuel had first encountered the modes,\footcite[279]{StewartMauriceEmmanuel18621939} a formative step which led him to seek out a historic method of accompaniment in the music of the distant past.
He did not accept the rhythmical theories posited by Solesmian theorists and adopted instead the mensural theory of the Medieval historian Georges-Louis Houdard (1860--1913).
\hlabel{int:emmanuel}%
It is not clear how any such mensural theory could have influenced Emmanuel's accompaniments, however, because he held that only the psalm tones could be accompanied.
On receiving the manual, Saint-Saëns quipped that `one does not accompany psalms'; Emmanuel rejoindered `If you had read my book as far as page 3 inclusively, you would have seen how I do not counsel accompaniment of the psalms, but that, if one must do it, one must employ the modes'.\footcite[162]{StevensonMauriceEmmanuelBelated1959}

\label{cc:emmanuel_octave}%
Emmanuel was nonetheless one of the very few harmonisers to recognise the desirability of arranging the accompaniment specifically for the octave in which the chant was to be sung.
Niedermeyer's rule that the melody should always be placed in the top part (see \cpageref{ln:niedermeyer_rules,ln:niedermeyer_rules_END} above) had up to this point been almost universally observed, with the result that the organ accompaniments invariably---and, it might be said, tediously---doubled the chanting of men's voices at the octave above.
Emmanuel, however, specified that the texture quoted in \cref{mus:emmanuel_traite_102} as being adapted to children's voices (`On suppose ici les versets chantés par des voix d'enfants. La hauteur de l'accompagnement s'y adapte.').\footcite[102]{EmmanuelTraiteaccompagnementmodal1913}
The accompaniment did not double the psalm tone, permitting greater freedom to the chanting as it meandered on occasion beneath the topmost accompanying part.
At the medial cadence, one recognises that the chanted `B'\kern 1pt\natural{} was treated as a dissonance, perhaps deliberately so in order to demarcate the accented syllable in `meam'.
But that dissonance also might have something to do with urging on the singers: Johner recommended dissonance for just such a purpose,\footcite[296]{JohnerNewSchoolGregorian1925} which arguably lent itself well to accompanying children or to an ensemble singing without the aid of a conductor.
%Perhaps it was simply a matter of the chant being sung at written pitch rather than down an octave, a tacitly commonplace result of chanting by a congregation.\footcite[373, 376, 380]{Lessmannanachronismemusicalaccompagnement2019}

While Emmanuel sought an authentic historical framework for his theory of accompaniment, a contrary stance was taken up by the fellow Bourgault-Ducoudray pupil Charles Koechlin (1867--1950), who recognised that accompaniment was inalienably an anachronistic endeavour.
In a similar argument to Griesbacher's, Koechlin opined that there was little sense in seeking historical rules to govern accompaniment when no such accompaniment had existed in the first place, and recommended instead `the most musical kind of anachronism', a type of modern accompaniment complete with passing notes, unprepared dissonances, modulations, and so forth.\footcite[382--3]{Lessmannanachronismemusicalaccompagnement2019}

While that sort of accompaniment was evidently appealing to those in the orbit of the Schola Cantorum, we should not forget alternatim practice, which continued to enjoy widespread use in French churches.
In contrast to the modern invention of diatonicism, there can be no doubt as to its historical authenticity, which as we have seen (\cpageref{ln:widor_alternatim,ln:widor_alternatim_END}) was still current at Saint-Sulpice in the very late nineteenth century.
Practice of that kind would appear to have inspired a genre of choral composition in which sung polyphonic parts alternated with monophonic chanting.
One example was composed by Louis-Lazare Perruchot (1852--1930), the \emph{maître de chapelle} of Monaco cathedral, who accompanied both the polyphony and the chanting.
Note how, in the latter case, the accompaniment quoted in \cref{mus:perruchot_interlineal} was governed by the rule whereby chords changed on the first notes of beamed groups.\footnote{\cite[1]{PerruchotMessediteAnges1910}; A separate choral part was also attached to \tsg{}. See \cite[2]{PerruchotSupplementMessedite1910}.}

\subsection{Bas and the `courageous' style}
\label{hl:bas_apodose}%
Among the first contributions to Solesmes's journal \emph{Revue grégorienne} in 1911 was a short series of articles by Bas, who took up the matter of changing chords less frequently.
With a view to systematising the approach, he divided up phrases into `protase' (antecedent) and `apodose' (consequent), harmonising the former in bare octaves and the latter with chords (\cref{mus:bas_antcons_116}).\footcite[116]{Bassimplicitedansaccompagnement1911}
It was a texture harking back to Ett's accompaniment of 1834 (see \cpageref{ln:ett_bare,ln:ett_bare_END} above) and was similar to that which Witt had advocated as his ideal in 1872 (see \cpageref{ln:witt_octaves,ln:witt_octaves_END} above), the latter having subsequently been revived in 1910 by Gastoué, who decided it was suited to large choirs and an organ registration consisting of Trompette and Clairon.\footcite[87]{GastoueTraiteharmonisationchant1910}
Bas acknowledged none of those musicians in his articles, and either arrived at the texture independently or elaborated on a technique that was circulating among practitioners.

Whatever the case may be, Bas levelled criticism at certain composers who he claimed changed chords more frequently than necessary because they were trying to make their accompaniments follow every rhythmic nuance in the chant.
Overwrought accompaniments were a far cry from the mellifluence Bas believed French audiences preferred.
And omitting the chant from busy accompaniments made matters even worse, because the next part down in the texture distracted the listener's ear with what sounded like an overactive counter-melody.
To substantiate these claims, Bas reproduced six anonymous accompaniments for `Kyrie fons bonitatis', comparing them with one of his own examples that used fewer chords.\footnote{\cite[143--9]{Bassimplicitedansaccompagnement1911a}.}
He naturally concluded that the true path to success was comprehended by his sustained accompaniment, quoted in  \cref{mus:bas_kyriesustained_149}, whose parts moved so infrequently as not to distract should the chant be omitted by the accompanist.
Comparison with Bas's earlier harmonisations (see again \cref{mus:bas_salve,mus:bas_epiphany_4,mus:bas_angelis_repertorio_15,mus:bas_angelis_kyriale_40}) shows clearly that by 1911 his style was characterised by part-writing that was much more sustained.\footcite[10]{BasKyrialeseuordinarium1906}
\nowidow[2]

Aside from those textural matters, Bas also embraced a broader concept of chant \mbox{harmony} that blurred the lines between diatonicism and major-minor harmony.
While at work for Desclée on the accompanied Proper of the Time, Bas remarked that his harmonisations contained some novel features, including major-minor harmony (`même du côte tonal').\fnletter{Bas}{Mocquereau}{19 June 1912}{\so{}}
The first volume, covering Sundays between Advent and Easter, was due for publication in time for 1 December 1912, the first Sunday of Advent, and by October the accompaniments had been engraved.
That left only the preface, which Bas wrote in Italian and translated into French, and which arrived at Solesmes in time for vetting.\fnletter{Bas}{Mocquereau}{18 October 1912}{\so{}}
From the first lines of the original and the translation, Bas made it clear that he viewed harmony and melody as equal partners (`Armonia e melodia sono due elementi d'uguale important'), and that he believed `la tonalità antica' was founded on the very same basis as modern harmony, a belief that permitted greater latitude in his choice of chords.
Harmony was to be `simple', a French term suggested in the markup to replace Bas's instinct first to use `pauvre', a literal translation of the Italian `povera'.\footnote{\covid{}\cite{BasPropriumtemporeAdventu1912}.}

Bas's admission of major-minor harmony anticipated by three years his foray into chant-based free composition.
In the same spirit as Gigout and Guilmant, Bas wrote a piece for organ solo that matched the style of an accompaniment to the Epiphany chant `Reges Tharsis' (\cref{mus:bas_compose}).\footcite[3]{BasAllaMessaed1915}
In fact, the tail end of the chant was printed as the first line of the piece, making it clear how the accompaniment was to segue into the composition.
While the one cannot be directly equated to the other, the postlude serves nonetheless as a witness to cross-fertilisation between accompaniments and solo literature.

Bas's conscription into military service at the outbreak of WWI interrupted his work on accompaniments.
He was drafted in to serve with the French territorial forces, in the 98\textsuperscript{th} and later the 102\textsuperscript{nd} Infantry Regiments, which saw battle at the Somme, Aisne and Oise.
Further details on Bas's function in the military have not yet come to light, but from the patchy correspondence he exchanged with Solesmes he appears to have been some kind of clerical functionary.
That was musically advantageous for a number of reasons, not least because it granted him access to the writing paper on which he penned three treatises on chant, including one on accompaniment, to which we shall now briefly turn.

The first, entitled \covid{}\emph{La sostanza dei modi gregoriani}, was sent to Quarr around 1916 with the macabre intent of preserving his thoughts on chant matters should he perish at the front.\fnletter{Bas}{Mocquereau}{[\emph{c}.1916]}{\so{}; The letter was written at Compiègne, Oise}
The second concerned the subject of chant accompaniment and was completed during 1917 before it was also was dispatched to England.
Bas hoped it could benefit the Solesmian monk and organist Leopold Alphonse Zerr (1879--1956) who was to correct the French translation:\footnote{The present author is grateful to Dom Cuthbert Brogan, abbot of St Michael's Abbey Farnborough, for confirming Zerr's year of birth.}

\simplex{Depuis quelques semaines nous avons moins à travailler dans notre bureau, et j'ai commencé un petit ouvrage pratique, où se trouve réuni en forme simple tout le résultat de mon expérience en fait de tonalité grégorienne en rapport à l'harmonie et partant à l'accompagnement. Ce petit travail est en français, mais naturellement dans un français pitoyable. Je m'adresse à l'amabilité du P.\ Zerr. Il pourrait faire les retouches nécessaires, et en même temps la connaissance du petit livre l'intéresserait. Après il y aurait le problème de trouver un éditeur, chose pas très facile en ce moment. Pourriez-vous m'en indiquer un~? Je pense que Desclée ne pourra pas s'en occuper dans les conditions actuelles.}
  {\letter{Bas}{Mocquereau}{15 May 1917}{\so}}
{For the last several weeks we have had less to do in our office, and I started on a small practical work, in which I have assembled in simple form the entire summary of my experience on the question of Gregorian \emph{tonalité} in its relation to harmony and from the perspective of accompaniment. This little work is in French, but naturally in very poor French. I appeal to the kindness of Fr Zerr. He could doctor it up, and at the same time the content in the little book might interest him. Then there would be the problem of finding a publisher, not an easy task at present. Could you recommend one to me? I think Desclée cannot take this on in the present climate.}
\noindent
We shall return to the subject of publishers in due course, but for the moment let us consider Zerr's interest in accompaniment, for his link to Bas predated WWI.
In 1909, Mocquereau had asked Bas to find for Zerr an organ teacher who could reside for a time in England: Bas's teaching commitments prevented his accepting the task for himself so he instead suggested Oreste Ravanello (1870--1938).
Ravanello's credentials surely made him an ideal choice since he was a practitioner in his own right with experience as \emph{maestro di cappella} at the Basilica of Saint Anthony, Padua.\fnletter{Bas}{Mocquereau}{4 February and 8 May 1909}{\so{}}
One year prior to Bas's suggestion, however, Ravanello had criticised Mocquereauvian rhythm, claiming that it was better suited to singing in French than in Latin (`serveno mirabilmente la lingua francese').\footnote{\covid{}\cite{RavanelloSullritmosull1908}; Adapted from its citation in \cite[56]{DelSordomonemicanellaccompagnamento2000}.}
That comment probably owed its existence to the view that Mocquereau's theory placed \emph{ictuses} on the last notes of groups, rather like the way the strong accent in French prose often falls on the last syllable (see \cref{ln:moc_lastsyllable}).
It is hardly surprising, therefore, that Mocquereau declined Ravanello as a suitable candidate as Zerr's teacher; but it is not altogether clear who might have taken up the position in his place.

Whoever was chosen to teach him, Zerr became sufficiently skilled to assume the position of organist at Farnborough Abbey, where, in the 1930s, he wrote some accompaniments of his own.
We might trace the lineage of the conjunct passage quoted in \cref{mus:zerr_ordo_2} to Bas's 1905 entry in the \emph{Paléographie} (on which, see \cpageref{ln:bas_paleo_rests,ln:bas_paleo_rests_END}), for Zerr's parts entered one-by-one, perhaps to coincide with successive \emph{ictuses}.\footcite[2]{ZerrOrdoadrecipiendum1937}
Note, also, how Zerr adopted the double-signature method, his publisher separating primary from secondary signatures with the conjunction `or'.
In a separate fragment brought out by the same publisher, Zerr adopted two different textures when accompanying a cantor versus accompanying a choir, the former being accompanied in three parts and the latter in four; though it is necessarily difficult to prove that those textures came from Bas directly since they could just as readily have been absorbed from the practice of others.\footcite[1]{ZerrMissaOrbisfactor1936}

The accompaniment manual Bas wrote at the front was not his first attempt at such a textbook.
He had first started drafting a similar book while preparing the accompanied Proper of the Time for publication in 1912.\fnletter{Bas}{Mocquereau}{16 May 1912}{\so{}}
But an entirely separate manual he was in the process of writing at the time, on musical form, caused him to question Mocquereauvian rhythm.\footnote{For Bas's discussion of the form of chant sequences and responses, see \cite[133--5]{BasTrattatodiforma}.}
His doubts had seemingly dissolved by the time he took up the matter once again while at the front, where he also conducted demonstrations of Mocquereauvian rhythm at a certain thirteenth-century Église de Saint-Yves in a fit of propaganda more musical than political (`j'exerce un peu de propagande pour la bonne cause').\fnletter{Bas}{Mocquereau}{18 December 1917}{\so{}}
The third of Bas's treatises to have been written at the front concerned chant rhythm which he intended as a short primer on the transcription of chant into modern notation.
It was scribbled on the back of an army ledger and was dispatched to Quarr for comments.
Bas also requested that Mocquereau add metronome markings to the music examples since he, understandably, did not have access to a metronome at the front.\fnletter{Bas}{Mocquereau}{20 February 1918}{\so{}}

Shortly after Armistice Day, Bas visited the Desclée branch in Brussels, but found that not only the publishing house but also the plates used to print his previous accompaniment books had been destroyed (`l'établissement de Tournai a été brûlé par les boches partants').
Bas was far from despondent, however, and took the opportunity to revise his previous accompaniments and to bring them up to date with more recent developments in Mocquereau's ideas.\fnletter{Bas}{Mocquereau}{5 December 1918}{\so{}}
Bas's approach to accompaniment had also evolved, and suggests that his time as a soldier had come to influence his process of harmonisation:

\simplex{Que penseriez-vous si l'accompagnement était \emph{très courageux}, c'est-à-dire trés transparent, et ne reproduisent [\emph{sic}] pas toujours le chant~?}
  {\letter{Bas}{Mocquereau}{2 February 1919}{\so{}}}
{What would you think if the accompaniment were \emph{very courageous}, that is to say very transparent and not always reproducing the chant?}
\noindent
Between February and April of 1919, Bas recomposed the accompanied Kyrial Desclée had published in 1906, reworking it from scratch (`en le retravaillant à fond').\fnletter{Bas}{Mocquereau}{9 and 22 February, 13 and 18 March, 10 and 21 April 1919}{\so{}}
Accompaniments to the Requiem mass and Sunday Vespers were dispatched to Solesmes the following December, whence they were forwarded to Desclée by Dom Le Floch,\footnote{\letter{Bas}{Le Floch}{9 December 1919}{\so}; \letter{Bas}{Mocquereau}{19 December 1919}{\so{}}}
The newly accompanied `Missa de Angelis', to be discussed below, followed in February 1920.\fnletter{Bas}{Mocquereau}{2 February 1920}{\so{}}

The `nuova armonizzazione' of the Kyrial appeared later that year.
The general texture, illustrated in \cref{mus:bas_qui_33}, is notable for the greatly increased proportion of unharmonised notes.
Chords were reserved for cadences in an application of the `protase'/`apodose' notion, while the density of the texture constantly varies, chords consisting of increasing numbers of notes as certain cadences are approached.
Although dissonances are prepared in the traditional manner, some, such as that in the tenor part at the end of the first line of \cref{mus:bas_angelis_35}, do not resolve until after a rest.
A particularly striking case also occurs in the Credo IV harmonisation, at `et homo factus est' (\cref{mus:bas_ethomo}).\footcite[33, 35, 78]{BasKyrialeseuordinarium1920}
There is no avoiding the termination of the previous phrase on a dissonance, but it is unclear whether this eccentricity originated with Bas or elsewhere; in any case there has been a clear divergence from his 1906 practice---note, for instance, how Bas now indicated the division of the chant between groups of singers.
One Teutophone writer described Bas's accompaniments as being the easiest to play of all those on offer in 1922, but judged them not demonstrative of much artistry (`Bas ist von allen Orgelbegleitungen die einfachste und leichteste, ohne viel Kunst').\footcite[25]{WeitzelFuehrerdurchkatholische1922}

Bas followed up his new accompaniments with a manual that codified a further topic he had broached in 1911: omitting the chant from the accompaniment.
When presented with an accompaniment such as the one quoted in \cref{mus:bas_sustained_kyriale}, the player was instructed to omit the chant entirely, and (rather than reducing the texture to three parts) to extemporise an uppermost part in the sustained style.\footcite[38]{BasKyrialeseuordinarium1920}
\Cref{mus:bas_sustained_traite} was Bas's ideal solution, which began with a more sustained organ part at the choir's entry before picking up the chant again at the end of the phrase.\footcite[144--5]{BasMethodeaccompagnementchant1923}
\nowidow[2]

The manual was published in Italian around 1920 and was published in a French translation undertaken by the Chartres priest Yves Delaporte (1878--1979) in 1921.\fnletter{Bas}{Mocquereau}{18 January 1921}{\so{}}
Around that time, however, the reservations concerning Solesmian rhythm which Bas had suppressed for nearly two decades finally came to a head.
Bas accused Mocquereau of placing the accent spontaneously on the `levé' and of being unable to explain the progeny of certain aspects of his theory.\fnletter{Bas}{Mocquereau}{5 January 1923}{\so{}}
With that, their collaboration ended almost as quickly as it had begun.
Although several accompaniment books by Bas were published by Desclée in later years, he ceased being Solesmes's pseudo-official harmoniser, a role that was taken up by the Benedictine monk to whom we shall now turn.


\section{A new approach to \emph{tonalité} at Solesmes}
\label{hl:three_groups}%
\subsection{Modal equivalence}
\label{ln:quarr_foundation}\label{cc:dd_equiv}%
During the late 1910s a new method of chant analysis surfaced at Solesmes which was, during the 1920s, adopted as the official \emph{modus operandi}.
It was borne of an analytical method devised by Jean-Hébert Desrocquettes (1887--1973), who was professed a monk of Solesmes on the Isle of Wight in 1911.
By then, the Solesmes community had moved to Quarr Abbey near Ryde, their lease on Appuldurcombe having expired in 1908.
It was at Quarr that a monastery was built according to plans drawn up by the Benedictine monk and architect Dom Paul Bellot (1876--1944).
A new Mutin-Cavaillé-Coll organ was installed on the gallery of the abbey church in 1912,\footnote{\qaa{} QAA-B-448.} over which Desrocquettes himself presided as organist from 1917.

Desrocquettes's interest in music led the monastic authorities to assign him to a chant-based paleographical project, just as they had assigned Delpech.
While in the process of transcribing chants for the 1934 \emph{Antiphonale Monasticum}, Desrocquettes believed he observed a previously unnoticed trait that governed how certain chants were made up, whereby the same phrases occurred at two or three different transposition levels.\footcite[108--109]{ClaireModalityWesternChant2008}
He observed, for instance, that `Pange lingua' sometimes began on `E', the semitone occurring between that pitch and `F'; and also that the very same chant sometimes began on `A', the semitone occurring between that pitch and `B'\kern 1pt\flat{}.\footcite[29]{DesrocquettesListeprincipalesequivalences1925}
He also recognised a third transposition level, whereby the semitone would occur between `B' and `C', though not in the case of `Pange lingua', since retaining the disposition of tones and semitones would require a prohibited `F'\kern 1pt\sharp{}.

Desrocquettes's observation proved seductive enough to lead him to a theory whereby the three transposition levels comprised different yet equivalent \emph{tonalités}.
Each could \mbox{establish} itself whenever its characteristic semitone was heard.
That postulate spawned another: a deuterus cadence could take place on `E', on `A', or indeed on `B', depending on the chant.
Hence Desrocquettes reckoned that a characteristic cadence on a specific pitch could also establish a given \emph{tonalité}.

In each of the \emph{tonalités}, Desrocquettes constructed a tetrachord of finals, naming it after its highest note, terming  `G'--`A'--`B'--`C' the \emph{Do} \emph{tonalité}, `C'--`D'--`E'--`F' the \emph{Fa} \emph{tonalité} and `F'--`G'--`A'--`B'\kern 1pt\flat{} the \emph{Si}\kern 1pt\flat{} \emph{tonalité}.
Either the characteristic semitone `E'--`F' or a deuterus cadence on `E' could establish the \emph{Fa} \emph{tonalité}, and so forth.
Protus cadences on `D' and tritus cadences on `F' were also said to establish the \emph{Fa} \emph{tonalité}.\footcite[208--209]{Desrocquettesaccompagnementmelodiegregorienne1923a}
Tetrardus cadences on `G', by contrast, established the \emph{Si}\kern 1pt\flat{} \emph{tonalité}.

It was and still is a confusing state of affairs, which Desrocquettes attempted to explain by the diagram reproduced in \cref{mus:three_tonalities}.\footnote{Supplement to \cite{Desrocquettesaccompagnementmelodiegregorienne1924}; English translation in \cite{PotironTreatiseAccompanimentGregorian1933} between pp. 110 and 111.}
Note that each horizontal row of letters signified pitches in each of the three \emph{tonalités}: the \emph{Do} \emph{tonalité} at the top, the \emph{Fa} in the middle and the \emph{Si}\kern 1pt\flat{} at the bottom.
The signifier of each one was printed as a slightly larger, emboldened uppercase letter, though these are not visually as distinct from their neighbours in the diagram as they might be.
Other pitches in uppercase denoted the tetrachord of finals in each \emph{tonalité}.
The square brackets and Roman numerals surmounting the tetrachords indicated the modal cadences Desrocquettes believed established a given \emph{tonalité}: as mentioned, protus cadences on `D' established the \emph{Fa} \emph{tonalité}, but protus cadences on `A' or `G' established the \emph{Do} or \emph{Si}\kern 1pt\flat{} \emph{tonalités} respectively.

Still, each \emph{tonalité} was comprised of more pitches than its tetrachord of finals, three adjunct pitches being shown in lowercase.
These were divided into two kinds.
To the left of each tetrachord were two so-called `continuous notes', which when added to the tetrachords converted them into hexachords similar to Guido's \emph{durum}, \emph{naturale} and \emph{molle} types.
To the right was a `supplementary note', which being a tone below its adjoining final was adapted to the formation of tetrardus cadences.
Considering the \emph{Do} \emph{tonalité}, then, tetrardus cadences on the note `G' were said to have recourse to the note `F' often enough for the latter note to be included as a `supplementary' note.
Owing to the mutual equivalence of the three \emph{tonalités}, the same was then said of the pitches `B'\kern 1pt\flat{} in the \emph{Fa} \emph{tonalité} and `E'\kern 1pt\flat{} in the \emph{Si}\kern 1pt\flat{} \emph{tonalité}.
But since `E'\kern 1pt\flat{} was prohibited outright by Desrocquettes's conception of modality (owing in no small part to Niedermeyer's influence), its appearance was thought to be little more than a theoretical quirk of `modal equivalence'.
Tetrardus cadences in the \emph{Si}\kern 1pt\flat{} \emph{tonalité} were therefore deemed impossible.

The `supplementary notes' were to be treated with much caution, for if they were to assert themselves with any deal of prominence, they could establish a different \emph{tonalité} altogether.
Likewise, should the characteristic semitone in a tetrachord of finals be heard, then the \emph{tonalité} associated with that tetrachord would be established in a process Desrocquettes called `modulation'.
Hence, the triangular glyph surrounding each `supplementary note' indicated that leftward motion was out of the question, even though the apex of the shape might suggest the opposite to be the case.
Directional arrows were drawn between characteristic semitones to demonstrate how one \emph{tonalité} could `modulate' to another, as, for instance, how the `E'--`F' semitone could establish the \emph{Fa} \emph{tonalité}, or how a `B'\kern 1pt\natural{}--`C' semitone could establish the \emph{Do} \emph{tonalité}.

\subsection{Permissible chords}
\label{hl:potiron_methode}%
Although the theory of modal equivalence was originally conceived as a tool for melodic analysis, before long it was applied to the accompaniment.
In spite of Desrocquettes's appointment as Solesmes's organist, there is insufficient evidence to conclude whether his training could have equipped him with the skills required to codify a harmonic method in this regard.
He was certainly on friendly terms with the organist René Lefebvre at Honfleur, but whether that organist was ever Desrocquettes's teacher is not altogether clear.
In 1920, Desrocquettes admitted that his own organ technique was not robust enough to provide music suitable for the offertory.
And when Mocquereau's eventual successor as \emph{maître de chœur} Joseph Gajard (1885--1972) indicated some years later that the organist Joseph Bonnet (1884--1944) was due to visit Solesmes, Desrocquettes recognised an opportunity to benefit from some informal tuition.\fnletter{Desrocquettes}{Gajard}{10 June 1920 and 26 December 1926}{\so{}}
Bonnet had been a pupil of Guilmant's at the Paris Conservatoire,\footcite[198]{OchseOrganistsOrganPlaying2000} and later became a Benedictine oblate.
\hlabel{int:guilmant}%
It was through Bonnet's insistence that Tournemire commenced \emph{L'Orgue mystique},\footcite[194]{ConnollyInfluencePlainchantFrench2013} and it was in a private meeting with Bonnet, Tournemire and Emmanuel several weeks before his death that Guilmant confessed to having turned against chant accompaniment and to agreeing with Gevaert's 1895 view that chant was not to be accompanied at all (see \cref{int:gevaert} above).\footcite[3]{EmmanuelTraiteaccompagnementmodal1913}

It fell not to Bonnet to apply `modal equivalence' to harmony, however, but to another organist, Henri Potiron (1882--1972).
Potiron had not always been on the best terms with Solesmian theorists, having railed against Mocquereauvian rhythm in 1912, dismissing the rhythmical signs as `useless and dangerous'(`inutiles et dangereux').\footcites[unpaginated `Avant-propos']{PotironMethodeharmonieapliquee1912}
That view was probably borne of encountering the chant at the Basilica of Sacré-C\oe{}ur, Montmartre, where Potiron had been appointed \emph{maître de chapelle} in 1911, succeeding Gabriel Mulet.\footcite[608--609]{BenoistSacreCoeurMontmartre18701992}
But during a visit to Quarr Abbey in April 1922, Potiron claimed to have simply misinterpreted Mocquereau's ideas, and performed a \emph{volte-face} in their favour as Bas had done.\fnletter{Potiron}{Mocquereau}{[24 April 1922]}{\so{}}
The early 1920s had been tumultuous time indeed for the Solesmes community, but its monks were permitted to return to France in 1922; Desrocquettes's stint at Saint-Pierre lasted only until 1925, however, when, on 1 September, he was ordered back to the Isle of Wight.\footnote{Document bearing the title `Foreigners resident at Quarr Abbey' dated 21 October 1931, \qaa{}~QAA-M-1391.}
From around 1927, Desrocquettes noted to his chagrin that he was no longer `à Quarr' but `de Quarr'.\fnletter{Desrocquettes}{Mocquereau}{27 January 1927}{\so{}}

Potiron contributed numerous articles to the \emph{Revue grégorienne} on divers subjects, and collaborated with Desrocquettes throughout the 1920s on a method of accompaniment that applied Mocquereauvian rhythm and `modal equivalence' to the chant repertory.
The method gained credibility from its promotion by Solesmes and Desclée prior to its being phased out in the 1930s for three reasons: the accompaniments became too dissonant for the some practitioners to accept; Desrocquettes's application of the system proved too sporadic for Potiron who struck out on his own to codify another approach; and Solesmes quietly moved away from Mocquereauvian rhythm following his death in 1930, thereby making accompaniments based upon it obsolete.

Potiron's first contribution to the \emph{Revue grégorienne} had necessitated Mocquereau to alert its readers to Potiron's change of heart:

\simplex{Dans sa \emph{Méthode d'harmonie appliquée à l'accompagnement du chant grégorien}, [Potiron] s'était séparé de nous sur la question du rôle de l'accent tonique latin dans le rythme et dans l'harmonie. Ce n'était qu'un malentendu, qu'une discussion amicale eut tôt fait de dissiper au cours d'un récent voyage à Quarr Abbey.}
  {\cite[Introduction to ][121]{Potironaccentmusicalmoderne1922}}
{In his \emph{Méthode d'harmonie appliquée à l'accompagnement du chant grégorien}, [Potiron] had separated himself from us on the question of the role of the Latin tonic accent in rhythm and harmony. This was only a misunderstanding which was quite soon dispelled during a friendly discussion on a recent trip to Quarr Abbey.}
\noindent
Joined to that statement was a list of personal credentials which Potiron had supplied Mocquereau in private correspondence.\fnletter{Potiron}{Mocquereau}{[May 1922?]}{\so{}}
The statement was no doubt meant to satisfy the \emph{Revue}'s readers that Potiron was one of their own.
And probably for the same reason, Desrocquettes claimed Potiron as a convert to Mocquereauvian rhythm.\footcite[155]{DesrocquettesCoursaccompagnementInstitut1924}
It is unlikely that Potiron could have been appointed as a teacher at the pro-Solesmes Parisian Institut grégorien (to which we shall turn below) had he not converted to Mocquereau's ideas.

One of the first public explanations of the Desrocquettes-Potiron harmonic method took place in New York at a chant summer school hosted by the American benefactor and children's pedagogue Justine Bayard Ward (1879--1975).
Under the aegis of her Pius X Institute of Liturgical Music, Ward and her staff tutored children in the fundamentals of music theory using a kind of Mocquereauvian method \emph{ad usum Delphini} devised by Ward herself.
She had previously hosted Mocquereau and Gajard in 1920 when both monks had provided classes to adult participants,\footcite[104--105]{HalavoyageNewYorkaisDom2019} though the reason for Gajard's presence was also to serve as a kind of chaperone to Mocquereau who could not manage alone on trips abroad on account of ailing health.
In 1922, that duty fell to Desrocquettes, who was also asked to provide a class on chant accompaniment.
Although Mocquereau briefly described the trip in the \emph{Revue grégorienne}, he did not describe Desrocquettes's class in much detail, other than to confirm that the `three harmonic \emph{tonalités}' of \emph{Do}, \emph{Fa} and \emph{Si}\kern 1pt\flat{} were among the topics discussed.\footcite[237]{MocquereauEtatsUnisAmeriquecours1922}
That statement evidently piqued the curiosity of the \emph{Revue}'s readers, and Desrocquettes was called upon to describe them in writing.

That `modal equivalence' should have been applied to harmony in the first place is not surprising, particularly when we take into account the belief that chant analysis had the potential to reveal an authentic and venerable method of accompaniment (see \cpageref{ln:wagner_reveal,ln:wagner_reveal_END} above).
Bas arguably succeeded in codifying a Solesmian method whereby Mocquereauvian rhythm determined the placement of chords; and the Desrocquettes-Potiron collaboration promised to extend the method to determine what notes should be used in such chords.
Desrocquettes's three \emph{tonalités} therefore offered the tantalising possibility of composing accompaniments that conformed not only to Mocquereauvian rhythm but also to a theory of \emph{tonalité} supposedly derived from the chant itself.
Thereby, the Solesmian accompaniments could reflect the chant in every possible way, as Desrocquettes noted:

\simplex{L'accompagnement doit être une traduction, une transposition, une projection, aussi objective, aussi fidèle et aussi simple que possible de la pure mélodie, dans l'ordre harmonique.}
  {\cite[170]{Desrocquettesaccompagnementmelodiegregorienne1923}}
{The accompaniment must be a transcription, transposition, and projection, made as objectively, faithfully and simply as possible of the pure melody, in the harmonic order.}
\noindent
Should the chant occupy the \emph{Fa} \emph{tonalité}, the chords could be derived from the same; and should the chant `modulate' to the \emph{Si}\kern 1pt\flat{} \emph{tonalité}, so could the harmony (pp.~171, 174).
%\footcite[171, 174]{Desrocquettesaccompagnementmelodiegregorienne1923}%this is same as above

\hlabel{ln:dd_tonal}%
But how could these \emph{tonalités} be distinguished from one another?
Niedermeyer's framework prohibiting notes foreign to the modal scale evidently inspired Desrocquettes and Potiron to prohibit notes foreign to a given \emph{tonalité}.
For example, only those chords made up of notes in the \emph{Fa} \emph{tonalité} were to be used to accompany that \emph{tonalité}.
A tautological maxim governed Desrocquettes's practice, however, when he insisted that the accompaniment should be tonal before it could be modal (`l'accompagnement grégorien, avant d'être modal, devra être \emph{tonal}').\footnote{\emph{Ibid}. 9, no.~6 (November--December 1924): 225; Also discussed in \cite[368]{Lessmannanachronismemusicalaccompagnement2019}.}
%\footcite[225]{Desrocquettesaccompagnementmelodiegregorienne1924a}
It might explain why, to arrive at chords permissible in each \emph{tonalité}, Desrocquettes arranged 5/3 chords above each note of ascending C major, F major and B\kern 1pt\flat{} major scales in a method not dissimilar to the \emph{règle d'octave}.
\Cref{mus:dd_fscale} shows the chords Desrocquettes believed applicable for the \emph{Fa}~\emph{tonalité},\footcite[130]{Desrocquettesaccompagnementmelodiegregorienne1924b} \cref{mus:dd_cscale} those for the \emph{Do} \emph{tonalité}, and \cref{mus:dd_bflatscale} those for the \emph{Si}\kern 1pt\flat{} \emph{tonalité}.\footnote{\emph{Ibid}. no.~6 (November--December 1924), pp.~221--3.}
\nocite{Desrocquettesaccompagnementmelodiegregorienne1924a}%[pp. 221, 223]
%same as two notes above...
Chords annotated with `a', `b', and `c' required special treatment because they were supposedly capable of `modulating' from one \emph{tonalité} to another.
As a safeguard, they were to be arranged as chord inversions in 6/3 position, or avoided altogether.
Chords annotated by asterisks contained `E'\kern 1pt\flat{} and were therefore inadmissible.

The chords and their mutual relationships have been plotted in \cref{tab:desrocquettes_chords}.
Those indicated \bullet{} were said to be `chords of repose', and were permitted without the harmoniser's needing to observe any special rules.
Chords marked by other glyphs were said to be `chords of movement'.
Those indicated \times{} required careful management since they contained `modulating notes', as observed in \cref{mus:three_tonalities}.
Likewise, diminished chords, marked \circ{}, were only to be used in 6/3 position.
Those indicated \diamond{} were supposedly capable of `modulating' too, because they reportedly suggested the harmony proper to certain cadences characteristic of a different \emph{tonalité}.

The mechanism by which such `modulations' were meant to take place was not explicated in any detail, though several inferences can be made.
By comparing \cref{tab:desrocquettes_chords} to \cref{mus:three_tonalities}, the bracketed cadences in the latter suggest that Desrocquettes believed, for instance, that A minor 5/3 chords were characteristic of protus cadences in the \emph{Do} \emph{tonalité}.
Should these have been used in the course of an accompaniment in the \emph{Fa} \emph{tonalité}, they would erroneously suggest that the accompaniment had modulated to the \emph{Do} \emph{tonalité}.

The logic behind `modal equivalence' does not stand up to critical examination, particularly when we consider that E minor chords were ruled out by Desrocquettes on account of their supposedly being equivalent to A minor chords.
But E minor chords only occurred in the \emph{Do} \emph{tonalité}, thereby making any reservations Desrocquettes voiced about the potential for `modulation' inexplicable.
Moreover, D minor chords were supposedly permitted as readily in the \emph{Do} \emph{tonalité} as they were in the \emph{Fa}, even though the characteristic protus cadence should have limited their use to the latter.
When Potiron took up the matter of `modal equivalence' for himself, he was more keen to explain away illogicalities in the system as quirks instead of tackling them for what they were---contradictions.\footcite[116]{PotironTreatiseAccompanimentGregorian1933}

\subsection{Systematising the three \emph{tonalités}}
\label{hl:potiron_threetonalities}%
Potiron's \emph{volte-face} stood him in good favour with the Solesmian authorities, who approved of his appointment in 1923 at the Institut Grégorien in Paris.\footcite[194]{BrunInstitutgregorienParis1923}
He taught modality and accompaniment and inculcated the theory of `modal equivalence' into a new generation of Catholic organists.
At the same time, he began codifying the theory himself, publishing on the three \emph{tonalités} in the Orléans-based \covid{}\emph{L'Orgue et les Organistes}, tackling the issues of chord placement and modal harmony in the July, August and September issues.\footcite[p.~5 n.~1]{Potirontheorieharmoniquetrois1926}
Following his first year at the Institut, Potiron collated his thoughts on accompaniment in a manual of his own.
It is hardly surprising to note that Desrocquettes placed a gushing advertisement for it in the \emph{Revue grégorienne}, saying it would be available from October 1924.
Desrocquettes also quoted from a panegyric letter of approval by Vierne,\footcite[pp. 154, 156]{DesrocquettesCoursaccompagnementInstitut1924} whose assessment of Potiron's manual was printed among its front matter:

\duplex{Le traité d'accompagnement du chant grégorien de M. Potiron présente un intérêt tout particulier: c'est la première fois qu'un musicien professionnel traite la question et du premier coup il le fait magistralement. Les chapitres traitant de la rythmique, de la modalité et de l'harmonisation applicable à cette modalité sont à la fois d'un homme tout à fait versé dans la matière spéciale du chant grégorien et d'un artiste pour qui la musique n'a pas de secrets.
Nous ne pouvions moins attendre du savant maître de chapelle du Sacré-C\oe{}ur de Montmartre après l'audition de la belle messe à deux orgues et ch\oe{}ur donnée le jour de Pâques et dont il est l'auteur. Les organistes soucieux de logique auront là un ouvrage capital et dont la documentation serrée les fera utilement réfléchir.}
  {\cite[p.~ix]{PotironCoursaccompagnementchant1925}}
{Mr Potiron's treatise on the accompaniment of plainsong is of quite peculiar interest; it is the first time that a professional musician deals with the subject and he has done it in masterly fashion. The chapters dealing with rhythm, with modality, and with harmonisation suitable to this modality are the work of a man who is not only a specialist in plainchant but also an artist for whom music has no secrets.
We could hardly expect anything less from the able \emph{maître de chapelle} of the Basilica of the Sacred Heart, Montmartre, after having heard his beautiful Mass for two organs and choir, performed on Easter Sunday. Organists who aim at consistency will have before them a first-rate work and one whose closely packed material will encourage them to think to good purpose.}
  {\cite[Adapted from][p.~vii]{PotironTreatiseAccompanimentGregorian1933}}
\noindent
Bonnet also supplied approving words, as did Mocquereau, who overcame an initial \mbox{reluctance} to approve of the manual:
\pagebreak{}
\hlabel{ln:desrocquettes_playing}%

\duplex{Je m'étais bien promis de ne plus donner de lettre d'approbation aux auteurs des Méthodes d'accompagnement de Chant grégorien, et voici que, malgré cet engagement, je sors de ma réserve et cède à vos affectueuses sollicitations. C’est que, tout bien considéré, votre cas est très spécial, et je regarde comme un devoir de reconnaissance de vous être agréable.}
  {\letter{Mocquereau}{Potiron}{2 July 1924}{published in \cite[xiii]{PotironCoursaccompagnementchant1925}}}
{I had firmly resolved never again to write letters of recommendation for authors of methods on plainchant accompaniment, and behold, in spite of my resolution, here I am, abandoning reserve and giving way to your affectionate solicitation. If I do so, it is because, every thing considered, yours is a very special case, and my gratitude compels me to yield to your request.}
  {Adapted from \cite[vi]{PotironTreatiseAccompanimentGregorian1933}}
\hlabel{ln:desrocquettes_playing_END}%
\noindent
One surmises that Mocquereau's reluctance stemmed from the ignominy of having had two previous harmonisers---Delpech and Bas---speak out against his rhythmic theories.
But Mocquereau's approval was probably also a requirement for any theory of accompaniment to be deemed official.
Mocquereau's report---that the accompaniments produced by Potiron's manual were equally as `soft and discreet' (`les accompagnements doux et discrets') as those Desrocquettes played each day at Solesmes---was therefore probably as positive a testimonial as Potiron could have expected to receive.\footnote{\cite{PotironCoursaccompagnementchant1925}, 1st ed., p.~xiii.}

Potiron's manual instituted two primary amendments to Desrocquettes's theory.
It acknowledged the conflict of definitions between musicians and monastic chant explorers when it came to `tonalité' and `modulation', terms Potiron rejected in favour of `groupe' and `changement de groupe' respectively.\footnote{\cite{PotironCoursaccompagnementchant1925}, 1st ed., p.~73 n.~1.}
It also adopted new terminology to describe the groups: out went the \emph{tonalités} of \emph{Do}, \emph{Fa} and \emph{Si}\kern 1pt\flat{} and in their place came the Roman numerals I, II and III.
While they were ordinarily represented as such, on some occasions Potiron preferred Arabic numerals instead, both cardinal and ordinal identifiers being used in text.
Potiron was not the last theorist to consider new terminology to describe the three \emph{tonalités}, and for the avoidance of doubt the terms coined by various other authors have been collated in \cref{tab:three_tonalities}.

Potiron's groups not only distanced the theory of `modal equivalence' from major-minor nomenclature but also solved a separate problem, namely how to refer to each \emph{tonalité} when the chant was transposed.
As we have seen, the epithets \emph{Do}, \emph{Fa} and \emph{Si}\kern 1pt\flat{} were imagined in conjunction with the untransposed chant as Desrocquettes had encountered it in preparing the \emph{Antiphonale}.
But those epithets did not suit transposed accompaniments.
In some cases, Desrocquettes used \emph{Do}, \emph{Fa} and \emph{Si}\kern 1pt\flat{} when the chant was transposed, doubtless confusing his readers; and in other cases, Desrocquettes transposed the epithets to suit the transposition of the accompaniment, with analyses of chants up a tone referring to the \emph{Re} \emph{tonalité}, and so forth.
It was a confusing state of affairs indeed, to which Potiron's numerals brought some much needed clarity.
Group I was always just that, no matter the transposition.
But Desrocquettes neglected to adopt Potiron's terms a year after their appearance, mixing them with his own fixed solfège system in descriptions of the `groupe modal de \emph{re}'.\footcite[144]{Desrocquettesexamenfinannee1926}

\subsection{Pedagogy at the Parisian Institut grégorien}
\label{hl:potiron_cours}%
It was not long after the appearance of Potiron's accompaniment manual, \emph{Cours d'accompagnement du chant grégorien}, that certain complaints arose regarding the lack of music examples.
One can hardly fault the complainants, since the dense technical matter (to which Vierne had made an oblique reference) made itself almost impervious to self-study.
Potiron accordingly revised the manual in an expanded edition that was published in 1927.\footcite[112]{PotironTreatiseAccompanimentGregorian1933}
It was reportedly translated into Dutch by one Vuillings de Hoelen (though no such copy was viewed by the present author),\fnletter{Potiron}{Gajard}{10 April 1928}{\so{}} and into English by one Ruth C.\ Gabain, who spent some time at Quarr consulting Desrocquettes on how best to translate its terminology.\fnletter{Desrocquettes}{Gajard}{undated card in the Desrocquettes archives}{\so{}}
One reviewer contended that it was the first book to deal with the accompaniment of Latin plainchant in the English language, since the Anglican scholarship which had appeared up to that point had largely been confined to the accompaniment of vernacular plainsong.\footcite[534]{L.ReviewAccompanimentGregorian1934}

Before the second edition went on sale, Potiron instigated two further measures to engage his readers.
The first was a correspondence course whereby readers of the \emph{Revue grégorienne} could send their accompaniments for correction by return.\footcite[120]{DesrocquettesCoursaccompagnementpar1925}
Notices advertising the course soon disappeared, however, the idea presumably having been for some reason scrapped.
Since the notices made no mention of fees, it could be that the venture was simply not worth Potiron's while, since he had many other demands on his time, including his teaching at the Institut grégorien and his playing at Sacré-Cœur.

The second was to collaborate with Desrocquettes on a practical supplement to his theoretical manual, an avenue previously followed by Lhoumeau in 1892 (see \cpageref{ln:lhoumeau_practical_supplement,ln:lhoumeau_practical_supplement_END} above).
The Potiron-Desrocquettes publication was to contain twenty-nine harmonisations along with an assortment of commentaries describing the method.
\emph{Vingt-neuf pièces grégoriennes} was prepared for Desclée in 1925, but for whatever reason it did not appear as one collated publication until Hérelle published it in 1929.
In the meantime, some of its accompaniments and corresponding `analyses détaillées' appeared piecemeal as part of the \emph{Revue grégorienne} in the so-called `Bulletin de vulgarisation grégorienne'.
As far as their \emph{mise-en-page} was concerned, the chant and accompaniment were parsed into Potiron's modal groups, as indicated by the Roman numerals placed beneath.
In some cases the accompaniment and related analysis were both attributed to Desrocquettes,\footcite[p.~224 n.~1, p.~227]{DesrocquettesAccompagnementAgnusMesse1925} but in others the accompaniments were attributed to both him and Potiron jointly.\footcite[182]{DesrocquettesHarmonisationIntroitGaudeamus1927}
As for the Hérelle publication, its \emph{mise-en-page} followed the precedent set by Solesmes's \ldo{} whereby the chant was set in quadratic notation above a transcription of the chant into modern notation and the accompanying parts.

Significant printing errors made their way into the Hérelle publication, which detracted from its pedagogical potential.
In some cases, horizontal \emph{episemata} were omitted from the transcription; in others, the pitches in the transcription did not match those in the quadratic notation, as one sharp-eyed reviewer pointed out.\footcite[227]{G.ReviewVingtneufpieces1930}
What that reviewer did not point out, however, was the plethora of inconsistencies in the designation of Potiron's modal groups.
In one case, an annotation that should have indicated a change of group was omitted entirely, meaning that a fundamental facet of the accompaniment referred to in the analysis was not indicated in the score.
The error was corrected in Desrocquettes's personal copy of the \emph{Vingt-neuf pièces} consulted by the present author at Quarr Abbey (\cref{mus:potirondesrocquettes_29pieces}), but even Desrocquettes himself seemed unsure as to where exactly group I took effect.\footcite[part~I p.~34, part~II pp.~67--73]{DesrocquettesVingtneufpiecesgregoriennes1929}
Two locations were proposed in pencil: mid-way through the second system, evidently on account of \pitch{4}\kern 1pt\natural{} in the chant; and at the end of the first system, though there is no clear rationale for this suggestion.

The complaints about the lack of music examples in Potiron's manual hardly concerned students enrolled at the Institut grégorien who no doubt benefited from live demonstrations.
Students in Potiron's accompaniment class were recommended to parse the chant into its constituent groups prior to harmonising it, and Potiron later demonstrated how the Alleluia for Ascension Sunday could be parsed in that way.\footcite[111]{PotironTreatiseAccompanimentGregorian1933}
Given that \cref{mus:potiron_parse} is an untransposed deuterus chant, we take `E' as being equivalent to \pitch{1}.
We may note how the characteristic semitone of group II, \pitch{1} \rightarrow{} \pitch{2}, occurs some three to four notes into the chant.
From there, Potiron evidently worked backwards in the conviction that the same group must have been in effect from the outset.
Arguably, that method was rather a dubious one, but it did not seem to trouble organists such as Vierne and Bonnet, assuming they delved deeply enough into Potiron's \emph{Cours} to encounter it for themselves.
The change to group III at \pitch{4} \rightarrow{} \pitch{5}\kern 1pt\flat{} was quite consistent with a change of group being effected by that group's characteristic semitone.
And the change back to group II in the next system was doubtless owing to a characteristic deuterus cadence on \pitch{1}, this being consistent with the tetrachord of finals in the middle row of pitches shown in \cref{mus:three_tonalities}, as discussed above.

Among the first cohort of students to encounter Potiron's method at the Institut was the Canadian religious Placide Gagnon (1880--1950).
The chief proponent of chant accompaniment in the Canadian Catholic Church had been the organist of Quebec cathedral Ernest Gagnon, who had placed the chant on top of a homorhythmic four-part keyboard texture which admitted certain quantities of sharping.
One book of accompaniments by Ernest stated that organists must follow exactly what the notation sets out, for Canadian organists were reportedly notorious for detracting from liturgical piety by using their own populist harmonisations.\footcite[unpaginated `Préface', p.~142 and \emph{passim}]{GagnonAccompagnementorguechants1903}
When the diocese of Quebec adopted Solesmes's rhythmical editions in 1915,\footcite[105--106]{N.restaurationchantgregorien1927} however, demand quickly grew for Solesmian accompaniments to match them, and Ernest's homorhythmic style fell out of fashion.
Placide had first encountered the Desrocquettes-Potiron method at the 1922 Summer School in New York, when he attended Desrocquettes's demonstration, whereafter he was granted leave to undertake further study with Potiron in Paris.
Following one year at the Institut grégorien, Placide returned to Canada to compose accompaniments of his own, as we shall see.\fnletter{Gagnon}{Gajard}{23 January 1931}{\so{}}

Potiron set the Alleluia for Ascension Sunday as the end-of-year examination in Placide's year.
One of the best solutions was published in the \emph{Revue grégorienne} some months later, and is here reproduced in \cref{mus:examen1925}.
Particularly with respect to cadences, this harmonisation is conspicuously lacking in deuterus characteristics: the harmoniser was apparently more intent on following the chords applicable to Potiron's groups than on indulging in \pitch{2} 6/3 \rightarrow{} \pitch{1} 5/3 progressions.
Certain corners of the harmonisation are unabashedly dissonant too, particularly where the harmoniser introduced passing notes and delayed the resolutions of suspensions.
Dissonances in the alto part at the end of the first system remain unresolved, and are redolent of the more relaxed approach to dissonance that Springer and his cohort had been popularising a decade earlier.
Desrocquettes supplied an analysis of the student's work, evaluating the chord marked by (a) as pushing at the boundary of acceptable dissonance.
He also complained that \pitch{5}\kern 1pt\flat{}---indicative of group III---was used too frequently when harmonising group II.\footcite[137]{Desrocquettesexamenfinannee1925}

The Canadian composer Eugène Lapierre (1900--1970) won first prize in the next year's examination with an accompaniment that was also printed in the \emph{Revue grégorienne}.
Lapierre resided in Paris from 1924 to 1928 to study composition with Vincent d'Indy and organ with Marcel Dupré (1886--1971).
That he should also have enrolled at the Institut grégorien speaks to its prestige in Catholic church-music circles.
Desrocquettes also commentated on Lapierre's accompaniment to say that although pitches not belonging to group II were used to accompany that group, their use above a pedal note made them admissible.\footcite[p.~142, n.~1 and p. 144]{Desrocquettesexamenfinannee1926}

Desrocquettes inadvertently fell foul of his own criticism, for in 1924 he had used pitches in group III to harmonise group II.\footcite[71]{DesrocquettesIntroitResurrexiAccompagnement1924}
Perhaps, owing to their being passing notes, he might have deemed them admissible.
But surely the same cannot be said of `C'\kern 1pt\sharp{} in the harmonisations of group II in \cref{mus:desrocquettes_agnus}.
Indeed, Desrocquettes described the harmonisation as being very daring (`bien osé'), and justified the appearance of `C'\kern 1pt\sharp{} by that sonority's having simply remained present in his ear from an earlier phrase (`m'est resté dans l'oreille').\footcite[pp. 225, 227]{DesrocquettesAccompagnementAgnusMesse1925}
It is an example of Desrocquettes's inconsistent, \emph{laissez-faire} attitude to his application of `modal equivalence' in the accompaniment.
His status as a monk of Solesmes possibly explains why such inconsistencies did not give rise to much doubt in his method, at least initially.

The Desrocquettes-Potiron theory came under threat when
Auguste Le Guennant (1881--1972) was appointed as a teacher at the Institut grégorien in 1925.\footnote{\covid{}\cite[297]{InstitutcatholiqueParis1975}.}
Rather than voice any opinions contrary to Mocquereauvian rhythm or to the Desrocquettes-Potiron theory of harmonisation, however, Le Guennant simply chose to apply his own method.
By 1927, Desrocquettes was concerned that Le Guennant's stance threatened the standing not only of Potiron but also of Solesmes itself.\fnletter{Desrocquettes}{Mocquereau}{23 May 1927}{\so{}}
Reiterating the concern two years later, Potiron nonetheless concluded that Le Guennant's non-adherence to Solesmes was borne not of malicious intent but of a lack of curiosity.\fnletter{Potiron}{Gajard}{23 May 1929}{\so{}}
Le Guennant was later credited with tutoring the French composer Maurice Duruflé (1902--86) in certain particulars of Solesmian rhythm which he went on to apply in his 1947 \emph{Requiem} op.~9.\footcite[126]{FrazierMauriceDurufleMan2007}
While it is beyond the scope of the present study to evaluate Duruflé's understanding of Solesmian rhythm, perhaps Le Guennant had tutored him in a theory of rhythm which had become out of date, hence the scorn from Solesmes itself.

In 1910, Le Guennant had produced a book of accompaniments which he claimed were based on Solesmes's rhythmical editions in modern notation.\footcite[p.~b]{LeGuennantVademecumparoissial1910}
It was an edition of easy accompaniments intended for parish organists, an evolving sub-genre which will be discussed here by way of a brief aside.
Le Guennant arranged his accompaniments in three parts, two accompanying the chant which was placed on top.
Further notable characteristics include tied notes, annotated fingerings and a transposition scheme that retained the pitch `A' as the dominant of all modes.
Other organists shared Le Guennant's desire for simple accompaniments, one such being a certain correspondant in the Toulouse-based journal \emph{La Musique sacrée} who remarked that it was common for parish accompaniments to be improvised.
Peter Wagner's accompaniments, the correspondent opined, were ill suited as examples of best practice because the disposition of parts was too difficult for a parish organist to improvise on their own.\footcite[47]{T.proposaccompagnementplainchant1911}

Louis Jacquemin, a teacher at the Petit Seminare de Saint Charles de Chauny at Aisne produced a set of easy accompaniments for the office that were largely in three parts.\footcites[14]{JacqueminAccompagnementsnouveauxtres1914}[Reproduced in][145]{Parisotaccompagnementmodalchant1914}
Others similarly designated for parish use were published in 1937 by an anonymous composer who reportedly received authorisation from Desclée's Tournai branch to reproduce the Solesmian rhythmic signs.\footcite[3]{PratiqueKyrialeparoissialaccompagnement1937}
It indicates that Solesmes had not relaxed its control over the manner in which its type was available to purchase, a thorny issue for many publishers as we have seen (\cpageref{ln:solesmes_type}).
Carlo Rossini (1890--1975), the Italian-born priest and choirmaster of Pittsburgh's Saint Paul's Cathedral, bore simplicity in mind when he arranged the text of the propers to be recited to tones.
For those, he provided a rudimentary accompaniment in a book whose preface acknowledged that church choirs were generally made up of volunteer singers who rehearsed once a week.
A simplified formula was therefore a requirement for the text of the Proper to be sung at High Mass.
Some melismatic chants such as alleluias were included too, but these required a separate accompaniment (\cref{mus:rossini_proper_56}).
They, however, were decidedly the exception rather than the rule in a publication that was geared to making matters as simple as possible for choir and organist alike.\footcite[unpaginated front matter, p.~56]{RossiniProperMassEntire1957}
A similar sort of publication was produced around the same time by the reverends Andrew Green (1865--1950) and Herman Joseph Koch (1892--1984), who also included a primer on the pronunciation of Church Latin and a calendar of feast days.\footnote{\cite[16--17]{GreenCompleteProperMass1956}; Green predeceased the appearance of the revised edition by approximately six years.}

\section{Modern modality}
\subsection{André Caplet and his influence on Solesmes}
While Le Guennant was deemed a threat to Mocquereauvian rhythm by Solesmes, another threat was in the form of dissenting voices from within the Benedictine circle itself.
The monk Jules Jeannin argued that the Desrocquettes-Potiron theory was not modal, but tonal.\footcite[11--12]{Jeanninimportancetiercedans1926}
His argument came to the attention of Desrocquettes, who signalled that Potiron would dispose of any concerns in his revised accompaniment book.\fnletter{Desrocquettes}{Mocquereau}{[1926?]}{\so{}}
The preface to that book downplayed Jeannin's argument by noting how Desrocquettes's major-minor terminology had little bearing on the theory as a whole.\footcite[pp.~xi, xii n.~1]{PotironTreatiseAccompanimentGregorian1933}

Desrocquettes's interest in major-minor harmony was probably given fillip following several meetings with well known composers.
Gustav Holst (1874--1934) stayed at Quarr from 12 to 17 August 1920, following the inaugural performance of \emph{The Hymn of Jesus}.
A copy of the score was reportedly inscribed to Desrocquettes,\footnote{\qaa{} QAA-M-1408} but cannot be located at Quarr today.\footnote{Fr Brian Kelly, Procurator of Quarr, to the present author, 20 May 2020.}
According to the composer's preface:

\single{As the free rhythm of plainsong cannot be expressed in modern notation, the Trombone and English Horn players are to study the manner in which this melody is sung by experienced singers.}
  {\cite[pp.~3, 8--9]{HolstHymnJesus1919}}
\noindent
There is little doubt that Holst was inspired by the Solesmes method as regards singing, for quasi-aleatoric figures in the orchestral writing set a freely chanted `Vexilla regis' in relief (\cref{mus:holst_sop}).
That is followed by a 7/5/4/2 chord in the string parts that accompany tenors and baritones (\cref{mus:holst_ten}), liberating them from strict rhythm.
Intriguingly, the chord just mentioned is of just the kind Desrocquettes was then introducing into his accompaniments, but the direction of influence (if any was specifically exerted in this respect) remains to be determined.
Whatever the case may be, the same sonority has outlived Holst, having been incorporated into an improvisation by Olivier Latry (b.1962) on `Salve Regina' that includes snippets of accompanied chant (quoted in \cref{mus:latry}).\footcite[27]{LatrySalveReginapour2010}

Memories of Desrocquettes's accompaniment at the time of Holst's arrival led some of the monks in later years to align his harmony with French Impressionists.
Claude Debussy (1862--1918) and his orchestrator André Caplet (1878--1925) both play parts in the following account from the 1980s:

\simplex{Solesmes a connu aussi cette vogue de l'accompagnement orne, surchargé. Jusqu'à la guerre de 1914, le Père Desrocquettes accompagnait le grégorien ainsi, avec des harmonies debussystes (Debussy est venue plusieurs fois à Solesmes) ou d'André Caplet, etc\ldots{}}
  {\cite[399]{Pinguetecolesmusiquedivine1987}, seemingly quoting either Eugène Cardine or Jean Claire}
{Solesmes also experienced this vogue of ornate, overloaded accompaniment. Until WWI, Fr Desrocquettes accompanied chant in this way with Debussian harmonies (Debussy came to Solesmes several times) or those of André Caplet, etc\ldots{}}
\noindent
It was not to be the last time Debussian harmonies were to appear in the narrative on chant accompaniment, for they crop up again in the following recollection by Willi Apel of an encounter with a group of seminarians:
\pagebreak{}

\single{When I mentioned my interest in Gregorian chant, one of them said, his face radiant with delight, `Oh, Gregorian chant is so wonderful in our church; we have an organist who makes it sound like Debussy.' I know that it does not always sound like that. In another church it might sound more like Vaughan Williams, and elsewhere like parallel organum. Invariably it will sound like `something' other than what it really is and what it should be. Moreover, the very variety of possibilities inherent in this practice is bound to weaken the catholicity of one of the most precious possessions of the Catholic Church.}
  {\cite[p.xii]{ApelGregorianChant1958}}
\noindent
Desrocquettes had indeed corresponded with Caplet and we shall turn to their exchange in due course, but first we must dispose of the myth that Debussy had ever visited Solesmes.
It was first placed on the record by Becket Gibbs, whose account claimed that Debussy visited Solesmes around 1893 or 1894, when he had heard Solesmes chanting.\footnote{\covid{}\cite[181--87]{dAlmendraModesgregoriensdans1950}.}
Edward Lockspeiser's painstaking researches in the Debussy archives yielded no evidence whatsoever that any such visit had ever taken place.\footcite[142]{LockspeiserNewLiteratureDebussy1959}
He charitably suggested that the name `Debussy' had been confused with a certain curé of Saint-Gervais by the name of `De Bussy'.\footcite[p.~171 n.~1]{LockspeiserDebussyHisLife1978}
While a more recent Francophone study has cast doubt on Lockspeiser's suggestion, it has confirmed nonetheless that no record of Debussy's having visited Solesmes is to be found in that monastery's archives.\footcite[10--19]{HalaSolesmesmusiciensannees2020}
The myth continued to abound in the Anglophone literature until recently, due in no small part to scholars who report the original account without conveying any of the doubts surrounding it.
Katherine Bergeron made one such report, for instance, which led Stephen Schloesser to suggest Debussy's visit was more than the fiction it had probably always been.\footcites[p.~168 n.~54]{BergeronDecadentEnchantmentsRevival1998}[See, for instance,][44 n.~53]{SchloesserVisionsAmenEarly2014}

We may in contrast be certain that Desrocquettes met Caplet, who visited Solesmes in 1924 and maintained a brief correspondence with Desrocquettes prior to his death.
Caplet was interested in understanding the rationale for placing chords on the `levé', and on 20 July 1924 Desrocquettes arranged an introduction to Potiron so the matter could be discussed in person.
The arrangement likely suited Caplet because Potiron was also then resident in Paris.\footnote{\letter{Desrocquettes}{André Caplet}{20 July 1924}{\bnf{} NLA-269 (240)}; Reproduced in \cite[104--105]{HalaSolesmesmusiciensannees2020}.}
Two days after Desrocquettes's introduction, Potiron invited Caplet to Sacré-Cœur, providing a list of the service times when he would be presiding at the organ.\fnletter{Potiron}{Caplet}{22 July 1924}{\bnf{} NLA-269 (662)}
It seems probable that Caplet intended to attend one of these services, for his autograph MS of \emph{Les prières} (in a version transcribed for the organ) bears the very same service times that Potiron had conveyed by letter.
Chant accompaniment was therefore not the only item to be discussed, and it seems that Caplet also sought Potiron's advice on organ registration.
Some stop names written into the MS appear to be in Potiron's hand, this being suggested by certain similarities between the foot serifs in the letter `P': those in `Pos.' (for `Positif') and `p' for `pianissimo' match the uppercase `P' in Potiron's signature, to name two examples.\footnote{See \bnf{} MS-20106, p.~2 and \emph{passim}.}%https://gallica.bnf.fr/ark:/12148/btv1b10315924j/f5.image

Caplet continued corresponding with Desrocquettes on the matter of chant accompaniment and appears to have offered several harmonic suggestions, including one snippet applicable to a harmonisation of Credo VI Desrocquettes and Potiron had just published in the September--October 1924 issue of the \emph{Revue grégorienne}.
The original cadence at `et homo factus est' was noted in the accompanying commentary to coincide with an alighting place in Mocquereau's `grand rythme'---signalling it in the accompaniment therefore required a more elaborate harmonisation.
Potiron also noted that the phrase's repetition provided further justification for a richer harmony at this point (`une harmonie plus chargée'), comprising more suspensions and a conjunct bass line.\footcite[pp. 189, 194]{GajardCredoVI1925}

Their solution (quoted in \cref{mus:credovi_revue}) was seemingly not rich enough for Caplet, however, for when Desrocquettes and Potiron came to publish the same creed in their 1929 \mbox{accompanied} Kyrial, they replaced the cadence in question with one of Caplet's.
The superscript numeral in \cref{mus:credovi_kyriale} draws the player's attention to the following footnote:
\pagebreak{}

\simplex{Cette formule finale (depuis \emph{ex Maria Virgine}, avec celles qu'elle a inspirées), nous la devons à André \textsc{Caplet}, auteur regretté du \emph{Miroir de Jésus}.}
  {\cite[86]{DesrocquettesAccompagnementKyrialeVatican1929}}
{We owe this cadential formula (from \emph{ex Maria Virgine}, and others which it inspired) to André \textsc{Caplet}, the late composer of \emph{Miroir de Jésus}.}
\noindent
It therefore seems unlikely that Caplet was responsible for the set of consecutive fifths at `Sancto', though his cadence struck Desrocquettes as being `perfectly within the Gregorian atmosphere' (`parfaitement dans l'atmosphère grégorienne'), and led Desrocquettes to voice a wish to establish a more modern framework for modality that was capable of capturing the same sort of conjunct, dissonant harmonisation that Caplet had demonstrated:

\simplex{J'aimerais terriblement faire des accompagnements, qui sans sortir des règles rythmiques et modales, oseraient tout dans le sense moderne de l'écriture.}
  {\letter{Desrocquettes}{Caplet}{9 February 1925}{\bnf{} NLA-269 (242) and digitised at \bnf{} IFN-53033966}}
{I would really like to make accompaniments which, without breaking rhythmic and modal rules, would brave everything in the modern sense of composition.}
\noindent
What exactly those rules amounted to is not clear.
Desrocquettes provided Caplet with his back catalogue of articles on `modal equivalence', adding the caveat that `groupe' should replace any instance of `tonalité', and `changement de groupe' any instance of `modulation'.
In February 1925, Desrocquettes invited Caplet to contribute some articles of his own to the \emph{Revue grégorienne}, but Caplet died only two months later.

\subsection{Desrocquettes's application of modern harmony}
\label{hl:bragers}%
Not long after Caplet's death, Desrocquettes began writing harmonisations of the psalm tones.
He completed these sometime in 1926 and added a preface---dated 11~November---stating them to be aimed at young organists.
It was probably for their benefit that Desrocquettes annotated certain cadences with the letters `S' and `D' to indicate how the alignment of certain chords changed depending on whether the word happened to be a spondee or a dactyl.\footcite[pp.~3, 18 and \emph{passim}]{Desrocquettesaccompagnementpsaumes1928}
Less than a month after writing his preface, Desrocquettes noted to Mocquereau that his style of accompaniment was reflective of 1926 but that it was in the process of evolving.\fnletter{Desrocquettes}{Mocquereau}{1 December 1926}{\so{}}
The harmonisations did not see the light of day until 1928, when Desrocquettes recorded his signature and the date 5~September~1928 on a copy currently held at Quarr Abbey.
Neither marginalia nor revisions were marked on either the preface or accompaniments,\footnote{\qaa{} QAA-Mu-59.} though this by no means implies that Desrocquettes had not departed from previously held principles.

Nevertheless, psalm tone harmonisations were published alongside a brief accompaniment manual in the form of a pamphlet, which outlined a thorough set of rules codifying how Mocquereauvian rhythm could be applied to the accompaniment.
The ideas had first appeared as a series of articles in the \emph{Revue grégorienne}, but when Desclée came to collate them Desrocquettes voiced his dismay at the thinness of the paper, requesting via Dom Le Floch that the publisher make amends.\fnletter{Desrocquettes}{Le Floch}{4 February [1928]}{\so{}}
Those supplications apparently fell on deaf ears, however, for the leaves of the copy owned by the present author are wafer thin indeed.
Desrocquettes proposed that chant rhythm not only dictated the placement of chords but also their vertical make-up, and recommended that dissonances and their resolutions could bring unity to neumes in particular.\footcite[5, 30--32]{Desrocquettesaccompagnementrythmiqueapres1928}
By striking a dissonance on the first note of a neume and by delaying its resolution to the last note, Desrocquettes re-established the procedure which Lhoumeau had described over three decades earlier (see \cpageref{ln:lhoumeau_feminine,ln:lhoumeau_feminine_END} above).

Contrary to Bas's desire to reduce the frequency of chord changes, Desrocquettes's accompaniments generally contained a greater frequency in the belief that each \emph{ictus} had to be marked by a change in at least one of the parts.
Certain notes of those quoted in \cref{mus:credovi_kyriale} were pointed to demonstrate where certain \emph{ictuses} fell, namely at the second syllable of `Maria' and at the first and last syllables of `etiam'.
Lapierre adopted a similar pointing system in his own accompanied Kyrial, published in 1949 (\cref{mus:lapierre_punctuation}).
In contrast to the Desrocquettes-Potiron Kyrial, however, Lapierre placed what he termed `The Dot' beneath the notehead.
Even though a cautionary note in his preface attempted to clarify that the player should not confuse such dots with staccato marks,\footcite[unpaginated preface and p.~8]{LapierreSimplifiedModalAccompaniment1946} a sight-reader could just as well have mistaken some dots of addition for rhythmic pointing (as, for example, the dot in the alto part at `Dei').

The connection between Desrocquettes's harmonic approach and Mocquereau's rhythm led Ward to encourage Desrocquettes to compose an accompanied Kyrial.
She wished to anticipate its final publication by bringing out several masses in America first, but the process was not straightforward because Desrocquettes needed first to send his compositions to Potiron for correction.
Thereafter, they needed to be recopied in preparation for engraving.
The process, as Desrocquettes noted to Mocquereau, was going to take some time.\fnletter{Desrocquettes}{Mocquereau}{[1926?]}{\so{}}
Hence it took until 1927 for Desrocquettes to send any of his accompanied masses to Ward.
By 27 January he had sent off masses I--IX,\fnletter{Desrocquettes}{Mocquereau}{27 January 1927}{\so{}} though he admitted a few days later that other engagements were slowing down his progress.

Among such engagements was Desrocquettes's tutelage, in the fundamentals of chant harmonisation, of one of Ward's lecturers.\fnletter{Desrocquettes}{Mocquereau}{30 January 1927}{\so}
During the Autumn of 1926, the Belgian musician Achille Pierre Bragers (1887--1955) had stayed at Quarr to study accompaniment with Desrocquettes, and had thereby diverted the latter's attention from his own projects.
Bragers had already accrued notable credentials, having graduated from the Royal Conservatory of Brussels in 1905 and from the Lemmens Institute (or the École Interdiocésaine de Musique Religieuse de Malines, as it was then known) in either 1907 or 1910, sources conflict.\footnote{One scholar contends that Bragers graduated from the Lemmens Institute in 1910, see \cite[p.~19 n.~10]{BrancaleoneGoldenYearsAmerican2019} and \cite[p.~19 n.~35]{BrancaleoneGeorgiaStevensinstitutionalization2012}; Whereas Bragers's obituary in \emph{The~Caecilia} asserts that he graduated in 1907. See \cite[169]{AchilleBragers18871955}.}
Following his graduation from the latter, Bragers moved to America where he became the organist and choirmaster at the Cathedral of Covington, Kentucky, whence he joined the faculty of Ward's Pius X School in 1922.\footcite[169]{AchilleBragers18871955}

Desrocquettes tutored Bragers in an approach to chant harmony that was not the modal approach to which Bragers had long been accustomed; namely, that the accompaniments were now to be comprised of modern harmony.\fnletter{Desrocquettes}{Gajard}{16 October 1926}{\so{}}
Ward's opinion of Bragers soured as a result, and by 1929 she complained that Bragers's use of dissonance `gave a character which was too modern and non modal to the melodies', calling it `torture for the ear'.
She communicated her sentiments to Mocquereau, who (in contrast to the approbation discussed on \cpageref{ln:desrocquettes_playing,ln:desrocquettes_playing_END} above) soon distanced himself from Desrocquettes's harmonic practice.

Ward summed up her own thoughts on how best the harmony could be tackled.
She proposed that accompaniment ought to use only those chords best fitting the mode: since little evidence exists to suppose Ward had any credentials as a harmoniser, her proposals must be taken to be those of a harmonic dilettante, even if they were not far removed from Niedermeyer's rules.
Ward considered the French-Canadian organist Conrad Bernier (1904--88) to have written accompaniments in a `serious, sincere, unpretentious' style which surely made them worthy of Solesmes's consideration.\footcite[Ward to Mocquereau, 13 December 1928; Mocquereau to Ward 22 December 1928; Ward to Mocquereau, 2 March 1929; Excerpts reproduced in][81--2]{CombeJustineWardSolesmes1987}
Bernier had been the organist at the Église Saint-Sacrement in Quebec until 1923 when he had won the Prix d'Europe, a grant from the Quebec government allotting him the means to take organ lessons in Paris with Bonnet.
Following his return to North America in 1927, Bernier took up a teaching post at the Catholic University of America, Washington.\footnote{\covid{}\cite[287--8]{MillerEncyclopediemusiqueau1993}.}
While conceding that Solesmes would receive Bernier's `experiments', Mocquereau warned that no accompaniment had yet satisfied him, a statement that contradicted his earlier approval of Potiron's accompaniment manual.
The sought-after traits were `gentleness and moderation'; but all Mocquereau could find was the `cold, dry, mechanical repetition of the chant' by the organ.
Ward held that, indeed, unaccompanied signing would be best of all, but admitted that some choirs required an organ's support.
She hoped that a choir's reliance on the organ could be removed through proper training, neglecting to acknowledge the possibility for the organ to dispense with the chant and to accompany in sustained chords instead.\footcite[Mocquereau to Ward, 26 March 1929; Ward to Mocquereau 16 April 1929. See][82--3]{CombeJustineWardSolesmes1987}
\nowidow[2]

On 9 May 1929, shortly after the above Ward--Mocquereau exchange, the Victor Talking Machine Company recorded excerpts from the Mass Ordinary chanted by around thirty women of Ward's Schola, whom Bragers accompanied on the organ.
8$^\prime$ and 4$^\prime$ stops may be discerned quite clearly from the recording of Credo I.\footcite{VictormatrixCVE47995}
In 1934, Bragers proposed 4$^\prime$ stops for accompanying children's voices,\footcite[55]{BragersShortTreatiseGregorian1934} though presumably this was on account of pitch rather than timbre.
For the Polish composer Feliks Rączkowski (1906--89), the use of 4$^\prime$ stops hinged neither on timbre nor pitch, but rather on the number of sung voices being accompanied:

\simplex{Akompaniament ma stanowić tło. Dlatego też jeśli organista śpiewa sam, winien użyć do akompaniamentu najwyżej 1 lub 2 rejestrów, łagodnych fletowych (8$^\prime$). Jeśli śpiewa chór lub wierni, można dołączyć flety 4$^\prime$.}
  {\cite[unpaginated `Uwagi Praktyczne', p.~41]{RaczkowskiMszeGregorianskie1957}}
{The accompaniment constitutes the background. Therefore, if the organist sings alone, he should draw a maximum of one or two stops, soft flutes (8$^\prime$), to accompany. If the choir or the congregation sings, 4$^\prime$ flutes may be drawn.}
\noindent
It was common practice for Polish organists to accompany themselves, hence presumably the advocacy for soft 8$^\prime$ flutes.
It is noteworthy how several composers such as Rączkowski relied on Solesmian transcriptions similar to those we observed above (\cpageref{ln:bas_138})---complete with \quaver~=~138 tempo indication, among other traits---in accompaniments composed in the 1950s (\cref{mus:raczkowski_angelis_41}).

Bragers's recording ought to bear witness to the dissonant style with which Ward took issue, but it is rather difficult to make out whether Bragers truly did follow a modern major-minor harmonic scheme.\footnote{\cite{DonovanCredoLU642014}.\label{fn:creed_recording}}
As for Bragers's use of dissonance, it was certainly more tame in 1929 than in an accompaniment of his published in 1937: the differences become all the more clear when comparing the recording of Credo I (see \cref{fn:creed_recording}) to \cref{mus:bragers_credo1}.
The latter contains a good deal more conjunct motion than had previously been the case, along with more dissonance.\footcite[93]{BragersAccompanimentVaticanKyriale1937}
Certain amendments made to Ward's method by other members of the faculty at the Pius X School led to her resignation from the board in 1931, and it was not until 1959 that the Boston musician Theodore Marier (1912--2001) managed to broker something of a rapprochement between Ward and the faculty.\footcite[17--22]{BrancaleoneJustineWardfostering2009}
Following Ward's departure, Julia Sampson took over the choir and set in train a second round of recording in 1933, Bragers accompanying on the organ once again.\footnote{\covid{}\cite{VictormatrixCS74994}.}
Perhaps Brager's increased use of dissonance might be explained by the change in directorship, assuming of course that Sampson was more permissive than her predecessor in that regard.
Bragers's publisher McLaughlin \&{} Reilly Co.\ advertised his published accompaniments to the American market in a leaflet claiming them to be in the `approved style' of Solesmes.
The publisher also went to great pains to convince prospective buyers that it would not be necessary to `unlearn' the accompaniments after a few years had elapsed.\footnote{Advertisement `Accompaniment to the Kyriale', \so{} in Desrocquettes--Cardine correspondence.}
Perhaps the tumultuous years around the turn of the century, when Solesmes's books were rendered out of date within a number of years, had made Americans wary of anything aligning itself too closely with Solesmes.

The English Benedictine monk Gregory Murray (1905--92) was anything but reticent when writing in support of Bragers's accompaniments, in November 1937.
In an issue of \covid{}\emph{Music and Liturgy}, Murray opined that players who found themselves `repelled by some of Dom Desrocquettes's quite justifiable discords' would find Bragers's accompaniments more agreeable.
The review was picked up by the American magazine \emph{The~Caecilia}, which was adequately placed to drive sales of the book in the American market.\footcite[479]{BritishCriticsPraise1937}
Socio-economic factors also contributed to the relative popularity of Bragers's accompaniments in North America during WWII in particular, when the Montreal-based newspaper \emph{Le~Devoir} acknowledged difficulties in acquiring European publications.
Its reporter recommended Bragers's accompanied Kyrial in the absence of others and also because it contained an accompanied Requiem Mass---a macabre notice for macabre times.\footcite[4]{Mercureaccompagnementsgregoriensmaitre1942}
The challenges facing the importation of European accompaniment books extended to Solesmian chant books too, prompting American editors to put out pirated versions of the \emph{Liber usualis} characterised by idiosyncrasies Desrocquettes explained away as stemming from local taste.\fnletter{Desrocquettes}{Cardine}{28 October 1946}{\so{}}

The exile of Polish nationals to America created demand for a bespoke genre of \mbox{popular} masses based on Polish hymn tunes.
Those by Jan Chojnacki are rather like contrafacts since they set various parts of the Mass Ordinary to a pastiche of Polish hymn tunes, including \emph{Witaj Krynico dobra wszelakiego}, \emph{Witaj Boże utajony}, \emph{Pójdź do Jezusa do niebios bram}, \emph{Kłaniam się Tobie} and \emph{Serce Twe Jesu miłością goreje}.
No pastiche was necessary for the creed, however, which was sung to Credo III.
Choral forces indicated by the Roman numerals `I' and `II' were prompted to take up successive phrases in alternation, \mbox{accompanied} by a four-part texture of Chojnacki's devising.\footcite[2, 9--14]{ChojnackiHymnTuneMassCongregational1960}

\subsection{The shock of the new: Novel methods in practice}
Jeannin and Ward were not the only figures to raise objections to Desrocquettes's method of accompaniment, for some critics concluded that his and Potiron's use of dissonance had patently gone too far.\footnote{\cite[373]{Lessmannanachronismemusicalaccompagnement2019}.\label{fn:lessmann_373}}
Desrocquettes acknowledged this criticism in the \emph{Revue grégorienne}:
\pagebreak{}

\simplex{Pour certains, nos accompagnements sont à ce point de vue un vrai scandale. Dans beaucoup de milieux, on a avoué qu'on«~aurait préféré moins de dissonances~», ou bien qu'«~on n'avait pas été habitué à entendre le grégorien accompagné avec tant de dissonances~».}
  {\cite[20]{Desrocquettesaccompagnementgregoriendissonances1931}; See also \cref{fn:lessmann_373}}
{For some, our accompaniments, from this perspective, are a real scandal. In many circles, it has been admitted that `we would have preferred less dissonance', or even that `we have not been used to hearing Gregorian chant being accompanied by so many dissonances'.}
\noindent
Potiron acknowledged the criticism too, admitting that he and Desrocquettes had probably changed chords too frequently.\footcite[220]{Potironproposaccompagnementgregorien1930}
In private, however, Potiron's relationship with \mbox{Desrocquettes} began to strain.
The monk blamed the musician for not properly editing the complicated, dissonant accompaniments; to make them simple and consonant now would require a new edition.\fnletter{Desrocquettes}{Cardine}{16 August 1931 and 21 August 1932}{\so{}}
Potiron learned of Desrocquettes's accusations and, while accepting his own role as corrector, argued that the turgidity had nothing to do with him.\fnletter{Potiron}{Gajard}{[24 December 1932?]}{\so{}}

Potiron struck out on his own in an accompanied Gradual that first appeared in 1933, dismissing Desrocquettes's request that the Sundays after Pentecost be reserved for him.
Ostensibly, Desclée required the proofs without delay,\fnletter{Potiron}{Gajard}{[12 July 1933]}{\so{}} and Potiron's preface attributed his sole authorship to the distance separating Paris and Quarr---geography was said to have made collaborating with Desrocquettes impossible.\footcite[unpaginated `Avant-propos']{PotironGraduelparoissialcontenant1933}
In truth, however, Potiron kept Desrocquettes out for musical reasons: not only were the monk's accompaniments too turgid for Potiron's taste, but his use of B\kern 1pt\flat{} in the harmony when it had not appeared in the chant was not a modal fact to which Potiron was willing to subscribe:

\simplex{Je ne peux plus supporter ses bémols ; je veux des lignes simples et claires et il est toujours touffu, et quand je lis un de ses accompagnements je ne peux que le refaire.}
  {\letter{Potiron}{Gajard}{[18 March 1933]}{\so{}}}
{I can no longer put up with his use of B\kern 1pt\flat{}: I want simple and clear lines and his are always dense, and when I read one of his accompaniments I have to rewrite it.}
\noindent
The last system quoted in \cref{mus:potirondesrocquettes_29pieces} bears witness to Desrocquettes's use of \pitch{4}\kern 1pt\flat{} without its having first appeared in the chant.
By 1933, Potiron admitted that his own views had become more puritanical (`je suis devenu plus rigoriste'),\footcite[p.~109, n.~1]{Potironbemoldansaccompagnement1933} which led to a revised accompaniment in which \pitch{4} was largely absent (\cref{mus:potiron_circumderunt}).
For Potiron, it was not solely a question of whether or not to include `B'\kern 1pt\flat{} in the accompaniment, for he had also arrived at a conclusion similar to Gevaert's by proscribing pitches in the accompaniment that had not appeared in the chant (see \cref{ln:gevaert_hexachordal}).
Potiron avoided the pitch `E' when writing the harmonisation quoted in \cref{mus:potiron_crastina} because that note is never sung.\footcite[18, 60--61]{PotironGraduelparoissialcontenant1933}

The approach stunned Cardine,\fnletter{Desrocquettes}{Cardine}{27 May 1933}{\so{}} who claimed that to avoid `E' was to make the accompaniment's modality somewhat undetermined between tritus and tetrardus.\footcite[236]{C[ardine]ReviewHenriPotiron1933}
On succeeding Mocquereau in 1930, Cardine began moving Solesmes away from the theory of `free musical rhythm' in favour of Gregorian semiology, and those accompaniments that had been conceived according to the former were increasingly being considered out of date.
The move provoked an ironic reaction from Desrocquettes:

\simplex{Je trouve déplorable qu'à Solesmes on semble ainsi se faire un jeu d'adopter des opinions et solutions en l'air qui semblent faites pour choquer les idées courants et décourager les solutions pratiques.}
  {\letter{Desrocquettes}{Cardine}{20 August 1933}{\so{}}}
{I find it deplorable that at Solesmes they seem to make a game of taking opinions and solutions out of thin air, which seem designed to shock present ideas and discourage practical solutions.}
\noindent
Desrocquettes's complaints fell on deaf ears, however, and in 1938 he was reassigned to a new Benedictine foundation in Las Condes, Chile.\footcite[111]{DusselHistoryChurchLatin1981}
His influence on accompaniment was dampened by that assignment until 1948 when he was recalled to represent Solesmes at the Pontifical Institute of Sacred Music in Rome.

\pagebreak{}
\hlabel{hl:potiron_kyrialeabrege}%
Potiron eventually replaced the 1929 accompanied Kyrial with \emph{Kyriale abrégé} in 1950, whose preface acknowledged the need for more simplicity (`le sens d'une plus grande simplicité').\footcite[4]{PotironKyrialeabregecontenant1950}
It was a second attempt for Potiron who aimed at a style different from Desrocquettes's:

\simplex{Le Kyriale de 1929 était certainement beaucoup trop chargé~; l'ami Desrocquettes en a bien convenu~; il est devenu compliqué en se «~perfectionnant~», nerveux, sensible, artiste même, mais pas objectif.}
  {\letter{Potiron}{Gajard}{30 November 1851}{\so{}}}
{The Kyrial of 1929 was certainly far too busy; our friend Desrocquettes was in complete agreement with that; he became complicated in `perfecting himself', nervous, sensitive, artistic, even, but not objective.}
\noindent
For Potiron, the ideal harmonic path continued to be in the avoidance of notes not in the chant.
The harmonisation quoted in \cref{mus:potiron_kyriale_9} forgoes E\kern 1pt\flat{} for that very reason, though the procedure appears not to have been possible with some shorter chants, including that quoted in \cref{mus:potiron_deo_9} where the note `E' occurs in the accompaniment without occurring elsewhere.
Yet, even though Desrocquettes had indeed admitted fault with the earlier harmonisations, he seemed not to have appreciated Potiron's rationale, calling the resulting effect `pure \emph{chinoiserie}'.\fnletter{Desrocquettes}{Henri Tissot}{31 March 1951}{\so{}}

Beginning in the 1930s, Joseph Yasser (1893--1981) attempted to promote a method of quartal harmony in preference to the tertian type promulgated by Desrocquettes, Potiron, Springer and Emmanuel (among many others).
The seed had been planted in 1932 with Yasser's theory of `Infra-Diatonic Harmony',\footnote{\covid{}\cite{YasserTheoryEvolvingTonality1932}, see chapter eight.} and by 1937 Yasser was applying it to the harmonisation of chant.
He deemed quartal harmony preferable owing to an assumption that the chant repertory was based on the pentatonic scale, quoting several examples from the \emph{Liber Usualis} to substantiate his point.\footcite[174, 181--2]{YasserMediaevalQuartalHarmony1937}
Through a convoluted method, Yasser determined that the principal chords suitable for the harmonisation quoted in \cref{mus:yasser_quartal_360} were the tonic dyad (comprising the pitches `E' and `A'), and the dominant dyad (comprising `D' and `G').\footcite[359--60]{YasserMediaevalQuartalHarmony1938}
The theory of quartal harmony languished for some two decades before Patricia Burgstahler took it up in a Master's thesis (\cref{mus:burgstahler_quartal_66}), wherein no fewer than fifteen species of pentatonic scales were outlined.\footcite[54, 66]{BurgstahlerAccompanimentGregorianChant1959}
In spite of Burgstahler's efforts, the system has once again returned to dormancy.

Whatever confidence Potiron might have had in his approach to harmony was certainly undermined by further developments in modality.
Faced with novel approaches posited by Jean Langlais (1907--91), who reportedly built on the modes to create new ones,\footcite[78]{KrellwitzUseGregorianChant1981} and Gaston Litaize (1909--91), who blazed his own trail, Potiron admitted that his own concept of modality was `plus sevère' and was in danger of seeming childish (`ma conception aurait l'air puérile').\footnote{\letter{Potiron}{Desrocquettes}{1 August 1958}{\qaa{} QAA-C-188}; The letter was miscatalogued as being from the Quarr Abbey abbot Dom Germain Cozien.}
Olivier Messiaen (1908--92), who expanded the modal horizons through his Modes of Limited Transposition, possessed a copy of a textbook Potiron wrote concerning accompaniment,\footcite{PotironLeconspratiquesaccompagnement1952} but on its consultation by the present author, only the first chapter was observed to have been cut open---the others remained sealed.\footnote{\bnf{}~VM~FONDS~30~MES-4~(23).}

\hlabel{hl:potier}%
It is difficult to reconcile Francis Potier's 1946 classification system for accompaniments with the wide range of approaches we have examined up to now.
Potier categorised accompaniments depending on whether they occurred before or after the Gregorian restoration---a period he fixed in the years 1905--1908---and whether accompaniments followed Solesmes's rhythmic or harmonic theories.
Those categories divided into two subcategories each: prior to the Gregorian restoration there were said to be `note contre note' (or so-called arrhythmic accompaniments), and those with so-called melodic notes (or rhythmical accompaniments); and following the restoration, there were reportedly rhythmical accompaniments that conformed with Solesmes's rhythmical principles and others that did not.
Some were also codified as being in conformity or otherwise with Solesmes's modal principles.
\nowidow[2]

Notwithstanding the various theories of chant rhythm Solesmes had adopted since the 1880s, Potier's bibliography is arguably biased towards French and Belgian sources while those from other linguistic traditions are decidedly under-represented.
Not only did Potier oversimplify the domain substantially by not accounting for the influence of Cecilianism on the use of cadential sharping, but certain value judgements inveigled their way into his descriptions of certain methods, undermining his impartiality; these include a mention of a `very defective' chant rhythm.\footcite[pp.~67, 92 \S{}115]{Potierartaccompagnementchant1946}
Moreover, the accompaniments Potier described as `arrhythmic' might better be labelled `homorhythmic', since the chord-against-note style and mensural schemes are not mutually exclusive.
Nonetheless, Potier's history contains a useful catalogue of the available technical literature up to 1946 and has been much expanded by the present author in \cref{ap:handlist}.
\noclub[2]

\subsection{Towards the sustained style}
Although Desrocquettes's preference undeniably swayed towards dissonant harmony, other composers did not share his view.
Placide Gagnon proposed three rules for accompaniment in 1938: first, that the harmonisation was to be consonant; second, that the part writing was to be clear and easy for an organist to play; and third, that chords were to be placed according to `double rhythm'.
The last was an idea Gagnon had picked up from one Père Lefebvre SJ and relayed to Gajard.
`Double rhythm' demarcated important notes in the `petit rythme' by changes in the soprano and tenor parts and those in the `grand rythme' by the alto and bass parts.\fnletter{Placide Gagnon}{Gajard}{8 September 1938}{\so{}}
It was not a novel idea by the 1930s, for Mathias's graduated stages had long since set the precedent for codifying theories of part movement according to a rhythmic theory.
Yet, the idea did not enthuse Gajard who was lukewarm about Gagnon's claim to outline binary and ternary groups not least because Solesmes was in the process of abandoning them.\fnletter{Gajard}{Gagnon}{6 November 1938}{\so{}}
\nowidow[2]

While changing chords on the `levé' was a noteworthy feature of Gagnon's \mbox{accompaniments} of the 1940s,\footcites{GagnonAccompagnementveprestemps1940}{GagnonMesseSSMartyrs1940} the example from 1944 quoted in \cref{mus:gagnon_dissonant_49} places chords preferentially on the first notes of binary and ternary groups.
One could hardly argue that the passage in question is consonant owing to the frequent intrusion of dissonances in the tenor part.\footcite[pp. viii, 4--5]{GagnonAccompagnementchantssaluts1944}
Note that `C'\kern 1pt\natural{} was written in anticipation of the same note in the chant.

The notion of `double accentuation' was taken up by an array of Belgian composers in the orbit of the Lemmens Institute.
Their understanding of it was quite different from Gagnon's `double rhythm', however, for it concerned chords struck either on accented syllables or on the first notes of groups, depending on how close one occurred to the other.
Under the directorship of Jules Van Nuffel (1883--1953), those Belgian composers brought out an accompanied Gradual by dividing the task between them, the division of labour being entabulated in \cref{tab:noh}.
In a preface, Van Nuffel provided a dizzying set of rules governing where chords were to be struck: in general, they depended on the type of neume, but the author anticipated a theoretical text that would describe the method in more detail owing to the sheer number of intricacies involved.
That theoretical text was published by Flor Peeters in 1949, who also provided rubrics for a more legato style with little part movement.
Minor chords were said to be preferable in chant accompaniments, for they were supposedly `in conformity with the modal and archaic character and general spirit of Plain Chant'.
Major chords were to be arranged as first inversion chords alone,\footcite[13--14, 22]{PeetersPracticalMethodPlainChant1949} though the rubric was evidently unsuitable to the composer of the accompaniment quoted in \cref{mus:noh_42}, whose tetrardus harmonisation did not shy away from using major chords in 5/3 position.\footcite[pp.~xi*--xiv*, 42]{KyrialeMissapro1942}
Perhaps the same assumption led Eugène Lapierre to posit in 1949 that plagal cadences were indicative of a `religious cadence', whereas perfect cadences were to be ruled out altogether.\footcite[pp.~12, 24]{LapierreGregorianChantAccompaniment1949}

It was seemingly to capture a certain numinous quality that the Australian priest Percy Jones (1914--92) used A major chords in the deuterus accompaniment in \cref{mus:jones_ictus_15}.
Certain cadences seemed to that composer to be more reminiscent of tetrardus ones:

\single{In Credo I., the use of the chord of A major may sound strange to some, accustomed to other accompaniments. But to me, this Credo is in the VIII. mode, except for the final phrase, and consequently the final note of every other phrase is the tonic, and the tonic of the VIII. mode requires a major chord as its harmony. Moreover, this strong eighth mode cadence at the end of each sentence is a true reflex of the radiant certainty accompanying the proclaiming of the truths of the Faith.}
{\cite[pp.~v, 15]{JonesHymnalStPius1952}}
\noindent
Perhaps that notion informed Jones's decision to terminate the `Amen' on an F\kern 1pt\sharp{} minor chord (note the general use of `C'\kern 1pt\sharp{}).
From the way vertical \emph{episemata} litter Jones's chant, the harmonisation appears to have been based on a Solesmian edition, but we can be confident that the idea of using `C'\kern 1pt\sharp{} did not come from Desrocquettes and Potiron, whose 1929 harmonisation of the same creed adopts the same transposition but terminates on a D major chord instead.\footcite[76]{DesrocquettesAccompagnementKyrialeVatican1929}

The trend to adopt a legato style arguably reached something of a peak in the 1940s when the Belgian organist
Jean Van de Cauter (1906--79) and others used very few chord changes indeed throughout the course of an accompaniment.
\Cref{mus:cauter_simple_10} illustrates a style in which no more than two accompanying parts made up chords that lasted for entire phrases.\footcites[10]{VandeCauterOrganumpulsantisad1944}[See also an early attempt at codifying such a sparse style in][160]{GastoueTraiteharmonisationchant1910}
It is doubtful that such a sustained style could have been possible without the developments by earlier theorists who permitted the chant to be treated as many dissonances over a select few bass notes.
The chant is therefore not always consonant with the bass part; hence, presumably, the two slurred notes in the first phrase can be interpreted as an appoggiatura.

The reforms to the Catholic Church liturgy instigated by the Second Vatican Council (Vatican II) also instituted reforms to Catholic Church music.
The Council relaxed restrictions on the vernacular and permitted it to supplant Latin in the Roman Rite.
Moreover, the regulation of instruments was made more permissive, such that any instrument `suitable for sacred use' was permitted to exercise its function with dignity for the edification of the faithful.\footcite[383--4]{HayburnPapalLegislationSacred1979}
Some dioceses adopted vernacular settings of plainchant, presumably because their congregations already knew the melodies, but others abandoned chant altogether and took on new music.
As the chant repertory began to fall out of use in parish churches, the once considerable demand for organ accompaniments dwindled.
