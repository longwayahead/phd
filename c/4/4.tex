\chapter{The Vatican's adoption of Solesmian chant books}
\section{\emph{Fin-de-siècle} accompaniment at Solesmes}
\subsection{Elaborating on free rhythm}
\label{sc:moc_paleo}%
By the end of the nineteenth century, the circulation of numerous competing theories of chant rhythm led some theorists to devise new methods of popularising their approaches.
Gaining the support of institutions such as the Schola Cantorum or Gigout's Institut d'orgue counted as one approach, but fending off theories such as Teppe's required a different strategy altogether.
\hlabel{ln:failsafe_experiments}%
A resurgence of Enlightenment thinking provided a new way of justifying notions concerning chant rhythm: just as eighteenth-century philosophers had sought truths about the natural world by devising scientific experiments, so music theorists devised musical `experiments' to provide fail-safe, objective conclusions in support of their theories.\footcite[197]{GreenMathisLussyTraite1994}
As we have observed (on \cpageref{ln:riemann,ln:lussy} above), writings by Lussy and Riemann propounded systems that claimed to reveal common characteristics underpinning musical expression in secular music, and it was not long before chant aficionados applied those systems to the sacred repertory.

One approach popularised by Lussy had been posited by Johann Mattheson (1681--1764) in 1739, when he parsed a minuet using commas, full stops, colons and semicolons to show how phrases could be divided into discrete units (\cref{mus:mattheson}).\footcite[For a discussion of the approach in the seventeenth century, see][41]{VialArtMusicalPhrasing2008}
Those symbols, however, were ill-equipped to handle the peculiarities of music notation, and so special symbols were used in addition, including a three-dot mark ($\because$) for a phrase's cadential notes (`unter ihren Schluss-Noten mit dreien Puncten').\footcite[p.~224, \S{}82]{MatthesonvolkommeneCapellmeister1739}

Surely the precision in Mattheson's method was what enticed Lussy to adopt a similar punctuation system to annotate expressive nuances previously relegated to the imagination.
Lussy's textbook proved popular on its appearance in 1874, and maintained that popularity through eight further editions, until 1904.
Much of the material on rhythm and phrasing was also published separately in the 1884 pamphlet \emph{Le rythme musical}.\footcite[52]{Lussyrythmemusicalson1884}

\label{sc:gevaert_dots}%
Lussy was not alone in looking to past centuries for certain methods.
In 1875, Gevaert provided a system of colons and dots reportedly used in Ancient Greek music (\cref{mus:gevaert_dots}),\footnote{\cite{GevaertHistoiretheoriemusique1875}, 1:350.} though he neglected to explain fully how they were meant to function in practice.
It fell to Van Damme to describe Gevaert's system in 1882 as one of pointing that could convey the rhythmic nuances which modern notation was reportedly incapable of representing:

\simplex{Les points placés au-dessus des notes marquent (comme dans la musique grecque) les temps forts du rythme. Par rythme je n'entends pas ici une division en temps isochrones, mais un rythme semblable à celui de la prose poétique, le \emph{tempo rubato}.}
  {\cite[p.~83, n.~1]{VanDammeaccompagnementplainchant1882}}
{The dots placed above the notes mark (as in Greek music) the strong beats of the rhythm. By rhythm I do not mean here a division into isochronal beats, but a rhythm similar to that of poetic prose, the \emph{tempo rubato}.}
\noindent
A similar system was used by the abbé Marcetteau in 1909 to parse the strong and weak beats of a metrical melody (\cref{mus:marcetteau-dots}).\footcite[26]{Marcetteaulogiquerythmemusical1909}
We shall return to the influence of such annotative systems later in this chapter.

Another approach was proposed by Lhoumeau in an attempt to demonstrate how closely Pothier's chant editions followed the manuscript tradition and how others fell short: why not publish reproductions of the original MSS?
Lhoumeau's rationale was based on a conviction that the general public, when faced with the source material, would decide for themselves that the \emph{Liber gradualis} was the most historically representative of all modern chant books.
Pothier was not convinced, however, being wary of the possibility that the untrained public could misinterpret the MSS.\footcites[130--31]{CombeHistoirerestaurationchant1969}[110--111]{CombeRestorationGregorianChant2003}
Perhaps Lhoumeau's proposal would do more harm than good?

In spite of those warnings, one of Pothier's fellow monks André Mocquereau pushed ahead with the plan.
He organised a group that travelled around Europe photographing select MSS for inclusion in the new publication entitled \emph{Paléographie musicale}, whose first volume of 1889 comprehended a facsimile of St Gall Codex 339.
The \emph{Paléographie} took on the aura of a scientific publication in which supposedly objective facts were presented with reference to the available evidence.
In a bid to assuage Pothier's concerns, an explanatory preface attempted to anticipate any potential misreadings on the reader's part by describing the contents that followed.
On Pothier's departure for Ligugé in 1893, Mocquereau became the principal of the chant restoration movement at Solesmes, and used the \emph{Paléographie}'s prefaces as vehicles for conveying his own opinions on how the MSS offered clues to performance practice.\footcites[14--15]{WaldenDomMocquereauTheories2015}[63--8]{BergeronDecadentEnchantmentsRevival1998}

%metonym/demonym
The adjective `Solesmes' ceased being metonymic for Pothier's free oratorical rhythm and came to be synonymous with a new theory of `free musical rhythm', Mocquereau's attempt at clarifying Pothier's method, that appeared piecemeal in successive instalments of the \emph{Paléographie}.
First, in a discussion on psalmody, Mocquereau argued that melodic and textual accentuation were separate phenomena because higher notes in a psalm tone were shown not to coincide with the accented syllables of every verse.\footcites[12]{MocquereauPaleographiemusicale1892}[For a discussion of how textual and musical accents are separate, see][20]{MocquereauPaleographiemusicale1896}
Then, he appealed to the authority of classical Greek and Roman writers, judging that by `the natural laws of melody and rhythm' the former had priority.\footcite[68]{MocquereauPaleographiemusicale1894}
And to this axiom, Mocquereau yoked methodologies derived from Lussy and Riemann.

\label{sc:solesmes_ictus}%
He devised a system of pointing to represent certain nuances, in a similar venture to Mattheson's.
The pointing represented where the so-called \emph{ictus} occurred in the melody, an elusive concept that Solesmian writers could not describe without recourse to metaphor.
In metaphorical terms, it was the point at which a skimmed stone contacted the surface of a pond: with each successive bounce the stone was robbed of more energy.\footcite[A metaphor of a golfer is provided in][303]{MocquereauPaleographiemusicale1901}
In musical terms, it was supposedly representative of the rhythm's metaphysical rise and fall.
An impulse (in the jargon of Solesmian writers) prompted the conductor's hand to rise, and so forth.
While the ensuing paragraphs will discuss the theory of the \emph{ictus} in more detail and how it influenced the practice of accompaniment, it should be noted that Solesmes retired Mocquereau's theory in the twentieth century in favour of Gregorian Semiology, a new theory of chant rhythm championed by Mocquereau's successor Eugène Cardine (1905--88).\footcite[77]{Viretchantgregorientradition2001}

\subsection{Towards a Solesmian accompaniment book}
Mocquereau's theory of the \emph{ictus} was applied in a new accompaniment book published at Solesmes.
The task of producing the book fell to one of Mocquereau's \emph{confrères} Antoine Delpech (1846--1909) who, prior to entering the Solesmes novitiate, had studied the organ with one Louis de Croze (d.1912).
Delpech became the titular organist of Limoux sometime around 1880 and worked up a small reputation as a composer of several organ pieces in Jean-Romary Grosjean's \emph{Journal des organistes}.\footcites[47]{SeveracDeodatSeveracmusique2002}[Also referenced in][451]{HalaSolesmesmusiciensSchola2017}

Delpech's first visit to Solesmes in 1887 coincided with a state-sanctioned expulsion (not the first) of the monks from Saint-Pierre.
Faced with no other option, they lodged temporarily with benevolent locals and took in their daily regimen of Offices at the nearby Benedictine convent of Sainte-Cécile where the nuns had been permitted to remain.
There, Delpech heard a method of accompaniment that piqued his curiosity, and on his return to Limoux petitioned Mocquereau to obtain an example.\fnletter{Delpech}{Mocquereau}{7 January 1888}{\so{}}
Abbess Mère Cécile Bruyère (1845--1909) conveyed it via Mocquereau, though Delpech reckoned despite his initial enthusiasm that it was not without fault:
\nowidow[2]

\simplex{Son accompagnement m'a plu en bien des points. Dans d'autres, il me semble qu'il ne seconde pas assez le rythme binaire ou ternaire des neumes. Les accords tombent q.q.\ fois sur des notes qui me paraissent plutôt notes de passage que réelles. D'autres fois encore, on aurait pu éviter de mettre des accords sur bien des notes ; cela rappelle un peu la vieille méthode.}
  {\letter{Delpech}{Mocquereau}{23 May 1888}{\so{}}}
{I liked her accompaniment in many ways. In others, it seems to me that it does not sufficiently follow the binary and ternary rhythm of the neumes. Chords are struck sometimes on notes that seem to me more passing notes than real ones. And elsewhere, chord changes could have been avoided on many notes; it reminds one a little of the old method.}
\noindent
By the `old method', Delpech no doubt referred to the chord-against-note style, which was widely considered out of date by 1888.
Following Delpech's entry into Solesmes the monastic authorities assigned him to the \emph{Paléographie musicale} project, sometime around 1888 or '89.

Along with facsimile work, Delpech was tasked with recruiting subscribers; his cousin the abbé Pratx is listed among those to the first volume.\footcite[p.~7*]{MocquereauPaleographiemusicale1889}
Delpech and Pratx pronounced an unfavorable judgement on one of Lhoumeau's accompaniments:

\simplex{Ah~! Son accompagnement. Pratx et moi l'avons joué et analysé. C'est un travail déplorable, à tous points de vue. Il défigure la mélodie, nuit au rythme, le contrarie ou le défigure. Cet homme ruinera notre œuvre. Pauvre P.~Pothier~!}
  {\letter{Delpech}{Mocquereau}{11 December 1893}{\so{}}; Delpech refers to `mon cousin Pratx' in various letters, including  \letter{Delpech}{Mocquereau}{7 January 1888}{\so{}} and \letter{Delpech}{Delatte}{May 1891}{\so{}}}
{Ah! His accompaniment. Pratx and I played and analysed it. It is a deplorable work in every respect. He disfigures the melody, works against the rhythm, contradicting or disfiguring it. This man will ruin our \emph{œuvre}. Poor Fr Pothier!}
\noindent
Some months previously, however, Delpech had been polite enough about Lhoumeau in his correspondence with the nun Mère de Vibraye (a superior in the congregation of Notre Dame de Cénacle, then resident at Versailles), when he stated that Lhoumeau's accompaniment manual of 1892 was `almost completely correct' save for the idea of \emph{arsis} and \emph{thesis} which Delpech dismissed as completely superficial (`une pure superfétation').\footnote{\letter{Delpech}{Mère de Vibraye}{18 March 1893}{Reproduction provided to the author by Père Patrick Hala on 6 August 2019}.\label{fn:delpech_vibraye}}

Delpech contended instead that chord changes were to take place on \emph{ictuses} without their needing to coincide with verbal accents.
His view on the matter accorded with \mbox{Mocquereau's} axiom that verbal and musical accents were distinct from one another.\fnletter{Delpech}{Vibraye}{21 April 1893}{see \cref{fn:delpech_vibraye}}
It was relayed to Mère de Vibraye who is responsible for having conveyed one of Delpech's accompaniments to Gigout around January 1894.
Gigout was reportedly quite complimentary of the accompaniment but raised objections about the placement of some chords.\fnletter{Delpech}{Mocquereau}{3 February 1894}{\so{}}
While visiting Paris soon thereafter, Delpech accepted Bouichère's invitation of a personal introduction to Gigout,\fnletter{Delpech}{Mocquereau}{8 February 1894}{\so{}} but reported to Mocquereau that Gigout's reception was frosty:

\simplex{Vu Gigout. Accueil froid, ennuyé, pénible. Je suis parti mal impressionné et quasi renvoyé, quoique poliment.}
  {\letter{Delpech}{Mocquereau}{[12 March] 1894}{\so{}}}
{Met Gigout. Cold, bored, difficult meeting. I left with a bad impression and felt almost dismissed, albeit politely.}

We shall never know whether or not Delpech had proposed that Gigout write accompaniments for Solesmes, but that proposition was certainly put to Étienne Hémery (1842--97), a composer today more associated with opera than church music even though he was the organist of Saint-Lô for several decades.
Hémery encountered the \emph{Liber gradualis} during the Christmas period of 1892, after which he sought correspondence on the matter with the monks of Solesmes.
That placed Hémery in the orbit of Delpech, who asked the composer to write a book of accompaniments to demonstrate the theory of the \emph{ictus} (`on demandait à Hemery d'écrire lui même l'accompagnement des Chants communs de Solesme \sic{}').
While Hémery's ailing health led him to refuse, he nonetheless agreed to coach Delpech in harmony and counterpoint so the project could be undertaken at Solesmes itself.
Weekly exercises were exchanged between them in an attempt to bring Delpech's compositional and harmonic techniques up to the required standard.\footcite[231--4]{HemeryEtienneHemerySa1898}

Delpech's unfavourable meeting with Gigout did not prejudice those in the Schola Cantorum against him.
Fernand de La Tombelle (1854--1928) met Delpech several months before the school was founded,\fnletter{Delpech}{Mocquereau}{9 March 1894}{\so{}} and Bordes even made him the following proposition: Bordes would agree to finance organ lessons for Delpech delivered by none other than Guilmant on condition that Delpech taught two chant classes, one at an orphanage attached to the Salesians of Ménilmontant and another for adults at Bordes's church, Saint-Gervais.
Delpech accepted and the half-price sum of 10~F. was negotiated for the organ lessons provided they took place at Guilmant's residence at Meudon.\footnote{\letter{Bordes}{Delpech}{July 1894}{\so{}}; Also discussed in \cite[120]{HalaSolesmesmusiciensSchola2017}.}
\noclub[2]

\subsection{Peter Wagner's views on chant accompaniment}
During the following Autumn, Delpech undertook further study at Fribourg with the academic and chant specialist Peter Wagner (1865--1931).\footnote{Wagner notes the first anniversary of Delpech's arrival in \letter{Wagner}{Mocquereau}{20 October 1895}{\so{}}.}
Abbot Delatte's words of gratitude to Wagner indicate that Delpech was receiving tuition in harmony.\footnote{\letter{Delatte}{Delpech}{22 September 1894}{\so{}}; The abbot's gratitude was conveyed via Delpech.}
By September, Delpech had seemingly gained enough confidence in his abilities that he believed the \mbox{collection} of accompaniments was ready to be announced:

\simplex{Je crois qu'il serait bon d'annoncer notre travail. Je vous ai dit que le \emph{Kyriale} est fini. Nous n'avons qu'à le revoir avec Mr Hémery, s'il est en état de suivre ce travail, sinon je le verrai moi-même et l'expédierai ensuite ici à Mr Wagner.}
  {\letter{Delpech}{Mocquereau}{21 September 1894}{\so{}}}
{I believe that it would be good to announce our work. I mentioned to you that the \emph{Kyriale} is finished. We only have to review it with Mr Hémery if he is able to undertake this work, if not I will have a look at it myself and send it on then to Mr Wagner.}
\noindent
But by October, an Eastertide Mass was still outstanding, and the accompaniments were not yet in a ready state to be published.\fnletter{Delpech}{Mocquereau}{25 October 1894}{\so{}}
In May, Hémery advised restraint by saying further improvements were yet to be made:
\pagebreak{}

\simplex{J'ai reçu une bonne lettre de St Lô. M.\ H[émery] tout en me félicitant de mes envois nouveaux, me recommande de ne pas me presser de publier des accompagnements. `\emph{Nous sommes encore dans la période des tâtonnements}.'}
  {\letter{Delpech}{Mocquereau}{9 May 1895}{\so{}}}
{I received a nice letter from St Lô. Mr H[émery], while congratulating me on my latest consignment, advises me not to rush into publishing accompaniments. `\emph{We are still in the phase of trial-and-error}.'}
\noindent
There is evidence that the trial-and-error phase led Delpech to consult Tinel, but the advice he received only conflicted with that already proffered by Wagner and Hémery.\fnletter{Delpech}{Mocquereau}{23 May \ny{}}{\so{}}
Since Delpech had avowed that the \emph{ictus} should take precedence over accented syllables, it is possible that the resultant process of harmonisation entered a state of quandary that was difficult to resolve.

Wagner took a dim view of Lhoumeau's practice of changing chords on unaccented syllables and also criticised the number of chords that led to an accompaniment's being too dense (`trop épais').\fnletter{Wagner}{Delpech}{6 February 1895}{\so{}}
He continued to ruminate on chant rhythm following Delpech's departure, and communicated the results of his deliberations to Solesmes:

\simplex{J'ai réfléchi en cette affaire beaucoup depuis le départ du bon P.\ Delpech. Espérons que bientôt vous verrez quelque chose d'imprimé sur mes idées d'accompagnement. Selon moi, la première et souveraine loi de l'accompagnement c'est~: s'adapter à la logique de la mélodie, la mettre en relief, faire ressortir ses effets mélodiques, mais pas les détruire~; faire oublier l'accompagnement par sa simplicité, sa modestie, de sorte qu'on ne croit entendre que la mélodie. Pour cela il faut laisser de côté tout ce qui rend l'accompagnement trop gros etc.}
  {\letter{Wagner}{Delpech}{20 January 1895}{\so{}}; Although the letter is arranged in the Wagner--Delpech correspondence, no fewer than five instances where Wagner refers to Delpech in the third person might in fact suggest a different intended recipient\hlabel{fn:wagnerdelpech_referral}}
{I have reflected on this matter a lot since the departure of the good Fr~Delpech. Let us hope that soon you will see something on my ideas on accompaniment in print. In my opinion, the first and sovereign law of accompaniment is the following: to adapt to the logic of the melody, to place it in relief, to bring out its melodic effects but not to destroy them, to hide the accompaniment by means of its simplicity, its modesty, so that one thinks one only hears the melody. For that, one has to leave out everything that makes the accompaniment too heavy, etc.}
\noindent
Delpech's sojourn at Fribourg led Wagner to abandon Pustet's `édition de l'Église' in favour of Solesmian editions, and to deem the \emph{Liber usualis} more historically accurate and therefore more appropriate for the university's `office académique'.\fnletter{Delpech}{Mocquereau}{18 November 1894}{\so{}}
Wagner's willingness to depart from the official chant edition left him in a vulnerable position, however, and to be regarded by some colleagues as a traitor (`il y a quelques professeurs de ma faculté qui me regardent comme un traitre').
Despite those German bishops who could opt to send their students to other universities, Wagner would stay the Solesmian course until his Damascene conversion to mensuralism in 1910.\footnote{\letter{Wagner}{Delpech}{15 December 1894}{\so{}}; \cite{JohnA.WagnerPeter}}

Perhaps those articles in the \emph{Paléographie} provided Wagner with the confidence to adopt Solesmian theories in the first place.
In 1895, his voice became the first among German scholars to question Haberl's theories.
Wagner's textbook \emph{Einführung in die gregorianischen Melodien} became the subject of a back-and-forth polemic with Haberl,\footnote{\letter{Wagner}{Delpech}{23 February [1896]}{\so{}}; This letter is catalogued as bearing the date 1890, which cannot be the case since it refers to the 1896 review cited in \cref{fn:haberl_review}.} who in turn launched \emph{ad hominem} attacks on Wagner, saying the academic relied on `lessons and teachings' (`empfangenen Lektionen und Unterweisungen') received from Mocquereau and Delpech.\footnote{\cite[123]{HaberlReviewEinfuehrunggregorianischen1896}; Copy consulted lacked pages 126--7.\label{fn:haberl_review}}
The drama caused quite a stir in Cecilian circles and led some prominent Cecilians to request copies of the textbook so they could form their own judgements.
Singenberger requested three copies from Wagner's publisher B.\ Veith for the American Cæcilienverein to evaluate the book,\fnletter{Wagner}{Delpech}{7 January 1896}{\so{}} and later published a review in that society's journal that took an opposing view to Haberl's, deeming Wagner's book an important study wrought of thoroughness and clarity.\footcite[4]{NeuePublikationen1896}
The support assuaged Wagner's concerns and probably those of his colleagues too, while also being a notable crack in the façade of Cecilian unity.
That fissure, Wagner hoped, would rally others to his side.

\hlabel{ln:wagner_reveal}%
Wagner's textbook contained some taciturn thoughts on accompaniment, borne of the notion that Medieval composers had concealed certain characteristics within the chant that could be revealed through harmony.
As we have seen, Wagner described such characteristics as `melodic effects', and their revelation, he believed, could establish a certain unity between chant and accompaniment:
\nowidow[2]

\simplex{Im Mittelalter wusste man nichts von einer Begleitung, hatte auch nicht das geringste Bedürfnis darnach. Die Sume musikalischen Inhaltes, die wir heute der Interpretation durch die Harmonie überlassen, mussten die alten Melodisten in die Melodie selbst legen. Diese vollzieht in sich selbst die vollständige Wiedergabe der Eingebungen der künstlerisch erregten Phantasie.}
  {\cite[26, 252 n.~1]{WagnerEinfuehrunggregorianischenMelodien1895}}
{In the Middle Ages, no one knew anything about accompaniment, nor had they the slightest need of it. The sums of musical content, which today we leave to harmony to interpret, the old melodists had to place in the melody itself. This accomplishes on its own the complete reproduction of the inspirations of the artistically excited imagination.}
\hlabel{ln:wagner_reveal_END}%
\noindent
Mocquereau objected to one of Wagner's accompaniments for containing harmonic progressions he deemed were too modern.
Wagner agreed to send accompaniments to Solesmes directly for vetting, a generous concession on Wagner's part considering Mocquereau's lack of formal harmonic or contrapuntal training.
But Wagner's belief that through analysing the chant a harmonic solution could be revealed probably trumped any reservations he would have had in similar arrangements.

Mocquereau attacked the use of V--I progressions in perfect cadences.\fnletter{Wagner}{Mocquereau}{29 December 1895}{\so{}}
Wagner parried by claiming an avoidance of V--I progressions made the cadence in \cref{mus:wagner_majorcadence} lack `something' (`il manque encore quelque chose').
Whatever it was, that `something' constituted the musical fault line along which French and German theories of modality were delimited.
On one side resided Niedermeyer's proposal to supplant V--I progressions with diatonic alternatives; on the other, the retention of cadential sharping.
Therein, Wagner held, lay the difference between French and German ears:
\pagebreak{}

\simplex{Si j'accompagne de la manière suivante [\cref{mus:wagner_minorcadence}] il faut reconnaître qu'il manqua aussi là quelque chose pour l'effet parfait d'une vraie conclusion. Mais à cause de la tonalité grégorienne je ne peux admettre le sol~\sharp{}. Mais plus loin je n'airai pas. Je regrette beaucoup que l'oreille française, comme dit le bon P.\ Delpech, ne peut pas gouter cela~;~mais c'est une erreur de croire que la progression de la 5\textsuperscript{e} à la tonique est moderne.}
  {\letter{Wagner}{Mocquereau?}{20 January 1895}{\so{}}}
{If I accompany in the following way [\cref{mus:wagner_minorcadence}] then one must recognise that it also lacks some perfecting factor of a true cadence. But owing to the \emph{tonalité grégorienne} I cannot admit the G\kern 1pt{\normalfont \sharp{}}. But I will not go on about it. I very much regret that the French ear, as the good Fr Delpech says, cannot abide that; but it is a mistake to believe that the progression from the 5\textsuperscript{th} to the tonic is modern.}
\noindent
Delpech's appraisal of the French ear may not be far fetched, for Gigout and others were already applying Niedermeyan diatonicism to their freely composed church music.
\mbox{Evidently}, however, Wagner subscribed to a modal tradition that was quite different.

\subsection{The \emph{Livre d'Orgue}: Solesmes's accompaniment book}
Wagner's dim view of diatonicism might explain why the task of proof-reading Delpech's harmonised collection was delegated to Hémery, who completed it between August and September of 1896.\footcite[234]{HemeryEtienneHemerySa1898}
A harmonised `Asperges me' was published in the \tsg{} in August 1897, and Wagner asked that a copy be sent to Fribourg so he could evaluate it for himself.\fnletter{Wagner}{Mocquereau}{30 August 1897}{\so{}}
It is not clear whether the harmonisation simply languished for eleven months following Hémery's involvement or whether further revisions were necessary to bring Delpech's harmonisation up to standard.
It may have been a calculated move by Delpech and Mocquereau to publish a single extract ahead of a larger publication to allow for further edits if a reviewer found a passage particularly unpalatable.

In an article published to complement the extract Mocquereau outlined his system of rhythmic pointing, which was placed above the transcription of the chant into modern notation to display whether an \emph{ictus} happened to be \emph{arsic} or \emph{thetic}.
In that way, it bore some resemblance to Lhoumeau's rising and falling arcs which represented essentially the same thing, though in 1907 Lhoumeau distanced himself from Mocquereauvian developments with the following comment:

\simplex{Les publications faites par le R.\ P.\ Dom Moquereau n'ont pas donné sur la question du rhythme grégorien ce qu'on en pouvait espérer. J'y retrouve bien toute la théorie dont je me reconnais l'auteur~; mais si je n'en recuse pas la paternité, je crois que l'enfant a singulièrement dégénéré ou qu'on me l'a changé en nourrice.}
  {\cite[2]{LhoumeauEtudeschantgregorien1907}}
{The publications by the Reverend Father Dom Mocquereau did not offer what one might hope on the question of the Gregorian rhythm. I find in them all the theory of which I recognise myself to be the author; but even if I do not deny its paternity, I think that the child has particularly regressed or that someone has changed it on me in the cradle.}
\noindent
Lhoumeau's belief that his theory essentially fell apart in Mocquereau's hands takes on a certain poignancy given the import Bergeron ascribes to `Mocquereau's Hands' in the \mbox{development} of Gregorian chironomy.\footcite[112--121]{BergeronDecadentEnchantmentsRevival1998}
And just as Lhoumeau had developed his system of arcs to show the course of a conductor's hand through the air, Mocquereauvian chrionomy did the same with a set of curlicues.

A precursor to that system of curlicues was Mocquereau's system of 1x2 Braille-like cells that were similar to those annotations of Gevaert's discussed above (\cref{sc:gevaert_dots}).
A dot in a cell's upper position was used for an \emph{arsis} while a dot in the lower position was used for a \emph{thesis} (\cref{mus:moc_punctuation}).
The cell at `1' shows an anacrustic \emph{ictus} that anticipates the cell at `2', showing a so-called primary \emph{arsic} \emph{ictus}.
In the latter case, dots in the upper and lower positions produce what looks like a colon, but seem reminiscent of Lhoumeau's practice of eliding successive arcs when the \emph{thesis} of one phrase coincides with the \emph{arsis} of the next.
The cell at `3' shows a \emph{thetic} \emph{ictus} which is prolonged by a further cell at `4'.
The cell at `5' shows a fresh rhythmic impulse with another primary \emph{arsic} \emph{ictus}, though this time it does not give way to a \emph{thesis} straight away, leading instead to the cell at `6' which shows a so-called secondary \emph{arsic} \emph{ictus}.
The cells at `7' and `8' show \emph{theses} as before.\footcite[127]{MocquereauNoteponctuationrythmique1897}

The pointing was applied to those of Delpech's accompaniments that were published subsequently by the Imprimerie Saint-Pierre and entitled \emph{Livre d'Orgue}, a title perhaps harking back to collections of versets by classical French composers such as Nivers discussed above (see \cpageref{ln:nivers_livredorgue}).
Solesmes's \ldo{} was published in four volumes plus a supplement of accompanied psalm tones: the first in February 1898; the second in July 1898; the third in May 1899; and the fourth in March 1900---the supplement of psalm tones bears the date April 1899.\footnote{For the date of publication of the first vol.\ see \cite[416, n.~35]{HalaSolesmesmusiciensSchola2017}; The cover pages of the remaining volumes were viewed by the present author.}
The first four volumes contained sections of the Mass Ordinary and a fifth was planned for around 1900 with further accompaniments, but---for reasons we shall discuss later---it never saw the light of day.

The `Asperges me' accompaniment heading Delpech's first volume (\cref{mus:delpech_asperges_1}) is laid out similarly to those harmonisations in Mégret's volumes.
A quadratic staff is placed above five-line staves bearing a transcription and the accompanying parts for the organist.
Chords are struck to coincide with \emph{ictuses}, but since most of the primary \emph{arsic} type are in alignment with the verbal accents the harmonisations seem almost regressive, particularly where a preponderance of \emph{ictuses} necessitated many chord changes.
Moreover, modern notation was not always sufficient to represent certain neumes and therefore an array of supplementary symbols was required.

\hlabel{ln:pointing_exceptions}%
Some neumes were considered important enough to draw away the primary \emph{arsic} \emph{ictus} from a nearby verbal accent.
A \emph{pressus} is said to divert the primary \emph{arsic} \emph{ictus} (and, therefore, the chord that accompanies it) owing to its being a repeated note: it is represented by the caret symbol (\cref{mus:delpech_caret_47}).
A note prolonged by a \emph{mora vocis} dot of addition is treated in like manner (\cref{mus:delpech_livre_2_87}), though it was simply transcribed as a crotchet without the need for any supplementary glyph.\footcite[1, 34, 47, 87]{LivreOrgueChants1898}
Prolongations by horizontal \emph{episemata} often coincide with \emph{ictuses}, as do \emph{distropha} and \emph{tristropha} which were amalgamated into one note of longer duration in the transcription, a triangular mark setting these latter apart from other longer notes.
Wavy lines are common annotations to signal the presence of \emph{quilismata}.
Metronome markings were \mbox{added} to each accompaniment, a noteworthy feature since they were probably borne of the same notion of a fail-safe, objective performance practice discussed above (see \cpageref{ln:failsafe_experiments}).\footnote{\cite[The annotations are also described in][64--5]{EllisPoliticsPlainchantfindesiecle2013}.\label{fn:ellis_annotations}}
\hlabel{ln:pointing_exceptions_END}%

While the \ldo{}'s preface is not attributed to Mocquereau, it repeats much of the same material Mocquereau had published in the \tsg{}.
It was expanded to include a discussion of the permissibility of accompaniment in the first place, and, contrary to what one might expect, it takes an ideological stance against accompaniment on the grounds of its being anachronistic.
Nonetheless, it concedes that the \ldo{} project had been undertaken to answer requests for an appropriate genre of accompaniment that could support less experienced voices:
\noclub[2]

\simplex{Disons-le sans détour : c'est à regret que nous l'avons entrepris, \& nous le publions seulement pour donner satisfaction à tous ceux qui nous le demandent depuis plusieurs années.

\parindent=10pt
A les en croire, outre qu'il est nécessaire de venir au secours des voix inexpérimentées de nos chantres, il est encore opportun de condescendre à cette déviation regrettable du goût général qui a créé, chez les fidèles, le besoin tout moderne d'entendre un accompagnement polyphone.}
  {\cite[p.~v]{LivreOrgueChants1898}}
{Let us say it bluntly: it is with regret that we undertook it, and we are only publishing it to satisfy all those who have asked us for it for the past several years.

\parindent=10pt
If one is to believe them, besides the need to come to the aid of the inexperienced singers among our number, it is still opportune to give in to this regrettable deviation of general taste which has created the very modern need among the faithful to hear a polyphonic accompaniment.}
\noindent
Although the preface tacitly intended the Solesmian accompaniments for use in parish churches, Solesmes itself was accused of modernism by the music critic Camille Bellaigue (1858--1930) as, contrary to the above moralising statement, it too made use of accompaniment.
Although the monks were surely anything but `inexperienced singers', Bellaigue reported that, during a visit to Saint-Pierre in 1898, the Offices he attended were accompanied by the organ.
While Bellaigue drew short of accusing Solesmes of hypocrisy, his account seems to suggest that the Solesmian community valued accompaniment on aesthetic grounds, rather than on purely practical ones:

\simplex{Si, partout ailleurs qu'à Solesmes, l'accompagnement du plain-chant est une faute nécessaire, à Solesmes c'est presque une heureuse faute.}
  {\cite[353--5]{BellaigueabbayeSolesmes1898}}
{If, everywhere other than at Solesmes, plainchant accompaniment is a necessary fault, then at Solesmes it is almost a happy one.}

\subsection{Organs and organists at Solesmes}
Bellaigue's account raises a question: what kind of instrument accompanied the chanting at Solesmes?
A history of organs at Solesmes was written by the present (2021) organist, Frère Yves-Marie Lelièvre, whose account is contained at the Solesmes archive and will form the basis of the following paragraphs.
In general, Solesmes conformed with the widespread trend in French churches to install two organs (on which, see \cref{sc:orgue_accompagnateur}), one on the gallery for improvisation and repertoire, and another nearer the altar to accompany singing.

Monastic accounts for February 1852 bear witness to a harmonium's having been sold to the nuns of Saint-Jean d'Angély for 135~F.
Presumably, it was rendered obsolete by a new organ which had been constructed for the abbey church between 1849 and 1850 by one Hippolyte Givelet, who later entered Solesmes's novitiate.
Guéranger mentioned the new `orgue au chœur' in a description dated 25 January 1850, though it is difficult to ascertain whether the organ was a worthwhile addition since the organist's playing reportedly left much to be desired (`cela relève un peu, bien que l'instrument soit faible et l'artiste aussi').
The organist was not identified: perhaps it was Givelet himself.
He was succeeded by one Frère Forgeois around 1854, about whom no further details were available.

Further accounts from December 1850 showed 4,400~F.\ allocated to finance a larger organ to match the growing number of professions.
That instrument was commissioned of the organ builder Fréderic Verschneider (1810--84) and the task of playing it fell to fifteen-year-old Nicolas Karren.
Meanwhile, the `orgue au chœur' retained its function as a Choir organ.
A further iteration of the grand orgue came about around 1856 when a new two-manual instrument disposed with twenty-five stops was built by Fréderic's younger brother Charles (1825--65), the 1850 instrument being sold to the minor seminary at Précigné.
Charles was already an organ builder of some experience and had been associated with the English inventor Charles Spackmann Barker and the Parisian church of Saint-Eustache.\footcite[pp.~255--6 \S{}10 n.~3]{OchseOrganistsOrganPlaying2000}
Solesmes therefore shows itself to have been decidedly à la mode by adopting two organs for its liturgy, following the lead of the principal cathedrals and parish churches elsewhere in France.
\nowidow[2]

A series of lay organists presided over the grand orgue following Karren's departure in 1859 until the conclusion of Camille Donay's tenure on 27 August 1866 when the responsibility for playing both organs fell to Forgeois.
The abbey church had been enlarged in June 1861, and the removal of a wall separating chancel from nave on 9 March 1865 incited the monastic authorities to reorganise the disposition of organs in the space.
One (presumably the Givelet instrument) was placed above the corbel arch in the north Choir, where it was first played on 23 August 1865.
In 1895, Cavaillé-Coll recorded its disposition as Bourdon~16$^\prime$, Salicional~8$^\prime$, Kéraulophone~8$^\prime$, Flûte~8$^\prime$, Prestant~4$^\prime$, Trompette~8$^\prime$, and this was probably the same instrument Bellaigue heard during his visit in 1898.

Dom Georges Legeay (1842--1903) became the first member of the monastic community after Forgeois to be appointed as an organist of Solesmes.\footnote{Frère [Yves-Marie] Lelièvre, `Note sur les premières orgues de Solesmes (1849--1867)', \so{}.}
Legeay's reputation had extended beyond the cloister in the 1870s when several collections of \emph{Noëls anciens} were published under his editorship in an edition with piano accompaniment.
His organ playing at Solesmes placed him at the centre of a dramatic stand-off between monks and gendarmes when the latter---following anticlerical orders---attempted to evict the monks from their monastery on 6 November 1880:

\simplex{Les moines se relèvent dans leurs stalles. Les gendarmes les engagent à sortir, à commencer par le dernier des novices, un manceau tout récemment entré. Il faut les arracher un à un. Les uns s'accrochent aux stalles, d'autres s'étendent à terre et se laissent porter comme des cadavres. Quelques-uns étendent les bras en croix. Pendant ce temps, continue le chant des antiennes et des psaumes, infatigablement soutenus par dom Legeay au grand orgue.}
  {\cite[p.~54 after the account by Étienne Cartier in \emph{Les moines de Solesmes -- expulsions du 6 novembre 1880 et du 22 mars 1882}]{SoltnerAbbayeSolesmesaux2005}}
{The monks stand up in their stalls. The gendarmes bid them to leave, starting with the last of the novices, a local who had only recently entered. They have to be taken out one by one. Some cling to the stalls, others stretch out on the ground and let themselves be carried out like corpses. Others open their arms in the shape of a cross. While this is going on, the chanting of antiphons and psalms continues, accompanied tirelessly by Dom Legeay at the grand orgue.}
\pagebreak{}
\noindent
Accompaniment by the grand orgue was presumably extraordinary and might be explained on this occasion by the access route to the organ gallery not being straightforward for the gendarmes to locate.
Legeay would later contribute to Mégret's volumes an accompaniment in free rhythm that did not reproduce the chant in the organ part.
\label{ln:novello_legeay}%
Instead, the accompaniment was notated according to a method similar to that adopted independently by Novello and Schmitt (see \cref{mus:legeay_4_14} and \cpageref{ln:repos_notation,ln:novello_notation_END} above),\footnote{Legeay points the reader to the `Graduel bénédictin p.~(112)', and it should be noted that the chant `Alleluia: Post partum' is found in \cite{PothierLiberGradualis1883} pp.~[112--113].} though in Legeay's case the beams were to indicate sustained notes and not the grouping of notes into neumes.
The accompaniment was therefore notated amensurally, and required the organist to change to a new chord only when the singer had arrived at a particular note.
Legeay outlined the rationale for his system in a footnote:
\noclub[2]
\hlabel{ln:legeay_group}%

\simplex{Les notes n'ont ici qu'une valeur purement conventionelle, entièrement dépendante du rythme de la partie vocale. La transcription en notes de musique de la mélodie du plain-chant reproduit aussi fidèlement qu'il est possible toutes les notes et leur groupement. La partie d'orgue est écrite de telle sorte que chaque accord est exactement placé sous la note du chant qu'il accompagne. Le groupe \legeayGroup{} indique que la note doit être tenue jusqu'à l'accord suivant.}
  {\cite[4:14]{Melodieschantgregorien1892}}
{The notes here only have a purely conventional value and are entirely dependent on the rhythm of the vocal part. The transcription into music [modern notation] of the plainchant melody reproduces all its notes and groupings as faithfully as possible. The organ part is written so that each chord is placed directly under the note of the chant that it accompanies. The group \legeayGroup{} indicates that the note should be held until the next chord.}
\noindent
The following account of the early years of the \emph{orgue de chœur} at Solesmes was provided in the 1980s, and makes clear that Legeay's style was eventually superseded:

\simplex{Lorsque l'orgue de chœur a été construit, nous avons fait changer trois fois le jeu d'accompagnement pour avoir exactement le son que nous désirons.}
  {\cite[401]{Pinguetecolesmusiquedivine1987}, quoting perhaps Jean Claire}
{When the \emph{orgue de chœur} was built, we had the playing style of the \mbox{accompaniment} changed three times to get exactly the sound we wanted.}

\subsection{A greater focus on dissonance}
There is no evidence to suggest Legeay's opinions on organ accompaniment of chant held any sway in the lead-up to publishing the \ldo{}.
Delpech reportedly fashioned the accompaniments to `better respect the character and suppleness of the chant';\footcite[p.~v]{LivreOrgueChants1898} or, in other words, to apply Wagner's theory of `melodic effects' to the accompaniment.
Nevertheless, harmonic questions still remained.
As mentioned previously, a preponderance of \emph{ictuses} left Delpech with no option other than to deploy a great number of chords, a stricture that begat accompaniments in the chorale texture which did little to preserve Wagner's notion of unity with the chant.
\nowidow[2]

Nevertheless, to at least one contemporary French commentator no opposition to Delpech's style was apparent.
In the opinion of one Pierre Aubry (1874--1910), stated in an address at the Parisian Institut catholique on 3 May 1899, Delpech's use of dissonance achieved the very unity Wagner had proposed.
Aubry recognised Niedermeyan diatonicism as the norm in the \ldo{} which, together with passing notes, so-called `artificial dissonances', and appoggiaturas (simple and double), meant that `the melodic line was respected and the plainchant conserved its veritable character'.
Aubry also recognised that certain terminations in the third, fourth and eighth modes seemed to follow rubrics set down by the Mechelen school,\footcite[98]{AubryOEuvrebenedictine1900} but it is difficult to tell whether they resulted from Tinel's advice.
We shall return to that matter later in the chapter.

For the moment, it may be noted that some theorists who had previously favoured the chord-against-note style began to change tack and offer more advice concerning the use of dissonance.
\hlabel{ln:lepage_dissonance}\hlabel{hl:lepage}%
In an 1895 accompaniment manual, the \emph{maître de chapelle} of the Église métropolitaine de Rennes abbé Louis Lepage (1852--1906) had provided no alternative to the chord-against-note style, but to his second edition of 1900 added a second volume detailing how to use `Notes Foreign to Chords'.
The inspiration for that addition probably resulted from a meeting with Delpech at the seminary of Rennes in January 1896, when, following an unclear chain of events, Delpech agreed to become Lepage's teacher.
The event was reported to abbot Delatte with Delpech's noting how useful Lepage's influence could be in propagating Solesmes's theories:
\noclub[2]

\simplex{Les élèves suivent avec zèle et entrain les cours de chant quelque multipliés qu'ils soient. Après les cours, je vois les organistes en particulier, ou bien M.\ l'abbé Lepage, le célèbre maître de chapelle de la cathédrale qui s'est gentiment constitué mon élève dès le premier jour. Je crois que ce sera une excellente recrue, car son influence est ici en rapport avec son talent d'organiste et de compositeur.}
  {\letter{Delpech}{Delatte}{31 January 1896}{\so{}}}
{Pupils follow the chant classes with zeal however many there may be. After class, I see the organists in particular, or even Fr Lepage, the famous cathedral \emph{maître de chapelle} who kindly made himself my pupil from the first day. I believe he will be an excellent recruit, because his influence here is set in relation to his talent as an organist and composer.}

Lepage's second volume reproduces the same description of Mocquereau's pointing as had been published in both the \tsg{} and the \ldo{}.
Furthermore, he included no less than an entire chapter of music examples with the \emph{mise-en-page} being the same as in the latter (an example is quoted in \cref{mus:lepage_85}).
Lepage's second volume was a joint venture between the Rennes-based publisher Bossard-Bonnel and the Imprimerie de Saint-Pierre.
We may assume that such an arrangement suited Solesmes, which could thereby maintain its guard on the proprietary musical type used for its chant editions.\footcite[85]{LepageTraiteaccompagnementplainchant1900}
A positive review of Lepage's method appeared in the March 1900 issue of \emph{Revue du chant grégorien} penned by one `A.~D.',\footcite[136]{D[elpech]ReviewLepageTraite1900} later unmasked as none other than Delpech in a 1906 bibliography of Benedictine publications.\footcite[41]{BibliographieBenedictinscongregation1906}
It is an example of that curious genre of panegyric, semi-anonymous reviews fawning over a publication brought to market by a member of the reviewer's own circle.

Lepage's practice certainly aligned with Delpech's, so there is little wonder why the monk's praise was so effusive.
Chords generally align with primary \emph{arsic} \emph{ictuses} which themselves ordinarily align with textual accents, save for the exceptions listed above (\cpageref{ln:pointing_exceptions,ln:pointing_exceptions_END}).
Notes between \emph{ictuses} were usually treated as various types of dissonance, including \emph{échapées}, auxiliaries, anticipations, passing notes and appoggiaturas.
In spite of the obvious similarities with the \ldo{}, however, divergences may be recognised on the quadratic staff.
First, \emph{porrecti} were typeset using a piece of type more curved than that previously used.
Second, \emph{strophici} were annotated with crescendo and descrescendo markings in a way not dissimilar to Lhoumeau's interpretation of the \emph{pressus}.
The triangular mark used by Delpech for \emph{strophici} might not annotate an accent, therefore, and might instead signify an expressive nuance to be produced by the voice, though confirmation of that practice is not to be found in the available relevant literature.

\subsection{The reception of Solesmes's \emph{Livre d'Orgue}}
Delpech's harmonisations were generally well received by critics.
In an approbative review in the \emph{Revue du chant grégorien}, Lhoumeau stated that his own views on rhythm were matched by Delpech's.\footcite[168]{LhoumeaulivreorgueSolesmes1898}
In private correspondence to Pothier, Lhoumeau went further by stating that Solesmes had not innovated on his own rhythmic theories and had adopted them unchanged.\fnletter{Lhoumeau}{Pothier}{3 June 1898}{\swf{} 153~(c)~203}
\hlabel{ln:latombelle_deuterus}%
A review by La Tombelle in \tsg{} was similarly congratulatory, though he questioned why harmonisations of deuterus chants arranged 5/3 chords on \pitch{6}, thereby reputedly destroying the chant's modality.
For instance, the E\kern 1pt\flat{} major chord quoted in \cref{mus:delpech_deuterus_23} was said to make the fourth mode indistinguishable from the eighth.\footcite[143]{LaTombelleNotesbibliographiquesLivre1898a}
La Tombelle's review of the \ldo{}'s second volume claims the foible had apparently been rectified:

\simplex{Si nous considérons ce travail au point de vue musical, nous serons porté[s] à préférer encore cette seconde livraison à la première, tant à cause de certaines recherches heureuses dans l'harmonisation que d'une logique plus rigoureuse dans le maintien de la modalité.}
  {\emph{Ibid}., no.~9 (September 1898): 216}
{If we consider this work from the musical point of view, we will be inclined still to prefer this second volume to the first, as much because of certain successful researches in harmonisation as because of more rigorous logic in the maintenance of modality.}
\nocite{LaTombelleNotesbibliographiquesLivre1898}%p.216
\pagebreak{}
\noindent
But there is musical evidence to contradict La Tombelle's change of stance: should the C~major chords quoted in \cref{mus:delpech_caret_47} not have elicited the same criticism as before?\footcite[23, 47]{LivreOrgueChants1898}
\hlabel{ln:latombelle_deuterus_END}%

\hlabel{ln:schola_accomps}%
The Schola Cantorum was partial to Mocquereau's pointing and employed it in a collection of its own, even though no explanatory preface was provided for a student of chant accompaniment to decipher them.
\emph{Melodiæ paschales} contains accompaniments by d'Indy, Guilmant, La Tombelle and Bordes for different parts of the Mass Proper for Easter Sunday.
Each composer's initials follow the portion of the Proper for which he was responsible.
The accompaniments demonstrate a wide variety of approaches that bring a certain disunity to the Proper as a whole.
Nevertheless, the collection provided one François Brun with the basis for an accompaniment manual that appeared first in the \tsg{} in 1909, and which examined the means whereby individual passages had been harmonised.
Brun's manual was published as a standalone leaflet in 1912, whereafter the Italian journal \emph{Rassegna gregoriana} took issue with the Schola Cantorum's practice of making chords coincide with metrical accents in hymns.\footcite[p.21 \S{}1 n.~2 and \emph{passim}]{BrunTraiteaccompagnementchant1912}
The Schola Cantorum had anticipated such criticism in the \tsg{} by claiming its personnel followed the `traditional' Solesmes school of Joseph Pothier:
\noclub[2]

\simplex{Nous suivons à la Schola les principes traditionnels de l'école de Solesmes, tels que Dom Pothier les a formulés il y a trente ans, tels que Dom Delpech les a appliqués dans le \emph{Livre d'orgue} publié à la célèbre abbaye. Que d'autres aient cru devoir changer depuis, c'est leur affaire, et non la nôtre.}
  {\cite[92]{Reponsespolemiquesgregoriennes1910}}
{We follow at the Schola the traditional principles of the Solesmes school that Dom Pothier formulated thirty years ago and that Dom Delpech applied in the \emph{Livre d'orgue} published at the famous abbey. Whether others believed that it needed to be changed in the meantime, that is their business and not ours.}
\noindent
One notes a certain proclivity for the Schola Cantorum to distance itself from \mbox{Mocquereauvian} theories of rhythm, doubtless owing, as we shall see, to the controversy stirred up at the end of the century.

Although no date of publication was stated in \emph{Melodiæ paschales}, the book was likely to have been published in 1898 following the first volume of the \ldo{} and perhaps in readiness for Easter Day on 10 April 1898.
That way, the chants of the Proper would still be fresh in the minds of \emph{maîtres de chapelle}.
The Ligugé monk Dom Jean Parisot (1861--1923) suggested in 1914 that the book had been published in 1889,\footcite[34]{Parisotaccompagnementmodalchant1914} but that seems improbable because the Schola Cantorum was not founded until 1894 and the pointing was not devised until at least 1897.
Guilmant's contribution offers a further piece of evidence to corroborate 1898 as the date of publication: his initials are prefixed with the name of the Canadian city `Montréal' where, on his second American tour, Guilmant played an organ recital at Saint George's church on 4 March 1898.\footnote{See \cite{ProgrammeOrganRecital1898}.}
That would have allowed ample time to mail an accompaniment to Paris and for it to appear in print by 10 April.
The copy consulted by the present author at the BNF is stamped with the year 1898, making that sequence of events all the more plausible.

In any event, Lhoumeau had already learned of the Schola Cantorum's collection by June of 1898 and compared the method of transcription to that in the \ldo{}.
While the rhythmic theory evidently aligned with his own, Lhoumeau considered neither the \ldo{} nor \emph{Melodiæ paschales} to represent an ideal method of transcription, a reservation he noted in a letter to Pothier:

\simplex{Les accomp\textsuperscript{ts} publiés par la Schola ne sont pas non plus mon idéal. D.\ Mocquereau leur a traduit le rythme d'une façon qui les brouille et leur fait commettre des erreurs.}
  {\label{fn:lhoumeau_schola}\letter{Lhoumeau}{Pothier}{3 June 1898}{\swf{} 153~(c)~203}}
{The accompaniments published by the Schola are not ideal either in my view. Dom Mocquereau transcribed the rhythm for them in a way that confuses them and causes them to commit errors.}
\noindent
Lhoumeau's original letter bears one notated music example illustrating a specific instance where Mocquereau's pointing differed from Lhoumeau's conception of the rhythm, the example matching a transposed passage in `Alleluia, Pascha Nostrum' (\cref{mus:latombelle_pascha}).
La Tombelle, the harmoniser, had dutifully changed chords according to most \emph{ictuses}, but Lhoumeau was unwilling to accept the result as correct, and indicated by way of barlines those notes on which chords could more effectively have been changed.

Lhoumeau considered Bordes to be the weakest harmoniser since his approach was said to be a bit blasé (`Bordes harmonise en \guillemotleft{} je m'en foutiste \guillemotright{}').
By contrast, Guilmant `did the best', perhaps owing to a more economical use of chords.
Guilmant, nonetheless, also followed Mocquereau's pointing (\cref{mus:guilmant_haec}), save at the beginnings of phrases when he appears to have followed his own rhythmic instinct.
\noclub[2]

One of Bordes's hymn harmonisations is notable in that each verse is individually pointed (\cref{mus:bordes_salvefesta_10}).\footcite[4--6, 9--11]{DIndyMelodiaepaschalesChoix1898}
Generally, primary \emph{arsic} \emph{ictuses} align with the verbal accents and as a result no two verses are pointed identically.
A far-reaching new approach to pointing is evident in a hymn from \emph{Melodiæ natales} (\cref{mus:guilmant_jesu}), a book of harmonisations of the Proper for Mass on Christmas Day.\footcite[5]{GuilmantMelodiaenatalesChoix1898}
Again, it was an enterprise of the Schola Cantorum, but this time the harmonising was entrusted to Guilmant alone.
It was deemed unnecessary to point each hymnodic verse individually, and unlike before each primary \emph{arsic} \emph{ictus} was not always aligned with a verbal accent or even with a note of longer duration.
The pointing was now contrived to suit the melody (or, at least, not to follow the verbal accents in every case), thus affecting where the harmoniser could change chords, as we shall now see.


\subsection{Mocquereau, d'Indy and the fate of Delpech}
Between the second volume of the \ldo{} (July 1898) and the third (May 1899), there began the trend of distributing the pointing not according to verbal accents but \mbox{according} to a different method altogether.
Chords were to be placed not on strong verbal accents but on weak syllables, often at the ends of words, causing the accompaniment to exhibit symptoms of the Lhoumeau effect (see \cref{sc:lhoumeau_effect}).
It is to this phenomenon that we shall now turn, not alone because it was responsible for an international fracas and an inconsolable rift between Mocquereau and Delpech, but also because the Solesmian propagation of the Lhoumeau effect continues to influence methods of accompaniment today.

\label{hl:dies_irae_mocquereau}\label{cc:moc_ictus}%
Delpech's harmonisation of `Dies irae' (quoted in \cref{mus:livre_diesirae_142}) in the fourth volume of the \ldo{} (March 1900) stirred up controversy in the international press on account of chords' being placed on the final, weak syllables of words.\footcite[142]{LivreOrgueChants1898}
As Katharine Ellis has observed, the pointing is conspicuously absent from above the chant,\footcite[64--6]{EllisPoliticsPlainchantfindesiecle2013} though the manner in which the accompanying parts are beamed makes the rhythmic intention clear.
So polarising was Mocquereau's new approach that the Turin-based composer and organist Giovanni Pagella (1872--1944) lamented how `hardened' in his stance the monk had become.
Pagella absolved Delpech of harmonic wrong-doing on account of monastic inferiors not being permitted to diverge from the views of their superiors.\footnote{Relayed in \letter{Delpech}{Delatte}{28 February 1901}{\so{}} and separately by a musician based in Turin in \letter{Giulio Bas}{Mocquereau}{21 April 1903}{\so{}}; Pagella's review appeared in the Turin-based periodical \covid{}\emph{Santa Cecilia}.}
The new approach drew the criticism of Wagner, who warned abbot Delatte of the necessity for caution.
A remarkable and still unpublished account in the Solesmian archives describes how Wagner was invited to Solesmes to present a case against Mocquereau's approach, which provoked Mocquereau to tender his resignation as editor of the \emph{Paléographie} and as Solesmes's \emph{maître de chœur}.
Delatte refused, however, and Mocquereau then requested to consult further theorists and to publish his findings on placing chords on weaker syllables.\footnote{`Affaire Wagner (vers 1896--1897)' dated 28 August 1929 among the Delpech archives at \so{}.\label{fn:wagner_affair}}
Pierre Combe fails to mention the incident which must have caused Mocquereau and Solesmes considerable embarrassment.\footcites{CombeHistoirerestaurationchant1969}{CombeRestorationGregorianChant2003}

\label{cc:dindy_rhythm}%
On the subject of chant rhythm, Mocquereau consulted d'Indy, arguably one of the foremost music theorists in France at the time.\footnote{\cite[For transcriptions of the available d'Indy--Mocquereau correspondence see][426--43]{HalaSolesmesmusiciensSchola2017}; At the time of writing, much of the Mocquereau--d'Indy correspondence had not yet come to light.}
Terminology proved of interest: d'Indy proposed the French terms `temps léger' and `temps lourd' instead of \emph{arsis} and \emph{thesis} because in his experience students were sometimes confused by the classical terms.
Notes taken by d'Indy's student Auguste Sérieyx (1865--1949) formed the basis of a 1912 composition treatise in which the terms `léger' and `lourd' are used to describe how rhythm was independent of meter.\footcite[26]{DIndyCourscompositionmusicale1912}
D'Indy admitted to coining the French terms after the German ones `schwer' and `leicht' had been introduced by Riemann.\footnote{\letter{D'Indy}{Mocquereau}{30 January 1901}{\so{}}; Some twenty-one letters between Riemann and Mocquereau dating from 1899 to 1914 are preserved in the Solesmes archives, though at the time of writing none was published. See \cite[410, n.~26]{HalaSolesmesmusiciensSchola2017}.}
But when the topic of `temps léger' and `temps lourd' arose in a manual of chant rhythm, Riemann's misattribution to Mocquereau of the French translation was quoted by Mocquereau himself, without due acknowledgement to d'Indy.\footnote{\cite[52 n.~1]{Mocquereaunombremusicalgregorien1908}. Riemann's involvement is also discussed in \cite[3 n.~11 and \emph{passim}]{WaldenDomMocquereauTheories2015}.}
\nowidow[2]

Perhaps d'Indy's erasure from the record was due to opinions he held on \mbox{accompaniment} that were probably disappointing to Mocquereau.
Apparently, Mocquereau asked for d'Indy's opinion on the Requiem Mass, perhaps to ascertain the validity of the controversial `Dies irae' harmonisation.
Chords, d'Indy wrote in reply, were to reinforce the accent whether it be `léger' or `lourd'---those placed on the former were liable to create syncopated accompaniments.\fnletter{D'Indy}{Mocquereau}{10 February 1901}{\so{}}
When composing accompaniments of chant, d'Indy was sensitive to a great many considerations that made it impossible to set out an immutable method.
And since harmony was necessarily governed by modern rules, he was also sceptical that an authentic method of accompaniment could be established in the first place:

\simplex{Faut-il vraiment établir une \emph{théorie d'accompagnement ne varietur}, chercher un \emph{système} qui ne sera forcément qu'une adaptation de notre pensée harmonique moderne aux rythmes libres anciens.

\parindent=10pt
N'est-ce pas contribuer à une déformation~?\ldots{} et en ce cas serait une besogne anti artistique\dots{}

\parindent=10pt
Et, que si \emph{il faut} accompagner harmoniquement, est-ce que la \emph{musique} ne devrait pas l'emporter dans cet accompagnement sur des règles que nous aurions, en somme, établies nous-mêmes ?'}
  {\letter{D'Indy}{Mocquereau}{31 March 1901}{\so{}}; \cite[Transcribed on p.~440 and reproduced between pp.~444 and 445]{HalaSolesmesmusiciensSchola2017}}
{Is it really necessary to establish an \emph{unvarying theory of accompaniment}, to seek a \emph{system} that will necessarily only be an adaption of our modern harmonic thinking to the ancient free rhythms?

\parindent=10pt
Does this not add to a deformation?\ldots{} and in that case it would be an anti-artistic task\dots{}

\parindent=10pt
And in this accompaniment, if \emph{it is necessary} to accompany harmonically, should \emph{musicality} not prevail over rules that, in short, we would have established ourselves?}

\pagebreak{}
The controversy might explain why the \ldo{}'s fourth volume contains markedly fewer accompaniments than its earlier volumes.
While Mocquereau was in the process of soliciting advice from theorists, Delpech sent a fifth volume of harmonisations to Wagner who returned the verdict that they ought to be published without amendment:

\simplex{Votre livraison est très bien faite. Je souhaite qu'elle puisse venir au jour telle qu'elle est, sans que vous soyez obligé d'adopter un système qui ne repose sur aucune donnée scientifique sérieuse.}
  {\letter{Delpech}{Delatte}{28 February 1901}{\so{}}}
{Your volume is very well done. I hope it can come to light as it is, without your having to adopt a system that is not based on any serious scientific evidence.}
\noindent
Wagner's view was forwarded by Delpech to abbot Delatte along with the request that the volume be sent to d'Indy for further comment.
Wagner's approbation also emboldened Delpech to request that Delatte inform Mocquereau that chords need not change at every \emph{ictus}, but Mocquereau proved unyielding.
The rationale for placing chords on weaker syllables was published by Mocquereau as an article in the seventh volume of the \emph{Paléographie musicale} entitled `The Role and Place of the Latin Tonic Accent in Gregorian Rhythm'.
It contained (among many other ideas) select music examples from classical and modern sacred repertories that were supposed to justify his method.\footnote{See \cref{fn:wagner_affair}.}
Mocquereau relies on an extract from Josquin Desprez's composition `Ave Christe immolate' to demonstrate how weak syllables were arranged on strong metrical beats, but by applying the principle of polyphonic syncopation to a monophonic repertory that had originated a thousand years earlier, Mocquereau is most assuredly guilty of anachronism.

A further example taken from the \emph{Oratorio de Noël} by Camille Saint-Saëns (1835--1921) serves only to confirm the anachronism and speciousness of Mocquereau's argument.
The movement `Et intendit mihi' was supposed to demonstrate how modern composers placed final syllables on the first, strong metrical beat of the bar,\footcite[7:32--3]{MocquereauPaleographiemusicale1901} but Mocquereau did not acknowledge the distinct possibility that Saint-Saëns had set the Latin text in the manner normative for a French one.
\label{ln:moc_lastsyllable}%
Neither did Mocquereau note that throughout the \emph{Oratorio} accented syllables are almost invariably aligned with strong musical beats.
Mocquereau's \emph{faux-pas} in quoting the movement was made all the more apparent in 1919 when Saint-Saëns himself passed the following judgement on d'Indy's perception of chant rhythm:

\simplex{Dans la musique du moyen âge, dont M.\ d'Indy donne des exemples et que l'on désigne sous le nom de \emph{plain-chant}, créée avant l'invention barbare de la mesure, j'ai beau chercher le rythme~;~c'est seulement l'absence de rythme que j'y trouve.}
  {\cite[13]{Saint-SaensideesVincentIndy1919}}
{In the music of the Middle Ages, of which Mr d'Indy gives examples and which one designates by the name \emph{plainchant}, created before the savage invention of \emph{la mesure}, in vain did I seek rhythm; but I find only the absence of rhythm.}

Mocquereau's rationale nonetheless convinced Delatte, with the result that Delpech's and Wagner's warnings fell on deaf ears and the fifth of Delpech's harmonised volumes was withheld from publication even though the process of engraving had already commenced.
Its proofs have not been observed in the Solesmes archive by the present archivist.\footnote{Père Patrick Hala to the author during a visit to the Solesmes archives in August 2019.}
On Delpech's expulsion from the \emph{Livre d'Orgue} project, the role of harmoniser passed to the former Solesmian monk and former Delpech pupil Louis Gregory Sergent (b.1870), then organist at Oosterhout abbey, Holland.\fnletter{Delpech}{Mère de Vibraye}{3 February 1903}{see \cref{fn:delpech_vibraye}}
Delpech had recommended in 1896 that Sergent follow Loret's organ method,\footnote{\letter{Delpech}{Mocquereau}{4 June 1896}{\so{}}; \covid{}\cite[20]{OchseOrpha[Caroline]NineteenthCenturyOrganTutors2007}.} and Sergent became confident enough in his own abilities to write a new accompaniment manual.
One catalogue describes it as being based on `new and precise principles' (`d'après des principes très nouveaux et très précis'),\footcite[4]{Joubertmaitrescontemporainsorgue1912} which also probably influenced Sergent's 1905 \covid{}\emph{Accompagnement du Credo des Anges d'après les Editions de Solesmes}, published in Paris by Haton.\footnote{\cite{BibliographieBenedictinscongregation1906} (1906), p.~159.}

Delpech's fate was already sealed, then, by the time an anticlerical law took effect in 1901 banning religious communities from France.
The Solesmes monks were faced with no alternative other than to emigrate, and settled on the Isle of Wight in England, first at Appuldurcombe House and later at Our Lady of Quarr (see \cref{ln:quarr_foundation}).
Meanwhile, Delpech was separated from the main community and was sent to the Abbey of Saint Michael at Farnborough.
Wagner recounted Delpech's fate to a certain Dom Émile Daval in 1903:

\hlabel{ln:wagner_viewsonmoc}%
\simplex{Je sais tout ce qui s'est passé, étant mêlé moi-même dans le combat qui s'est livré à Solesmes, il y a 2 années, et qui a finit par le bannissement du bon P.\ Delpech à Farnborough, et la victoire des idées vraiment folles du P.\ Mocquereau qui les contient avec un entêtement fanatique.

\parindent=10pt
Vous verrez leurs caprices encore mieux quand elles seront transplantées dans l'accompagnement.
Il place les accords nouveaux sur les syllabes faibles, forçant ainsi les chantres qui ne comprendront du reste, jamais ces idées, et surtout ceux qui ne savant pas le latin, à appuyer les syllabes faibles, à les accentuer. Mais il n'y a pas de transaction possible là-dessus avec le P.\ Mocquereau.

\parindent=10pt
Je prévois cependant que sa théorie fait le plus grand tord à Solesmes ; lui est son plus grand ennemi, plus dangereux que une douzaine de Haberl, car il commence déjà à falsifier et fausser les mélodies vieilles, en les arrangeant d'après ses idées préconçues et \emph{tout arbitraires}.}
  {\letter{Wagner}{Émile Daval}{21 February 1903}{\so{}}}
{I know all about what happened, having been embroiled myself in the fight that took place at Solesmes two years ago which ended in the banishment of the good Fr Delpech to Farnborough and the victory of the truly mad ideas of Fr Mocquereau, who maintains them with a fanatical stubbornness.

\parindent=10pt
You can see their fickleness much better when they are applied in the accompaniment. He places new chords on weak syllables, thus forcing singers who will never understand these ideas (and especially those who do not know Latin) to stress the weak syllables, thereby accenting them. But there is no possible negotiation with Fr Mocquereau on this point.

\parindent=10pt
I therefore predict that his theory causes the greatest damage to Solesmes. He is his own worst enemy and is more dangerous than a dozen Haberls, because he already begins to falsify and distort the old melodies by arranging them according to preconceived and \emph{completely arbitrary} notions.}
\hlabel{ln:wagner_viewsonmoc_END}%
\noindent
Daval's reply to Wagner was intercepted by one Dom Athanase Logerot (1840--1908) who drew the matter to Delatte's attention.
Logerot was aware of rumours circulating that Solesmes itself was divided on the rhythmic question, and determined that discussing the matter with those outside the monastic community would be improper.\fnletter{Logerot}{Delatte}{12 March 1903}{\so{}}
Daval therefore raised the matter with Delatte directly, and argued Delpech's case by asking the abbot to reconsider Mocquereau's pointing.
That which had been added to the recent \emph{Liber usualis} was, in Daval's words, `infinitely unfortunate' (`chose infiniment regrettable').\fnletter{Daval}{Delatte}{10 March 1903}{\so{}}
Daval's supplications amounted to nothing, however, even though other French musicians were making their own cases against Mocquereau's system.
Widor regretted that rhythmic signs were invented according to whim and without much consideration for the historical facts:

\simplex{Voici qu'aujourd'hui se forme, chez les Bénédictins mêmes, une autre école plus hardie, plus ambitieuse, plus autoritaire, déclarant que le rythme est soumis à certaines lois par elle-même édictées, qu'il faut accentuer telle note de préférence à telle autre, inventant des signes, empruntant à la fois aux neumes et à notre système moderne, voulant imposer au monde un régime dont il est difficile de dire s'il est plus traditionnel ou plus novateur.}
  {\cite[58]{Widorrevisionplainchant1904}}
{And now today, among the Benedictines themselves, another more daring, more ambitious, more authoritarian school is taking shape. The school proposes that the rhythm be subject to certain laws prescribed by that selfsame school, that one note must be accented in preference to another, inventing signs, borrowing both from the neumes and our modern system, wanting to impose on the world a regime that makes it difficult to say whether it is more traditional or more innovative.}
\noindent
Claiming that Mocquereau had invented the notion of the \emph{ictus}, Delpech distanced himself not only from his prior view that chords were to change on \emph{ictuses} but also from his own harmonisations in the \ldo{}:

\simplex{Remarquez que dans le \emph{Livre d'Orgue}, il y a changement d'harmonie sous chaque ponctuation rythmique etc. Ces accompagnements sont insupportables. Je vous assure que je ne me vante de les avoir faits.}
  {\letter{Delpech}{Mère de Vibraye}{12 July 1906}{see \cref{fn:delpech_vibraye}}}
{Notice that in the \emph{Livre d'Orgue} there is a change of harmony under each rhythmic punctuation etc. These accompaniments are unacceptable. I assure you that I do not brag about having composed them.}
\noindent
When Delpech was called to review an accompaniment based on the same system, he wrote to Delatte to clarify whether the theory of the \emph{ictus} was still in vogue:

\simplex{Je désirerais savoir exactement~: 1.\ Si en me rangeant du côté de M.\ d'Indy, au point de vue de l'harmonisation, je me mets au nombre de vos adversaires.

\parindent=10pt
2.\ Si, pour vous, l'accent a cessé d'être \emph{toujours} à l'\emph{arsis}. C'est d'après l'idée de l'\emph{arsis}, temps fort, que j'ai écrit les deux premières livraisons du \emph{Livre d'orgue}, où j'ai fait concorder le temps fort \emph{harmonique} avec le temps fort de la déclamation (\emph{Arsis}).

\parindent=10pt
J'ai refusé de continuer quand on a bouleversé capricieusement cette manière d'entendre l'harmonisation des mélodies grégoriennes.}
  {\letter{Delpech}{Delatte}{25 July 1905}{\so{}}; Also printed in \cite[421--2]{HalaSolesmesmusiciensSchola2017}}
{I would like to know exactly: 1.\ If I place myself among your adversaries, from the harmonic point of view, by siding with Mr d'Indy.

\parindent=10pt
2.\ If, for you, the accent has ceased \emph{always} to be on the \emph{arsis}. It is according to the idea of the \emph{arsis}, \emph{temps fort}, that I wrote the first two volumes of the \emph{Livre d'Orgue}, where I made the \emph{harmonic} \emph{temps fort} coincide with the \emph{temps fort} of the declamation (\emph{Arsis}).

\parindent=10pt
I refused to continue when this way of hearing the harmonisation of Gregorian chant was capriciously upset.}
\noindent
While Delpech's interest in accompaniment abated thereafter, one letter to the Farnborough abbot Fernand Cabrol (1855--1937) bears witness to his enduring disquiet.\fnletter{Delpech}{Cabrol}{13 July 1908}{\so{}}


\section{Towards a new official chant edition}
\subsection{Vatican approval for Solesmes}
The French government's exiling of religious communities coincided with a period of significant upheaval in Catholic church music.
Not only had the monopoly previously granted to Pustet expired, but the Holy See also ruled in the letter `Nos quidem' (dated 17 May 1901) that the task of restoring the chant repertory was thenceforth to be delegated to Solesmes.\footcite[182--191]{HayburnPapalLegislationSacred1979}
As we have seen (\cref{sc:cecilian_resistance}), Haberl and his circle proved resistive to that ruling and continued to use Cecilian chant editions and organ accompaniments as they had before.
Outside that circle, however, musicians were divided.
A nonplussed cohort took little heed of the new papal directives, at least initially, while a proactive cohort took up the mantle of applying them to musical composition.
Prior to examining how `Nos quidem' and other papal decrees exerted changes on the musical traditions in Catholic worship, it is first necessary to consider two early advocates of the Vatican's updated stance.

Wagner received papal assent on 7 June 1901 for a new school of chant at the \mbox{University} of Fribourg, for which he placed advertisements in French and German periodicals, tailoring their content to suit populations with discrete interests.
No mention is made of Regensburg in the French advertisement, for instance, which focuses instead on the papal backing the school had received.
It also notes the proposed curriculum, which was divided into theory---history of chant aesthetics and manuscript studies---and practice---chanting, accompanying and choral directing.\footcite[Notice dated 10 August 1901 in][10]{WagnerAcademiegregorienneFribourg1901}
By contrast, the Teutophone advertisement tackled the Regensburg quandary, albeit in a subtle way:
\pagebreak{}

\simplex{Die gregorianische Akademie zu Freiburg i[n] d[er] Schweiz ist bischer die einzige Kirchenmusikschule deutscher Zunge, in welcher der traditionelle Choral gründlich und nach den Resultaten der neuesten wissenschaftlichen Forschung gelehrt wird. Der Besuch dieser Schule bildet demnach unter den obwaltenden Verhältnissen des geeignetste Mittel, sich auf die gregorianische Restauration durch intensive praktische wie gelehrte Arbeit vorzubereiten.}
  {\cites[106]{WagnerGregorianischeAkademieFreiburg1904}[150]{WagnerKleinereMitteilungen1904}}
{The Gregorian Academy in Fribourg, Switzerland, is the only German-\linebreak{}language church music school in which the traditional chant is taught thoroughly and according to the results of the latest scientific research. Attending this school is therefore, under the prevailing circumstances, the most suitable means of preparing for the Gregorian restoration through intensive practical and learned work.}
\noindent
By the academy's third semester, in 1903, further advertisements broke down the course into six classes, the number of registered pupils here being noted in parentheses: history of chant (8), theory (8), reading and accompaniment (9), critique of chant editions (4), semiography (4), and practical exercises (16).\footcite[35]{WagnerGregorianischeAkademieFreiburg1903}
It is hardly surprising that more demand existed for practical classes than theoretical ones, because the papal decrees of the early years of the twentieth century placed a particular emphasis on practical aspects of church music.
Those aspects grew in importance with each passing year as further decrees reinforced earlier bans on secular genres, requiring church musicians to adopt various approved repertories in their stead.

Perhaps with a view to meeting the demands of such musicians, Wagner edited and published a chant book of his own.
The chants themselves were reportedly of Germanic origin and laid claim to a heritage quite distinct from the Latin repertory which had been taken as the basis of Roman chant books.
There is little doubt that a Germanic book would have appealed to German-speaking congregations, but the venture would have been for naught had the quadratic notation been illegible and the Latin rubrics incomprehensible.
Wagner therefore brought out a version in modern notation and with German rubrics.\footcites[p.~iii]{WagnerKyrialesiveOrdinarium1904}[p.~iii]{WagnerKyrialegewoehnlichenMessgesaenge1904}
Nor were transcription and translation Wagner's only tactics in the interest of popular dissemination, for he also composed a complementary book of accompaniments so that choirs could benefit from the support of organs or harmoniums.\footnote{\covid{}\cite{WagnerOrgelbegleitungKyrialenach1904}.\label{fn:wagner_1904accomps}}
Was this a further ploy to draw Germanic audiences away from Regensburg?
Perhaps, though Wagner was inevitably assisted in that regard by the Vatican which caused Cecilians to lose substantial ground to Solesmes.

Wagner was not alone in advocating the Vatican's new stance on plainchant.
Another advocate was found in the Italian composer Giulio Bas (1874--1929) who, for some twenty years, enjoyed the \emph{de facto} status as pseudo-official harmoniser for Solesmes.
Bas had studied the organ with Marco Enrico Bossi (1861--1925) and counterpoint and composition with Josef Gabriel Rheinberger (1839--1901),\footcites[95]{GiulioBas1907}[42]{ScraperJosefGabrielRheinberger2006} and therefore could hardly have enjoyed a more prestigious professional training.
He became better acquainted with the chant repertory on his appointment as director of a Teano-based Schola Cantorum, and subsequently as \emph{maestro di cappella} at the Venetian Basilica of Saint Mark.
In tandem with the latter post, during the winter of 1902 Bas began composing chant accompaniments which the Turin-based publisher Marcello Capra (1862--1932) published in monthly instalments under the title \emph{Repertorio di melodie gregoriane trascritte ed accompagnate con organo od armonium}.
Each was cheaply priced at 50 \emph{centesimi}.
The instalments were to encompass all the first class feasts in the church year---rather an undertaking to achieve in one go---but the monthly routine afforded Bas the time to compose as he went along.
It also permitted Bas the flexibility to alter his approach when some journalists levelled criticism at his style of accompaniment, as we shall see.

The first of Bas's instalments, containing the Proper for All Saints, was released on 15~October~1902, in good time for the feast at the beginning of November.\footcite[p.~153]{Pubblicazionigregoriane1902a}
It received a positive review in the recently launched, pro-Solesmian periodical \emph{Rassegna gregoriana}, a publication for which Desclée's Roman branch was responsible and of which Bas later became the editor.
Bas's transcription of the chant into modern notation made less use of the \emph{mora vocis} dot of addition than the reviewer was expecting, however, leading to the following comment:

\simplex{La melodia è assai bene trascritta, secondo le regole da noi proposte e seguite. Però non sarebbe stato male introdurre qua e colà qualche \emph{mora vocis} di più, dove il senso logico della melodia pareva richiedere. L'accompagnamento d'organo procede bene, semplice, diatonico.}
  {\emph{Ibid}., no. 11 (November 1902): p.~171}
{The melody is very well transcribed, according to the rules we propose and follow. However, it would not have been bad to introduce a few more \emph{mora vocis} here and there, where the logical sense of the melody calls for them. The organ accompaniment proceeds well, simple, diatonic.}
\nocite{Pubblicazionigregoriane1902}%p.~171
\noindent
Bas's accompaniments for the Feast of the Immaculate Conception were already in print by the time the review appeared, having been released in November, again in good time for the feast day in December.
The third instalment, for the Purification of the Blessed Virgin Mary, followed shortly thereafter and proved acceptable to reviewers in the Milanese \emph{Musica Sacra} and the Turinese \emph{Santa Cecilia}.
Both lavished praise on Bas's style of accompaniment and---in contrast to the \emph{Rassegna}, but with the concurrence of at least one German reviewer---noted the practicality of his transcriptions.\footnote{\emph{Ibid}., 2, no. 1 (January 1903): cols~43--4; \cite[179]{Besprechungen1902}.}
\nocite{R.Pubblicazionigregoriane1903}%cols 43--4

\subsection{Bas's allegiance with Mocquereau}
Positive verdicts on Bas's accompaniments were not forthcoming from those French journalists who were vociferous in defending Solesmian methodologies.
The \emph{maître de chapelle} of Poitiers cathedral, Clément Gaborit, suggested that Bas's `numerous rhythmic faults' resulted from placing chords elsewhere than on the `levé'.
Moreover, Gaborit tried to prove his point by using barlines to analyse Bas's method of placing chords: the two redactions quoted in \cref{mus:bas_gaborit_19} outline two different rhythmic results, the upper showing the rhythm as Bas had treated it, and the lower showing how Gaborit believed it should have been treated.\footcite[p.~19, n.~3]{Gaboritnouveaumanuelgregorien1903}
Since Bas's original accompaniment could not be consulted for the present study, Gaborit's claims must continue to await evaluation.
Yet, it is notable that one contemporary theorist arrived at a similar conclusion to Gaborit's via the same analytical procedure.
Louis Laloy (1874--1944) added the barlines in the passage quoted in \cref{mus:bas_allsaints} to demonstrate, to his own satisfaction if not necessarily to everyone else's, that Bas's accompaniment produced a syncopated effect:

\simplex{Que résulte-t-il de là~? Une mesure syncopée, où le temps fort est réduit à une croche, tandis que le temps faible en a deux~; un rythme brisé, assez familier à notre musique, mais qui surprend dans le chant grégorien, si paisible et si grave.}
  {\cite[547]{LaloyQuelquesmotsrythme1903}; It is likely that, in the process of adding barlines, an accidental is omitted from the second bar.\label{fn:laloy}}
{What follows from this? A syncopated bar where the \emph{temps fort} is reduced to one quaver while the \emph{temps faible} has two; a broken rhythm rather familiar in modern music, but one that stands out in Gregorian chant which is so peaceful and solemn.}
\noindent
Laloy offered an alternative transcription of the same passage (quoted in \cref{mus:laloy_dominum}), opining that the two quavers on the syllable `mi' are thetic and should therefore receive a chord.
That analysis evidently captured Mocquereau's attention who reproduced it in the \emph{Paléographie}.
Mocquereau nonetheless steered clear of voicing his own opinions on harmonic matters, at least in the public arena, opting instead to leave them in the hands of established commentators.\footcite[7:169--70]{MocquereauPaleographiemusicale1901}

Still, Mocquereau did not shy away from holding forth in private, and sought to establish a line of communication with Bas directly.
The Italian dispatched a telegram to Capra to suspend engraving future instalments of the \emph{Repertorio} before the \emph{tête-à-tête} could take place, requesting of Mocquereau that future transcriptions be sent from Solesmes directly so that his accompaniments might better conform to Mocquereau's ideas.\fnletter{Bas}{Mocquereau}{17 December 1902}{\so{}}
Mocquereau did not restrict his recommendations to the transcription of chant melodies alone, voicing several opinions on the matter of accompaniment in a memorandum dated January 1903:
\pagebreak{}

\duplex{Le rôle de l'accompagnement, relativement au rythme est de suivre le rythme de la mélodie grégorienne. L'accompagnement doit marcher du \emph{même pas} qu'elle, s'appuyer où elle s'appuie. La place ordinaire des accords est donc toute indiquée sur les touchements. Mais, étant données la souplesse infinie, la marche, le vol aérien, la spiritualité du rythme grégorien, l'accompagnement est tourjours pour lui un danger~; c'est le revêtir d'une lourde cuirasse. Le plus léger, le plus subtil sera le meilleur. Mieux n'en vaudrait pas du tout. Dans toutes nos grandes exécutions, nous l'avons toujours repousé.}
  {\cite[238--9]{CombeHistoirerestaurationchant1969}\label{fn:moc_memo1}}
{The role of the accompaniment, relative to the rhythm, is to follow the rhythm of the Gregorian melody. The accompaniment must proceed \emph{at the same pace} as the melody, rest where the melody itself rests. The ordinary place of chords, therefore, is entirely indicated in the rhythmic alighting places. But given the infinite suppleness, movement, soaring flight and spirituality of the Gregorian rhythm, accompaniment is always a threat to it; it is akin to cloaking the melody in heavy armor. The lighter and more subtle, the better. No accompaniment would be best of all. In all our major performances, we have always eliminated accompaniment.}
  {Adapted from \cite[209--210]{CombeRestorationGregorianChant2003}\label{fn:moc_memo2}}
\noindent
While it is possible that Solesmian mores had changed since Bellaigue's visit to Saint-Pierre in 1898, one should not overlook the fact that monastic life had been thrown into quite considerable disarray by the community's exile to England.
It is difficult to ascertain whether a harmonium was available to accompany the chanting at Appuldurcombe, but a Mutin-Cavaillé-Coll \emph{orgue de chœur} was installed there around 1903.\footcite[31]{HaleFrenchTreasureIsle2017}
Eliminating accompaniment outside Lent and Advent might therefore have been borne of necessity rather than of a change in doctrine.

We might take the mention of `rhythmic alighting places' to be analogous to those \mbox{\emph{ictuses}} that proved so contentious in Delpech's accompaniments.
Let us not dismiss the possibility that Mocquereau, by establishing contact with Bas, was seeking Delpech's replacement, or at least someone more willing to apply Solesmian rhythm to accompaniments without igniting public opinion.
Sergent had already commenced a harmonisation of the Kyrial from scratch, but Bas warned Mocquereau that the venture was amateurish and could undermine Solesmes's authority:
\pagebreak{}

\simplex{Ce qui vient de Solesmes doit être indiscutablement fort, et votre \emph{Livre d'Orgue} ménace d'être indiscutablement faible, comme l'œuvre d'un amateur maladroit.}
  {\letter{Bas}{Mocquereau}{11 January 1904}{\so{}}}
{What comes from Solesmes must be unmistakably strong, and your \emph{Livre d'Orgue} threatens to be unmistakably weak, like the work of a clumsy amateur.}
\noindent
The task of harmonising the Kyrial was thereafter reassigned solely to Bas, who hoped his involvement in the \ldo{} would extend to harmonising other portions of the chant repertory too.\fnletter{Bas}{Mocquereau}{8 and 10 October 1903}{\so}
He stepped into his new role as semi-official Solesmian harmoniser, and by January 1904 was using Desclée's Kyrial in modern notation as the basis for his accompaniments.
These transcriptions did not always offer answers to his rhythmic questions, however, and Bas continued to probe Mocquereau for further advice.\fnletter{Bas}{Mocquereau}{24 January 1904}{\so{}}

Not only was a newly harmonised Kyrial essential to superseding Delpech's harmonisations, but Mocquereau's evolving ideas on rhythm had rendered the pointing in the \ldo{} obsolete.
In 1904, a distinction was no longer drawn between \emph{arsic} and \emph{thetic} \emph{ictuses}, at least as far as the pointing was concerned, and the colon-like annotation was therefore discontinued, its place being taken by single dots of the kind illustrated in \cref{mus:solesmes_pointing_26}.\footnote{\cite[26]{KyrialeseuOrdinarium1904} (Desclée \textnumero{}~576).}
Comparing them to the \ldo{}'s pointing (of which an example is reproduced in \cref{mus:delpech_caret_47}), we note that, irrespective of the form the pointing took, the placement of \emph{ictuses} underwent few changes.
But even that was set to change as the new chant edition promised by `Nos quidem' modified the chants in subtle ways, the effect of which being discussed in more detail below (\cref{sc:omitted_note}).

Bas acknowledged the need, in tandem with annotative differences, for a stylistic approach predicated on more simplicity.\fnletter{Bas}{Mocquereau}{2 April 1903}{\so{}}
Practically speaking, he seems to have accepted the proposition in Mocquereau's memorandum that the accompaniment should rest when the chant itself rests by anticipating accented notes with such rests.
\Cref{mus:bas_salve} illustrates one of Bas's early forays into applying Solesmian rhythm to his accompaniments, one which would lead to his routinely placing chords on unaccented syllables.
Although the chant was reportedly one of Pothier's fabrications (`la melodia è tutta sua'), the accompaniment could hardly have embodied greater opposition to his opinions.\footcite[cols 179, 181--2]{Salvematermisericordiae1903}
Gaborit's incomprehensible suggestion that syncopation could be avoided by placing chords on \emph{ictuses} led Bas to commit the very error of which d'Indy had warned Mocquereau two years earlier.
The persistent refusal of Bas's accompaniment to engage with the verbal accents is explicable only in terms of the incompatible definitions of \emph{arsis} and \emph{thesis} held by metricians and musicians (on which, see \cref{sc:lhoumeau_effect}).
Nor was the predicament lost of Bas himself, who went as far as to raise with Mocquereau the question of why accompaniments of syllabic chants such as \emph{Victimæ paschali} should not simply observe the verbal accents.\fnletter{Bas}{Mocquereau}{19 January 1903}{\so{}}
Whereas in later correspondence Bas downplayed his concerns as a temporary fit of foolishness, they continued to bubble beneath the surface of his relationship with Mocquereau, leading around 1920 to its foundering.\footnote{\letter{Bas}{Mocquereau}{7 November 1903; 28 March 1906; 10 July and 26 August 1907}{\so{}}.}

In spite of harbouring doubts about Mocquereauvian rhythm, Bas engaged in some propaganda on Solesmes's behalf and became something of an ambassador for Mocquereau's rhythmic theories in Italy.
A pamphlet in Italian was published dealing with chant performance practice according to Solesmian rhythm,\footnote{\covid{}\cite{BasNozionidicanto1904}.} of which the proofs of a French translation bear some marginalia that include the suggestion to Frenchify Bas's forename as Jules.\footnote{See the handwritten `Notions du Chant Grégorien' among Bas's correspondence in \so{}.}
Bas's early thoughts on applying Solesmian rhythm to the accompaniment were aired in an article he contributed to a chant method by the Benedictine monk Gregorio María Suñol y Baulenas (1879--1946).
The events leading to the article's appearance were not without certain complications, however, since it was not included in the original Spanish edition of Suñol's method, but rather in its subsequent French translation.\footcite[Note the absence of Bas's contribution in][195--7]{SunolMetodocompletosolfeo1905}
\hlabel{int:rich_syllabic}%
When the article did appear, it stated that chords were to be placed preferentially on the \emph{ictus} and that so-called rich harmonisations were ideally suited to accompanying syllabic chants---plainer harmonisations were reported to be preferable for accompanying melismatic chants.
\hlabel{int:rich_syllabic_END}%
A separate category was created for what an Anglophone translator termed `festooned melodies', a kind of chant that continually circles back to the same pitch, thus requiring a special type of accompaniment with as few chord changes as possible.\footcite[pp.~153, 158--60]{SunolTextBookGregorian1930}
Bas's private reservations notwithstanding, his public adoption of Mocquereau's ideas was taken by Solesmian apologists as proof that those ideas must be correct, Laloy being among the first to broadcast the matter, followed shortly thereafter by the \emph{Paléographie} which did not pass up the opportunity to claim Bas and the American cleric Norman Dominic Holly as converts.\footcites[547--8]{LaloyQuelquesmotsrythme1903}[7:154--7]{MocquereauPaleographiemusicale1901}

\subsection{Bas's revised accompaniments}
Until his break with Mocquereau, Bas was apparently prepared to secure Solesmes's continued support, nearly any cost.
Was Bas's adherence to Solesmian rhythm financially motivated?
There is no doubt that he aired his financial grievances often in correspondence with Mocquereau.
To make matters worse, the \emph{Repertorio} failed to make a convincing impression on the Italian clergy and was under threat of folding despite the positive reviews it was receiving in the press.
Bas's attempts to drum up more support for the publication amounted to little if anything at all: even though he sent the first instalment free gratis to one hundred Italian seminarians, only three subscribed.
Although the SCR decree of 1894 had banned `theatrical motives, variations and reminiscences', Italian musicians were rather slow to change their customs.
Not even `Nos quidem' sparked enough interest for Italian musicians to interrupt the use of secular music in the liturgy.\footcites[141]{HayburnPapalLegislationSacred1979}[27--8]{JaschinskiRenewalCatholicChurch2010}

The situation led the \emph{Repertorio} into dire straits since Bas could no longer justify financing the project with personal funds.
His monthly organist's salary (reported as 50~F.) only just covered the monthly outlay on printing costs of 30~F.\fnletter{Bas}{Mocquereau}{27 July 1903}{\so{}}
Pleas were placed in the \emph{Rassegna} to attract further subscribers by mentioning that the cost of publication had not yet been recouped (`le spese della pubblicazione non sono coperte per nulla'),\footcite[cols 319--20]{R.RepertoriodiMelodie1903} but that was hardly a convincing advertisement for a venture that did not enjoy much demand, at least not in the domestic market.
Fewer than half of the eighty total subscribers were Italian, and the \emph{Repertorio} instead found a small niche for itself abroad, particularly in those places where chant was sung but where no accompaniments were readily available.
One subscriber, for example, required his or her instalments to be dispatched to faraway Santiago del Chile.\footcite[col.~375--6]{MonteroAmericaLatinaSantiago1903}
Faced with impending financial ruin, then, Bas appealed to Mocquereau for assistance, asking whether Desclée might be convinced to take on the publication.
Mocquereau proved amenable to the request and Desclée began bearing the financial and productive burdens of Bas's accompaniments sometime during the Autumn of 1903.

The transfer provided Bas not only with financial relief but also with the opportunity to revise the accompaniments that had appeared prior to his collaboration with Mocquereau.\fnletter{Bas}{Mocquereau}{1 September 1903}{\so{}}
Among the changes to the Office of the Purification is a chord placed on the second syllable of `Domini' (\cref{mus:bas_gaborit}), just as Gaborit had suggested;\footcite[p.~11]{BasPurificationeMariaeVirginis1904} and among those to the Office of All Saints is the transcription quoted in \cref{mus:bas_laloy} that follows Laloy's suggestion.\footnote{\emph{Ibid}., `Festum omnium Sanctorum,' \emph{ibid}., p.~51}
\nocite{BasFestumomniumSanctorum1904}
These revised accompaniments came to the attention of Heinrich Bewerunge (1862--1923), the Professor of Church Chant and Organ at St Patrick's College Maynooth who, as a staunch opponent of Mocquereau's theories, did not consider Bas's rhythms satisfactory.
Speaking of the Epiphany accompaniment (comprising the first instalment in Desclée's first volume), he queried whether chords should not coincide with accented syllables.\footcite[222, 251--2]{McCarthyHeinrichBewerunge18622015}
Bas's reasons for placing chords on the second syllables of `stellam' and `ejus' (\cref{mus:bas_bewerungecritic}) eluded him.\footcite[5]{BasEpiphaniaDomini1904}
\hlabel{hl:stanbrook}%
The same gripe was communicated to the Stanbrook nun Dame Laurentia McLachlan (1866--1953) which probably informed the brief discussion of accompaniment in Stanbrook's 1905 \emph{Grammar of Plainsong}.
That publication was intended for the Archdiocese of Birmingham, having been requested by archbishop Edward Ilsley,\footcite[209]{MuirRomanCatholicChurch2008} and was among the first manuals to introduce Solesmes's theory of the \emph{ictus} into the Anglophone discourse.\footcite[35, 62--3]{StanbrookGrammarPlainsongTwo1905}
Bewerunge offered no answers to the questions he posed, musing instead: `Is it not truly wonderful what queer things men can do out of theoretical considerations?'.\footnote{\cite[478--9]{BewerungeNoticesBooks1904}.\label{fn:bewerunge}}

Prior to returning to Bewerunge's review, we must acknowledge a potentially thorny issue concerning the order in which Bas's \emph{Repertorio} was published.
We have already observed how three instalments respectively for the feasts of All Saints, Immaculate Conception and Purification had appeared in 1902, and in December Bas advertised the volumes set to appear in 1903 in the following order: 1.\ Purification of the Blessed Virgin Mary (2 February); 2.\ Easter (12 April 1903); 3.\ Ascension of Jesus (21 May 1903); 4.~Pentecost (31 May 1903); 5.\ Corpus Christi (11 June 1903); 6.\ Ss Peter and Paul (29~June); 7.\ Assumption of the Blessed Virgin Mary (15 August); 8.\ Nativity of Blessed Virgin Mary (8 September); 9.\ All Saints' (1 November), 10.\ Immaculate Conception (8~December); 11.\ Christmas (25 December); 12.\ Epiphany (6 January).
The ninth and tenth instalments in that list corresponded to those that had already been published in 1902, a fact Bas acknowledged by following them with the Italian word `uscita', or `released'.\footcite[p.~189]{R.Pubblicazionigregoriane1902a}
To that confusion may be added a further change made to the ordering of instalments: when Desclée took over the \emph{Repertorio}, the first volume was rearranged to place Epiphany as the first instalment.
Hence, when Bewerunge described the Epiphany accompaniment, he reviewed it as the first to appear from the Desclée press.
The entabulated contents of Bas's \emph{Repertorio} in \cref{tab:repertorio} therefore correspond to the Desclée publications, and are not to be confused with any of those instalments printed by Capra.\footnote{Bas later reported having dispatched harmonisations for a ninth series in \letter{Bas}{Mocquereau}{11 November 1909}{\so{}}.}
\nowidow[2]

Bewerunge took further issue with Bas's practice of `leaving a few notes here and there unaccompanied' because, in his view, an accompaniment ought to be unobtrusive.
Bringing in a new chord after a rest would divert the ear from the chant and produce an unsatisfactory effect, so he believed.\footnote{See \cref{fn:bewerunge}.}
But Bas probably owed the passage illustrated in \cref{mus:bas_epiphany_4} to Mocquereau's memorandum, since chords are reserved for accented notes alone.\footcite[4]{BasEpiphaniaDomini1904}
\hlabel{ln:bas_paleo_rests}%
Bas was allotted space in the \emph{Paléographie} to deliver a more detailed explication of his method but failed to note whether the organ was to remain silent during rests or whether the chant was to be accompanied at the unison.
Instead, he proposed various strategies for handling successive \emph{ictuses} which included a method of managing the part writing so that the introduction of parts would coincide with a succession of accents (\cref{mus:bas_paleo}).\footcite[332]{Basrythmeharmonieleurs1905}
\hlabel{ln:bas_paleo_rests_END}%
Bewerunge was in agreement with Bas's explication and conceded that chords should indeed be changed on \emph{theses}, but he also perhaps conjured up the metrician/musician dichotomy when querying where exactly those \emph{theses} occurred:

\single{The natural place for [a] change of harmony is on a thesis, there is no doubt of that. In practice I would, however, allow anticipations and retardations, whenever they are fairly easily intelligible. It is very common in plainchant to have the main note preceded, on the beat, by an appoggiatura. If you bring in your harmony on this appoggiatura you get often very harsh suspensions. But the main question is, where are the \emph{theses}?}
  {Bewerunge to McLachlan, 27 October 1905, cited in \cite[251--2]{McCarthyHeinrichBewerunge18622015}}
\noindent
On the harmonic substance of Bas's accompaniments, Bewerunge could not abide the tendency to harmonise the deuterus cadence `F' \rightarrow{} `E' with D minor \rightarrow{} E minor harmony, preferring deuterus accompaniments to terminate on A minor harmony instead.
His preference illustrates that consensus had not yet been reached on the subject of deuterus harmonisations.
As we have seen (\cpageref{ln:latombelle_deuterus,ln:latombelle_deuterus_END}) La Tombelle had previously aired a similar reservation in connection with deuterus harmonisations in the \ldo{}.
But in contrast to Bewerunge's reservations, Gaborit appreciated Bas's new style, though he admitted that the accompaniments could do with being more `full-bodied' (`plus corsée') for the sake of choral support---perhaps the sparse texture did not agree with him.
Gaborit mentioned that a full-bodied style was being proposed by a certain organist of Strasbourg cathedral, to whose accompaniments we shall turn in due course (\cref{sc:mathias}).\footnote{Gaborit's correspondence to Bas is quoted in \letter{Bas}{Mocquereau}{15 August [n.y.]}{\so{}}.}

\subsection{The Vatican commission}
\label{sc:new_edn}%
Although Italian seminarians proved themselves ambivalent to chant in the early years of the twentieth century, `Nos quidem' was unquestionably a harbinger of a new era in Catholic Church music.
Cardinal Giuseppe Melchiorre Sarto (1835--1914) had kept abreast of Solesmes's researches during the 1890s and committed the first draft of the ground-breaking \emph{motu proprio} `Tra le sollecitudini' (TLS) to paper in 1893.\footcites[186--7]{CombeHistoirerestaurationchant1969}[162]{CombeRestorationGregorianChant2003}
Following his election as Pope Pius X in August 1903 the draft was revised and published on 22 November.
It stipulated that vocal music was to be considered as the music most befitting of the Catholic Church and also made pronouncements on the use of instruments in the liturgy.
An outright ban was placed on pianos, drums and cymbals, and orchestras were only to be permitted with `the explicit permission of the local Ordinary'.
The organ, by contrast, was deemed the church instrument \emph{par excellence}, provided, of course, that it was properly played.
\tls{} permitted preludes, interludes and the like provided that they were appropriately solemn, and also permitted accompaniments provided that they did not drown out the singing:

\duplex{Siccome il canto deve sempre primeggiare, così l’organo o gli strumenti devono semplicemente sostenerlo e non mai opprimerlo.}
  {\cite[\S{}\S{}15--16]{Trasollecitudini22}}
{Since the singing must always be the chief thing, the organ and the orchestra may only sustain and never crush it.}
  {\cite[228--9]{HayburnPapalLegislationSacred1979}}
\noindent
Saint-Saëns repudiated the rationale behind the ban on percussive instruments because, in his view, cymbals and drums could be orchestrated with sufficient decorum to warrant a place in the liturgical orchestra.
He also mused how depictions of such instruments in sacred imagery surely provided ample justification of their retention.
Since \tls{} provided only a general outline for the new musical topography, Fauré was obliged to conclude that applying it to church music would come down to a matter of personal opinion.
D'Indy and Guilmant were nonetheless pleased by the directive, the former because the Schola Cantorum had already gone some way towards adopting it, and the latter because it spelled the end of marching music in the Nuptial Mass.\footcite[183--4]{MurisMotupropriomusique1904}

A further \emph{motu proprio}, `Col nostro', followed in April 1904 which detailed the appointment of a papal commission to oversee the production of that chant book profiled in `Nos quidem'.
The book itself was to be free of copyright so that any publisher irrespective of nationality could disseminate the official chants, provided, of course, that the approved melodies were not altered in any way.
The chants, as prepared by Solesmes, were to be vetted and approved prior to their publication by the commission, led by Pothier.
Although Haberl and his circle were invited to take part, they reportedly declined the invitation.\footnote{\cites[318]{CombeHistoirerestaurationchant1969}[285]{CombeRestorationGregorianChant2003}.}
Combe, who asserted that no German expert responded to the call (`Aucun des Allemands invités à titre d'experts n'avait répondu à cet appel'), remains misleading on this point, since there were indeed Germans who did accept the Vatican's invitation.
Wagner, Raphael Molitor, and Horn joined with French experts including Gastoué and Mocquereau to thrash out the way forward.

The commission met several times in Rome during the spring and summer of 1904, and once again at a seminal gathering at Appuldurcombe hosted by Solesmes from 6 to 9 September.\footcite[256--60]{HayburnPapalLegislationSacred1979}
The Solesmians tried steering the commission towards adopting the \emph{Liber usualis} as the basis for the Vatican's new edition, but experts expressed such doubts about the authenticity of its rhythmic signs that Pothier's \emph{Liber gradualis} was settled on instead.
With Solesmes's entreaties falling on deaf ears some hubbub erupted in the months that followed, leading to Delatte's withdrawing from the commission outright, followed shortly thereafter by Mocquereau who tendered his resignation on 17 July 1905.
The task of preparing the Vatican Edition continued in spite of those departures, without Solesmes's direct involvement.\footcite[109--111]{EllisPoliticsPlainchantfindesiecle2013}

While Solesmes had transferred the copyright of the chants to the Vatican,\footcite[206]{MuirRomanCatholicChurch2008} the same was not done for the rhythmic signs.
There had been a danger since at least 1895 of other publishers swooping in to produce chant books in the same typeface Solesmes had designed for its own use: Wagner had warned of the potential for confusion if Pustet were to have started down that path.\fnletter{Wagner}{Mocquereau}{28 December 1895}{\so}
\hlabel{ln:deberny}\label{sc:deberny}%
Solesmes therefore maintained a jealous guard over the type used to print its rhythmical signs.
As we have seen (on \cpageref{ln:deberny_impetus,ln:deberny_impetus_END} above) the type was forged neither at Solesmes nor by Desclée, but by a third party, the Parisian type foundry Deberny~\&{}~C\textsuperscript{ie}, which advertised a special `Casse de plain-chant' in a splendid brochure showing off excerpts from the \emph{Liber gradualis} with a kind of double-impression printing.
Black neumes are set on red rastrations in a display of the five available point sizes: \linebreak{}32, 40, 48, 84 and the gargantuan 120.\footcite[part iv, pp.~210--14]{DebernyCielivrettypographiquespecimen}
The last was probably intended not for chant books but for prompt sheets displayed in a prominent location to remind an ensemble of singers of common chants without their needing to find the relevant page.
One such sheet of responsories in Pustet's characteristic notation was displayed above the organ console in Regensburg cathedral in the early years of the twentieth century.\footnote{See the photograph cited in \cref{fn:renner_photo}.}

\hlabel{ln:solesmes_type}%
As Mocquereau's theories of chant rhythm evolved so too did the requirements for typographical symbols, and Deberny was tasked with forging the relevant type necessary to print them.
Solesmes put stringent controls in place to thwart potential pirates; so stringent, in fact, that even Desclée's Roman branch was unable to purchase \emph{episemata}, \emph{orisci} or rhythmical dots for the \emph{Rassegna} without Bas requesting the necessary permissions from Mocquereau.\fnletter{Bas}{Mocquereau}{8 February 1904}{\so{}}
The situation was vexing to one Jules Combarieu who, in a review of the \emph{Liber usualis}, complained that he could not provide music examples since he did not have access to the type required to print them.\footcite[93--4]{EllisPoliticsPlainchantfindesiecle2013}
\nowidow[2]

\hlabel{pg:scr_signs}%
The Vatican commission approved for publication the first extract of its chant edition in 1905 which comprised the Kyrial, but prior to discussing it in more detail we must first evaluate Desclée's version with added rhythmical signs.
It had potentially broken the clause in `Col nostro' forbidding editors to alter the chants, causing some consternation that prompted the commission to weigh in on the matter.
It affirmed that signs could indeed be added to versions of the Vatican Edition,\footcites[286]{CombeHistoirerestaurationchant1969}[253]{CombeRestorationGregorianChant2003} a finding corroborated by the SCR which confirmed the legality of Desclée's publication.\footcite[158--9]{BergeronDecadentEnchantmentsRevival1998}
But sufficient confusion continued to abound for the American organist Caspar Petrus Koch (1872--1970) to take up the matter with Pothier's monastery directly.
Koch had been steeped in the Cecilian tradition, having received his musical training first at the Amerikanische Cäcilien-Verein under Singenberger (at Saint Francis College, Wisconsin) and later at the Regensburg Kirchenmusikschule.\footnote{\cite[198--200]{SkerisMusicasacraArchdiocese2010}; A discrepancy in Koch's educational history is evident in \emph{Grove Music Online} which states he attended Saint Francis College, Joliet, Illinois. See \cite{GotwalsKochCaspar}; For descriptions on his passage to Regensburg and to the Kirchenmusikschule, see Koch's obituary in \emph{The Pittsburgh Press}, 25 June 1933, p.~20.}
The Saint-Wandrille monk Dom Lucien David (1875--1955) responded on Pothier's behalf, noting that the SCR had done little more than to permit the signs `invented by Dom Mocquereau'.
David noted, however, that \emph{episemata} had no basis in historical fact and that neither the SCR nor the Vatican commission had explicitly approved their use.
They had also steered clear of drawing conclusions on chant rhythm and performance practice.\footnote{Dom Lucien David to Caspar Petrus Koch, 20 December 1906, original reprinted in \cite[46--7]{editionsrythmiquesSolesmes1921}; Translation and discussion in \cite[277--9]{HayburnPapalLegislationSacred1979}.}
\noclub[2]

\subsection{Mathias's graduated stages}
\label{hl:mathias}\label{sc:mathias}%
While the Vatican Kyrial was at an advanced stage of preparation, the commission elected to convene a congress of international experts to consider the implications of \tls{} on chant performance practice.
Haberl believed that such a gathering was premature because, at the time, the new chant edition had not yet been published; Pothier did not approve it for publication until the congress was underway.\footcites[419, 421]{CombeHistoirerestaurationchant1969}[372, 374]{CombeRestorationGregorianChant2003}
Haberl was nonetheless among some fifty-six members of the organising committee that convened the congress in Strasbourg between 16 and 19 August, and at which the debates on a wide range of chant-related topics were considered in Francophone and Teutophone sessions.\footcites[7--8]{VogeleisFestschriftInternationalenKongress1905}[Report referenced in][258]{GeyerviemusicaleStrasbourg1999}
The tenor of their debates on chant accompaniment will be explored in the following paragraphs, but it should be noted that the topic was considered separately by each cohort, and as a result the delegates attending a session conducted in French could not necessarily make their opinions known in the parallel session conducted in German.
Any bias arising in one session and conflicting with ideas raised in the other might be recognised as a side effect of this crude division along linguistic lines.

The organist of Strasbourg cathedral mentioned by Gaborit, Franz Xaver Mathias (1871--1939), chaired the Teutophone session.
He outlined a system of his own design that attempted to codify how a player might handle greater or lesser accents by using greater or lesser motion in the accompanying parts.
In short, a greater accent required either a greater amount of motion or an excursion to a harmonically remote chord.
A lesser accent, by contrast, required less motion and for the harmony to remain static.\footcite[75]{VogeleisFestschriftInternationalenKongress1905}
In one way, Mathias's system may be considered analogous to Mocquereau's theory of chant rhythm because it too analysed the chant to determine a codifiable method of performance.
In another way, however, Mathias's system diverged from theories of rhythm because it depended on them for the purposes of the accompanist alone.

The system was codified in nine ascending stages of part movement, each being designed to mark a greater accent than the last.
They were first outlined in a series of journal articles appearing between 1902 and 1903 and are described briefly below---the reader is invited to consult Mathias's examples quoted in \cref{mus:mathias_graduated} in conjunction with the description of each stage.
The first three stages concern parts moving between chords that share the same harmony: first, a single inner part moves to another note in the same chord; second, several inner parts move in like manner or a bass part traverses the interval of an octave; and third, a bass part moves to a different chordal note producing, say, the progression `A' 6/3 \rightarrow{} `F' 5/3.
The next three stages concern parts moving between chords of different harmonies: fourth, by inner parts; fifth, by the bass part; and sixth, by most parts in conjunct motion.
The final three stages concern more energetic motion: seventh, by most parts moving to a new harmony in disjunct motion; eighth, by changing to a chord that is not necessarily harmonically related; and ninth, by using dissonances such as suspensions and anticipations.
Should such dissonances occur in inner parts, they are used to smooth over certain chord changes, Mathias classifying them as dampening, blunting and hardening (`Abdämpfung, Abstumpfung, Verbreterung de Akkordwechsels').
But should they occur in the top or bottom parts, the effect is said to be quite different, Mathias claiming that they produce the strongest accents of all which are sharp and cutting (`scharf und schneidend').\footnote{\cites[62--4]{MathiasChoralbegleitung1903}[39--41]{MathiasChoralbegleitung1905}.\label{fn:mathias_choralbegleitung}}
Even unprepared dissonances may form part of the ninth stage, a subject Mathias also discussed during the congress:

\simplex{Da unter den Fachleuten auch Meinungsverschiedenheiten über den Wechzel zwischen Konsonanzen und Dissonanzen in der Choralbegleitung obwalten, erörterte Dr.\ Mathias auch diese Frage~; er hielt unvorbereitete Dissonanzen nicht bloß für berechtigt, sondern bezeichnet sie als dem Charakter des Chorals geradezu entsprechend.}
  {\cite[170]{H[orn]internationaleKongressfuer1905}}
{Since there are also differences of opinion among the experts about the alternation between consonance and dissonance in chant accompaniment, Dr.\ Mathias discussed this question also; he not only considered unprepared dissonances to be correct, but described them as being almost in keeping with the character of chant.}
\noindent
Perhaps that might explain why, at the second syllable of `magnam' quoted in \cref{mus:mathias_gratias}, the first quaver is treated as an accented passing note.
It is perhaps an example of the `scharf' dissonance, even though the pitch class is prepared in the preceding chord's bass part.
The tenor note \emph{g} on the same syllable becomes more like the `Abdämpfung' dissonance, not solely because it is relegated to an inner part but also because at the second quaver of the syllable `nam' it becomes dissonant.\footnote{\covid{}\cite[16]{MathiasOrgelbegleitunggebraeuchlichstenMess1903}.}
\nowidow[2]

Mathias's dizzying system seems to unravel somewhat at certain cadences, where he seems more intent on striking a dissonance than resolving it on a given chant note.
The tenor part at the end of the line illustrates one such one example, owing to its resolution's coinciding with neither a new note nor a new syllable.
The cadence might instead be emblematic of so-called beautiful cadences (`schönen Kadenzen'),\footcite[124--5]{MathiasChoralbegleitung1903a} though it is necessarily difficult to judge how Mathias managed the inevitable tension between his nine stages and greater aesthetic endeavours.
\hlabel{hl:chassang}%
He was not alone in attempts at codifying dissonance, however, and an attempt at demarcating \emph{ictuses} was trialled at around the same time by Pierre Chassang (1855--1933), then \emph{maître de chapelle} of Avignon's minor seminary.
Although Chassang's accompaniment quoted in \cref{mus:chassang_dissonances} is not really in compliance with theories of free rhythm (because it admits triplets in the transcription), Chassang nevertheless appears disposed to treating each \emph{ictic} note as a dissonance by the underlying chord changes.\footcite[110]{ChassangManuelaccompagnateurchant1904}
Later, Chassang admitted that \emph{ictuses} need not always require a new chord,\footcite[21]{Chassangaccompagnementchantgregorien1917} but his method nevertheless joined Mathias's as being among the first to use dissonance to demarcate points of rhythmical activity.

\label{sc:gastoue}%
Amédée Gastoué, the chant teacher at the Schola Cantorum,  recognised the merit in Mathias's graduated stages and considered them to have great practical potential.\footcite[p.~80 with attribution to Gastoué on p.~81]{GastoueRevueChoralbegleitung1905}
Using them to arrive at an acceptable accompaniment remained a challenge, however, because their use depended on the verisimilitude of the associated rhythmic theory.
Should the theory prove faulty, then any accompaniment based on it would also find itself vulnerable.
Gastoué's gripe with Mathias's examples did not concern the stages themselves but rather the faulty chant edition that Mathias had used in his illustration of them, this being the chant book in use at the diocese of Strasbourg (`sur les éditions fautives en usage au diocèse de Strasbourg').\footcite[385--6]{GastoueBibliographie1904}
The transcription of its chants into modern notation was also called into question owing to its being based on a proprietary rhythmic scheme devised by the cathedral's \emph{maître de chapelle}, abbé Joseph Victori (1871--1935).\footnote{For a description of Victori's involvement in the musical life at Strasbourg, see \cite[264--5]{GeyerviemusicaleStrasbourg1999}; Victori's dates of birth and death are noted in \cite[55]{Periodiques1936}.}
Victori's scheme seems to be quite arbitrary and was apparently not described anywhere, thus diminishing the pedagogical value of any accompaniments Mathias based on it.

The Strasbourg congress therefore provided Mathias with an opportunity to rectify vulnerabilities in his previous adhesion to Victori's scheme.
New accompaniments of Mass chants were recorded for posterity in a pamphlet published by Pustet, whose foreword credits the \emph{Paléographie musicale} with supplying the rhythmic framework.
Mathias revealed that \emph{melismata} were beamed according to where \emph{ictuses} were marked, and the player was advised to lengthen notes immediately preceding \emph{quilismata} to produce a beautiful and light effect (`schönste und leichteste').
The accompaniments encompassed the feasts that had occurred while the congress was in progress, each day's proceedings having been anticipated by chanted High Mass.
The feasts included the Octaves of Saint Lawrence and Assumption, but a miscellany of other accompaniments were also included in the pamphlet which were probably intended for the lecture-recitals (`Praktische Übungen im Chor Vortrag') where experts demonstrated various different styles of chanting.\footcite[11]{VogeleisFestschriftInternationalenKongress1905}
Some parts of the Ordinary were also included, and by comparing \cref{mus:moc_gloria_bonitatis} (Mocquereau's method of pointing in 1904) to \cref{mus:mathias_moc_39} (Mathias's method of harmonising in 1905), one notices how chords were placed at almost every dot marking an \emph{ictus}, a similarity that is perhaps too great to be explained away as mere coincidence.\footcites[unpaginated `Vorwort', pp.~3, 39]{MathiasOrgelbegleitungfuerInternationalen1905}[12]{KyrialeseuOrdinarium1904}

Mathias also seemed to reinforce where \emph{ictuses} occurred by the arrangement of parts on the staff.
There is some evidence to suggest that he preferred shorter, tied notes to longer ones to make \emph{ictuses} more obvious to the player, even if the difference might not have been obvious to a listener.
The bare octaves at `Cum Sancto' almost hark back to the Cecilian practice discussed above (\cpageref{ln:cecilian_octaves}) and deserve some consideration because there is also evidence that Delpech used them in the \ldo{}.
In Delpech's case, the technique might have been inherited from Wagner who also used it as a cliché to provide relief from a persistently four-part texture or when his capacity for harmonic invention failed.
Whatever the reason for their retention, bare octaves continued to be a useful weapon in the arsenal of Teutophone accompanists in particular, and we shall return to another instance of Wagner's use of them below (\cref{ln:wagner_bareoctaves_jubilo}).

In light of Mathias's pledge to follow Solesmian rhythm, the \emph{Paléographie} labelled him a `convert'---recalling Bas and Holly---and reproduced music examples from the Pustet pamphlet without describing anything about his stages.
It is therefore most unlikely that the \emph{Paléographie} enlightened anybody as to Mathias's attempt at establishing a connection between Solesmian chant rhythm and harmony.\footnote{\cite{MocquereauPaleographiemusicale1901}, 7:336--41.}
The foreword to Mathias's pamphlet is dated 21 July 1905, a mere three days after Mocquereau's resignation from the Vatican commission which probably churned up controversy among the congressional delegates.

The session parallel to Mathias's took a decidedly anti-Solesmian stance: it was chaired by Gastoué who later dismissed `la nouvelle école de Solesmes' outright,\footcite[For a discussion of Gastoué's criticism of placing chords according to Solesmes rhythm, see][364--5]{Lessmannanachronismemusicalaccompagnement2019} showing himself to be critical of placing chords on \emph{ictuses}.
In Gastoué's opinion, chords were to be placed instead on the first notes of neumes.\footcite[35--6]{VilletardCongresinternationalchant1905}
But the view ruffled some delegates' feathers and one even took to the floor to argue Mocquereau's case.\footcite[167--8]{BrenetCongresinternationalchant1905}
The English Benedictine Thomas Anselm Burge (1846--1929) witnessed the exchange first-hand, and identified Gastoué's interlocutor as none other than Bas.
The Italian's protests failed to stir the other delegates to his side, however, before a frosty Gastoué brought the session firmly to a close.\fnletter{T[homas] A[nselm] Burge}{Delpech}{10 October 1905}{\so{}}

\section{Accompanying the Vatican Edition}
\subsection{Publishers: their œuvres and manœuvres}
Solesmes's version of the Vatican Kyrial was one of at least thirty-two in circulation by the Autumn of 1906.\footcite[The numerals adjacent to Kyrials in the \emph{Revue du chant grégorien} serve as a sort of index of the published versions. See][p.~29 and \emph{passim}]{Bibliographiegregorienneeditions1906b}
The extent of certain publishers' interests in the matter did not stop at producing chant books, for the possibility of driving sales led some to publish complementary accompaniment books.
Congresses provided networking opportunities for publishers to solicit the necessary harmonisations from experts, and it was under these circumstances that Henri Delépine (1871--1956), the Arras-based priest and founder of the publishing house La Procure générale de musique religieuse, approached Wagner.

Delépine was not alone in employing the tactic: Capra had established himself at the centre of Turinese chant-based deliberations with the periodical \emph{Santa Cecilia}, gaining for himself a commercial foothold in the chant restoration movement there.\footcite[22--5]{Carolimusicasacraperiodici2017}
Capra convened a conference of his own in the same city from 6 to 8 June 1905, making himself its secretary and entrusting his own printing house with publishing the official congressional report.\footnote{\covid{}\cite{AttiVIIcongresso1905}.}
One commentator noted the obvious conflict of interest but admired the report all the same for its laudable impartiality (`con lodevole imparzialità').\footcite[852]{ReviewCapraAtti1905}
Capra's conference hosted a discussion of accompaniment led by Bas,\footcite[47]{ReviewCapraAtti1905a} so it seems that transferring the \emph{Repertorio} to Desclée had not affected Bas's standing with his fellow countryman.
That such conferences were convened purely for commercial interests was something of an open secret.
When another was convened years later (ostensibly to discuss church music style), it was recognised as being a convenient advertising platform for publishers to advertise their wares.\fnletter{Henri Potiron}{Joseph Gajard}{n.d.}{\so{}}
Nonetheless, the commercial interests at play had little bearing on the relationship between publishers and chant experts, which, by all accounts, proved to be a symbiotic one.
\nowidow[2]

Wagner completed his accompaniments for Delépine at extraordinary speed, sending off the first tranche less than a fortnight after the Strasbourg congress was brought to a close and the remainder a day later.\fnletter{Wagner}{Delpech}{18 and 28 August 1905}{\so{}}
It has not been possible to ascertain whether these accompaniments constituted a simple rehashing of those published in 1904,\footnote{See \cref{fn:wagner_1904accomps}.} but whatever the facts as to their origin, the combination of Delépine's shrewd business acumen and Wagner's celerity made Arras the first publisher to bring an accompanied Vatican Kyrial to market.
Given Wagner's views discussed above in connection with Mocquereauvian rhythm (\cpageref{ln:wagner_viewsonmoc,ln:wagner_viewsonmoc_END}), it is hardly surprising that Wagner avoided the \emph{Paléographie musicale}'s pronouncements on rhythm in his preparation of his accompaniments.
He did not place chords on the second notes of \emph{salici} or \emph{scandici}, and surely had Mocquereau in mind when castigating some of Pothier's rhythmic `disciples' for their arbitrary meddling:

\simplex{Le R\textsuperscript{me} Abbé de S\textsuperscript{t} Wandrille a eu des disciples qui n'ont pas compris que le très grand mérite de leur Maître était précisément de n'avoir pas de système. Ils ont cru pouvoir ajouter quelques nouveautés à son enseignement; malheureusement, s'il en est qui peuvent être utiles, d'autres sont manifestement dangereuses et arbitraires.}
  {\cite[p.~iii]{WagnerOrdinariumMissaejuxta1905}}
{Father Abbot of Saint-Wandrille had followers who did not understand that the very great merit of their \emph{Maître} was precisely not to have a system. Those followers thought new elements could be added to his teaching; unfortunately, if some of them are potentially useful, others are clearly dangerous and arbitrary.}
\noindent
The book's preface, dated 30 October 1905, couched Wagner's methodology in plain terms: chords were placed on the first notes of neumes and, presumably for the sake of variety, different accompaniments were provided when the chant was to be repeated, such as at `Kyrie eleison'.

One journalist recognised Wagner's accompaniments as being simple enough for less practiced organists to navigate, though in some places that simplicity reportedly made the harmonisation a bit lean (`un peu maigre').\footcite[77]{Bibliographiegregorienneeditions1905}
Simplicity was evidently the watchword because composers could not assume that the musicians taking up their books would know how to handle them.
Moreover, the style of organ playing adumbrated by \tls{} was to be strictly ecclesiastical, free from profane and, specifically, theatrical characteristics, and could well have informed compositional decisions to dispense with gaudy superfluities.
\label{ln:wagner_bareoctaves_jubilo}%
It could explain why some of Wagner's harmonisations flit between two, three and four audible parts while others commence in bare octaves (\cref{mus:wagner_jubilo_42}).\footcite[42]{WagnerOrdinariumMissaejuxta1905}
Although Wagner's book benefited from being the first of its kind to market, it did not enjoy that unique position for very long.
Three more had appeared by the beginning of 1906, one by the Belgian trio Desmet, Desmet and Depuydt, a second by Horn and a third by Mathias, which we shall discuss in turn.

One of the most striking aspects of the Belgian book concerns the adoption of filled-and-void notation, as popularised at the Lemmens Institute.\footcites[The Lemmens Institute opened new buildings on 5 November 1903 to celebrate its twenty-fifth anniversary in October. See][66]{Guillaumeproposmusiquereligieuse1906}[p.~xviii]{GodenneMalinesjadisaujourd1908}[The school's twenty-fifth anniversary has been erroneously placed in 1908 in][18]{RobijnsJaakNikolaasLemmens1981}
\label{sc:desmet_1906}%
As we have seen (\cpageref{ln:desmet_dupuydt}), Aloys Desmet had notated rests using crotchet rests instead of obliques in 1892, but in 1906 obliques were common, along with some other notational novelties.
The different melodic groups in the chant book were set apart from each other laterally on the staff and liquiescent neumes were placed in parentheses (as illustrated in \cref{mus:desmetdupuydt_notation_97}).
The wedge-shaped glyph was intended to indicate a \emph{mora vocis} and suggests that Mocquereau's edition might have had some influence on their approach.\footcite[unpainated approbation pp.~5, 97]{DesmetCommunepluriumconfessorum1910}
The three harmonisers provided two sets of cadences---one diatonic and the other sharped---for certain deuterus chants (`des finales altérées et non altérées'), no doubt to avoid prejudicing warring factions against their accompaniments.
The book was well received by a reviewer who predicted that the trio's efforts would `continue to enhance the merit of Mechelen's École de musique religieuse' (`qui rehaussera encore le mérite de l'école de musique religieuse de Malines').\footnote{\covid{}\cite{DesmetOrganumcomitansad1906}; \cite[391]{SwolfsBulletinbibliographiqueinternational1906}.}
And in the opinion of another reviewer, the accompaniments were simple enough for novice organists to play, and choirs would benefit from the unobtrusive organ part.\footcite[117]{Bibliographiegregorienneeditions1906a}
In a letter of approbation dated 25 April 1907, the then Cardinal Archbishop of Mechelen, Désiré Joseph, declared the Desmet-Desmet-Dupuydt accompaniments as being in conformity with Pothier's theory of rhythm, and recommended therefore that they be adopted in his diocese.
The declaration anticipated by three years the outright banning of the pipe organ from participating in Belgian Low Masses, from 1~January~1910.\footcite[80]{NouvellesmusicalesBelgique1910}

In contrast to the Belgians, Horn indicated neumatic groups with slurs and reportedly abandoned his earlier \emph{mise-en-page} (quadratic notation surmounting the accompaniment) for filled-and-void notation.
He apparently followed Mocquereauvian rhythm which required him to signal \emph{morae vocis} dots of addition by adding stems to certain noteheads,\footcite[14:116--117]{Bibliographiegregorienneeditions1906a} and accents by the use of carets.\footnote{\covid{}\cite{HornOrganumcomitansKyriale1906}.}
A reviewer was complimentary of the layout, but took issue with the thirty-seven percent of Horn's accompaniments that were transposed; allegedly, the transpositions made singing from the official chant book an impossible task.
The same reviewer also bemoaned a lack of registration indications and suggested that Horn might include them in a revised edition.\footcite[cols 245--6]{MantuaniKunstundKunstgeschichte1906}
Whether or not Horn was made aware of the review is not certain, though for future editions he did not accommodate the reviewer's suggestions.
The third edition was not entirely as the reviewer had described the first, however, particularly since the accompaniments were notated in ordinary quavers and not in filled-and-void notation.
Perhaps the caret symbols quoted in \cref{mus:horn_kyriale_1} were supposed to represent certain accents; Horn provided no relevant words of explanation.
He nonetheless permitted his inner parts a certain amount of contrapuntal freedom, which, in contrast to Wagner's, were rather more disjunct.\footcite[1]{HornOrganumcomitansKyriale1932}

\subsection{The double-signature method and polemics on diatonicism}
\label{sc:transpositions}%
Not only were composers of accompaniments required to bear in mind accessible vocal ranges for amateur choirs, but they also had to ensure that their accompaniments did not stray into a part of the keyboard's range where the texture became muddy or unclear.
Balancing the practical potential of a choice of transpositions with the added cost of a larger volume was surely a reason why so few accompaniment books offered any such choice.
Bas was among the first to offer a selection of transpositions in the second volume of the \emph{Repertorio}, where two signatures are notated on some staves (\cref{mus:bas_angelis_repertorio_15}),\footcite[15]{BasMissaAngelis1904} illustrating for the first time the procedure described by Haberl some years earlier (see \cpageref{ln:haberl_transposition,ln:haberl_transposition_END} above).
Since both signatures had to apply to the same staff notes, the resulting transpositions lay necessarily a chromatic semitone apart.
\emph{Ficta} accidentals applying to the primary signature were placed before the relevant notes in the usual way; those applying to the secondary signature were placed above or below.
Novel though the method may have been, it provided only for the smallest possible variation in transposition, and merely spelled out an inherent general capacity of staff notation that---to all except the most inexperienced of organists---ought to have been glaringly obvious.
Bas made no mention of his transposition gimmick in the advertisements viewed by the present author, but touted instead the convenience of his harmonisations for less practiced musicians.
He stressed that his accompaniments were simple and easy to play, which (along with the addition of dynamics) reportedly resolved the `difficult problem of accompanying' (`risolve il difficile problema dell'accompagnamento').\footcite[col.~154]{Pubblicazionigregoriane1904}

Mathias's Strasbourg pamphlet led to his writing an accompanied Kyrial, in which he too offered a choice of two signatures.
Unlike Bas, however, Mathias did not indicate secondary accidentals, and left it to players to infer the relevant one for themselves.
Given that his harmonisations are diatonic, Mathias required players to do so only for the transposed equivalent of `B'\kern 1pt\flat{}, deeming no words of explanation necessary in the book's preface (\cref{mus:mathias_transposition_44}).
The double-signature method offered several advantages and disadvantages.
First, Mathias could include transpositions without increasing his book's page count.
Second, relatively inexperienced players were equipped to offer a choice of lower and higher options to a choir.
His accompaniment book was certainly intended for such players and included several `easy cadences' so that portions of recited text could be brought to a conclusion in a simple idiom.
At the same time, no precedent was set for providing a second signature when the first consisted of neither sharps nor flats; we might settle the matter now by assuming Mathias considered seven-sharp or seven-flat signatures beyond the technical abilities of amateur players.
Conceivably, such signatures could have been notated in parentheses, leaving players to grapple with the tacit presence of all-natural signatures.
A further disadvantage concerned the necessarily limited set of transpositions by a chromatic semitone up or down.
To fulfil a promise that his harmonisations would `suit the compass of all voices', Mathias reprinted select accompaniments at different transposition levels, labelling the choices `a', `b', and so forth.
Nowhere is Mathias more verbose in offering alternatives than in an appendix containing responses to `Ite missa est' and `Benedicamus Domino' where each is iterated up to four times.
Many of the iterations are also provided with secondary signatures.\footcite[pp.~29--30, 44, 5*]{MathiasOrganumcomitansad1906}

In the wake of the Vatican Kyrial, demand swelled not just for fully notated accompaniment books but also for methods of accompaniment.
It is hardly a coincidence that Mathias's articles describing his nine stages of part movement were published in textbook form in 1905,\footnote{See \cref{fn:mathias_choralbegleitung}.} as well as in a French translation.\footcite[p.~`a' n.~1]{LeGuennantVademecumparoissial1910}
This new textbook was also translated into English by Bewerunge in 1907, but was not widely disseminated, if at all.
The copy extant at the Russell Library in Maynooth up to 1993 is now no longer accounted for.\footcite[p.~252 n.~191]{McCarthyHeinrichBewerunge18622015}
An Anglophone translation of Niedermeyer's \emph{Traité} appeared in 1905, intended for the benefit of the English and American Catholic markets.
On the grounds that Niedermeyer `treats of plainsong accompaniment, and not of ritual', the translator suggested that the book might also be of interest to Anglican musicians.\footcite[pp.~iii--iv]{NiedermeyerGregorianAccompanimentTheoretical1905}
\nowidow[2]

While the diatonic approach was gaining more prevalence in the English-speaking world, its tenets were not universally accepted in some parts of Europe.
As we have seen, Desmet, Desmet and Dupuydt were not alone in offering alternative sharped cadences, and Mathias also permitted those who found diatonicism too `crude' to make whatever chromatic adjustments to his harmonisations that they wished.\footcite[unpaginated introduction]{MathiasOrganumcomitansad1906}
Mathias was drawn into a polemic on diatonicism by the priest and Cecilian composer Franz Nekes (1844--1914), whose arguments against diatonic harmony stemmed from a certain unwillingness to depart from the myth of Palestrinian authority, a mare's nest that nevertheless continued to beguile Cecilian composers long after the Vatican had sided with Solesmes.
Nekes pitted himself against the Vatican's decrees that relegated the accompaniment to a status beneath that of the chant.\footcite[564]{T.ReviewNekesKyriale1906}
Just how starkly he deviated from \tls{} may be understood with respect to Wagner's view on the matter, who held that at points where melody and harmony were in conflict, the former was always to prevail.\footcite[p.~v]{WagnerOrdinariumMissaejuxta1905}
By contrast, Nekes argued that accompaniments were works of art in their own right and should therefore be granted equal status.\footcite[104]{NekesUeberChoralbegleitung1904}
The reluctance to follow those of his peers who capitulated to navigating the Vatican's new musical topography left Nekes increasingly marginalised as the twentieth century progressed.\footcite[83--4]{WagnerFranzNekesund1969}

In comparing Nekes's harmonisations to Mathias's, the \emph{CVK} was predictably rather complimentary of the former's approach.\footnote{\cvk{3388}.}
Other enclaves of the German press were not so forthcoming with praise, however, and tended to side with Mathias.\footcite[156]{Literarisches1905}
In spite of those rebuttals, Nekes remained committed to chromaticism long after the dust had settled, using many more sharps in deuterus harmonisations than French critics were willing to accept (\cref{mus:nekes_sharps}).\footcite[94]{NekesKyrialesiveOrdinarium1912}
One such complained that many truly bizarre cadences (`plusieurs cadences véritablement bizarres') were not in keeping with the Gregorian \emph{tonalité}.\footcite[178--9]{Bibliographiegregorienneeditions1906}
Another took the same stance when reviewing a later accompaniment, though noted without irony that conventional harmonic rules were well observed, and concluded that Nekes's work obviously proceeded from a good musician and was not merely the result of a hapless Gregorianist straying beyond his domain.\footcite[99]{ReviewNekesMissae1910}
Indeed, Nekes maintained that one ought to be a good composer prior to turning to compose accompaniments,\footcite[104]{NekesUeberChoralbegleitung1904} but his musical aptitude did little to sway ardent diatonicists to his side.

Nekes was not alone in maintaining a preference for cadential sharping.
The Belgian composer François Johanns also used sharps, particularly in dominant \rightarrow{} tonic progressions of the type quoted in \cref{mus:johanns_50}.
Several other features of Johanns's style are notheworthy too, such as the notating of certain chant notes in small type so that by means of their omission the chant could gain independence from the accompaniment.
Some phrase endings were marked with `rall', while others were followed by a comma above the next barline to indicate that the value of a `temps faible' was to be added to the note preceding the barline.\footcite[pp.~iii--iv, 50]{JohannsAccompagnementsKyrialeou1909}
Despite maintaining sharping in their accompaniments, Nekes and Johanns admittedly drew short of the type of chromaticism we have observed in some nineteenth-century accompaniments (compare, for instance, \cref{mus:wincenty,mus:wincentysixth}), though a revival of that genre of chromaticism was trialled at the beginning of the next decade and will be considered in the next chapter.

\subsection{The revising of obsolete Solesmian accompaniments}
As we have seen (\cpageref{pg:scr_signs}), the Vatican commission had voiced its tolerance for the rhythmical signs Solesmes had added to its versions of the Vatican Kyrial.
Desclée published no fewer than three versions with rubrics in Latin, English and French, while a fourth presented the chant in modern notation.
But subtle changes to certain chants rendered Solesmes's previous chant books and their associated accompaniments obsolete.
The updated chant quoted in \cref{mus:solesmes_vatican_28star} omitted one note from the third syllable of `Kyrie' in conformity with the Vatican Edition.\footcite[28*]{KyrialeseuOrdinarium1905}
Solesmes had no option other than to contend with the Vatican commission's approved revisions which made the accompaniments shown in \cref{mus:delpech_angelis_34,mus:bas_angelis_repertorio_15} out of date.
A review of Bas's accompanied Kyrial commented that he had been obliged to revise the accompaniments previously published in the \emph{Repertorio} for this very reason,\footnote{See \cref{fn:bibkyriale}.} and the fruits of his labours in that regard may be recognised in \cref{mus:bas_angelis_kyriale_40} where the same note was omitted.\footcite[40]{BasKyrialeseuordinarium1906}
Incidentally, Bas now placed secondary accidentals in parentheses, thereby creating a stronger semiotic link to the similarly parenthesised secondary signatures.

Given that Solesmes's versions were based on the Vatican Kyrial, Desclée was technically correct to advertise Bas's accompaniments as conforming to the Vatican Edition, when in truth they were well known to be based directly on Solesmes's rhythmical editions, as one wry quip in the \emph{Revue du chant grégorien} attests:

\simplex{M.\ Bas se conforme toujours, pour les appuis harmoniques, aux indications rythmiques des éditions rythmées de Solesmes, dont il s'est fait, on le sait, le champion.}
  {\cite[178]{Bibliographiegregorienneeditions1906}\label{fn:bibkyriale}}
{Concerning the harmonic stresses, Mr~Bas follows as ever the rhythmic indications in the rhythmed editions of Solesmes, of which he makes himself, as we know, the champion.}
\noindent
Be that as it may, there is another notable difference between Bas's new Kyrial accompaniments and those which had appeared in the \emph{Repertorio}, concerning where chords were placed.
\label{sc:omitted_note}%
Could the omitted note have changed Mocquereau's interpretation of the chant's rhythm?
Perhaps, though Desclée's version in modern notation was not available for the present author to evaluate the hypothesis.
\hlabel{ln:liber_138}%
A transcription of the chant from 1924 (quoted in \cref{mus:solesmes_liber_1924_39}) might offer some clues, since it bears the same \quaver{}~=~138 tempo indication as the \ldo{} while also beaming the chant in the same way Bas did in 1906.\footcite[39]{Compendiumgradualisantiphonalis1924}
Provided the beaming and the initial quaver rest were also in force in the 1906 version, it may corroborate the assertion that Bas's \emph{modus operandi} consisted of faithfully reproducing Solesmes's rhythmed transcription and choosing chords to fit.

When Bas learned from Desclée's agent in Rome Auguste Zucconi that Wagner's book was outperforming his own in the French market, he traced the reason back to the \emph{Revue du chant grégorien} wherein advertisements for Wagner's accompaniments were placed among others for the Solesmes-Desclée chant books.
Bas complained that the general public was being led to believe that Wagner's accompaniments were based on those chant books when in fact they were not, Bas's were.\footnote{\letter{Bas}{Mocquereau}{27? February 1908}{\so{}}; Although the letter in question is typewritten, the second numeral in the date is only partially struck.}
Mocquereau could probably do little about it, for that periodical was not under Solesmes's control; its contributors, in fact, had long shown themselves to be wary of Mocquereauvian rhythm.\footcite[p.~42 n.~50, p.~72]{EllisPoliticsPlainchantfindesiecle2013}
Solesmes would not establish a periodical of its own until the \emph{Revue grégorienne} was started in 1911---until then, Solesmes relied on other periodicals to advertise its books.

Another accompanied Kyrial `conforming to the Vatican edition' was prepared by Leo Peter Manzetti (1867--1942), master of music at St Peter's Cathedral in Cincinnati, Ohio.
Its preface stated that the `Benedictine method of Gregorian chant' (by which Solesmes's was presumably to be inferred) had guided the process of composition.
Manzetti's New~York-based publisher J.\ Fischer \& Bro.\ brought out two versions of the Vatican Kyrial, both without rhythmical signs, the one in quadratic notation and the other in modern notation.
The degree to which Manzetti's accompaniments followed the transcription, however, is not altogether clear.\footcite[p.~29* and \emph{passim}]{KyrialesiveOrdinarium1906}
He reserved a more detailed discussion of his method for an advertised publication entitled \emph{Method of Accompanying Gregorian Melodies} which seemingly never saw the light of day.
\hlabel{ln:bas_138}%
Norman Holly posited in the \emph{Repertorio} that Manzetti followed Mocquereau's teachings and Bas's practice,\footcite[49]{HollyLetterEditor1906} a claim supposedly held up by the passage quoted in \cref{mus:manzetti_angelis_48} which incorporates a similar transcription to that quoted in \cref{mus:bas_angelis_kyriale_40}, including the \quaver~=~138 tempo indication.\footcite[unpaginated preface and p.~48]{ManzettiOrganAccompanimentKyriale1906}
Note how Manzetti placed his secondary accidentals in parentheses to the right of the note to which the accidental pertained, which (when printed on a busy staff) surely invited trouble for unobservant sight-readers.

It is doubtful that Manzetti had access to Bas's accompanied Kyrial prior to composing his own, for a copy of the first tranche of Manzetti's accompaniments was already in a reviewer's hands when the same reviewer evaluated Bas's accompaniments.\footcite[179]{Bibliographiegregorienneeditions1906}
We may set any accusation of plagiarism aside, therefore, and seek a more plausible explanation for how two accompaniments may be so similar.
One reason could be that Manzetti simply based his accompaniments on the transcriptions into modern notation published by Desclée.
A notable difference in Manzetti's accompaniments actually lends the notion some credence, because chords were placed where vertical \emph{episemata} divided groups of four beamed quavers into two groups of two (see \cref{mus:solesmes_liber_1924_39}).
Assuming both Manzetti and Bas followed the rubric that chords were to be placed on each \emph{ictus} (whether demarcated by beaming or by \emph{episemata}), then an unavoidable similarity must have resulted in their accompaniments because each composer was obliged to place a chord on the same note.

Manzetti's book was well received by an American periodical which seemed more willing than some of its European counterparts to give Mocquereau's ideas the benefit of the doubt.\footcite[571]{PublicationsReviewedManzetti1906}
By contrast, the \emph{Revue du chant grégorien} took a predictably dim view of Manzetti's application of Solesmian rhythm.\footcite[30--31]{Bibliographiegregorienneeditions1906b}
\hlabel{cc:burgess}%
The accompaniments perplexed the Anglican plainsong pedagogue Francis Burgess (1879--1948) who questioned why Manzetti should deliberately place chords `on a subsidiary stress':
\pagebreak{}

\single{Thus the weaker thing is helped at the expense of the stronger; but it is impossible to avoid feeling that the result is fantastic, that it produces a conflict between the words and the accompaniment, and that it seems to subvert the fundamental law of musical rhythm.}
  {\cite[80]{BurgessTeachingAccompanimentPlainsong1914}}
\noindent
At the time, few descriptions of Mocquereauvian rhythm existed in the Anglophone literature, even in spite of the Plainsong and Medieval Music Society's having followed, since its foundation in 1888, developments in the chant restoration movement at Solesmes.
A delegation of around twenty British musicians and clerics visited Saint-Pierre from 24 to 26 August 1897---among whom was the organist Rev.~George Herbert Palmer (1846--1926)---but,\footcite[164]{Halarestaurationplainchantdans2016} as we have seen, Mocquereau's theory of the \emph{ictus} did not mature until several years after their visit and was not well known to any except perhaps certain Anglo-Catholic specialists with more than an incidental connection to the continent.
As we shall see in chapter five, however, Solesmes's displacement to the Isle of Wight made visits by English musicians more convenient and thereby gave rise to detailed descriptions of Solesmian theories of chant rhythm in the English language.
\hlabel{cc:burgess_END}

\subsection{Extracts from the Gradual}
Producing the relatively circumscribed repertory of the Kyrial presented no serious financial challenges to publishers, whose accompaniment books seldom exceeded 10~F.
The Kyrial constituted what we here term an extract from the Gradual, and soon there arose the issue concerning how best to publish a complete accompanied Gradual that was not prohibitively expensive.
The first problem facing publishers was the requirement to await the Vatican commission's approval of the chant repertory.
Fascicles containing approved chants for the Common of the Saints began circulating during 1906,\footcite[6]{GrospellierCommunesanctorumedition1906} allowing publishers to get a head start on engraving their own versions.
Some began publishing what they had engraved and in many cases the Common of the Saints became the next extract after the Kyrial to receive organ accompaniments.
But the manager of the Vatican Press intervened in April 1907 to halt premature publications, and publishers were then obliged to await the completed Gradual before they could bring their own versions to market.\footcite[287--8]{HayburnPapalLegislationSacred1979}
\nowidow[2]

The Gradual, which was not ready until 1908, ran to 900 pages.
How could accompaniments---which evidently required more space---be provided in an affordable format?
One solution was to divide the complete edition into extracts, but some of these were still too large, such as the Proper of the Time which some publishers broke down further into two or three volumes.
Another solution was to publish abridged extracts that dispensed with less common chants in the name of offering a more affordable publication.\footcite[50--52]{WeinmannOrgelbegleitungGradualeRomanum1911}
We shall return to abridged extracts later, but for the moment let us consider how composers tackled accompanied extracts and how publishers divided up the material between various volumes.

Harmonisation itself was an obvious bottleneck in the publication of accompaniments.
Awaiting too long the completion of the thousands of new harmonisations could mean a publisher was slow off the mark in securing purchases by one diocese or another.
Conceivably, a diocese that had adopted Wagner's Kyrial accompaniments would prefer to await Wagner's accompaniments of the Proper of the Time so that the music in its liturgy would be relatively cohesive.
Wagner's accompanied Proper of the Time was divided among three volumes, the first from Advent to Lent, the second from Lent to Easter, and the third from Easter to Advent, and by 1911 they were in circulation along with two more volumes comprising the Proper of the Saints.
Considering Wagner had one more volume to compose before his accompanied Gradual could be deemed whole, there is little doubt that such an enormous task required a long span of time to complete.
Wagner's publisher Delépine mitigated the delays by attracting subscriptions at 5~F.\ apiece for one hundred of the most recently engraved pages as the process of composition was going along.
A player could also subscribe only to those accompaniments in which he or she was most interested.\footcite[unpaginated frontmatter bearing the title `Mode de souscription']{WagnerPropriumTemporepremier1908}
When Wagner's volumes were eventually completed, they ranged in price from 5~F.\ to 14~F.\ depending on the page count, but a buyer could also opt to purchase the volumes unbound at a slightly reduced cost.

Wagner's accompaniments were put into circulation in quite a different manner from those of Bas, who used the \emph{Repertorio} as a kind of proving ground for his own accompanied Common of the Saints.
When those accompaniments were later published by Desclée in collated form, their pagination and layout were independent of what had gone before.
This was in marked contrast to Delépine's subscription model, which provided finalised pages in advance of the finished product.
Like Delépine's model, however, the \emph{Repertorio} also generated revenue as the process of composition was in train, no doubt a boon for Bas and Desclée who were not obliged to await the final published volumes before reaping financial benefits from the accompaniments.
But Bas's progress trailed Wagner's, such that by 1911 Desclée had published only the accompanied Common of the Saints, setting itself at a disadvantage to its French rival.

By then, even Pustet lagged behind Delépine, since Mathias had completed only the Common of the Saints and one volume of the Proper of the Time, comprising Advent to the sixth Sunday after Epiphany.
The preface to that volume is dated Candlemas Day 1910 and incorporates some introductory words in four languages: German, English, Italian and French.
While the translations into English and Italian comprehend much of the same material as the German, that into French is about half its length and omits the reference to Mathias's accompaniment method.
\hlabel{ln:mathias_omit_chromatic}%
It also omits the permission granted to players to alter the accompaniments chromatically as they saw fit.
Perhaps French booksellers were simply unwilling to stock a German-language textbook; perhaps, also, Mathias recognised how the French ear favoured diatonicism, and suspected that any allusion to chromaticism would prejudice the French market against his accompaniments.
He continued to offer double signatures and provided parenthesised secondary accidentals, but only when they pertained to the chant part.
When accidentals occurred in an accompanying part, secondary accidentals were not notated at all, leaving the player to arrive at the correct secondary accidental in all cases (\cref{mus:mathias_secondary_89}).\footcite[unpaginated introductory remarks, pp.~50, 89]{MathiasOrganumcomitansad1936}

Karl Weinmann (1873--1929), Haberl's successor as director of Regensburg's \mbox{Kirchenmusikschule}, compared those accompanied Graduals by Wagner, Bas and Mathias, noting that Mathias's was more expensive than it should have been owing to its inclusion of multiple transpositions of some chants.
At a reported 40~M., Wagner's was expensive too, placing it beyond the reach of smaller and less financially endowed choirs.
But with dimensions of 27cm~x~19cm, it was at least comfortably sized to fit on an organist's music desk.
The same could not be said of Bas's, which at 35cm~x~27cm was deemed unwieldy.

Pustet had commercial interests in mind when producing an  accompanied `little Kyrial' (`Kyriale parvum'), which excised eight of the eighteen Mass Ordinaries from the accompanied Kyrial by Mathias discussed above.
Those Ordinaries were said to be surplus to the requirements of Alsation congregations, and so a thinner book at a reduced price proved to be the obvious commercial step.\footcite[50--51]{WeinmannOrgelbegleitungGradualeRomanum1911}
\hlabel{ln:springer_omit}%
\hlabel{int:springer}%
The venture was not an isolated one: the organist of Emaus Abbey, Prague, Max Springer (1877--1954), omitted the Sundays after Epiphany and Pentecost from his accompanied `Graduale parvum' but included accompaniments for the Sundays in Advent and Lent.
Those on which organ playing was prohibited by ecclesiastical decree are marked `silent organa' or `non pulsantur organa', whereas those for Gaudete and Laetare Sundays are either marked `Organis comitantibus' or not at all.
Springer included these accompaniments for the benefit of less experienced choirs and also for use in rehearsal.\footcite[1, 4, 8, 11, 50, 53, 58]{SpringerOrganumcomitansad1910}
Having started out as a staunch diatonicist, Springer underwent a conversion to moderate sharping, the implications of which will be examined below (\cref{sc:springer_chromatic}).

The father-son duo August Wiltberger (1850--1928) and Karl Wiltberger (1876--1954) also used sharping in their abridged Gradual, dividing the task of harmonising the chants between them.
August took on the Proper of the Time, the Proper of the Saints (pagination followed by asterisks) and the Common of the Saints (pagination in square brackets), whereas Karl took on the Votive Masses and the Missæ pro aliquibus locis.
Karl also turned his hand to editing his father's harmonisations.
The accompaniments were advertised at less experienced organists, allowances being made in the preface for experienced players to substitute certain passages with their own so-called `artistically designed forms' (`künstlerisch gestalteten Begleitungsformen').
The Wiltbergers' Düsseldorf-based publisher Schwann had previously published Nekes's accompanied Kyrial, and so the Mass Ordinary was not harmonised anew.
In a similar manner to Nekes, the Wiltbergers harmonised some protus and tetrardus cadences \pitch{2} \rightarrow{} \pitch{1} with \pitch{7}\kern 1pt\sharp{}.
The pitch \pitch{3}\kern 1pt\sharp{} was also a common feature of terminal deuterus cadences (\cref{mus:wiltberger_deuterus_60}).\footnote{\cite{WiltbergerOrganumcomitansad1910}, p.~[60].}
The Wiltberger accompaniments were arguably a more viable option for amateur organists with proclivities for sharping than Mathias, since in the Wiltbergers's book the musical material did not require any editing.

\subsection{Fragments}
The demand for accompaniments grew steadily as more Catholic church musicians sought to align their practice with the Vatican.
In some markets, though, demand was not sufficiently strong enough even for abridged extracts to become saleable propositions.
To appeal to such markets, publishers brought out what we here term fragments, slight publications containing perhaps only a single accompanied Ordinary or Proper.
Today, such fragments provide insights into the musical requirements of a particular region or religious order; when they were published, they were useful devices to whet the appetites of potential customers, and also responded to the needs of musicians who could not afford more expensive publications.

An accompaniment of the `Missa de Angelis' by the Dutch organist Peter Johannes Joseph Vranken (1870--1948) follows a similar chord placement routine to those of Bas and Manzetti, to say nothing of reproducing the well-nigh ubiquitous \quaver~=~138 tempo indication (\cref{mus:vranken_angelis_7}).\footcite[7]{VrankenMissaDuplicibusAngelis1910}
Vranken was nonetheless able to keep the number of pages in his fragment to a minimum by providing a single harmonisation for repeated lines of the chant, simply instructing players (by way of an italicised Roman numeral) to repeat a given line.
Vranken's `Christe' accompaniment is arranged largely in three parts, perhaps with a view to setting it apart from the four-part texture of the `Kyrie'.
Even though Weinmann considered Vranken's accompaniments to be sometimes empty and unsatisfactory (`die Begleitung mitunter leer und unbefriedigend klingt'),\footcite[52]{WeinmannOrgelbegleitungGradualeRomanum1911} that view was evidently not shared by one American publisher, who anthologised them in a hymn book for Cathedral and Parish musicians.\footcites[7]{BurtonChoirManualCathedral1914}[Reproduced in][65]{JoksContemporaryUnderstandingGregorian2009}

Sometime after 1913, the Spanish publishing house Boileau brought out a fragment of the same chant with an accompaniment by the Benedictines of Besalú, Girona (\cref{mus:boileau_angelis_15}).\footcite[1]{BenedictonosdeBesaluMisaAngelisconforme}
Even though the identity of the composer (assuming there was only one) was not indicated, a likely candidate is the Benedictine monk Dom Maur Sablayrolles (1873--1956) who had joined the abbey of En-Calcat in 1891 and was thereafter appointed as the abbey's organist and \emph{maître de chœur}.
When the congregation was forced into exile during the early years of the twentieth century, it moved south into Spain, settling at Besalú.\footcite[see paragraph 19]{Dauzetcongresmusiquesacree2009}
Sablayrolles thereafter undertook paleographical research on Catalonian and Spanish manuscripts, and described Mocquereauvian rhythm in various widely disseminated publications.\footcite[51]{PujolElsmonjosbenedictins2019}

Sablayrolles's adherence to Mocquereau's theory might explain why the Besalú accompaniment was so similar to those by Bas, Manzetti and Vranken in style; he also worked with Suñol, translating his chant manual into French.\footcite[47]{MassotiMuntanerAproximaciohistoriareligiosa1973}
Bas remarked in 1909 that a certain Spanish priest had been granted permission to reproduce his accompaniments, but there is little evidence to name Sablayrolles definitively as he.\fnletter{Bas}{Mocquereau}{11 November 1909}{\so{}}
Sablayrolles was a composer in his own right, and brought out accompaniments to various Spanish chants in 1912 that were largely in three parts (\cref{mus:sablayrolles_parts}).\footcite[31]{SablayrollesAlleluiaPsalliteDeo1912}
Nevertheless, the similarity of the above fragment to those accompaniments by other figures close to Solesmes lends credence to the notion that Solesmes's transcriptions were governing how such accompaniments were to be composed.
Any individual flair on the part of a composer was apparently subdued by that apparently pervasive rubric requiring consonant chords to be placed where indicated by vertical \emph{episemata}.
In short, a consensus on accompanying style had largely been arrived at, in Solesmian circles at least.

In parts of Europe where musical traditions had not previously lent themselves to chanting, fragments proved indispensable to introducing the chant repertory at a reasonable price.
The director of music at Ljubljana cathedral, Stanko Premrl (1880--1965), mused in the early years of the twentieth century that chanted masses were a rarity indeed in Slovenia, and as such there was little demand for the accompaniment books of Mathias, Springer, Nekes and Horn.
A common vehicle for disseminating music among Cecilian musicians in Germany had been the musical supplement, typically a short composition added to the verbal content of one periodical or another.
They proved themselves to be useful vehicles in Slovenia too, where the periodical \emph{Cerkveni Glasbenik} offered to domestic musicians affordable accompaniments in fragment form.
The first such was written by the Czech expatriate Anton Foerster (1837--1926), who from 1877 had been the principal of Slovenia's Orglarska šola, a music school for young organists.\footcite[179--80]{SkuljSurveyEvolutionSlovene1995}
But Foerster's transcription quoted in \cref{mus:foerster_asperges_17} shows itself divorced from the latest developments in French and German practice, it being based not on an equalist approach but on a quasi-mensural rhythmic scheme that transcribed lozenge-shaped neumes as shorter notes, and so forth.\footnote{\cite[17]{FoersterAspergesmeTrad1906}; For Foerster's accompanied responses see the appendix to the 1907 volume.}
Foerster indicated that the intonation was either to be unaccompanied or accompanied in bare octaves, the latter indication bearing witness, perhaps, to the spread of Cecilian practice across Europe.
Assuming that to be the case, the instances of cadential sharping hardly seem out of place.

For all of Foerster's attempts to promote chant in Slovenia, his accompaniments did not spark much interest.
In 1909, Frančišek Kimovec (1878--1964) took up the mantle to offer an accompanied `Missa de Angelis' to persuade domestic musicians to introduce at least one Mass Ordinary into their annual repertoires.\footnote{\covid{}\cite{KimovecMissaAngelisCod1908}.}
The \emph{CVK} noted how Kimovec's was a sustained style, but also that many so-called `deviations' were present in his harmonisations (`so viele Abweichungen in der Harmonisierung der Chorals').
Another contributor stated that `various corrections' (`Es sind verschiedene Korrekturen vorzunehmen') would be required to bring them up to standard.\footnote{\cvk{3871}.}
Maybe those reviewers took issue with the passage quoted in \cref{mus:kimovec_angelis}, on the basis of which Premrl found Kimovec guilty of close consecutive octaves and fifths.
But Premrl's redaction---which suppresses the musical context that presumably governed Kimovec's contrapuntal decision-making---suggests the carping of a narrow-minded grammarian.
The admittedly poor sonority of the first dyad was a very likely concession to better sonority in the chords immediately preceding it.
Kimovec's options were also presumably limited in terms of where chords could be placed in the sustained style.
Considering sporadic Cecilian resistance to musical neologisms, it is quite possible that the reviewers' pedantry was simply that and nothing more.
\noclub[2]

Kimovec's accompaniment was regarded in its day as a pioneering attempt at promoting chant in Slovenia, and according even to Premrl it heralded the beginning of a new era.\footcite[6--7]{PremrlMissaAngelis1909}
Premrl and Kimovec later collaborated on a collection of harmonised introits and communions for twenty-four first class feasts, publishing them as supplements in \emph{Cerkveni Glasbenik}.\footnote{For Anglophone biographies of Kimovec and Premrl, see \cite[pp.~182 n.~27, 183 n.~28]{SkuljSurveyEvolutionSlovene1995}.}
Not only could choir directors purchase each instalment at a reduced cost, but the editors also added Slovene translations of the Latin.
Some chant notes quoted in \cref{mus:kimovec_communio_7} are treated by the harmonisers as dissonances; the use of pedal point is also rather a progressive inclusion.
Whereas sustained chords are used consistently, the same cannot be said of the `B'\kern 1pt\natural{} in \cref{mus:kimovec_sharp_34}, which is anticipated in the accompaniment before becoming audible in the chant, quite a forward-looking gesture.
Neither is there much consistent about the contrapuntal imitation in the bass part quoted in \cref{mus:premrl_imitate_14}, which may be viewed as little more than a sporadic venture.\footcite[7, 34, 14]{KimovecIntroitusCommunionespro1909}

The Premrl-Kimovec accompaniments were collated and published, in 1910, as a single volume,\footnote{\covid{}\cite{KimovecIntroitusCommunionespro1910}.} but an advertisement in \emph{Cerkveni}'s October issue stated that the individual Propers could still be purchased at a reduced price.\footnote{Zimovec also publishes another fragmented accompaniment in 1911 entitled \emph{Missa pro defunctis cum responsorio `Libera'}. See \cite[226]{SlebingerSlovenskabibliografijaza1913}.}
The advertisement also advised organists against prejudicing themselves against the difficulty of performing chant,\footcite[80]{P[remrl]Oglasnik1910} indicating that Premrl and Kimovec were still fighting an uphill battle at introducing chant into all  liturgical corners of the domestic market.
Some circumstantial evidence points to an increase in interest by Slovene church organists in the repertory, namely that each periodical's issue dedicated a section to discussing the dispositions and technical features of new organs that were being introduced into Slovene churches.

Quite apart from encouraging accompaniments in the first place were certain fragments published as supplements to a composer's previously issued accompaniment books.
Nekes composed at least five such supplements to his op.~46 Kyrial of 1906 in a kind of continuing series, providing alphabetised entries that covered other parts of the Gradual, including op.~46a, \emph{Missa pro defunctis};\footnote{\cvk{3609}.} op.~46d, \emph{Missæ in summis festis} (containing ten Propers, among which are Christmas, Epiphany, Easter Sunday, Ascension, Pentecost Sunday, Corpus Christi, and All Saints);\footcite[7]{Organaria1910} and op.~46e, \emph{Commune Sanctorum}.\footcite[95]{BesprechungenVerschiedeneKompositionen1911}
His fragments appeared in the years following 1906, and their being supplements might explain why one of Nekes's biographers erroneously dated the publication of his accompanied Kyrial to 1908.\footcite[205]{WagnerFranzNekesund1969}
\nowidow[2]
