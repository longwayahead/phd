\chapter*{Conclusion}
\addcontentsline{toc}{chapter}{Conclusion}
Our findings on chant accompaniment have brought to light several aspects of its theory and practice.
The motives behind the development and adoption of various theories had previously gone unnoticed, not least because of their being distributed in a disorganised cache of material written in twenty or so languages.
Histories are therefore often limited to specific musical or linguistic traditions, whereas the theories discussed therein tell only part of the story.
The influence of an oral tradition, though certainly exerted, had not previously been acknowledged, nor had the effect of Cecilianism and the chant restoration movement at Solesmes, which the present dissertation has discussed at some length.
The historical accounts and archived correspondence presented here (much of it for the first time) bear witness to original and illuminating methodologies that contextualise the many newly evaluated printed accompaniment books and theoretical manuals.

Approaches to accompaniment which had been adopted prior to the nineteenth century have been shown to have been rather sporadic.
The reforms instituted by eighteenth-century antiquarians of music history (who sought pious alternatives to popular genres, see \cpageref{ln:reform_restart_history,ln:reform_restart_history_END} above) succeeded in detaching the history of music from what had gone before.
Although the methods of accompaniment subsequently recommended by Cecilian authorities were deemed authentically venerable for use in the church, our study of the available source material (\cref{sc:cecilianism}) has revealed that theorists in the nineteenth century re-invented the wheel, ignorant of their methods' place as modern exponents of musical practice.
This made such methods no more venerable than the popular genres they were trying to replace.
Following the Cecilian movement's approval by the Vatican, accompaniments devised by Haberl and Hanisch became the de facto standard in Catholic churches and have been shown to have been disseminated around Europe, North America and also to South Africa.
\nowidow[2]

Whereas Cecilian theorists retained sharping at cadences, during the 1850s Niedermeyer led the charge in Francophone countries against admitting black-note pitches other than `B'\kern 1pt\flat{}.
The notion that such diatonicism was a trait inherent in the chant became highly influential, and music historians scoured classical texts for evidence of a truly historical practice.
Their well-nigh occult attitudes to chant rhythm were the substance of intense debates and led theorists to analyse the chant repertory for clues to a method of accompaniment they hoped was simply hidden under their noses.
If one could be found, then chant and accompaniment could, in their view, satisfactorily be unified.

Some theorists also used accompaniments as a means of popularising their rhythmic theories.
Gigout has been shown as particularly adept at writing accompaniments according to various different schemes (see \cref{cc:gigout_teppe,int:gigout_megret}), ostensibly to demonstrate their practical value but in reality to prove the wide applicability of the diatonicism of Niedermeyer, with whom he shared a familial connection.
Solesmes also benefitted from such propaganda, and we have seen how composers in the Benedictine circle contrived accompaniments to popularise its methodologies.
In the age before recorded media, they made the repertory accessible to choir directors, organists and singers, who could pick up and use a Solesmian accompaniment book without any training in quadratic notation, or indeed in chant rhythm.
To be sure, the same could also be said of Cecilian accompaniments, but the new notational path followed by Solesmes was widely hailed as the key to performances in free rhythm.
So seductive was Solesmes's typography, in fact, that theorists expended huge energies in devising new strategies to represent free rhythm in their accompaniments.
Novel approaches to the notation of harmony sought to free accompaniments from the shackles of metre: the quadratic-harmonic and filled-and-void notational styles pioneered by Schmetz and Van Damme respectively (see \cref{cc:schmetz,cc:filled_and_void}) were directly inspired by Solesmian notation and Pothier's theory of free rhythm.

The application of free rhythm in the accompaniment soon transcended notation as some in the Benedictine circle determined that chords could be placed on important notes (such as the first notes of neumes) or at particular syllables.
This gave rise to what we have termed the `Lhoumeau effect' (\cref{sc:lhoumeau_effect}), whereby chords were placed on the unaccented syllables of words.
The effect was exacerbated when Mocquereauvian rhythm introduced the controversial notion of the \emph{ictus}, requiring chords to be placed on unaccented syllables more frequently than before.
It elicited warnings from international authorities on music such as d'Indy and Widor who held that the syncopated effect was incompatible with chant (\cref{cc:dindy_rhythm}).
Initially, Mocquereau turned a deaf ear and maintained his course, but eventually matters came to a head at Solesmes.
The account (on \cpageref{ln:wagner_viewsonmoc} above) detailing Mocquereau's offer to resign as Solesmes's \emph{maître de chœur} provides a new perspective on the tension at Solesmes, as the abbot was seemingly obliged to refuse the offer and to row in behind Mocquereau's ideas.
After all, any visible crack in the façade of Benedictine practice might have dissuaded the Vatican from shunning Regensburg and vouchsafing Solesmes's chant editions---this political landscape no doubt influenced the abbot's decision.
Not only does Wagner's account illustrate that there is more to the politics of plainchant at the \emph{fin-de-siècle} than Katharine Ellis and others have acknowledged, but it also offers an explanation as to why the dubious progeny of the `Lhoumeau effect' has never before been challenged.
It continues to disfigure accompaniments today.
\noclub[2]

Few theorists have challenged Niedermeyer's rule requiring the chant (which was believed to be just like any other melody) to be placed in the top part of the keyboard texture.
Emmanuel was one of the sole figures who took into account the intended voice type (see \cref{cc:emmanuel_octave}).
Other theorists paid little attention to the accompaniment of mens' voices in their range: they are, in numerous cases, accompanied at the octave above.
Whereas Niedermeyer's rules enjoyed widespread popularity, the diatonic theory which they espoused did not go far enough for some theorists.
Gevaert parsed individual chants to determine which of three hexachords the chant traversed, and limited his harmony to the notes of a given hexachord (\cref{ln:gevaert_hexachordal}).
The idea was ressurected seemingly independently by Desrocquettes and Potiron in the 1920s (\cref{cc:dd_equiv,hl:potiron_methode}), who codified a complex theory of chant harmony based on the belief that such hexachords limited the notes available for chord construction.
Potiron made the system even more stringent in the 1930s by omitting `B'\kern 1pt\flat{} and `B'\kern 1pt\natural{} when they were liable to conflict with instances of those notes in the chant.
In the 1950s, he made the system stricter still by forbidding the use of notes in the accompaniment that did not appear in the melody.
Reducing the notes available to an accompanist stands in stark contrast to the practice of some German theorists at the beginning of the twentieth century, who chose to admit all the notes of the chromatic scale because, they believed, it permitted them to reflect the conjunct nature of the chant in an equally conjunct---though highly chromatic---accompaniment (\cref{cc:chromatic}).
Both cohorts evidently existed at opposite ends of the harmonic spectrum, though the preference for less active accompaniments was not a novel phenomenon.
It arose as the Gruppenbegleitung style (see \cpageref{ln:gruppenbegleitung} above) and again during the 1860s when it was decided to interpolate 6/3 chords between 5/3 chords (see \cpageref{cc:populus_conjunct} above).

In the present study, no attempt has been made of functional analysis of chant harmonisations; if such analysis is possible, then we await an appropriate methodology for that purpose.\footnote{At a late stage in the preparation of this dissertation the author became aware of the following analytical study: \cite{ShironishiPlainchantAccompanimentModal2021}, but the writer's use of a restrictive range of sources has led to certain conclusions with which the present author begs to disagree.}
Nor was a consideration of the Anglican practice of accompanied plainsong within the scope of the present study, yet the exile of the Benedictine community to England put Anglican musicians into closer contact with the theory and practice we have discussed throughout this dissertation.
The accompaniment manuals written by Francis Burgess and John Henry Arnold clearly owe a debt to Solesmian practice in their placement of chords and choice of harmonies (see \cpageref{cc:burgess,cc:burgess_END} above and Arnold's entry on \cpageref{cc:arnold_entry} below), which may in the future be evaluated further.
By all accounts, the fertilisation of Anglican methodologies corroborates the interpretation that chant accompaniment played host to adaptable methods, and the recent arrangements of chant accompaniment for synthesiser suggest it to be capable of withstanding further developments in the future.
