\chapter{General prehistory and Germanic antiquarianism}
\section{The role of the organ in the performance of plainchant}
\subsection{Early performance practice}
Countless musical instruments are depicted in iconography and manuscripts from the Middle Ages, yet there is scant convincing evidence to suggest that organs actively participated in the Latin liturgy prior to about the twelfth century.
It seems probable that writers in later centuries mistook early allegorical descriptions of musical instruments for evidence that they exercised performative roles, and it is most likely that the musical adornment of the liturgy was restricted to monophonic chanting.\footcite[23]{McKinnonMusicalInstrumentsMedieval1968}
Unisonous chanting later burgeoned into the more elaborate routines \emph{organum} and \emph{falsobordone} that allowed singers to derive multiple voice parts from a chanted tenor part.
Those voice parts were later permitted to exercise more independence as contrapuntal and rhythmic techniques gradually developed.
The polyphonic \emph{lingua franca} initiated by Guillaume Du Fay (1397--1474) and his Renaissance successors was perceived by later generations as having reached full maturity at the hands of Giovanni Pierluigi da Palestrina (1525--94), whose compositions supposedly captured an austere, reverent and pious \emph{Weltanschauung} that musicians of the late-nineteenth-century Cecilian movement in particular sought to recreate for themselves.
Although such musicians sought approaches to polyphony and cadence construction in the musical practice of antiquity, their researches, as we shall see, were conducted with rose-tinted views of the past that blinded them to the historical facts subsequently established by modern musicology.

From the twelfth or thirteenth centuries another common form assumed by the chant repertory was the alternatim verset, a polyphonic composition played on the organ in alternation with certain sung verses of a canticle or hymn, or instead of passages of a Mass Ordinary.\footcite[287--9]{vanWyeRitualUseOrgan1980}
The sung portions could either be chanted monophonically or set in a more elaborate polyphonic form depending on the solemnity of the feast, but given that the organ verses were played and not sung the import of their verbal texts necessarily remained a matter for the imaginations of worshippers.
Various sixteenth-century rubrics deemed this a vulnerability, and required that omitted texts be recited during the polyphonic organ verset;\footcite[93]{PrasslAnmerkungenzurOrgelbegleitung2012} such rubrics were continually reiterated until the nineteenth century on account of the popularity and longevity of alternatim practice.\footcite[See, for example, ][358--9]{Enchiridionsanctorumrituum1856}

In the seventeenth century, the alternatim verset was given particular impetus in France by the Rouen organist Jean Titelouze (1562 or '63--1633), who brought out \emph{Hymnes de l'Eglise} in 1623 and eight sets of versets on Magnificats in 1626 that were arranged according to the eight church modes (`suivant les huit tons de l'Église').\footcites{TitelouzeHymnesEglisepour1623}{TitelouzeMagnificatoucantique1626}
\hlabel{ln:nivers_livredorgue}%
Collections of versets were also published by Guillaume-Gabriel Nivers (\emph{c.}1632--1714) in 1665, who, in his \emph{Livre d'orgue}, assigned the chant melodies of some versets to the reed stops of the pedal organ and polyphonic parts to subsidiary registrations played by the hands.
The term \emph{accompagnement} therefore came to describe the organ stops used for such polyphonic parts, as witnessed by the following dictionary entry of 1690:

\simplex{\textsc{Accompagnement}, en termes \linebreak{}d'Organiſ\kern -0.5pt tes, ſ\kern -0.5pt e dit de divers jeux qu'on touche pour accompagner le deſ\kern -0.5pt ſ\kern -0.5pt us, comme le bourdon, la monſ\kern -0.5pt tre, la fluſ\kern -0.5pt te, le preſ\kern -0.5pt tant, \&c.}
  {\cite[unpaginated entry under `ACC']{FuretiereDictionnaireuniverselcontenant1690}}
{\textsc{Accompaniment}, in the vocabulary of organists, is used for the various stops drawn to accompany the \emph{dessus}, such as the \emph{bourdon}, \emph{montre}, \emph{flûte}, \emph{prestant}, and so forth.}

\pagebreak{}
\noindent
From the eighteenth century in France, accompaniment of chant was most commonly supplied by wind and string instruments that doubled sung chant verses at the unison or octave.
Serpents, ophicleides, bassoons, trombones, cellos or double basses were used as accompanying instruments depending on local customs and the solemnity of the feast.
There is some evidence to suggest that those instruments might have been involved in the polyphonic parts of an organ verset, but it should be noted that the organ's function was, in the first half of the century at least, to play polyphonic versets and not to accompany voices.\footcite[10--11]{HillsmanInstrumentalAccompanimentPlainChant1980}

\subsection{The Lutheran chorale and thoroughbass practice}
Alternatim practice was widespread in the Catholic churches of sixteenth- and seventeenth-century Germany.
Although some Protestant churches also retained it, reformers envisaged new roles for church music generally and for the organ specifically.\footcite[3--31]{SoehnerGeschichteBegleitunggregorianischen1931}
The \emph{Erfurt Enchiridion}, published by Martin Luther in 1524, includes the two Latin chant hymns `Veni redemptor gentium' and `Veni Creator Spiritus' transformed and translated into the respective chorales `Nun komm, der Heiden Heiland' and `Komm, Gott Schöpfer, Heiliger Geist'.
Congregational vernacular hymnody and folk songs were substituted for other parts of the Latin service, and polyphonic music (either sung or instrumental) was used to introduce the sermon as the high point of the liturgy.
Organ music also anticipated the congregation's involvement in the chorales with chorale preludes, whose function was effectively to line-out the tune.
Chorale preludes were (and still are) short settings for organ solo that set the tune as a \emph{cantus firmus} amidst contrapuntal matter: the genre originated in the chorale variations of the Dutch organist Jan Pieterszoon Sweelinck (1562--1621) and was subsequently developed by German composers of Protestant backgrounds.
As for the chorales themselves, their tunes were transferred from the tenor part into the top part of the texture for the first time in Lucas Osiander's \emph{Fünfftzig Geistliche Lieder und Psalmen mit vier Stimmen} of 1586.\footcite[169]{FrenchChorale2003}
This marked a decisive step towards the consolidation of what will be termed the `chorale texture' below.

Transferring the tune into the top part was complemented by the emergence of the basso continuo in Italy during the closing decades of the sixteenth century.
One of the first witnesses to the basso continuo's being used in church music at all was probably the 1601--1603 publication of embellished \emph{falsobordoni} by the Sistine Chapel singer Giovanni Luca Conforti (1560--1608), in which the psalm tones with \emph{abbellimenti} were provided with a bass part.\footcites[230--31]{BradshawPerformancePracticeFalsobordone1997}[p.~xiii]{BradshawGiovanniLucaConforti1985}[5]{BradshawGiovanniLucaConforti2012}
A similar publication by Francesco Severi (d.1630) arranged the tones into `intonatione' (a simple harmonisation of the tone) and `falso bordone' (a more elaborate form not unlike those in Conforti's publication), pairing each with a rudimentary figured bass.\footcite[1--6]{SeveriSalmipassaggiatitutte1615}
Lodovico Grossi da Viadana (\emph{c.}1560--1627) applied the nascent accompanying technique to the mass in `Missa dominicalis', publishing his accompaniment in the 1607 collection \emph{Concerto ecclesiastici, libro secondo}.\footcite[pp.~32, 90]{SoehnerOrgelbegleitunggregorianischenGesang1936}
In the nineteenth century, the Italian campaigner for church music reform Pietro Alfieri (1801--63) published a collection of psalm-tone basses with the claim that two of them were about three centuries old (`antiche forse tre secoli').\footcite[pp.~4--5, 7]{AlfieriAccompagnamentocollOrgano1840}
Although the two in question (both provided in \cref{mus:alfieri_antiquated}) are similar to Severi's `intonatione' bass parts, Alfieri's first-tone bass makes greater use of conjunct motion and 6/3 chords.

The first Germanic example of basso continuo practice might have been imported from Italy by Gregor Aichinger in 1607;\footnote{\covid{}\cite{AichingerCantionesecclesiasticaetre1607}.} and by 1640 the practice was sufficiently established in the north for the Protestant composer Johann Crüger (1598--1662) to publish a book of chorale melodies with bass parts.\footnote{\covid{}\cite{CruegerNewesvollkoemlichesGesangbuch1640}.}
German Catholic organists also adopted the basso continuo, but the majority of extant seventeenth-century sources of Latin chants with bass lines was confined to the manuscript tradition.
\emph{Orgelbücher} were compilations of the chants sung at a specific church or religious house where the resident organist had probably experimented with improvising and realising the basses for some time prior to committing them to paper, either as an aide-mémoire or as a model for his or her successors.
The individualistic nature of the \emph{Orgelbuch} repertory made it vulnerable to quirks in an organist's taste, with the result being that in some cases a given book remains the sole witness to a particular style or method of organ-accompaniment.
In other cases, \emph{Orgelbücher} reflect developments in the basso continuo and reveal how closely some musicians matched their accompaniments of chant to the accompaniment of melodies in other genres.
One must be wary of drawing definite conclusions from the repertory as it stands, however, because various collections continue to yield new discoveries: in 2006, for instance, an \emph{Orgelbuch} was discovered at the Cistercian monastery of Stams in the Austrian state of Tyrol,\footcite[110--12]{Herrmann-SchneiderChoralgesangmitGeneralbassBegleitung2010} and further procedures adopted by unknown composers may yet come to light.
With that caveat in mind, Leo Söhner's survey of select \emph{Orgelbücher} is summarised in the following paragraphs.

First, the most rudimentary procedure assigned to each chant note its own bass note (`Akkordische Begleitung'), to which the organist could apply the rules of thoroughbass realisation.
The description `note-against-note', often levelled at the resulting chordal texture,\footcite[95]{PrasslAnmerkungenzurOrgelbegleitung2012} scarcely does justice to the variety of ways in which the bass and inner parts can be handled.
Three approaches to bass motion are identified by Söhner: the first uses sequences of 5/3 chords, causing the bass to move predominantly by fourths and fifths; the second interpolates 6/3 chords, producing more conjunct motion; and the third uses bass notes in oblique motion, anticipating cadences with 6/4 \rightarrow{} 5/3 harmony.

\hlabel{hl:imitationenimbass}%
Second, a more contrapuntal procedure constructed an imitative bass line (`Imitationen im Baß') from the intervallic shape of the chant.
Söhner identified one MS in which that particular procedure was used as perhaps having been assembled by a continuo musician.\footcite[The MS in question is also described in][66]{FellererCodXXVII841926}
The intervals traced out by the bass could be derived either from the first phrase of the chant or from its entirety, necessitating a certain degree of contrapuntal imagination to make the bass fit.
One chant note could be matched to one bass note, to several bass notes in shorter durations, or to rhythmically complex bass lines composed in florid counterpoint.
An accompaniment could flit from one texture to another at a new verse depending on the composer's inclination.
\noclub[2]

\hlabel{ln:gruppenbegleitung}
Third, a so-called grouped procedure (`Gruppenbegleitung') permitted an organist to accompany by groups of sustained chords instead of by individual bass notes.
Numerous chant notes (which sometimes comprised extensive melismata) were matched to a few well selected bass notes depending on how the composer intuited the potential harmony of the chant.
Distinguishing between essential and inessential notes formed the basis of this procedure, as did permitting more chords than those of the 5/3 and 6/3 varieties.
Söhner identified 6/4 chords as being particularly useful in retaining the same bass part in passages where a great number of chant notes could be accompanied by an economical use of chords.
The reduced bass motion was claimed by Söhner to have suited organists with less developed pedal technique.

\hlabel{ln:keyboard-textures}%
Fourth, the procedure of the melodically independent bass part (`Melodisch selbständige Baßführung') was recognised in \emph{Orgelbücher} dating from about the second half of the eighteenth century.
The intervals between successive bass notes were filled in with scalic passages, octave leaps, arpeggiations, and elaboration by auxiliary notes.
Textural padding in the accompaniment would probably have been idiomatic in the accompaniment of voices by continuo musicians, and it is perhaps unsurprising that such devices were also used in the accompaniment of plainchant.\footcites[pp.~44--7, 84--111]{SoehnerGeschichteBegleitunggregorianischen1931}[36--8]{SoehnerOrgelbegleitunggregorianischenGesang1936}

\subsection{Parity with continuo practice}
The development of the basso continuo led the French-language theorist Jean-Jacques Rousseau (1712--78) to recognise two definitions of the term \emph{accompagnement} in 1768: the addition of upper parts to a figured bass and the addition of a bass line below a melody.\footcites[6]{RousseauDictionnairemusique1768}[47]{JambouDoscategoriascanto2007}
The organ accompaniment of voices was seemingly common enough in France by 1766 that the organ builder Dom Bédos de Celles (1709--79) specified organ registrations so that singers would not be drowned out by a loud organ.
Dom Bédos held that the accompaniment ought only to ornament and support a dominating singing voice (or, indeed, singing voices), and proposed that many voices, perhaps choirs, perhaps congregations, might be accompanied by a proportionate array of foundation stops (`des jeux de fonds proportionnés').
Strong voices might be accompanied by three or four 8$^\prime$ stops, and weak ones by two of the same.
Moreover, loud (presumably solo) singers might be accompanied on the last registration, whereas for quieter singers a soft flute (`un petit bourdon') sufficed.\footcite[533]{deCellesartfacteurorgues1766}

Both of Rousseau's definitions are recognisable in Portuguese accompaniments published in 1761 and composed by one José de Santo António for the Basilica at the Palace of Mafra.
Some chants were set in the bass part to which the composer added figures, and others (such as the psalm tones) were seemingly intended to be sung above a provided bass line.
Although António's accompaniments have been described as `note against note',\footcite[118--19]{CardosoEmbuscapeculiar2012} chains of suspensions are a notable feature of the second `Christe' from \emph{Missa duplex: Das Primeiras Classes Mayores} and necessitate considerable inner part movement (\cref{mus:antonio_christe}).
The \emph{Missa duplex} was sung on double feasts by numerous choirs and was accompanied by no fewer than six separate pipe organs (`e acompanhaõ o Côro ſ\kern -0.5pt eis Orgaõs') arranged in the chancels and transepts of the Basilica---the organs described by António pre-date those presently installed.
Rubrics governing the use of one, two, four or six organs were provided depending on the solemnity of the feast,\footcite[89--90]{VazSixOrgansBasilica2015} and the unparalleled spatial arrangement must have presented quite a considerable challenge to the \emph{mestre da capela}.
In a bid to coordinate the various forces at his disposal, António devised the rubric that the organists should lift their hands at all commas and rests (`levantem ſ\kern -0.5pt empre as maõs do Orgão em todas as virgulas , ou pauzas').\footcite[p.~2 and unpaginated `Advertencias' I, XXIII]{SantoAntonioAcompanhamentosmissassequencias1761}

The first Spanish source of notated accompaniments is reported to date from the eighteenth century at the convent of Santo Espíritu de Jerez de la Frontera in the Spanish port city of Cadíz.\footcite[39]{JambouDoscategoriascanto2007}
Unfortunately, further information about the performance practice in that convent is sparse, but some details about Spanish accompanimental practice in the following century have survived.
\hlabel{hl:eslava}%
The nineteenth-century composer Hilarión Eslava (1807--78) admitted that chant was not generally accompanied in Spanish cathedrals (`no se acompaña generalmente') where sung polyphonic music was probably more prevalent, but he states that accompaniment was common in lower churches where it complemented the hymns `Pange lingua' and `Salve regina'.
Two accompanimental procedures appear to have been in use in Spain: the first placed the chant in the bottom part above which chords were realised; the second comprised a chordal texture that did not require the chant to sound continuously in any one part.
According to Eslava, the former was more widespread and easier to play (`más común y fácil'), whereas the latter was said to be more perfect but much more difficult (`más perfecto, aunque es también mucho más difícil').\footcite[46--9]{EslavaMuseoorganicoespanol1853}
The second procedure matched a recommendation by the Spanish theorist José de Torres (\emph{c}.1670--1738) during the previous century that the melody should be left out of a continuo accompaniment.\footcites[95--6]{TorresReglasgeneralesacompanar1736}[990--92]{Torresfacetadesatendidaquehacer2013}
\noclub[2]

The sixteenth-century Spanish conquest of the Aztec empire had introduced Spanish liturgical customs to Mexico where organ music was played at the cathedral of Oaxaca from 1544.
Towards the end of the colonial period, the nun Sor María Clara del Santísimo Sacramento possessed an undated notebook which is not only a witness to the propagation of the alternatim verset to Mexico but also suggests that the accompaniment of psalm tones inherited customs from Spain.
In seventeenth-century Spanish alternatim practice, the pitch of an ensuing sung verse had been provided to singers by means of a short introductory passage for organ called \emph{la cuerda};\footcite[239]{NelsonAlternatimPractice17thCentury1994} the introductions preceding the Mexican psalm tones bear the same epithet and doubtless served a similar purpose.
The psalm basses in María Clara's notebook are annotated using obliques (for the same pitch class above the bass) and figures, which together produce a chordal texture that does not always comprehend the notes of the psalm tones in any one part. Instead, the tones move freely in the texture and bear a resemblance to Eslava's more complicated procedure.\footcite[pp.~xiii, xxi]{JohnsonCuadernoTonosMaitines2005}
\noclub[2]

The parity between echt continuo practice and the accompaniment of chant by the basso continuo furnishes ample evidence that, wittingly or otherwise, church musicians seized developments in secular genres for their own benefit.
The Italianisation of church music caused some organists to produce accompaniments that mimicked the operatic \emph{recitativo secco}, while others composed elaborate keyboard textures that far exceeded chordal accompaniments in complexity.
One such elaborate accompaniment was written in Salzburg, leading Söhner to make the cautious assertion that it might have been the work of Michael Haydn;\footcite[pp.~112--17, 195--6, and `Notenbeilagen' pp.~18--21, 49]{SoehnerGeschichteBegleitunggregorianischen1931} and a separate example with a pianistic texture was headed `Für Geübtere' leaving no doubt that it was intended for experienced players.\footcite[p.~5 and `Notenbeilage II']{WagenerBegleitunggregorianischenChorals1964}
\hlabel{hl:knecht}%
By the end of the eighteenth century, chord progressions and cadences typical of major-minor harmony became the \emph{sine qua non} of chant accompaniment, notwithstanding the claim of Justin Heinrich Knecht (1752--1817) that one of his harmonisations corresponded entirely to the spirit of the second mode (`entspricht dem Geiste der zweiten Kirchentonart ganz'; \cref{mus:knecht_secondmode}).\footcite[60--61]{KnechtVollstandigeOrgelschulefur1798}
Apart from the melody itself, the only recognisable characteristics of the second mode are its transposition up a fourth and concluding cadence as described by Adriano Banchieri (1568--1634) in 1604;\footcite[41]{BanchieriOrganosuonarino1605} otherwise, the harmony constitutes what we would expect to find in a nineteenth-century hymn tune in F major.

Contemporaneous developments in keyboard textures---such as those scalic passages, octave leaps, arpeggiations, and that procedure of ornamenting by auxiliary notes discussed above (on \cpageref{ln:keyboard-textures})---permitted the accompaniment a greater amount of independence from the chant.
\hlabel{hl:neubig}%
Such a style appears to have been popular in the German diocese of Limburg where, in 1844, Johann Nikolaus Neubig developed a genre of arpeggiated accompaniments requiring special organ registrations.
For chordal accompaniments, Neubig deemed the Principal and Viola di Gamba too heavy (`zu stark'), and recommended instead the soft 8$^\prime$ stops Gedackt, Bourdon, Flûte traverse, Rohrflöte, Salicional and Quintatön.
For the `arpeggio Begleitung' (\cref{mus:neubig_arpeggio}) he recommended the 4$^\prime$ stops Salicional, Principal, Spitz-Flöte and even the 2$^\prime$ Flageolet.\footcites[pp.~iii--v]{NeubiggregorianischeGesangbei1844}[54--61]{WagenerBegleitunggregorianischenChorals1964}
%\noclub[2]

Chanting in the Polish liturgy did not escape the impact of opera, particularly with regard to chromatic harmony and instrumental and vocal virtuosity.
The practice of ornamenting chants was exemplified by Jan Jarmusiewicz (1781--1844) whose `Sanctus' of 1834 (\cref{mus:jar_sanctus_55}) adopts the kind of operatic vocal technique that, in 1841, no less a figure than Richard Wagner blamed on the influence of church orchestras.\footcites[55]{JarmusiewiczChoralgregoryanskirytualny1834}[pp.~70--72 and `Notenbeilage' XVI]{WagenerBegleitunggregorianischenChorals1964}[Wagner's essay was published some thirty years after it was written in][337]{WagnerEntwurfzurOrganisation1871}[See also its translation in][341]{WagnerPlanOrganisationGerman1966}
Chromatic chants were included by Michał Marcin Mioduszewski (1787--1868) in a pioneering Polish catalogue of church music considered easy for the people to sing (`ograniczyłem się do tych tylko, które są i łatwe do śpiewania dla ludu').\footcite[pp.~5, 142]{MioduszewskiSpiewnikkoscielnyczyli1838}
The Polish composer Wincenty Gorączkiewicz (1789--1858) included Neapolitan, diminished and augmented sixth chords that were arguably more chromatic than some more conservative historians of church music would at that time have permitted (\cref{mus:wincenty,mus:wincentysixth}).\footcite[6, 13]{GoraczkiewiczSpiewychoralnekosciola1847}
The Polish pedagogue Jan Galicz, in an organ method of 1861--3 published in Vilnius, anticipated some congregational entries by bridging two phrases of a chant melody with scalic filler.\footcite[9--13, 21]{GaliczSzkolanaorgany1861}


\subsection{Stylistic innovations among German theorists and antiquarians}
The papal encyclical `Annus qui' of 1749 took a dim view of the secularising influence that popular genres exerted on the liturgy by denigrating the `terrible noise which comes from [organ] bellows and which expresses more thunderous din than the sweetness of song'.\footcite[96]{HayburnPapalLegislationSacred1979}
\hlabel{ln:reform_restart_history}%
Some methods of Lutheran chorale accompaniment also voiced criticisms of music that undermined the sacred spirit of the liturgy with major-minor harmony, instrumental textures or both.
The organist and music historian Jacob Adlung (1699--1762) was among those advocating for a reform of church music, and a widespread movement to redraft the principles underpinning Protestant and Catholic church music began to take root.\footcites[681--82]{AdlungAnleitungmusikalischenGelahrtheit1758}[159--60]{GravePraiseHarmonyTeachings1987}
Harmonisers and music theorists sought alternatives to major-minor harmonisations, and antiquarians of music theory argued that contemporary practice should recapture what they held to be the \emph{Palestrinastil}.

The abbé Georg Joseph Vogler (1749--1814) was a vocal detractor of Bach's music: he levelled particular criticism against a harmonisation of `O Haupt voll Blut und Wunden' saying that, in his opinion, the major-minor approach to harmonising that modal melody obscured its `phrygian' character.
Vogler offered an alternative harmonisation that retained an A minor key centre rather than modulating to those keys related to C~major,\footcite[160--72]{VoglerChoralSystem1800} though he must have forgotten something Bach had not: namely that \emph{c}$^\prime$ is the dominant of the third mode.
Nonetheless, Vogler's staid harmonisations of chorale melodies succeeded in driving a coach and horses through the complex textures favoured by other composers.
His approach to church music reform influenced others to devise new theories of harmonisation based on what they too believed were more authentic principles.
In effect, the reformers restarted the history of church music, causing the period under consideration to be disinherited from the principles of organ tuning first laid down by Arnolt Schlick (\emph{c}.1460 to some date after 1521).\footcite[129--34]{LindleyEarly16thCenturyKeyboard1974}
\hlabel{ln:reform_restart_history_END}%
\nowidow[2]

\hlabel{hl:stehlin_tonarten}%
The reforms led musicians to seek out (or indeed to imagine) more appropriate, austere methods in the practice of antiquity.
There was considerable debate about what constituted appropriate church style among clerics and music theorists who disagreed on various conflicting schemes.
One such scheme was proposed in 1842 by Sebastian Stehlin (1800--77), who parsed chants according to three separate note-groups analogous to the Guidonian hexachords.
Chants were shown to mutate from one hexachord to another when they occupied different positions in the scale, yet despite Stehlin's pioneering historicism his own harmonic practice---and that of his collaborator Simon Sechter (1788--1867)---appears to have made little use of it.
Rather, the music examples in his treatise incorporate raised semitones that were seemingly not justified aside from an unsupported allusion to the Renaissance practice of \emph{musica ficta causa pulchritudinis}: when a 7--6 or 2--3 suspension coincided with textual conclusions to form a cadence, the Renaissance singer was to make the 6 a major interval or the 3 a minor interval, whether it was notated or not.\footcite[84, 96]{BentMusicaRectaMusica1972}
Stehlin described the modes according to the major-minor keys that he deemed best fit, and thereafter recommended that chromatic harmony be applied to a chant harmonisation.
The eighth mode was likened to the modern key of C major (`Diese Tonart wird in der modernen Musik C dur ausgedrückt'), to which a harmonisation would modulate when the chant exhibited certain modal traits (\cref{mus:stehlin_etexpatre_31}).\footcite[pp.~iii--iv, 11 and `Beilage' p.~31]{StehlinTonartenChoralgesangesnach1842}
\hlabel{ln:nisard}%
Such a concession to modernity gave the Belgian theorist abbé Théodule Normand (1812--88; here referred to by his nom de plume Théodore Nisard) ample reason to describe Stehlin's harmonic system as belonging to \emph{la tonalité moderne} rather than to \emph{la tonalité ancienne};\footcite[col.~88]{NisardAccompagnementplainchant1854} we shall discuss the differences between those two approaches to \emph{tonalité} in the next chapter.
\hlabel{hl:stehlin_hexachord}%
In a subsequent book, Stehlin mooted the possibility of basing the harmony of an accompaniment on the hexachords themselves by deriving dyads from an eight-note scale of `G' including both `F'\kern 1pt\natural{} and `F'\kern 1pt\sharp{} set in contrary motion with itself (\cref{mus:stehlin_hexachords_54}).\footcite[54]{StehlinNaturgesetzeimTonreiche1852}
In spite of Stehlin's practice continuing to rely on major-minor harmony, his neo-Guidonian experiment anticipated by some years attempts by other theorists to reveal harmonic approaches in Guido's hexachords, as we shall see below.

It is unlikely that some accompaniments were intended to be sight-read by singers, who instead probably applied the tacit rubric that leading note \rightarrow{} tonic formulæ were preferred at cadences.
Joseph Adam Homeyer (1786--1866) repeatedly contradicted `C'\kern 1pt\natural{} in a protus chant by using `C'\kern 1pt\sharp{} in his accompaniment (\cref{mus:homeyer_libera_113}).\footcites[113]{HomeyerAltarundResponsoriengesang1846}[62--3]{WagenerBegleitunggregorianischenChorals1964}
Another example by J.~N. Basilius Schwarz (1779--1862 or '63) seems to corroborate the intimations of presumptive sharping among some country choirs, because otherwise his figured bass part would conflict with the chant printed above it (\cref{mus:schwarz_avemarisstella_39}).\footcite[There appear to be two undated editions of Schwarz's book, one published in landscape format in which the example is printed on p.~29, and the following published in portrait format:][39]{SchwarzChoralwieer}
Catholic organists adopted a similar technique to their Protestant counterparts by anticipating chanted verses with introductory preludes that established the mode or starting pitch, and perhaps also the vocabulary of an ensuing harmonisation.
The texture of Schwarz's introductory `Cadenz' would incline the singers to expect an accompaniment using similar chords, whereas the contrapuntally conceived, inter-versicular `Zwischenspiel' provided the singers with a brief tacet and respite from the incessant sequence of chords.

\hlabel{ln:niedermeyer_firstmention}%
An accompaniment diverging from the characteristics of the melody was eventually deemed untenable, and several attempts at unifying melody and accompaniment were popularised in France and Germany alike.
\hlabel{hl:schneider}%
As we shall see, a theory of diatonic harmony was proposed during the 1850s by the Swiss composer Louis-Abraham Niedermeyer (1802--61) and the French theorist Joseph d'Ortigue (1802--66), but subsequently the same process appears to have been initiated quite independently in Germany by Ludwig Schneider (1806--64) in 1866.
The first of Schneider's eleven rules dismissed the tenets of major-minor harmony entirely in favour of a new scheme based on the diatonic properties of the modes:\footcite[93]{BainHildegardBingenMusical2015}
\hlabel{ln:niedermeyer_firstmention_END}%
%\noclub[2]

\simplex{Nur dieselben Töne, welche in der Tonart des Gesangstückes vorkommen, dürfen zur Harmonie verwendet werden (leitereigene Töne); leiterfremde Töne, somit alle Ausweichungen in andere Tonarten mittelst Semitonien, somit alle unwesentlichen Vorzeichen \sharp{}, \flat, \natural{} sind zu meiden, den Fall des Tritons und der falschen Quinte ausgenommen.}
  {\cites[pp.~iii--iv]{SchneiderGregorianischeChoralgesaengefuer1866}[98--9]{WagenerBegleitunggregorianischenChorals1964}}
{Only those very pitches which are part of the scale of a vocal piece can be used to harmonise it; pitches not part of the scale, and with them any move to another key by way of semitones, as well as accidentals {\normalfont \sharp{}}, {\normalfont \flat{}}, {\normalfont \natural{}} are to be avoided, except the tritone and the diminished fifth.}

Schneider's manner of harmonising cadences points to a salient difference between Germanic and French diatonic theories at this time.
According to the former, the protus cadence `E' \rightarrow{} `D' could be harmonised using A minor \rightarrow{} D minor harmony (\cref{mus:schneider_piejesu_131}),\footcite[131]{SchneiderGregorianischeChoralgesaengefuer1866} raising an avowed and frankly un-Germanic preference for dysfunctional perfect cadences constructed of white notes alone.
According to the latter, by contrast, C major \rightarrow{} D minor harmony was preferred in order to steer clear of any semblance of perfect cadences, imagined or otherwise.
It was believed that A minor \rightarrow{} D minor harmony could impress upon the listener that dominant \rightarrow{} tonic harmony was still implied even though `C' had not been sharped.\footcites[42]{NiedermeyerTraitetheoriquepratique1859}[Also discussed in][190--92]{LessmannRezeptiongregorianischenChorals2016}
The subtle distinction drawn by each side was emblematic of differing approaches to diatonicism that were split along linguistic and geographical lines---we shall return to each one below.

\subsection{Notions of the \emph{Palestrinastil}}
Some nineteenth-century theorists equated the \emph{Palestrinastil} with an aesthetic ideal rather than with the abstract paradigm of Palestrina's contrapuntal technique as codified by Fux.\footcite[18, 67]{GarrattPalestrinaGermanRomantic2002}
Cecilian composers parsed Palestrina's music for vestiges of a more plain, austere style that could be used to inform new compositions of church music that were distanced from popular or dramatic works.
The result of their researches led to polyphony being confused with monophony and homophony,\footcite[191]{EllisInterpretingMusicalEarly2005} and to the notion that Palestrina's music, in its reported stateliness, was no different from plainchant.
The notion endured until at least the early years of the twentieth century,\footnote{The notion was prevalent enough in London for one English historian of church music to discredit it as nothing more than hearsay. See \cite[pp.~115--116 n.~*]{BurgessTextbookPlainsongGregorian1906}.} and might explain why some composers took to accompanying certain passages of chant in bare octaves rather than with chords.
Consecutive octaves are rare in Palestrina's music, with one notable instance---in the `Agnus Dei II' of the \emph{Missa Papae Marcelli}---more likely to have been a concession to writing for seven parts (three of which being in canon) than an aesthetic principle.\footcite[p.~69 see bar 6 between the first alto and first bass parts; Lockwood does not mention the consecutive octaves in his commentary]{LockwoodGiovanniPierluigiPalestrina1975}

\hlabel{ln:ett_bare}%
The reform movement initiated by Adlung extended to Bavaria where King Ludwig I commissioned the composer Caspar Ett (1788--1847) to establish a less theatrical style of church music.\footcites[60]{HutchingsChurchmusicnineteenth1967}[p.~114 where the author mistakes Ludwig I for Ludwig II]{MuirRomanCatholicChurch2008}
Ett brought out a volume of chant accompaniments with a simplified texture (\cref{mus:ett_easterintroit_40}), returning to the `chorale texture' and not to the `Für Geübtere' approach we saw above.
Major-minor chord progressions pervaded the figured bass part,\footcite[40]{HauberCanticasacrausum1834} leading Söhner to describe them as redolent of an ecclesiastical \emph{Biedermeier} style.\footcites[124]{SoehnerGeschichteBegleitunggregorianischen1931}[37]{WagenerBegleitunggregorianischenChorals1964}
\label{ln:cecilian_octaves}%
They were sufficiently popular for the prominent Cecilian composer Franz Xaver Witt (1834--88) to bring out a new edition in 1869, to which he added parts of the Mass and some four-part polyphony for good measure.\footcite[unpaginated `Vorrende']{HauberCanticasacra1869}
We shall return to Witt's ideal of accompaniment later.
\noclub[2]
\hlabel{ln:ett_bare_END}%

\hlabel{hl:benz}%
By the mid-century, the view that accompaniments could be modelled on Renaissance polyphony led Johann Baptist Benz (1807--80) to publish a collection in which passages for voices in unison alternate with passages for voices in SATB harmony (\cref{mus:benz_1850_41}).\footcite[1, 41]{BenzHarmoniasacraGregorianische1850}
Benz's use of bare octaves was probably an attempt to reconcile chant accompaniment with notions of an old-fashioned but desirable vocal style.\footcites[37]{BenzHarmoniasacraGregorianische1851}[1, 28]{BenzHarmoniasacraGregorianische1864}
The ATB parts were perhaps delegated to the organ in the absence of other singers, the same versatility being adopted by Franz Xaver Reihing (1804--88),\footcites[76]{ReihingCantionalechorioder1855}[Cited in][92]{WagenerBegleitunggregorianischenChorals1964} and separately by Michael Hermesdorff (1833--85) who advertised the dual function of his accompaniments in their title.\footnote{\covid{}\cite{HermesdorffHarmoniaCantusChoralis1865}.}

\label{int:mettenleiter}%
In 1854, the bishop of Regensburg Valentin Riedel gave his approbation to the opinion of Johann Georg Mettenleiter (1812--58) that accompaniments should derive their diatonic nature from Renaissance and Baroque models:

\simplex{Der harmonischen Begleitung der gregorianischen Choralgesänge auch nur solche fortschreitende Harmonienfolgen in Anwendung kommen dürfen, die rein diatonischer Natur sind, und sich auf die Gesetze der Theorie, sowie auf die vollendetste Praxis der grossen contrapunctischen Meister -- in Melodie und Harmonie des 15. und 16. sowie der ersten Hälfte des 17. Jahrhunderts etc. stützen, und sich ihr anschliessen.}
  {\cite[p.~xxxxvi {[\emph{sic}]}]{MettenleiterEnchiridionchoralesive1854}}
{Hence only those chord progressions can be used to harmonically accompany Gregorian chant which are diatonic in nature while in melody and harmony being based on and following the laws of theory, as well as the most perfect practice of the great masters of \linebreak{}counterpoint -- of the 15\textsuperscript{th}, 16\textsuperscript{th} and the first half of the 17\textsuperscript{th} centuries etc.}
\noindent
Mettenleiter's harmonisations adopted a consonant approach to harmony using 5/3 and 6/3 chords with 4--3 suspensions and sharped notes not present in the chant part.
Mettenleiter was also seemingly not averse to modulating to different key areas, and appears to flat certain pitch classes when the melody occupies a different position in the scale (\cref{mus:mettenleiter_enchiridion_8}, p.~8). One might suppose that the composer envisaged a different type of harmony for such passages, but confirmation of that process is difficult to glean from the available music examples.
Mettenleiter's experiments attracted praise from some of Europe's most celebrated musicians.
The Belgian theorist François-Joseph Fétis (1784--1871), whose influential but inconsistent views on plainchant will be discussed in \cref{sc:fetis_inconsistent} below, was one such;\footcite[89]{WagenerBegleitunggregorianischenChorals1964} and another was the Hungarian composer Franz Liszt (1811--86), whose own campaign for church music reform had led to his seeking more austere methods of chant harmonisation.\footcite[88--9]{MerrickRevolutionReligionMusic1987}
Liszt followed contemporary developments on the subject closely enough to have a copy of the Niedermeyer-d'Ortigue treatise in his possession.\footcite[13]{DufetelReligiousWorkshopGregorian2014}
One of Liszt's contemporaries Anton Bruckner (1824--96) turned to chant for melodic material for various motets, and, perhaps as a preparatory step, harmonised `Veni creator spiritus' in a minor key with an abundance of 5/3 and 6/3 chords (\cref{mus:bruckner_venicreator_524}).\footnote{\cite[524]{GoellerichAntonBrucknerLebens1936}; See also \emph{A-Wn} Mus.Hs.39743 and \emph{A-Wn} Mus.Hs.19721.}

\section{Cecilianism}
\label{sc:cecilianism}%
\subsection{The Haberl circle}
\label{sc:haberl_circle}%
Begun in the 1860s, the first collected edition of the works of Palestrina reached its total of thirty-three volumes by 1907.\footcite[150]{HayburnPapalLegislationSacred1979}
Its editors were Theodore de Witt, Franz Espagne, Franz Commer, Johannes N.\ Rauch and the German priest Franz Xaver Haberl (1840--1910), who acted as editor-in-chief from 1879.
Those men were powerless to prevent the Romantic aesthetics of their age from influencing their understanding of Palestrina's music, which they believed to have numinous characteristics.
Their perception of Palestrina's style led nineteenth-century composers to seek in his compositions a kind of church music worthy of reproduction.\footcite[2]{BayreutherSituationdeutschenKirchenmusik2010}
Such musicians organised their endeavours into what became the Cecilian movement, so named after the patroness of musicians St Cecilia who was believed by some to represent spiritual music as distinct from more popular genres.\footcites[5, 7]{Marchiwhomfireburns2015}[Some writers have expressed doubts about St Cecilia's true attachment to music. See, for instance,][570]{FloodStCeciliaMusic1923}

Haberl led the charge by communicating his own notions of the style to parish organists in the journal \emph{Fliegende Blätter für katholische Kirchen-Musik}, founded in 1866.
Although its title may have invited comparisons to the Munich-based satirical magazine \emph{Fliegende Blätter}, Haberl's journal gained a reputation in strongholds of Cecilianism as an unquestionably serious endeavour.
It disseminated Cecilian ideals in an affordable package of articles and musical supplements,\footcite[1]{WittAufruf1866} while also serving as the vehicle for conveying the \emph{Cäcilienvereins-Kataloge} (\emph{CVK}), a numbered index of church music the Cecilian authorities approved on stylistic grounds.
Their often quite lengthy commentaries on approved items stand as testaments to Cecilian idealism in the nineteenth century.
The index includes several accompaniment books that will form the basis for discussions in the remainder of this chapter.

Another venture to disseminate Cecilian ideals was set in motion by Franz Xaver Witt who founded the Allgemeiner Cäcilien-Verband für Deutschland (ACV) in 1868, a society seeking to rejuvenate tenets thought to underpin Renaissance church music.\footcite[38--40]{GarrattPalestrinaGermanRomantic2002}
The official organ of the ACV was a separate journal, more technical than Haberl's, \emph{Musica sacra~:~Beiträge zur Reform und Forderung der Katholischen Kirchenmusik}.
Protracted articles on the accompaniment of chant appeared frequently amongst others dealing with the minutiæ of church music aesthetics and style.
\noclub[2]

The year 1868 also coincided with attempts by some ecclesiastical authorities to reconcile inconsistencies in church music practice.
Although the rise of Ultramontanism incited several dioceses to abandon proprietary liturgical customs in favour of those sanctioned by Rome, the movement had not yet gained the support of every bishopric.
Not only were rubrics liable to differ from one diocese to the next but the chants in use were also vulnerable to editorial mischief.
Sharping or flatting was sometimes effected by editors of chant books with no clear editorial motive, necessitating subsequent editors to rely on further accidentals to avoid outlining the prohibited intervals of the augmented fourth and diminished fifth, or to avoid leaps of the same.
We shall return to that vulnerability below (\cref{sc:modulation}).
A further vulnerability concerned the lack of available chants for certain feast days added to the ecclesiastical calendar by the Vatican, leading composers to turn their hand to composing chants in modern idioms to fill lacunæ in their dioceses' requirements.
We shall return to this new repertory below since it also sparked a considerable demand for accompaniments.

In the run up to the First Vatican Council (1869--70), the topic of a teetering church music practice was broached by one Fr Loreto Jacovacci who proposed a total reform of the church's chant books and the abolition of gaudy modern music from the liturgy.
He opined that the Medicean Gradual of \emph{c}.1614 should be taken as the basis for a new official edition and that its adoption should be made obligatory in all cathedrals and collegiate churches.\footcite[149]{HayburnPapalLegislationSacred1979}
The editorship of the Medicean Gradual had been falsely attributed to Palestrina by many scholars including Palestrina's nineteenth-century biographer Giuseppe Baini,\footcite[93--5]{BainiMemoriestoricocritichevita1828} but that falsehood was not acknowledged as such by the ecclesiastical authorities before the end of the century.

In the meantime, the commercial potential in printing chant editions bearing Palestrina's name was not lost on the Bavarian music publisher Friedrich Pustet.
His firm was already one of the primary publishers of Cecilian music editions and periodicals, which proved to be rather a calculated manoeuvre since the printing contract for a folio edition of the new chant book was then awarded to him---Haberl supplied new chants for those feasts added after 1615.
The terms reached between Pustet and the Vatican were ostensibly quite simple: at his own financial risk, Pustet would prepare each page for approval by a Vatican commission; in return, the Sacred Congregation of Rites (SCR) would grant Pustet a thirty-year monopoly to safeguard his investment, this being formalised in a decree dated 1 October 1868.

That form of the agreement was short-lived, however, because socio-economic pressures exerted by the Franco-Prussian War led not only to the premature conclusion of Vatican I but also to well-nigh insurmountable economic challenges for Pustet's firm.
He therefore solicited further protections from the SCR for smaller, more affordable editions to tide him over until the folio edition was complete.
Two further decrees were issued on 11 March 1869 and 12 January 1871 to protect two such chant books, one in octavo format.\footcites[pp.~xix--xx, p.~xx n.~10, p.~ 69]{EllisPoliticsPlainchantfindesiecle2013}[For the SCR's decrees see][150--4]{HayburnPapalLegislationSacred1979}
The folio edition finally saw the light in 1873 and served as the basis for Cecilian accompaniment books until the early years of the twentieth century.

\hlabel{ln:witt_octaves}%
Witt was among the first to publish organ accompaniments to Pustet's chant editions and began with the Mass Ordinary.
His procedure will be further explicated below, but for the moment let us consider two methods he claimed were derived from antiquity.
The first considered bare octaves to be most authentic since that style was reputedly used by the Greeks and early Christians (`Die altgriechische, wie die altchristliche Begleitung war die durch Consonanzen').
Witt made one concession to modernity, however, by permitting cadences in more parts (\cref{mus:witt_appendix}).\footcite[p.~iv \S{}3, pp.~99--100 ]{WittOrganumcomitansad1872}
It was thus that Witt accompanied the chanted parts of an instruction course for choir directors at Saint Gall in 1872, later reporting the incredulous surprise of attendees at the result.\footcite[26]{WittMeineCaecilienfahrt18721873}
\hlabel{ln:witt_octaves_END}%
In 1874, Heinrich Oberhoffer (1824--85) reckoned that an accompaniment in bare octaves did little to assist singers in maintaining pitch.
Probably for similar reasons did Oberhoffer advocate for D major chords in proximity to sung `B'\kern 1pt\natural{} because chords containing `F'\kern 1pt\natural{} would cause out-of-tune singing, in his view at least.\footcite[82, 101]{OberhofferSchulekatholischenOrganisten1874}

The second considered accompaniment by the organ tolerable as a necessary evil for choral support, but inferior to accompaniment by stringed instruments which, Witt claimed, could communicate nuances beyond the capabilities of an organ's steady wind supply.
The Freising-based choral director Johann Nepomuk Kösporer (1828--1900) dutifully arranged an accompaniment of chant for two violins, two cellos and double bass, of which a performance on 20 February 1877 was described by one journalist as `extremely effective' (`außerordentlich wirkungsvoll').\footcite[44]{WalterUmschauFreising1877}
Another composer also wrote a freely composed Mass for soprano and alto voices with the accompaniment requiring either an organ or an ensemble made up of violins, viola, cello, double bass and two horns---the organ part was simply a reduction of the orchestral parts.\footcite[1 and \emph{passim}]{HabertMessefurSopran1870}
It is not clear whether the composer's rationale was purely aesthetic, however, or whether financial considerations might have influenced the decision to delegate instrumental parts to the organ.

\subsection{The `system of passing notes' in Germany}
The papal brief \emph{Multum ad commovendos animos} of 16 December 1870 elevated the ACV to the status of an official Catholic corporation with its own cardinal protector.\footcite[128--9]{HayburnPapalLegislationSacred1979}
Although a music school did not open in Regensburg until 1874, the tacit authority bestowed upon the musicians in that city provided reason enough for others to seek inspiration in the performance practice there.
The Cardinal Archbishop of Cologne Johannes von Geissel sent one of his chaplains, Friedrich Koenen (1829--87), to Regensburg in 1862 to receive a kind of informal tuition from Witt.
Casual though the arrangement was, it was undoubtedly influential because Koenen later established a Cologne-based arm of the ACV with a choir numbering fifty boys and sixteen men.\footcite[48--50]{HoevelerKardinalErzbischofPhilippus1899}

Among the techniques reportedly passed on to Koenen by Witt was a new procedure of accompaniment that differed from Metteneleiter's chorale-textured, consonant approach: fewer chords than chant notes were to be used to produce a more flowing texture.
Dissonance in chant accompaniments was no longer considered a flaw because Witt believed its prevalence in Palestrinian polyphony gave it sufficient assent for use in other music.
Cologne was among those dioceses using its own chant edition, for which Koenen wrote accompaniments using the so-called `system of passing notes' (`das System der durchgehenden Noten').\footnote{\covid{}\cite{KoenenKyrialesiveCantus1876}.}
\hlabel{fn:competing_claim}%
A competing claim to the system was made by Belgian theorists, whose method will be discussed below in chapter three, but German journalists took no notice of international developments when they credited Witt alone with the first use of passing notes.\footcite[col.~503]{BaeumkerKirchenmusikMelodiesGregoriennes1880}
\hlabel{fn:competing_claim_END}%
Witt codified his method in 1872:

\dualcolumn{Sie ist leichter spielbar, weil eine Menge Noten keinen eigenen Akkord erhalten;}{It is easier to play because many notes have not their own chord;}
%\ParallelPar
\pagebreak{}

\dualcolumn{Sie entspricht mehr der Einfachheit des Chorales und ist weniger monoton aus demselben Grunde;}{It suits the simplicity of the chant better, and is therefore less monotonous;}
\ParallelPar
\dualcolumn{In den Melodien selbst sind nicht alle lauter Haupt- (betonte), sondern viele sind `durchgehende' Noten und das spricht ganz entscheidend für meine Theorie;}{In the melodies themselves all the notes are not of equal importance (accented); many are `passing notes,' and this is decisive for my theory;}
\ParallelPar
\duplex{Sie lässt die Melodie mehr hervortreten; denn eine Melodie über einem liegenbleibenden Akkord hebt sich viel gewaltiger ab und kommt viel mehr zur Geltung.}
  {\cites[p.~v]{WittVorwortzurOrgelbegleitung1872}[See also the same preface printed separately in][52]{WittVorwortzurOrgelbegleitung1872}}
{It allows the melody to be more \linebreak{}prominent, for a melody over a held-down chord stands forth much more boldly and is therefore more effective.}
  {\cite{HaberlMagisterChoralisTheoretical1877}, 1st ed. (English) from 4th ed. (German), 238; \cite[pp.~iii--iv of the Anglophone preface]{WittOrganumcomitansad1881}}
\noindent
The `passing notes' system was therefore applied at melismatic passages: \cref{mus:witt_continuous_30} demonstrates how the tenor part is set in contrary motion with the chant while the other parts function more like pedal notes.
Witt anticipated the terminal cadence by beginning in bare octaves before branching out into more parts and including a sharp.\footnote{\cite{WittOrganumcomitansad1872}, 1st ed., 30.}
The book's entry in \emph{CVK} indicates that sharped pitches occur rarely enough for their omission to be justified on the part of a player.\footnote{\cvk{126}.}

Since Witt's accompaniment book catered for Pustet's Mass Ordinary alone, it was left to other composers to provide accompaniments for the remainder of the Gradual.
In contrast to Witt's begrudging admission that the organ could indeed be tolerated, Haberl considered it a \emph{sine qua non} because it was believed to fortify the solemnity of a service.\footcite[134--5]{HaberlMagisterChoralisTheoretisch1864}
Haberl was faced with a choice of accompanimental systems but eventually settled on Witt's because it retained `perfect harmonic closes' at cadences.

Haberl's involvement in the accompaniment books discussed below largely remained that of an editor: while he also transcribed Pustet's chants into modern notation, he left it to other composers to harmonise them.\footnote{\cite{HaberlMagisterChoralisTheoretical1877}, 1st ed. (English) from 4th ed. (German), 237--8.}
For clarity, attributions of select accompaniment books are given in \cref{tab:pustet-accomp}.
The accompaniments to introits, offertories and communions from the Proper and Common of Saints were delegated to the Regensburg cathedral organist Joseph Hanisch (1812--92) and were published in 1875.
A second book by Hanisch was published in 1876 and received the enthusiastic endorsement of the \emph{CVK}, though that is hardly surprising given Haberl's influence on that index.\footnote{\cvk{248} and \cvk{282}.}
Hanisch was considered a kind of modern-day Palestrina figure by Haberl, who, in 1883, reckoned Hanisch's keyboard practice was worthy of record for the benefit of musicians everywhere:
\noclub[2]

\simplex{Jene so viel bewunderte Gabe des Hrn. Hanisch, fliessend, dramatisch und schwungvoll die Melodieen des gregorianischen Chorals zu begleiten, ist hier für alle diejenigen fixirt.}
  {\cite[unpaginated `Aus dem Vorworte zur ersten Auflage']{HanischOrganumcomitansad1883}}
{That most admired gift of Mr Hanisch to accompany the Gregorian chant melodies in a flowing, dramatic and lively manner is here recorded for all.}

In 1887, the young Max Reger (1873--1916) held quite a different view, and considered it farcical that the under-winded pipe organ in Regensburg cathedral could be deemed fit for the seedbed of Cecilianism.
Reger's account of Hanisch's playing is hardly consistent with Haberl's endorsement, judging it too fast for the reverberant acoustic.\footcite[79]{AndersonMaxReger18732012}
Although Reger had not started learning the organ yet,\footcite[p.~217; Reger took piano lessons from 1884 and organ lessons from 1888]{AdamsModernOrganStyle2007} his statement that the cathedral organ was unfit for purpose might not be without merit.
It had been built in 1839 by the Regensburg builder Johann Nikolaus David Heinßen (1797--1849) and placed behind the High Altar, but by order of King Ludwig I it was designed to be a modest instrument, no bigger than necessary to accompany singing while maintaining the audibility of clergy on the altar.
\hlabel{ln:renner_photo}%
Although a disposition of the instrument has not yet come to light, a photograph of the organ console taken in the early years of the twentieth century shows a single manual with a limited compass and about six stops on the right-hand jamb: the organ was probably therefore disposed with about a dozen in total.\footnote{\cite[35--7]{DittrichZurGeschichteOrgeln2010}.\label{fn:renner_photo}}

Johann Baptist Singenberger (1848--1924), a former pupil of Hanisch's who later became director of the American arm of the Cäcilienverein, maintained that his teacher's playing was a model for liturgical worship, but voiced sentiments similar to Reger's concerning the state of the cathedral organ:

\simplex{Ich betrachte Hanisch [als] das Muster eines Organisten für den liturgischen Gottesdienst. [\dagger{}] Effekthascherei ist ihm ferne, und könnte ihm eine solche auf der herzlich schlechten einmanualigen Domorgel auch nichts helfen. Und doch, wer immer beim Gottesdienste im Dome in Regensburg sein Orgelspiel hört, bewundert den Meister; man glaubt eine Orgel von 2 Manualen zu hören. [\ddagger{}] Diese frische und fließende Stimmbewegung, dieser Wechsel in Harmonie und Rhythmus, diese geist- und gemütvölle Erfindung und Verwendung der Motive[,] verbunden mit einer natürlichen, gewandten Registrirung, im engsten Anschluße an die betr[effenden] liturgischen Gesänge, bilden die Vorzüge unseres Meisters, eines wirklichen Beherrschers der Königin der Instrumente.}
  {\cite[1]{SingenbergerHerrJosephHanisch1891}; Reproduced in Haberl's eulogy in \cite[105--106]{HaberlJosephHanischDomorganist1893}}
  {I class Hanisch as the model organist for liturgical worship. [\dagger{}] He is far from a showman, and such would not have helped him on the sincerely poor single-manual cathedral organ. And yet, whoever hears his organ playing at the church service in the cathedral in Regensburg admires the master; one thinks one hears an organ with two manuals. [\ddagger{}] His fresh and flowing part-movement, changes in harmony and rhythm, spirit and soulful invention and use of motifs, combined with a natural, skilful registration in the closest connection to the liturgical chant possible, are virtues of our master, a true ruler of the king of instruments.}
\noindent
When Haberl reprinted Singenberger's eulogy some years later, he suppressed the section between \dagger{} and \ddagger{} that called into question the esteem of the cathedral organ.\footcite[p.~i n.~*]{HaberlOrganumcomitansquod1900}

Hanisch provided accompaniments to general responsories (`In qualibet Missa cantatur et respondetur') for the second edition of Witt's accompaniments of the Ordinary.
It was naturally anticipated that accompaniments would require more space in a printed volume than chant melodies, but Hanisch's responsories took up even more space than usual because they were reprinted in different transpositions.\footcites[p.~70*]{Gradualetemporesanctis1871}[106--107]{WittOrganumcomitansad1876}
The space required for accompaniments also proved to be a concern for Haberl and Hanisch who were obliged to omit graduals, alleluias and tracts from the Proper and Common of Time.
The explanation offered for those omissions was that choirs would otherwise become too reliant on the accompaniment, but one suspects that the true reason was simply to avoid exorbitant printing costs.
\hlabel{ln:haberl_transposition}%
Haberl recognised nonetheless that a choice of transpositions could benefit organists, and outlined a procedure whereby a player could transpose up or down by a chromatic semitone (from, say, three flats to four sharps or from two sharps to five flats) without the need for supplementary printed matter.\footcite[p.~iv]{HaberlOrganumcomitansad1875}
Accidentals could be raised or lowered by the player depending on whether they were transposing up or down.\footnote{\cite{HaberlMagisterChoralisTheoretical1877}, 1st ed. (English) from 4th ed. (German), 238; \cite{HaberlMagisterChoralisTheoretical1892}, 207.}
The procedure was an erudite compromise and was revived in the next century, as we shall discuss below (\cref{sc:transpositions}).
\hlabel{ln:haberl_transposition_END}%
It did not seem to satisfy the bishop of Castabala Louis Aloysius Lootens (1827--98), however, who also lamented that Haberl and Hanisch's accompaniments did not adopt the same dominant for every mode.\footcite[p.~403 n.~1]{Lootenstheoriemusicalechant1895}


Haberl and Hanisch followed up their accompaniments of the Gradual with those of the Vesperal in two sections issued respectively in 1877 and 1878.
Although the Gradual received a positive review in the \emph{CVK},\footnote{\cvk{345}.} the Vesperal drew criticism from the pedagogue Peter Piel (1835--1904) who held that the accentual hierarchy of repeated pitches required chords to change; Hanisch, by contrast, had retained the same chord.
Piel then critiqued the harmonic progressions since they were reportedly written without harmonic direction.
This, together with Piel's reservations about dissonant upper auxiliary notes, amounted to quite a castigating assessment of Cecilian practice.\footnote{\cvk{438}.}
Piel's views duly came to the attention of Witt, who revised his accompanied Mass Ordinary to rectify potential vulnerabilities, including a false relation in the `Dies iræ' accompaniment.\footnote{Compare the accompaniment at `flammis acribus addictis' in \cite[2nd ed., p.~90]{WittOrganumcomitansad1876} with \cite[3rd ed., p.~90]{WittOrganumcomitansad1881}.}


\subsection{Reforms, revisions and refinements, 1880--1900}
Cecilian belief in the historical accuracy of the Medicean Gradual conflicted with another approach to chant scholarship that was gaining traction in France and Belgium.
The mounting evidence against the Medicean edition gained from paleographical research led some theorists to consider its melodies faulty and to resolve in favour of competing chant editions to Pustet's.
Pope Leo XIII attempted to stem the tide with \emph{Romanorum pontificum} in April 1883, a decree reiterating the Catholic Church's stance in favour of Pustet's offerings.
While the decree also placed a moral obligation on bishops to adopt the Pustet editions in their dioceses, it stopped short of banning other editions outright which were permitted for the purposes of theoretical and `archaeological' research.\footcite[159--161]{HayburnPapalLegislationSacred1979}

The effect of the decree was immediate.
Not only did it bolster Pustet's reputation (who reproduced the decree among the front matter of future editions),\footcite[pp.~iii--vi]{Gradualetemporesanctis1884} but it also caused a run on the remaining Haberl-Hanisch accompaniment books.
The unprecedented demand caused the second edition to sell out entirely and required either a reprint or a new edition.
Faced with that choice, Haberl settled on the latter and brought out three further volumes of Gradual accompaniments between 1883 and 1884.\footnote{\cite[82]{HaberlvollstaendigeOrgelbegleitungGraduale1895}; The edition's contents are listed in \cref{tab:haberlhanisch-second}.}
Haberl's preface asserts that Hanisch's practice had not changed since Piel had voiced his criticism, and the available evidence supports that assertion: for example, dissonant upper auxiliary notes continued to be a notable feature of Hanisch's style from the very first accompaniment (\cref{mus:hanisch_upper_1}).
Nevertheless, refinements were made to other aspects of the accompaniments in the simplification of certain tricky passages by reducing the number of chords and by rearranging parts so organists could accompany without using pedals.
Haberl and Hanisch also modified the transposition of certain chants and, probably in response to the demands of organists, added accompaniments to Eastertide alleluias.\footcite[2nd~ed., unpaginated `Vorwort zur zweiten Auflage' and p.~1]{HanischOrganumcomitansad1883}

By all accounts, Pustet's resourcefulness allowed him to recognise the commercial potential in providing musicians with more options.
Accompaniments for graduals, alleluias and tracts were not yet readily available,\footnote{\cvk{1732}.} and following Hanisch's death in 1892 Haberl recruited one of Hanisch's former students, the Hitzkirch musician Josef Schildknecht (1861--99), to harmonise those portions of the Proper of the Time.
Schildknecht's work was duly published by Pustet as a supplement to the Haberl-Hanisch second edition (an inventory is reproduced in \cref{tab:schildknecht-supplement}), he having previously scored a success with his 1891 collection of \emph{recto tono} settings of the Proper of the Mass which gained widespread popularity because they suited choirs with little time for rehearsal.
Simplified settings satisfied the liturgical requirement that texts of the Proper ought to be chanted, and the organ accompaniments for the sake of choral support no doubt proved quite helpful.\footcites[`Vorwort und Empfehlung von Arnold Walther' in][p.~ii]{Schildknecht178Kadenzenfuer1891}[Cited in][64--5]{HornbachnerOrgelbewegungundOrgellehre2013}
The liturgical nature of graduals, alleluias and tracts meant they were usually performed in two combinations: either as gradual--alleluia or (during Advent and Lent) as gradual--tract.
\hlabel{ln:schildknecht_modulation}%
For the supplement, Schildknecht provided interludes to smooth over harmonic changes when the one did not share the same mode as the other.
Useful though the preludes undoubtedly were, Piel's opinion in the \emph{CVK} censures interludes following graduals such as that reproduced in \cref{mus:schildknecht_supplement_[60]},\footnote{\cite{SchildknechtOrganumcomitansad1892}, pp.~ii--iii, v--vi, [60].} since the alleluia or tract that followed it were both in the same mode.
Surely, so Piel argued, the interlude was to be considered otiose;\footnote{\cvk{1732}.} but perhaps Schildknecht was also concerned with providing short organ pieces in the chorale prelude idiom for the benefit of less experienced choirs.
\hlabel{ln:schildknecht_modulation_END}%

Demand arose for such preludes, interludes and postludes which was met by the choirmaster of Saint Gallen Johann Gustav Eduard Stehle (1839--1915), who brought out short contrapuntal pieces by Cecilian composers that supposedly matched the stylistic properties of the chant repertory:
\noclub[2]

\simplex{Stilgerechte Vorspiele über die detreffenden Choralmotive werden den Herren Organisten eine höchst willkommene Erscheinung sein -- das Vorspiel soll zum Cantus passen, wie ein Prolog zum nachfolgenden Stück; eine absolute, aber eigentümlich schwierige Anforderung der `Stileinheit'.}
  {\cite[unpaginated `Vorbemerkung']{StehlePraeludiaorganiad1892}}
{Stylish preludes to the relevant chants will be a most welcome inclusion for organists -- the preludes should match the chant, like a prologue to the following piece; an absolute, but peculiarly difficult, requirement for `stylistic unity'.}
\noindent
Stehle claimed that the compositions in his book were written to suit the Haberl-Hanisch second edition, presumably because they took account of the transpositions of the accompaniments.
Stehle did not account for differences between composers' approaches, however.
One prelude to the protus introit `Gaudeamus omnes' is marked \emph{Maestoso} and is intended for full organ registration (`Kraftvolle Registrierung'), whereas another is marked \emph{Langsam}, `Nicht zu stark' and `Gebunden', arguably producing quite a different \emph{Affekt}.\footnote{Compare preludes by Peter Piel, Joseph Schildknecht, Jacob Quadflieg and J. Breitenbach numbered 71--5 on pp.~49--51 and how they match up with the chant accompaniment in \cite[2nd~ed., p.~27]{HanischOrganumcomitansad1883}}
Singenberger proposed that registering an accompaniment should be different to registering an interlude:

\single{We would suggest that a registration be employed for the interludes different from the one used for the accompaniment. For the latter avoid a too loud registration which would induce the singers to scream and consequently, sing `flat'.}
  {\cite[unpaginated preface]{SingenbergerOrganAccompanimentCantate1912}}
\noindent
The SCR weighed in on the matter in July 1894, reminding organists to `preserve the sacred character' of the liturgy in their preludes and `decorously to support and not drown the chant' with their accompaniments.\footcite[p.~141 \S{}6]{HayburnPapalLegislationSacred1979}

A further supplement to the Haberl-Hanisch second edition was required when the Vatican instituted reforms to the Roman Breviary by adding feast days.
In December 1883, officials also modified the layout of certain neumes leading Witt to revise his Ordinary accompaniments.\footcite[viii]{WittOrganumcomitansad1885}
These changes anticipated wider reforms undertaken in 1884 that brought about the reversion of the Breviary's rubrics and the number of syllables in certain liturgical texts to the format proposed during the seventeenth-century pontificate of Urban VIII.
Feast days introduced into the ecclesiastical calendar by later pontiffs were also incorporated into the nineteenth-century edition.\footcites[264--5]{BatiffolHistoirebreviaireromain1893}[For an Anglophone translation, see][286--8]{BatiffolHistoryRomanBreviary1898}
Though Pustet produced a new chant edition of the Gradual in 1884 to bring his offerings up to date, the accompaniments issued by Haberl and Hanisch in 1883 and 1884 incorporated neither the updated neumes nor the enlarged calendar of feasts, hence the necessity for a further supplement.

The demand for such accompaniments was not met until after Hanisch's death when Jacob Quadflieg (1854--1915), the organist and choir director at the Marienkirche in Elberfeld and another of Hanisch's former students, assumed the mantle under Haberl's editorship.\footcite[For a brief Anglophone description of Quadflieg's credentials see][22]{DeacyContinentalOrganistsCatholic2005}
Accompaniments to the introits, offertories, and communions of the added feasts were composed by Quadflieg together with the chants of the Easter and Pentecost octaves (an inventory is reproduced in \cref{tab:quadflieg-supplement}).\footnote{For a review of Quadflieg's supplement, see \cvk{1720}.}
The pagination of the supplement was contrived so as not to conflict with the Haberl-Hanisch second edition, or indeed with Schildknect's supplement.

Quadflieg included contrapuntal preludes based on the first intervals of each harmonised intonation.
Instructions direct the player either to forgo using pedal (`s[ine] P[edal]') or instead to bolster its registration (`Pedal hervortr[etend]').
\Cref{mus:quadflieg_supplement_(45)} shows the sole prelude for which Quadflieg provided specific registrations, making clear the distinction between the left-hand chant snippet and the right-hand contrapuntal texture.
Passages most likely to be sung by solo voices (such as the intonations of chants or psalms) are harmonised in fewer parts---one recognises in particular the use of bare octaves---probably so the organ's involvement could be made discreet.
By contrast, the remainder of the accompaniment is probably intended to support a choir and is written mostly in four parts.\footcite[pp.~iv, (45), 103]{QuadfliegSupplementumadOrganum1894}

Quadflieg then took over the revision of Hanisch's accompaniments,\footnote{Although the present author could not consult the third edition, its preface was accessible because Haberl reprinted it in the June issue of \emph{Musica sacra}.}
the rationale for the execution of which being Piel's criticism of Hanisch's compositions that had raised doubts about the appropriateness of the official Cecilian style.
Eventually, even Haberl was forced to admit that improvements could be made, and permitted Quadflieg to incorporate subtle changes to certain neumatic groupings.
Preludes were newly composed by Quadflieg to anticipate the now-harmonised intonations of introits, offertories and communions.

\hlabel{ln:schildknecht_large_noteheads}%
Along with those musical changes and additions came some typographical updates: namely, that engraving replaced movable type, permitting the chant to be printed in larger note heads and the vertical alignment of notes on the staff to be rectified.
In previous editions, longer note values were centre-aligned which was probably little more than a typographical quirk inherited from editions of vocal polyphony where the parts were read independently.
But the alignment became enough of a nuisance for keyboard players who were required to read four or more parts simultaneously, hence the decision to move to left-aligned bars.
In a reduction of a polyphonic mass by Hanisch, the keyboard part had been typeset in larger notes, perhaps so an amateur répétiteur would not become disoriented,\footcite[1 and \emph{passim}]{HanischZweiLateinischeMessen1870} and a similar scheme had been used in Hanisch's accompanied Mass Ordinary too, perhaps to distinguish between numinous chant and terrestrial accompaniment.
\hlabel{ln:schildknecht_large_noteheads_END}%
Hanisch's Ordinary accompaniments ostensibly competed with Witt's, for their first edition appeared in 1888, the same year Witt died.
They therefore might have been a project for Hanisch to rectify vulnerabilities or omissions in his colleague's edition.\footnote{\covid{}\cite{HanischOrganumcomitansad1888}.}
Be that as it may, Hanisch's accompanied Ordinary was revised in 1893, one year after he himself had died, though it is unclear whether Hanisch had worked on the revisions in previous years.
One reviewer declared that Hanisch's accompanied Ordinary `enjoys implicitly the approbation of the SCR' and that its influence on church music practice was indisputable.\footcite[78]{ReviewOrganumcomitans1893}

Those feasts added to the 1884 Roman Breviary were newly harmonised by Quadflieg, causing the volume to increase by some one hundred leaves.\footcite[82--4]{HaberlvollstaendigeOrgelbegleitungGraduale1895}
Schildknecht's graduals, alleluias and tracts were apparently not incorporated into later volumes of accompaniments, though that composer produced a standalone volume of accompaniments to the Mass Ordinary in three parts for organ or harmonium.\footnote{\covid{}\cite{SchildknechtAllerleichtesteBegleitungOrdinarium1897}.}
Several feasts were added after 1895 that required a further revision of the accompanied Gradual which was published in 1900.
The feast of Anthony Maria Zaccaria, who was canonised by Leo XIII on 27 May 1897, was among those feasts harmonised by Quadflieg: \cref{mus:quadflieg_zaccaria_ap1} demonstrates how Quadflieg's preludes were now longer than those he had composed in 1894.
Haberl suggested that students ought to study them as organ pieces in their own right.
The texture of the accompanied intonation was reduced to three parts, but the choral accompaniment remained in four.
Typesetting the chant in larger noteheads now appears to have been standard procedure.\footnote{\cite{HaberlOrganumcomitansquod1900}, 4th ed., p.~1 of `Appendix ad Organum comitans'.}


\subsection{The rise and fall of Cecilian influence}
Given Regensburg's authority in the domain of Catholic Church music, its status as a hub of international repute for music pedagogy was formalised in 1874 with the establishment of the Katholische Kirchenmusikschule.
The school offered an eight-month course in \mbox{aesthetics}, liturgy, history of music, chant (and its harmonisation), score reading, conducting, repertoire and singing.
The circumstances surrounding its foundation are confused by two conflicting narratives.
One describes it as a joint venture between Haberl and Witt,\footcite[8]{BayreutherSituationdeutschenKirchenmusik2010} the other as Haberl's single-handed achievement.
\nowidow[2]

Liszt's continued interest in developments in church music style had brought him to Regensburg in 1868 where he met both Witt and Haberl.
In a subsequent letter to Haberl dated 22 November 1876 Liszt requested copies of Witt's accompanied Ordinary as well as the Haberl-Hanisch accompanied Gradual.
It has not been possible to ascertain whether Liszt intended his letter deliberately to coincide with the feast of St Cecilia, though whether happenstance or not it nonetheless failed to stir Haberl into action---the accompaniments in question were never supplied.
Haberl's indolence has been attributed to a strained relationship with Witt, lending sketchy credence to the idea that Haberl founded the Kirchenmusikschule on his own.
Clarification probably lies in the papers of Witt and Haberl held at the Bischöfliche Zentralbibliothek in Regensburg, where Witt's correspondence numbers some 30,000 letters.
At the time of writing, however, neither they nor Haberl's correspondence have been catalogued.\footcite[152, 170, 180]{LibbertFranzLisztund2001}

In the US, the seeds of church music reform were first planted as early as 1838 when the first American Cecilian Society was established in Cincinnati by John Martin Henni.
The venture all but petered out soon thereafter, but was revived when Henni was appointed archbishop of Milwaukee in 1844.
There, with the assistance of the Austrian priest Joseph Salzmann, Henni established the Seminary of Saint Francis in 1856; but  the American Civil War further hampered progress and placed on hold any plans for a national movement of musical reform.
It was not until after that war that Salzmann managed to return to Europe where he solicited funds for a Catholic Normal School.

The Normal School's first roll of nineteen students in June 1870 coincided with \mbox{developments} in Regensburg which led to Pustet's being afforded the Vatican's protection.\footcite[9--10]{DamianHistoricalStudyCaecilian1984}
In a bid to popularise the Cecilian movement in the US, Salzmann sought Witt's advice on finding a musician to lead it.
Witt recommended his former pupil, the Swiss musician Johann Singenberger (1848--1924), who assumed the directorship of Wisconin's Catholic Normal School from 1873.\footcite[6]{GrabrianMilwaukeeWisconsinAmerica1973}
Under the anglicised name John Singenberger, he also taught seminarians to be choral directors and organists.\footcite[167]{JohnsonCrosierFrontierLife1959}
A desire to foster an American analogue to Regensburg's Cäcilienverein led Singenberger to convene the first annual congress of the newly stylised Amerikanische Cäcilien-Verein in Milwaukee on 17 June 1874.

Pustet established two branches of his printing firm in America to capitalise on the spread of Cecilianism there, the first in New York in 1865 and the second in Cincinnati in 1867.\footcites[224]{MuirFullPantingHeart2004}[129]{MuirRomanCatholicChurch2008}
He was therefore well placed when Singenberger's society elected to start a journal of its own in 1874.
Like its equivalent in Germany, the American journal disseminated articles on church music style and performance practice in German; but unlike it, some articles were also included in English.
The German articles were probably intended for the large German-speaking population that had settled in Northeastern American cities following the socio-economic upheaval of the German Revolutions in 1848--9, and German remained the primary language of the journal which was entitled \emph{`Cæcilia': Vereinsorgan des Amerikanischen Cäcilien Vereins}.
By 1878 the society had attracted some 3,000 members,\footcite[220]{OgasapianChurchMusicAmerica2007} and Pustet went on to publish translations of German textbooks previously published in Germany.
Singenberger remained at the head of the Amerikanischen Cäcilienverein for some fifty years, during which he exercised a considerable influence on Catholic Church music in the US.\footcite[1]{BliedThreeArchbishopsMilwaukee1955}

Among certain chant books for which accompaniments were written by Singenberger are several by the German composer Joseph Hermann Mohr (1834--1892) who should not be confused with Joseph Franciscus Mohr (1792--1848), the Austrian priest and author of the Christmas hymn \emph{Stille Nacht}.
J.\ H.\ Mohr's chant book \emph{Cäcilia} contained transcriptions of chant melodies as well as practical rubrics for their use during the Mass or Office.\footcite[notes on p.~231]{MohrCaciliakatholischesGesang1874}
The chant book enjoyed considerable popularity, running to a thirty-sixth edition by 1909.
The accompaniments composed by Singenberger were published in a separate volume and were claimed to be stricter in tonality than accompaniments by other Cecilians such as Witt and Hanisch (`in der Tonalität strenger als Begleitungen aus Witt, Hanisch etc').\footcite[unpaginated `Vorbemerkungen']{SingenbergerOrgelbuchosephMohr1888}

It was not the first time a composer had harmonised one of Mohr's chant editions.
In 1877 Mohr's name alone appeared on a book of accompaniments that were in fact composed by Heinrich Oberhoffer, with preludes and postludes being composed by numerous other musicians including Piel.\footnote{\covid{}\cite{MohrOrgelbegleitungCantate1877}; \cvk{354}.}
In 1878, Piel composed accompaniments to another of Mohr's editions that mentioned Piel's involvement only in the preface; again, the title page bears Mohr's name alone.
As shown in \cref{mus:mohrpiel_cantiones_56}, Piel annotated certain tenor notes with `d' and `s' depending on whether they were to be played by the right hand (\emph{dexter}) or the left (\emph{sinister}).\footcites[A description of `d' and `s' is provided in][233--4]{Directoriumchoriad1874}[See also][unpaginated `Preface' and p.~56]{MohrCantionessacraeCollection1878}
Such was also his practice when composing accompaniments for Mohr's \emph{Ordinarium Missæ}, whose sixth edition was published in 1884 with chant transcribed in modern notation.\footcite[3 and \emph{passim}]{MohrOrdinariumMissaesive1884}
Piel's careful directions for dividing inner notes between the hands extended to situations where an optional pedal part was indicated (\cref{mus:piel_mohrordinarium_2}), doubtless to the benefit of inexperienced players.\footcite[unpaginated `Vorrede' and p.~2]{MohrOrgelbegleitungMessbuechleinund1888}

The US was not alone in receiving exported Cecilian church music practice.
German missionaries brought Cecilian ideals and chant books to the South African province of KwaZulu-Natal where one Franz Pfanner established a Trappist monastery in 1882.
Another German Trappist monk, Willibald Wanger (1872--1943), published not only \emph{Scientific Zulu Grammar} but was also responsible for translating the bible into that language and for editing a chant book to match.
Accounts of Wanger's life subsequent to his mission are unreliable, but it seems that controversies churned up by his translations into Zulu of biblical documents forced his return to Germany.
An outspoken critic of the Third Reich, he was executed in 1943.\footcites[410]{BallingApostleSouthAfrica2016}[The date of Wanger's death is reported as 1944 in][p. 244 n.~21]{HaggardDiaryAfricanJourney2000}

\label{sc:wanger}%
Wanger's chant book was published in Germany in 1894 with rubrics in both German and Zulu,\footcite[34 and \emph{passim}]{WangerInncwadiYamagamaOkuhlabelela1894} and included accompaniments intended for choral directors and `die Organisten'.
Since no organs existed in KwaZulu-Natal at that time, the preface suggested that the harmonium could serve as an alternative.
The style of accompaniment bears some resemblance to Witt's and Hanisch's, particularly where some bass notes (styled as breves) endure for multiple chant notes, and where inner parts follow the chant in similar or contrary motion (\cref{mus:willibald_20}).\footcite[unpaginated `Vorrede' and p.~20]{WangerOrganumcomitansad1894}
The Roman numerals serve to designate one of a number of permutations of choral forces described in the book's preface: I for solos and II for choir; I for boys' choir and II for girls' choir; or I for boys' and girls' choirs and II for men's choir.\footcite[p.~vii]{WangerInncwadiYamagamaOkuhlabelela1894}
Soloists are often accompanied in three parts while choirs are accompanied in four; oblique lines are used in exceptional cases to indicate tacets.

\hlabel{ln:recht_links}%
Just like in other Cecilian accompaniment books, the annotations `r' and `l' indicated the division of the tenor part between the hands: `recht' for the right and `links' for the left.
Singenberger adopted `r' and `l' too, but the fact that they stood for the English words `right' and `left' was little more than convenient happenstance.\footcite[unpaginated preface]{SingenbergerOrganAccompanimentCantate1912}
Indicating which of the hands to take the tenor part was not solely a concern of chant accompaniments, since such indications were useful in elementary organ methods and were also used in some cases to apportion pedal notes between the feet.\footcite[See, for instance, Quadflieg's contributions in][78--83]{Diebold100groessereund1896}
\nowidow[2]

The Austrian composer and founder of the Österreichischen  Cäcilien-Vereins Johann Evangelist Habert (1833--96) relied on such an annotative scheme in his chant accompaniments to indicate the division of a pedal part between the feet,\footcite[56]{HornbachnerOrgelbewegungundOrgellehre2013} probably as an instructional aid for less competent organists.
The passage quoted in \cref{mus:habert} details which part of the foot should play the bass note, whether it be the heel (`a' for `Absatz') or the toe (`s' for `Spitze').\footcite[16]{SchmetzHarmonisierunggregorianischenChoralgesanges1894}
A slur-like mark was later used to indicate the substitution of one foot for another.
The method was used in 1881 to annotate accompaniments printed in Habert's journal \emph{Zeitschrift für katholische Kirchenmusik} but was also applied to elementary compositions for the organ.\footcite[2]{HabertOrgelcompositionen1877}
Harmonisations of Habert's included `Asperges me', `Responsorien zur Messe', `Missa, vulgo de Angelis' and `Missa in Dominicis per annum', all of which having come to light by 1885.\footcite[469--70]{E.KirchlicheCompositionenJohannes1885}

The proximity of Bohemia to Germany posed fewer geographical challenges to the spread of Cecilian ideals.
Although the Czech diocese of Hradec Králové produced its own chant book in 1896 rather than adopting Pustet's Gradual, Czech practice was also to use two separate choirs for the sake of more diversity (`K vůli větší rozmanitosti').
The relevant indications were `S' for the first choir (of sopranos and altos, or solo voices), and `T' for the second  (of tenors and basses, or \emph{Tutti}).\footcite[324]{Oltarpoucnamodlitebni1896}
\hlabel{hl:orel}%
It was the Czech musician and pedagogue Dobroslov Orel (1870--1942) who first used chant as a pedagogical device in that diocese, and whose manual on the subject includes references to numerous German accompaniment books including the Haberl-Hanisch editions.
Moreover, Orel's book includes cadential formulæ and several accompaniments from the Roman Vesperal by a Czech composer Fr.~Jirásek, who was probably the church musician Františk Jirásek (1856--1906).\footcite[p.~96 n.~438, pp.~100--102]{AndrsovaDobroslavOreljeho2019}
Although the accompaniment appears to use the chorale texture, a curious major seventh after the intonation in \cref{mus:orel_jirasek_66-7} is most unusual: perhaps it was supposed to prepare an instance of `B'\kern 1pt\natural{} in the chant; perhaps also the bass part was simply a typographical error and was meant to be \emph{e} instead.\footcite[66--7]{OrelTheoretickopraktickarukovetchoralu1899}

During the thirty-year monopoly granted to Pustet's Gradual, Haberl and the Vatican were quick to dismiss the evidence against the accuracy of the Medicean edition that historians and paleographers were placing on record.
Older melodies than the ones supposedly edited by Palestrina were shown to exist in earlier sources, a fact that had untold consequences for methods of chant harmonisation.
Such figures as the distinguished Austrian conductor and composer Heinrich von Herzogenberg (1843--1900) argued that chant pre-dated the invention of harmony and was therefore incapable of being harmonised.\footcite[135--7]{HerzogenbergPielHarmonieLehre1890}
Haberl used the preface of Schildknecht's supplement to rejoinder that Pustet's edition was based on the Medicean, and that accompaniments remained perfectly admissible since they were based on what was believed to be a Palestrinian approach.\footcite[p.~iv]{SchildknechtOrganumcomitansad1892}
\hlabel{hl:uriarte}%
Haberl's bullish stance was questioned by the Spanish composer Eustoquio C. de Uriarte (1863--1900) who found it difficult to distinguish, in Haberl's \emph{Magister choralis}, between sound doctrine and unsubstantiated claims.\footcite[149]{UriarteTratadoteoricopracticocanto1890}

By 1899, mounting evidence against the accuracy of the Medicean edition led the SCR to remove (without fanfare) \emph{Romanorum pontificum} from the official inventory of decrees in force and, by extension, to dissolve the ardent protections afforded by the Vatican to the Medicean edition.\footcites[175]{CombeRestorationGregorianChant2003}[166]{HayburnPapalLegislationSacred1979}
It led to several barbs being launched against Haberl as the most notorious perpetuator of the Palestrina myth, including one by the Italian priest Carlo Respighi.\footcite[6]{RespighiNuovoStudiosu1900}
A certain Raphael Molitor conducted his own researches into the matter and determined that Baini's claims about Palestrina were without merit.\footcites[pp.~v, 241]{MolitorNachTridentinischeChoralReformRom1901}[For an Anglophone discussion of Molitor's involvement see][205]{MuirRomanCatholicChurch2008}[See also the charge that Baini's lack of critical thinking belied his welter of claims in][34--5]{LockwoodGiovanniPierluigiPalestrina1975}
It was later discovered that the Medicean Gradual had in fact been published under the direction of Felice Anerio (\mbox{\textit{c.}1560--1614}) and Francesco Soriano (\mbox{\textit{c.}1549--1621}),\footcite[208--9]{HileyGregorianChant2009} but even though the tide had turned against him Haberl refused to abandon his course.

The morass of decrees and privileges published and granted between 1868 and 1871 sowed enough seeds of confusion by century's end that few knew exactly when Pustet's monopoly was set to expire.
The Vatican confirmed the expiration date to be 1 January 1901,\footcites[pp.~xix--xx, 69]{EllisPoliticsPlainchantfindesiecle2013}[For the SCR's decrees in question, see][150--4]{HayburnPapalLegislationSacred1979} and distanced itself from the Medicean edition by adopting a competing Gradual edited by French monks that will be the focus of our discussion below.
\label{sc:cecilian_resistance}%
Some Cecilians maintained their previous course and opted not to follow the Vatican's new direction.
Haberl even weighed in against the Vatican's stance, but for that conduct he was upbraided by the SCR in a brief dated 18 February 1910.\footcite[282--3]{HayburnPapalLegislationSacred1979}
Cecilians in Germany and the US who perpetuated false and prejudicial views of the new official edition received cautions for their actions which were the subject of mirth in some French publications.
An extended footnote to a French translation of the decree reports rather gleefully that Haberl had agreed to acquiesce to the Vatican's demands and not to print further polemics on rhythm in \emph{Musica sacra} or \emph{Fliegende Blätter}.
Haberl also exhorted Cecilians to follow his example and to obey the wishes of the Holy See.\footcites[82]{decisionromainerythme1910}[For the brief in Italian and in an alternative French translation, see][31--3]{editionsrythmiquesSolesmes1921}
Today, \emph{Grove Music Online} records Haberl as `one of the pioneers of modern musicology',\footcite{HaberlHaberlFranzXaver} but his perpetuation of musical falsehoods in the face of contradictory evidence surely makes it impossible to justify that view any longer.
\nowidow[2]
