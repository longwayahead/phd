%For a clickable link in the toc to arrive here correctly
\phantomsection
%For the Table of Contents
\addcontentsline{toc}{chapter}{Postscript\hspace*{20pt}Recent developments}
%Suppress default chapter details in TOC
\addtocontents{toc}{\protect\setcounter{tocdepth}{-1}}
%Suppress default PDF bookmarking behaviour
\hypersetup{bookmarksdepth=-2}
%Change "Chapter" to "Postscript"
\renewcommand{\chaptername}{POSTSCRIPT}
%Suppress chapter numeral
\renewcommand{\thechapter}{}
\chapter{Recent developments}
%Restore default chapter details in TOC, for bibliography and appendices, etc
\addtocontents{toc}{\protect\setcounter{tocdepth}{1}}
%Restore default PDF bookmarking behaviour
\hypersetup{bookmarksdepth=1}
% rendered obsolete and ... by bolt from the blue
%Nova organi harmonia
%
Despite the far-reaching effects of Vatican II on church music, the practice of accompanying plainchant on the organ has not entirely died out.
In fact, it remains today as much a living tradition as before the Catholic Church introduced its reforms.
While some monastic foundations adapted their musicking to suit the vernacular,\footcite[114--117]{LynchSingNewSong2019} others---such as the Benedictines---have largely maintained a Latinate liturgy.
The Solesmian organist Jean Claire (1920--2006), who had studied with Potiron before succeeding Gajard as Solesmes's \emph{maître de chœur}, reported having adopted the sustained style in his own accompanying:

\simplex{Cet accompagnement correspond exactement à notre sensibilité actuelle~: les accords sont tenus longuement, il n'y a un changement d'accord que lorsque cela est vraiment indispensable.}
  {\cite[345, 397, 400]{Pinguetecolesmusiquedivine1987}}
{This accompaniment matches exactly how we feel about it now: chords are held for a long time, and a change of chord only happens when it is really necessary.}
\noindent
While Claire preferred unaccompanied chant, the usefulness of accompaniment to support the monks justified its retention.
During the 1980s at Solesmes, it was more likely to be an aspect of feast days (`je n'accompagne que pour les grandes fêtes'); while on other days, Claire directed the Schola.
It should be noted that accompaniments written by the religious Ferdinand Portier (1914--2009) and published during the 1980s at Solesmes are not necessarily representative of Claire's practice, Portier not having been a monk of Solesmes.
With that being said, however, the chord changes in the passage quoted in \cref{mus:portier_ictus_16} were evidently contrived to coincide with \emph{ictuses}: note also the additive procedure at `Kyrie'.
Portier's accompaniments are therefore not dissimilar to those published by Solesmian harmonisers at the beginning of the twentieth century (compare, for instance, to \cref{mus:bas_angelis_kyriale_40,mus:vranken_angelis_7}).\footcite[unpaginated `Avertissement', p.~16]{PortierLibercantualiscomitante1981}

The notion lives on that accompaniments ought to align themselves with characteristics inherent in the chant, resurfacing in 1985 when Luciano Migliavacca (1919--2013) contributed his thoughts on the matter to the International congress at Subiaco.
He proposed to limit the accompaniment to notes which had already appeared in the chant: we might take the Xes in the passage quoted in \cref{mus:migliavacca} to mean that the appearance of notes in the accompaniment was premature, either because they had not yet appeared or because their use in the accompaniment coincided with their first occurrence in the chant.
On the use of dissonance, Migliavacca made the following remark:

\simplex{Quanto al tipo di accordi possibili, ogni consonanza e dissonanza può essere valida, purché scaturisca come logica conseguenza armonica della melodia.}
  {\cite[133, 137]{MigliavaccaArmonizzazionemodaleCanto1985}}
{As for the type of possible chords, every consonance and dissonance can be valid, as long as it arises as a logical harmonic consequence of the melody.}
\noindent
A similar notion was voiced in 2000 when Federico Del Sordo (b.1961) derived a theory of accompaniment from writings by the semiotician Umberto Eco (1932--2016).\footnote{\cite[43]{Ecostrutturaassentericerca1968}.}
Del~Sordo suggested that chant is as divisible into segments as language is divisible into monemes.
As in linguistics, accompaniments reportedly contained musical monemes, though they were considered without meaning until an organist could bring some meaning to them through harmony.\footcite[53]{DelSordomonemicanellaccompagnamento2000}

Viewing the chant through the prism of the word (`postrzeganie melodii przez pryzmat słowa') informed the method by Mariusz Białowski (b.1971) of Ponań.
Cadences were to be harmonised first by matching the scale degree on which they occurred with the relevant chord indicated by the Roman numerals quoted in \cref{mus:bialowski_scale}.
The cadence on \pitch{4}\kern 1pt\flat{} quoted in \cref{mus:bialowski_harmonisation} therefore required a chord to be built on the same pitch.
The rationale for using \negpitch{6} was not made clear, however.
Chords were to be used as sparingly as possible and only to properly accompany the text.
As a result, it was not simply a matter of harmonising a melody but of producing bespoke accompaniments for each verse if the text were different.
Białowski made a semiological analysis of the chant in adiastematic notation to benefit from certain nuances that were reportedly lost by its representation in quadratic notation.\footcite[20, 24--9]{BialkowskiAnalizasemiologicznomodalnaswietle2012}

\hlabel{int:marier}%
Approaches to analysing accompaniments has left something to be desired in some instances, particularly where the analyst attempts to glean information from a contrapuntal accompaniment by using harmonic, major-minor methodologies.
The analyst responsible for \cref{mus:marier_analysis} arguably missed the mark by considering \emph{f}$^\prime$ at `mea' and `Galilaeam' as passing sevenths when they might better be understood as minor tenths above the bass part.
And why were these chords also labelled `4/3'?
There is little doubt that the analyst in question derived a false sense of security from the theory of chord inversion, which would take the tenor note \emph{g} to be the root of the chord---the Roman numerals placed beneath the bass staff serve to corroborate that inference.\footcite[144, 148]{AtwoodInfluencePlainchantLiturgical2014}
\hlabel{int:marier_END}%

Recent recordings from Notre Dame de Fontgombault bear witness to the sustained style that Claire had adopted, and which continues to be the preferred medium of accompaniment in certain monastic settings.\footcite{TheAbbeyofNotreDameIntroitBenedictaSit2001}
Quite apart from those settings, however, is the modern trend to appropriate chant for quasi-spiritual contexts.
Some compact discs purporting to be meditation aids incorporate chanting with synthesiser accompaniment with long, sustained chords.
The album by the recording engineer Dan Gibson entitled \emph{Illumination: Peaceful Gregorian Chants} combines chanting by the Gregorian Schola of the Pittsburgh Latin Mass Community with an accompaniment arranged for synthesizer, strings and aleatoric bird song by the film composer Daniel May.\footcite{GibsonIntroit2014}
The demand for such recordings was recently proven strong once again with the 2020 Decca release of \emph{Light for the World}, an album of chanting with a similar accompaniment, which placed fifth in the UK charts.\footcite{LightWorld}
It was the brainchild of the music producer James Morgan, who recorded the Poor Clares of Arundel, Sussex UK.\footcite{ArundelPangeLingua2020}
One reporter for \emph{The Guardian} relayed the following from one of the recorded nuns, Sister Gabriel, who suggested that the CD was designed to meet a demand in the public at large for the `need to zone out and find a place of peace'.\footcite{MoorheadSwingoutsisters2020}
