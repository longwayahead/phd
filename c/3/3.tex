\chapter{Free rhythm: \emph{Zeitgeist} and zealotism}
\section{Early applications in Belgium and Germany}
\subsection{Gevaert's hexachordal accompaniment}
\hlabel{int:gevaert}\label{hl:gevaert_1895}\label{ln:gevaert_hexachordal}%
Although Gevaert had advocated in 1856 for accompaniment in parish churches, by 1895 he had arrived at completely the opposite position and dismissed accompaniment altogether (`Le meilleur accompagnement du plain-chant ne vaut rien').
Accompaniments were to be permitted solely for the sake of choral support, provided they were played in unison with the chant.
The interval of a perfect fifth could also be interspersed `here and there'.\footcite[125]{Gevaertmelopeeantiquedans1895}
Gevaert's late-century view did not arrive \emph{ex nihilo}: during the 1870s he had courted the idea that an appropriate style of accompaniment might be revealed through analysing the chant.

By considering sources from antiquity, Gevaert determined that Fétis's definition of \emph{tonalité} was little more than specious.\footcite[73]{ChristensenStoriesTonalityAge2019}
The music historian Louis-Albert Bourgault-Ducoudray (1840--1910) concurred with Gevaert's determination, and Fétis's pronouncements began yielding to a new wave of scholarship that sought answers to musical quagmires in classical texts.\footcite[59]{Bourgault-DucoudrayEtudesmusiqueecclesiastique1877}
The notion that Monteverdi had signalled a shift towards major-minor harmony fell out of favour when such nineteenth-century thinkers shifted their attention to theorists such as Heinrich Glarean, whose `mutual interchange of modes' provided enough justification for a presumed gradual shift from Lydian to Ionian modes and from modality to major-minor tonality, although that was simply a case of replacing one specious construct with another.\footcite[129]{GlareanDodecachordon1965}
\nowidow[2]

When Gevaert succeeded Fétis as director of the Brussels conservatory in 1871, he became a catalyst for a more fundamental revision of the history of music.\footcite[375]{PouginBiographieuniversellemusiciens1881}
Gevaert's revisionism extended to Belgian church music, a domain in which Fétis's authority had long exercised a considerable influence, writing in 1875 that once-graceful performances of chant melodies had been replaced by a heavy style of chanting, now practically ubiquitous.\footcite[390--92]{GevaertHistoiretheoriemusique1875}
%(`on verra comment des mélodies d'un contour gracieux ont pu se transformer en ce chant lourd que l'on entend dans nos églises').
The Belgian priest Pierre-Jean Van Damme (1832--98) reported on just such a performance practice at Ghent's Groot Seminarie when he was appointed a teacher there in 1869, and set out  to seek a better alternative in Rome and Germany.\footcite[14--15]{RobijnsJaakNikolaasLemmens1981}
In Rome, Van Damme was disappointed to encounter operatic secular music being used in the liturgy, and all he found in Germany was what he called a `terribly old defective routine' (`une vieille routine fort defectueuse').
The last was reportedly used by Heinrich Oberhoffer and Johann Baptist Benz, among others, whose methods of chant performance we discussed in chapter one.
Van Damme also encountered Witt during a brief visit to Regensburg, but took a dim view of that Cecilian's performances of polyphony and chant, suspecting Witt knew little of Ancient Greek music (`il m'a semblé n'être pas extraordinaire en plain-chant, et n'avoir aucune idée de la musique grecque').
On vocalising that suspicion in 1887, Van Damme ignited vigorous protests from the seedbed of Cecilianism.\footcite[53--4]{VanDammeromanhistorique1887}

On his return to Belgium in 1870, Van Damme sought Gevaert's tutelage,\footcite[p.~13 n.~1]{VanDammeaccompagnementplainchant1881} and the ensuing collaboration led to a new book of chant accompaniments that sought alternatives to the `monotony and heaviness' of chordal, consonant harmonisations (`des principales causes de la monotonie et de la lourdeur').
From 1871, they experimented with a new procedure comprised of fewer chords and more passing notes which the musicologist Benedikt Leßmann has recently deemed a stepping stone on the path to free rhythm in French-speaking countries.\footcite[139]{LessmannAllTheseRhythms2008}
As discussed above (\cpageref{fn:competing_claim,fn:competing_claim_END}), Witt made a competing claim that he had been the first to introduce dissonances into the accompaniment and that Gevaert had plagiarised the idea (`sowie durch Gevaert's Nachamung meines Systemes').\footnote{The claim was first placed on the record in 1872 prior to Witt redoubling his efforts to out-manoeuvre Gevaert in an addendum dated 1881. See \cite{WittOrganumcomitansad1881}, 3rd ed., pp.~iv, viii.}
Van Damme refuted the claim on Gevaert's behalf by pointing out Nisard's use of dissonance during the 1850s.\footcite[35]{VanDammelegende1886}
The claim is made all the more tenuous in the light of the well-nigh universal use of dissonance we observed in the eighteenth century, though it is quite possible that the musical reforms instituted by Adlung's circle divorced nineteenth-century practice from those of earlier centuries, leading music theorists to overlook a praxis that was seemingly in widespread use before their own time.

\label{ln:gevaert_new}%
Reducing the number of chords in an accompaniment was one of two items topping Gevaert's agenda, the other being a new proposal for chant harmony.
Where Niedermeyer had restricted harmony to notes of the modal scale, Gevaert restricted harmony to the notes of an active hexachord.
The passage of the `Te Deum' quoted in \cref{mus:gevaert_vademecum} first traverses the hexachord `G'--`A'--`B'--`C'--`D'--`E', leading Gevaert to avoid using `F' in the accompaniment since it was not a component of the hexachord.
Even though the pitch `D' did not occur in the chant at this point, its tacit membership of the hexachord made its use in the accompaniment acceptable.
When the chant was said to mutate hexachords to that comprising `C'--`D'--`E'--`F'--`G'--`A', Gevaert began using the pitch `F' in the accompaniment.
On the subject of chords, Gevaert made the following statement:
\pagebreak{}

\simplex{Quant à la douceur de l'harmonie, mon accompagnement, si je ne me trompe, se distingue avantageusement des procédés d'harmonisation employés jusqu'à ce jour. A ceux qui s'offusqueraient du grand nombre d'accords de quinte (sans tierce), de l'emploi assez fréquent de dissonances de passage, je rappellerai que ces particularités résultent nécessairement de ma manière d'envisager le plain-chant.}
{\cite[`Preface' p.~6, Body matter p.~1]{GevaertVademecumorganistecontenant1871}}
{As for the softness of the harmony, my accompaniment, if I am not mistaken, is an improvement on those harmonisation methods used up to today. To those who would take issue with the great number of fifth chords (without thirds) and by the fairly frequent use of passing dissonances, I would remind that these particularities necessarily result from my way of considering plainchant.}
\noindent

Witt therefore generalised too much when he mistakenly described Gevaert's system as restricting the harmony to notes in the chant itself (`Jene Töne, welche in der Melodie nicht vorkommen, selbst wenn sie in der diatonischen Reihe liegen, dürfen auch nicht in der Begleitung vorkommen').\footnote{\cite{WittOrganumcomitansad1872}, 1st ed., p.~v; \cite[Witt might also have been responsible for the following review which commits the same error][82]{UnsignedReviewVademecumorganiste1871}.}
While that might have been the case in certain instances, in reality Witt's statement was only a half truth, one to which Van Damme brought little clarity in later writings.\footcite[70]{VanDammeUeberChoralOrgelbegleitung1872}
Van Damme's proposal that Gevaert's approach to chant harmony should be mixed with `la richesse de Bach' sowed even more confusion,\footcite[14]{VanDammeaccompagnementplainchant1881} and, some time after his collaboration with Gevaert, Van Damme set out to publish some `harmonised plainchant' for unison voices and an obbligato organ part that could not have been more at odds with Gevaert's practice.
Those harmonisations alternated with four-part choral sections transposed up by a perfect fourth accompanied by the organ `ad libitum' (\cref{mus:vandamme_pange}).
Rather an eccentric feature of Van Damme's production is that modulating interlude joining two sections, a technique he no doubt absorbed from Germanic practice (see \cpageref{ln:schildknecht_modulation,ln:schildknecht_modulation_END} above).\footcite[1--2; The eighth instalment in the series refers to the author's \emph{Enchiridion chorale ad Vesperas} published in 1874 by C. Poelman of Ghent which makes it possible to date the series after Van Damme's collaboration with Gevaert]{VanDammePangelingua}

\subsection{Free rhythm in antiquity?}
A gradual turning away from the \emph{cantus martellatus} style was accelerated by a new theory of chant rhythm proposed by the Solesmes monk Dom Joseph Pothier (1835--1923), who held that all chant notes were of equal duration and their accentuation was decided by the Latin text.\footcites[204--205]{Pothiermelodiesgregoriennesapres1880}[16]{RayburnGregorianChantHistory1964}
The anthropologist Émile Burnhouf (1821--1907) contended that pronunciation of Latin had indeed become more equalised around the turn of the first millennium, but that such equalism stemmed in fact from Latin's falling into disuse compared to vernacular languages.\footcite[80]{Burnhoufchantsegliselatine1887}
Solesmes's first abbot Prosper Guéranger had been critical of chanting that did not distinguish between strong and weak syllables,\footcite[`Approbation du très-révérend père abbé de Solesmes' in][p.~x]{GontierMethoderaisonneeplainchant1859} and approved the opinion of Augustin-Mathurin Gontier (1802--81) that chant rhythm could be discerned from the \emph{mise-en-page} of neumes.\footcite[12]{Gontierplainchantsonexecution1860}
Unquestionably, Guéranger's belief that melodic structure and performance practice could be revealed through appropriate research informed his decision to task Pothier and Dom Paul Jausions with a paleographical survey of the available MSS.\footcite[16, 34; Combe's account was first published in issues of the \emph{Etudes grégoriennes} from 1963 to 1968]{CombeRestorationGregorianChant2003}
It was a pioneering venture that built on the work of Danjou and other such historians, but it was also controversial because the meaning imputed to various symbols by generations of Solesmian researches was subject to criticism by other scholars, who proposed alternative interpretations.

\hlabel{ln:rhythmus_numerus}%
One vulnerability in chant scholarship as a whole was signalled by Gevaert, who reported that the Greek word for rhythm---\greekfont{ῥυθμός}\rmfamily, derived from \greekfont{ῥέειυ}\rmfamily, or `flow'---had been incorrectly translated into Latin by Roman philosophers.
Their preference for the translation `numerus' led the Carolingian music theorist Hucbald onto thin ice when he described rhythm and number as being synonymous with one another (`Qu\ae{} canendi \ae{}quitas rhythmus gr\ae{}ce, latine dicitur numerus').\footnote{\cite[228]{GerbertScriptoresecclesiasticimusica1784}; \covid{}\cite{BaileyCommeratiobrevistonis1979}; \cite[For a French translation, see][27]{VanDammeaccompagnementplainchant1881a}.}
This was in spite of `numerus' being more properly a translation of \greekfont{αριθμóς}\rmfamily, meaning numeral or number.\footcite[1]{GevaertHistoiretheoriemusique1881}
Here is not the place to critique Gevaert and others' views on this obscure matter, but Pothier's unquestioning adoption of the French term `nombre' as a synonym of `rythme' was to have profound implications for plainchant interpretation.\footcite[19]{Pothiermelodiesgregoriennesapres1880}
In developing Pothier's theory, his successor at Solesmes Dom André Mocquereau (1849--1930) deployed the term `nombre' as a determining factor in his own method of chant rhythm, using it in the title of a book on the subject.\footcite[57]{Mocquereaunombremusicalgregorien1908}
\hlabel{ln:rhythmus_numerus_END}%

The return to Medieval principles had recently been given fillip when a group of musical antiquarians convened the 1882 Arezzo Congress to mark a suppositious anniversary of Guido d'Arezzo's birth.
Its delegates turned to Guido's \emph{Micrologus} to inspire their understanding of equalist performances of the chant repertory according to a sense of textual declamation.\footcite[13, 43]{RuellecongreseuropeenArezzo1884}
They concluded that musical notes functioned in groups of ones, twos and threes and that accented notes occurred at the beginning of each group, rather like the frequency of accented syllables in Latin (`in harmonia sunt phthongi, id est soni, quorum unus, duo, vel tres aptantur in syllabas').\footcites[34--5]{DArezzoMicrologus1904}[French translation in][38--41]{RuellecongreseuropeenArezzo1884}
Whether Guido's equivalence between music and Latin pronunciation practice was intended to be anything other than a useful analogy is not clear, yet it was taken by nineteenth-century historians to mean a proportional theory of chant rhythm was what Guido had intended.\footcite[54--5 70]{PaliscaHucbaldGuidoJohn1978}
Whatever the rationale for Guido's comment, it certainly aroused a great deal of interest among musical antiquarians who were nonetheless required to tread a fine line between academic curiosity and musical heresy on account of the official status of Pustet's chant editions.\footcite[39]{BergeronDecadentEnchantmentsRevival1998}
One Fr Juget was therefore quite brave to note that accompaniments of plainchant ought to `return to the traditional rhythm of the neumes' (`revenir au rythme traditionnel des neumes'), but little consensus was reached on how such a return could be set in train (`La question de l'accompagnement du chant liturgique sur l'orgue n'a donné lieu à aucune résolution').
%\footcite[32, 41, 43]{RuellecongreseuropeenArezzo1884}

Pothier's equalist-accentualist theory garnered widespread support when the fruits of Solesmes's paloegraphical research first appeared as the \emph{Liber gradualis} in 1883, a chant book destined not only for use at Solesmes but also by the public at large.
The chant repertory it contained contradicted that officially sanctioned by the Holy See (see above in chapter one), and was presented in a new style of music notation designed by Pothier himself.
The characters were fashioned after their appearance in medieval MSS and were cast for use by the Belgian publisher Desclée (`en dessinant de sa propre main les caractères que MM.\ Desclée-Lefebvre ont fait graver').\footnote{\cite[37]{Schmidttypographieplainchant1895}; Katharine Ellis makes a compelling case for `Schmidt' being a pseudonym of Auguste Pécoul's, a collaborator of Pothier's who might have been writing at Pothier's instigation. See \cites[9]{EllisPoliticsPlainchantfindesiecle2013}[58]{BergeronDecadentEnchantmentsRevival1998}.}%\cpageref{pg:pustet_privilege}
The Imprimerie de Saint-Pierre was not yet equipped to print bulky music books (a matter to which we shall return in \cref{sc:publishing_libergradualis,sc:deberny}), but several notable pamphlets were printed in-house with the new repertory, including (also in 1883) \emph{Céremonial de vêture} and \emph{Chants pour le salut du T.~S.~Sacrement}.\footcite[14]{OuryimprimerieAbbaye18801979}

\subsection{The passing notes style in Belgium}
\label{hl:lemmens_passingnotes}%
To rectify the shortcomings in church music practice identified by Gevaert, the Church in Belgium established a systematic approach to training its musicians.
That took the form of a school of church music to which Belgian dioceses could send their local musicians.
Orpha Ochse has attributed the foundation of a school of church music in the Belgian city of Mechelen to Lemmens himself,\footcite[171]{OchseOrganistsOrganPlaying2000} but it appears that in spite of Lemmens's brief tenure as its director, the impetus for the venture actually came from Van Damme.
It was he who convinced bishop Henri Bracq of the need for such a school and who made several visits to London during 1876 and 1877 to recruit Lemmens as its director.
Lemmens then undertook to publish chant accompaniments that could serve as examples for Belgian organists, though the majority of these was only published after his death.\footcite[Joseph Duclos, `Essai sur la vie et les travaux de Lemmens' in][p. xxix]{Lemmenschantgregoriensa1886}

\label{ln:lemmens_update}%
The Salle Érard was the venue for a lecture on 18 February 1878 at which Lemmens provided two updates to Fétis's system of accompaniment.
The first was to admit `F'\kern 1pt\sharp{} on the unexplained and inexplicable proviso that the accompaniment remain modal (`à la condition de rester \emph{modal}').\footcite[58]{Bourgault-Ducoudraynouveausystemepour1878}
Fétis had adopted cadential sharping himself by the 1860s (see \cref{sc:fetis_sharping}), and Lemmens held that, when presented with the sequence of notes `G'~\rightarrow{}~`F'~\rightarrow{} `G', Medieval singers solmised them as \emph{sol}~\rightarrow{}~\emph{mi}~\rightarrow{} \emph{sol}, according to the rule of \emph{musica ficta causa pulchritudinis}.\footcites[3, 45--6]{Lemmenschantgregoriensa1886}
Lemmens reserved cadential sharps for the terminal cadence, the refrain in Alleluia `Pascha nostrum' receiving the sharped pitch (\cref{mus:lemmens_sharped}) in contrast with the same cadence at the end of the verse (\cref{mus:lemmens_nosharped}).\footcite[16--17]{LemmensChantsliturgiques1884}
Lemmens's stance on sharps made his practice inadmissible to Niedermeyer's pupils, including Eugène Gigout who, in 1876, claimed a lack of evidence supported cadential sharping, and who observed rather wryly that some schemes resulted in an accompaniment that was `un peu fantaisiste'.\footnote{\emph{Le Ménestrel}, 8 December 1878, p. 15 referenced in \cite[181--2]{OchseOrganistsOrganPlaying2000}; \cite[19]{GigoutPartiepratique1876}.}

Lemmens's second update was to tackle that chord-against-note style Gevaert had criticised and to seek the `true rhythm' of chant instead.
As is evident from \cref{mus:lemmens_sharped,mus:lemmens_nosharped}, Lemmens's harmonisation method involved imposing strictly proportional time values on the notes of the chant.
\hlabel{int:lemmens_britishmuseum}%
He claimed authority for this from a British Museum MS consulted on 11 August 1876, but foreclosed verification by withholding all information on the MS in question.
Lemmens held that a consonant, chorale-textured accompaniment could not distinguish between notes of greater or lesser relative importance, particularly when the chant notes were of equal duration.\footcite[1--2]{LemmensChantsliturgiques1884}
Nor, in Lemmens's view, was chant rhythm subordinate to verbal accentuation:
\pagebreak{}

\simplex{Des auteurs, très savants sans doute, mais peu musiciens, se sont imaginé que le rhythme du chant grégorien est le même que celui de la parole dans le discours oratoire~: mais il y a juste autant de différence entre ces deux rhythmes, qu'il s'en trouve entre la parole et la musique.}
  {\cite[61]{Lemmenschantgregoriensa1886}}
{Some authors, most learned no doubt, but not musicians, have subscribed to the notion that the rhythm of Gregorian chant is the same as that of speech in oratorical discourse~; but there are just as many differences between these two rhythms as there are between speech and music.}
\noindent
In the interest of keeping a congregation together, for instance, Lemmens proposed the scheme for psalm chanting quoted in \cref{mus:lemmens_pothier} that subordinates the eighth psalm tone and the doxology to melodic variations and crisp modern rhythms, presumably of Lemmens's own imagining:

\simplex{Ce phrasé là deviendrait de suite populaire et comme il est \emph{presque} mesuré, le peuple n'avait pas de difficulté à rester \emph{ensemble}.}
  {\letter{Lemmens}{Pothier}{21 December 1879}{\swf{}~191~(6)}; The Lemmens--Pothier correspondence will be published in the forthcoming third volume of \emph{Solesmes et les musiciens}. Patrick Hala to the present author, 2 April 2020}
{This phrasing would become popular straight away and as it is \emph{almost} mensural, the congregation would have no difficulty in staying \emph{together}.}
\noindent
Moreover, Lemmens held that Gregorian chant (as opposed to plainchant in general) was not completely consonant, and was in fact made up of a diverse array of dissonances:

\simplex{Le chant de saint Grégoire lui-même est rempli de dissonances~: \emph{notes de passage}, \emph{appoggiatures}, \emph{portamenti}. En les éliminant du plain-chant moderne, on en a fait disparaître la vie, et, par une conséquence toute naturelle, on a été logique en accompagnant un \emph{chant mort} par une \emph{harmonie également morte}.}
  {\cite[4, 120]{Lemmenschantgregoriensa1886}\hlabel{ln:lemmens_chants}}
{The chant of Saint Gregory himself is full of dissonances: \emph{passing notes}, \emph{appoggiaturas}, and \emph{portamenti}. By eliminating them from modern plainchant, we removed its vitality, and, as a natural consequence, were logical in accompanying a \emph{dead chant} with an \emph{equally dead harmony}.}
\noindent
However specious Lemmens's claim might have been, it afforded him the possibility to divide a chant melody into what he called real and inessential notes, the former being consonant and the latter dissonant.
The former were to be placed on strong beats of a bar which he sometimes annotated by crosses (pp.~96--7), whereas the latter were relegated to metrically weaker positions.
Chords placed on strong beats would therefore align with real notes, while dissonant, inessential notes bridged the gap between successive chords.\footcite[14]{Lhoumeauharmonisationmelodiesgregoriennes1884}

One imagines that Lemmens's ideas formed part of his teaching at the École de musique religieuse, for which the curriculum was prepared and signed by Lemmens himself on 20 August 1878.
The courses on offer included theology, liturgy, church Latin, singing, aesthetics, performance, history, organ, piano, harmony, counterpoint and the diatonic accompaniment of chant (`de diatonische begeleiding van den Kerkzang').
Tuition fees amounted to 400~F.\ per year, or 450~F.\ inclusive of room and board, and each Belgian diocese was to send a quota of students.\footcite[Prospectus printed in][34, 39--42]{ErensJaakLemmensstichter}
Prior to the first intake on 2 January 1879, Lemmens received Leo XIII's approval for the school in a private audience on 13 November 1878.
But a January start was awkward for several reasons, not least because teaching commenced with seven months remaining in the academic year rather than all ten, thereby requiring a commensurate reduction in fees.

The École took an active role in promoting agreeable models of church music in Belgium and further afield.
Lemmens used his position to draw the pontiff's attention to inconsistencies in Pustet's chant books, making plain his view on the distinction between plainchant and Gregorian chant and criticising the Medicean edition for comprising a repertory at odds with that supposedly codified under Pope Gregory I.\footcite[14--15, 17--18]{RobijnsJaakNikolaasLemmens1981}
Van Damme established a Belgian analogue to the Germanic Cäcilienverein, the Société de Saint Grégoire, on 28 September 1880, a society seeking to promote the restoration of church music along historical lines.\footcite[490]{GrijpEenmuziekgeschiedenisNederlanden2001}
Lemmens was made the society's chairperson but his untimely death on 30 January 1881 led Van Damme to assume the chair out of necessity.
A new journal was set up in Ghent to promote the views of the society, entitled \emph{Musica sacra~: organe de l'\'{E}cole interdioc\'{e}saine de musique religieuse et de la Soci\'{e}t\'{e} de Saint-Gr\'{e}goire}.\footcite[98]{LorenzoChanoineVanDamme1898}
As for finding someone to replace Lemmens as director of the École, Van Damme approached Edgar Tinel (1854--1912) on 9 Feburary who assumed the post on 3 March, a mere two days after his nomination was approved by Cardinal Deschamps.\footcite[189--90]{TinelEdgarTinelrecit1923}
The relative haste of Tinel's appointment might suggest that high priority was given to the École's endeavours in Belgium, though a replacement was also probably a necessity owing to the academic year's being well underway.

Antonin Lhoumeau (1852--1920), a priest and later a collaborator of Pothier's, acknowledged Lemmens's system as a pioneering one, but regretted that the result was rhythmically quite arbitrary.
While the École turned away from the system altogether following Lemmens's death,\footcite[279]{LhoumeauRhythmeexecutionaccompagnement1892} his undisputed influence as a performer and pedagogue led to the school's name being changed to the Lemmens Institute.
\hlabel{hl:tinel_organ}%
\hlabel{ln:tinel_registration}%
Sadly, little is known of the Institute's operation under Tinel's directorship, though scraps of evidence of Tinel's views on accompaniment have survived from the 1890s when he was most concerned with balance:

\simplex{Si l'organiste accompagne trop fort, faites-lui observer amicalement que le texte sacré a le pas sur la musique.}
  {\cite[49]{Tinelchantgregorientheorie1895}}
{If the organist accompanies too loudly, point out to him in a friendly way that the sacred text takes precedence over the music.}
\noindent
Several accompaniments by Tinel have also been preserved, for which the 8$^\prime$~Salicional on the Récit with box shut was to be used when accompanying a cantor.
An 8$^\prime$~Flûte or 8$^\prime$~Bourdon was deemed suitable for accompanying a choir---a discreet registration indeed.\footcite[4:1]{Melodieschantgregorien1892}
It seems Tinel maintained a connection to Solesmes in the same era, and was kept abreast of developments in the rhythmic ideas at that monastery by Mocquereau (see \cref{sc:france} below):

\simplex{Tous mes remerciements au Révérend Père Dom Mocquereau pour son livre sur l'\emph{Accent tonique}! Voilà un ouvrage hautement significatif. Puisse-t-il ouvrir les yeux à beaucoup!}
  {\letter{Edgar Tinel}{Mocquereau}{20 March 1894}{\so{}}}
{Many thanks to Reverend Father Dom Mocquereau for his book on the \emph{Tonic Accent}! This is a highly significant work. May it open the eyes of many!}
\hlabel{ln:tinel_registration_END}%

Along with the Conservatories of Brussels and Liège, the Lemmens Institute through its four-year course produced many of the next generation of Belgian organists who assumed teaching and organist positions in Belgium, France, Ireland and further afield.\footcite[\S{}\S{}4.2, 4.4]{DeacyContinentalOrganistsCatholic2005}
Theirs was a tradition insulated from those in France, Germany and Austria, and their chant accompaniments set themselves apart by being typographically distinct from those of other countries, as we shall see.\footcite[11]{G.EdgarTinelessai}
\hlabel{ln:desmet_dupuydt}%
Notable Belgian pedagogues included Alphonse Desmet (1864--1944; alternatively spelled Alfons), who succeeded Alphonse Mailly as professor of organ at the Brussels conservatory in 1903 in preference to Joseph Jongen (1873--1953).\footcite[66]{WhiteleyJosephJongenHis1997}
Alfons's younger brother Aloys (1867--1917; alternatively Aloïs) succeeded Tinel as the director of the Lemmens Institute in 1909.
And Aloys's colleague Oscar Depuydt (1858--1925; alternately De~Puidt) collaborated with the Desmet brothers on accompaniments of the Vatican Kyrial and Gradual in the next century (see \cref{sc:desmet_1906} below).
\noclub[2]

\subsection{Filled-and-void notation}
\label{hl:filled_and_void}\label{sc:publishing_libergradualis}\label{cc:filled_and_void}%
The Lemmens Institute's influence on matters of accompaniment is most apparent in the characteristic style of notation used by those of its teachers and pupils who sought an alternative to metrical notation.
In supplements to the first three issues of \emph{Musica sacra}, Van Damme published a harmonisation for the organ of the Requiem Mass notated mensurally with dotted barlines and slurs.
In spite of the accompaniment's \emph{mise-en-page}, a performance direction indicated that it was to be performed with `all the freedom inherent in plainchant' (\cref{mus:vandamme_requiem}; `avec toute la liberté que comporte le plain-chant').\footcite[1]{VanDammeOrdinariumMissaeMissa1881}
Van Damme's stance on the appropriateness of modern notation to represent chant was set to change in 1883 when he encountered an example of Pothier's notation.
Proofs were circulating as least as early as September 1882, at the Arezzo congress,\footcites[105]{CombeHistoirerestaurationchant1969}[88--9]{CombeRestorationGregorianChant2003} though Van Damme had not attended and probably did not encounter the notation until the \emph{Liber gradualis} was published.
\nowidow[2]

It was enough to convince Van Damme, in January of 1883, that verbally-oriented rhythm of the kind Pothier described as oratorical (`le rhythme oratoire') was the way forward,\footcite[179]{Pothiermelodiesgregoriennesapres1880} and that Pothier's archaeological finds were even worthy of consideration as works of art.
Van Damme wished to publish some excerpts to show off the notation to \emph{Musica sacra}'s 1,600 subscribers, and reported to Pothier that its Ghent-based publisher C.\ Poelman was in the market for procuring the type:

\simplex{Il est vrai que mon imprimeur n'a pas jusqu'ici les caractères typographiques de la notation traditionelle du plain-chant, mais il est tout disposé à se les procurer. En attendant, ne pourrions-nous pas imprimer les exemples [?], sur des feuillets à part, comme cela [est] si pratique pour les gravures intercalées dans le texte? Cette impression pourrait se faire soit à Solesmes, soit chez MM.\ Desclée. Cela me causerait quelques frais supplémentaires, mais je suis en mesure d'y faire face et je les supporterai volontiers pour obtenir votre collaboration.}
  {\letter{Van Damme}{Pothier}{25 January 1883}{\swf{}~175~(4)}}
{It is true that my printer has not yet had the typographical characters of the traditional plainchant notation, but he is quite prepared to procure them. In the meantime, could we not print the [?] examples on separate sheets, as that would be a practical way for the engravings to be inserted in the text? This printing could be done either at Solesmes or at Desclée. It would cost me some extra funds but I am able to manage that and will gladly bear them to secure your collaboration.}
\noindent
\hlabel{ln:deberny_impetus}%
Desclée had received the necessary new type from the Parisian foundry Deberny \&{} C\textsuperscript{ie}, though is it unclear whether Desclée was responsible for commissioning the type or whether that impetus came from Solesmes.
There is evidence to suggest that strict controls were put in place by the authorities of Solesmes to regulate further sale of the new type---we shall return to that point and its implication for publishers below (\cref{sc:deberny}).
\hlabel{ln:deberny_impetus_END}%

Poelman evidently had little success in procuring the type for himself---perhaps Solesmes suspected piracy---and was forced to improvise.
A new approach to representing the \emph{Liber gradualis}'s neumes was required so harmonisations would not be encumbered by metrical baggage, as Van Damme explained:
\pagebreak{}

\simplex{Je voudrais bannir de la notation du plain-chant harmonisé cet élément qui lui est étranger, et tout-à-fait antipathique~: \emph{la mesure}. Je voudrais revenir, sous ce rapport, au \emph{principe de la notation primitive}.}
  {\cite[30--31]{VanDammeUtilitepratiqueGraduel1883}}
{I would like to banish that element from the notation of harmonised plainchant which is foreign to and completely at odds with it: \emph{meter}. I would like to revert, under this heading, to the \emph{principle of the primitive notation}.}
\noindent
A new system was cobbled together from the movable type Poelman had to hand: modern notation was taken as the starting point but stems were dispensed with to create a new notational system of filled and void notes.
One may observe from its first use, cited in \cref{mus:vandammepothier}, that filled notes transcribed the chant, these being little more than former quavers or crotchets; stubs where they formerly connected with stems may still be discerned.
%p.31

The filled notes of the chant were arranged at the top of the four-part texture and grouped to represent the original neumes.
Chords were struck on the first notes of such groups, a void note enduring for two or more filled notes and ties retaining their conventional function.
Lhoumeau opined that it was in fact possible to assign harmonies to notes other than those at the beginnings of neumes,\footcite[15]{Lhoumeauharmonisationmelodiesgregoriennes1884} yet there was nothing systematic about his approach, save for some contrary motion in the tenor part.

Van Damme went on to develop filled-and-void notation in an accompanied \emph{Ordinarium Missæ} for Ghent, where he was involved in diocesan administration.\footcite[39]{CollinPierreJeanVan1979}
Stems were used in exceptional cases to distinguish between two parts occupying the same pitch (\cref{mus:vandamme_ordinarium}), the chant once more being arranged into groups of two and three notes.\footnote{See \cref{fn:vandamme_ordinary}.}
Void notes were worth two filled notes; and should three filled notes be notated above one chord then they were arranged as a triplet in a noteworthy departure from Pothier's doctrine of equalism.
Supplementary to those notational novelties was Van Damme's proposal that performances should be laden with nuances to bring life and movement to the chanting (`plein de vie et de mouvement')---the singer was well advised not to dwell on notes enclosed within parentheses.

Accompaniments were to proceed not from one note to the next but from group to group, and Van Damme understood those binary and ternary groups---so called after the number of notes they comprised---as the fundamental rhythmic units (`notre accompagnement procède, en règle générale, non par notes uniques, mais par groupes binaries ou ternaires').
They were said to function like musical syllables (recalling Guido's analogy), and the hierarchy of `syllables' in a phrase was said to be capable of representation by a system of nested arcs, though no such system was demonstrated by Van Damme in practice.\footcite[23]{VanDammeUtilitepratiqueGraduel1883a}
Another novelty of Van Damme's interpretation of grammatical rules concerned syllables of lesser importance which were annotated with a zero to warn the singer not to accent them.
A horizontal line connected to one side of the zero indicated which of the adjacent syllables was the more important.\footnote{\cite[pp.~iii--v, 7; Although no date of publication is printed on the text itself, its imminent publication is announced in the September--October 1884 issue of \emph{Musica sacra}]{VanDammeOrdinariummissaeordinariire1884}.\label{fn:vandamme_ordinary}}
Van Damme followed that accompanied \emph{Ordinarium Missæ} with other books for Vespers and Lauds, which were nearing completion by the beginning of 1885.
Those accompaniments were notated along similar lines, one notable innovation being a cross that marked certain beats as metrical, or strong (\cref{mus:vandamme_vespers}; `La petite croix indique la place des accents métriques ou principaux temps forts').\footcites[p.~43 n.~1]{Kornmullerprincipalesobligationsmaitres1885}[pp.~iv, 13; The text in question, along with \emph{Psalmi vesperarum~: les psaumes des Vèpres pour tous les Dimanches et fêtes de l'année}, is announced in the September 1885 issue of \emph{Musica sacra}]{VanDammeVesperaelaudesvespertinae1885}
%\noclub[2]

Van Damme continued his foray into notational experimentation in \emph{Musica sacra} by expanding on his idea that performances of chant should be nuanced.
He codified a system that used `p' and `s' to show primary and secondary accents,\nocite{VanDammeaccentuationlatinau1886b} and `a' and `t' to mark accented notes and notes of transition.
The accented note was always to receive a new chord (or, at the very least, a new bass note) whereas others, such as those coinciding with unaccented beats (`le temps levé'), would permit the accompaniment to rest.
5/3 chords were said to be particularly useful for demarcating strong accents while 6/3 chords were considered more appropriate for weaker accents.
\nocite{VanDammeaccentuationlatinau1886a}
In a departure from Nisard's practice where the array of available dissonances was limited to passing notes,\footcite[37]{Nisardvraisprincipesaccompagnement1860} Van Damme described a certain strongly accented note in descent as a long appoggiatura (`lange Vorschlag').
Auxiliary notes, anticipations and suspensions (`le retard') could also be used, as could a type of the last that ascended by a minor second.\footcite[75--7]{VanDammeaccentuationlatinau1886}
Van Damme coined the term \emph{Sprachbegleitung}, or Speech Accompaniment, to describe the process,\footnote{Pierre-Jean Van Damme, `De l'accentuation du latin au point de vue du chant liturgique', \emph{Musica sacra} 5, nos 8, 9 \& 10 (March, April \& May 1886), pp.~60, 69, 77.}\nocite{VanDammeaccentuationlatinau1886} though it ought not to be confused with the vocal technique \emph{Sprechgesang}, even if both concepts concern themselves with the expressionistic delivery of text.\footcite[229]{KnustRichardWagnerCreative2015}
Nevertheless, Van Damme was more than ten years ahead of Engelbert Humperdinck who introduced the term \emph{Sprechgesang} with his 1897 opera \emph{Königskinder}, and more than twenty years ahead of Arnold Schönberg, whose use of \emph{Sprechstimme} in \emph{Pierrot lunaire} of 1912 was so termed only subsequently, by Alban Berg.\footcite{GriffithsSprechgesang}

As other Belgian composers adopted filled-and-void notation for their accompaniments, it became less common for ternary groups to be notated as triplets.
Two and three filled notes were eventually considered as equivalent to one void note depending on the context.
\label{sc:desmet}%
Moreover, where Van Damme had left the staff blank to indicate rests, as early as 1892 Aloys Desmet preferred a crotchet rest (\reflectbox{\quaverRest{}}) for this purpose instead.
\hlabel{ln:tinel_verticals}%
Other practitioners used what Tinel termed `vertical bars' (`Les barres verticales marquent des silences pour l'orgue comme pour le chant'),\footcite[1:18, 20; The word `barres' is absent from the first volume but appears in vol.~4 p. 1]{Melodieschantgregorien1892} though they may more properly be considered obliques because, in contrast to conventional rests, they could be extended horizontally on the staff for as long as a rest was required.
A near-horizontal oblique in \cref{mus:oblique_1907} covers the duration of five filled notes,\footcite[11]{AspergesmeVidi1907} and in certain cases could also indicate a resting part when more than one occupied a staff.
The versatility of the notation led Wanger to deploy obliques in the book of accompaniments to his Zulu chant book (see \cref{sc:wanger} above).
\nowidow[2]
\hlabel{ln:tinel_verticals_END}%


\subsection{Freely rhythmed accompaniments in Germany}
\hlabel{hl:schmetz_1884}%
Pothier's textbook on rhythm, \emph{Mélodies grégoriennes d'après la tradition} of 1880, and its subsequent German translation in 1881 by the Emmaus monk Ambrosius Kienle (1852--1905), aroused interest in free rhythm among Francophone and Teutophone musicians alike.\footcite{PothiergregorianischeChoralseine1881}
The appearance of the \emph{Liber gradualis} inspired Paul Schmetz (1845--97), a former pupil of Peter Piel's, to devise a new notational method to represent harmonic accompaniments in free rhythm without recourse to modern notation.\footcite[243]{PaulSchmetz1897}

Schmetz recognised that Pothier's \emph{Liber gradualis} transmitted certain neumes that were verifiable among manuscript sources while being conspicuously absent from Pustet's editions.
In 1884, he devised a set of symbols that could stand in for the missing neumes in any chant book: they included a horizontal line for the \emph{pressus}, a wavy line for the \emph{strophicus}, a caret for shortened notes (`für verkürzende Formulen') and a dot for lengthened notes, after the `mora ultimæ vocis'.\footcite[17]{SchmetzDomPothierLiber1884}
A similar system was used in a joint Piel-Schmetz publication of the Mass Ordinary (\cref{mus:schmetzpiel_ord_51}) in which notes receiving emphasis were annotated by several signs: a circumflex for single notes, a horizontal bar for several notes, and dots for notes at which the voice was required to renew its emphasis.\footcite[pp.~v, 51]{PielOrgelbegleitungOrdinariumMissae1887}
One advertisement for those Mass Ordinary accompaniments credited Schmetz with their invention.\footcite[unpaginated advertisement]{OrgelbegleitungOrdinariumMissae1888}
Dominant \rightarrow{} tonic progressions were a hallmark of Piel's harmonic style, as the presence of `F'\kern 1pt\sharp{} at `Altissimus' attests.
To clarify the discussion on chant harmonisation in his harmony treatise, Piel deployed the same annotations and stated that dominant \rightarrow{} tonic progressions could indeed be used at intermediate cadences.
Sharping was therefore an ordinary component of Piel's practice, and his permissiveness also extended to \pitch{3}\sharp{} in terminal chords at deuterus cadences.\footcite[240--44]{PielHarmonieLehreUnterbesonderer1903}
\nowidow[2]

Ignaz Mitterer (1850--1924) favoured the Piel-Schmetz approach because quadratic notation could be preserved in the chant part.
Similarly, Schildknecht's compromise in typesetting the chant part in larger noteheads (see \cpageref{ln:schildknecht_large_noteheads,ln:schildknecht_large_noteheads_END} above) did not go far enough for one critic who considered it to be an inferior alternative to retaining the quadratic notation in the first place.\footnote{\cvk{1732}.}
A belief circulated among certain theorists that to retain quadratic notation for the chant part was to retain a distinction between chant and accompaniment, to set in relief the sacred connotations of the one from the secular implications of the other.
\hlabel{hl:horn_notation}%
Michael Horn (1859--1936) could well have subscribed to that notion since he is among those who retained quadratic notation for the chant in his accompaniments.\footcite[p.~3 and \emph{passim}]{HornOrdinariummissaeorgano1898}
Theorists such as Ambrosius Kienle certainly went to some effort to link quadratic notation to numinism when he associated it with entering a church via the sanctuary.\footcites[13]{KienleChoralschuleHandbuchzur1899}[Evidently cited from a previous edition of Kienle's work in][31--2]{MohrEinleitungundQuellennachweis1891}
And perhaps the notion might explain why Mohr's \emph{Ordinarium Missæ} set unaccompanied intonations by the priest or cantor in quadratic notation, whereas the accompaniments by Piel were set in modern notation.

Although such symbols as those devised by Schmetz were evidently quite useful in conveying more information than was otherwise represented by notational systems, the notation of accompaniments was still considered by some to be inherently deficient because rhythmic nuances were difficult to convey.
Presumably, that view prompted Schmetz to typeset an accompaniment of Piel's in the same quadratic notation that Desclée had used to engrave melodies for the \emph{Liber gradualis}.
\label{cc:schmetz}%
Using that notation (and with Desclée's support), Schmetz arranged chords on treble and bass staves (\cref{mus:schmetz_deogratias_35}; note the G2 and F4 clefs).
Perhaps Schmetz believed that an accompaniment notated in such a way would permit the player to follow the chant's nuances better than an accompaniment in modern notation, since the accompanist could apply the same rhythmical rules to the chords as a singer did to the chant.
Two of Piel's harmonisations of `Benedicamus domino' were transcribed, a bass part being added to match their neumatic layout.
If a three-note neume occurred in the chant then a three-note group would be arranged in the bass, and so forth.

Apparently, only certain chords were indicated, for the void inner notes at the starts of phrases showed an arrangement of the parts that would lead naturally to the following chord, thence to the chord after that, and so on.
5/3 chords were the default unless indicated otherwise by a bass figure.
A nonsensical 5/4 chord near the end of the first line stands as a testament to the notational method still being in its conceptual infancy, though the inner part \emph{g} seems clearly to be a typographical error for \emph{f}.
A symbol at the beginning of the second system, comprising two vertical strokes plus one horizontal stroke, presumably stood in for \sharp{} that would produce a dominant \rightarrow{} tonic chord progression, in complete accordance with Piel's practice at intermediate cadences.
Since it was left to the imagination of the reader to fill in the missing notes, a realisation has been provided in \cref{mus:schmetz_transcription} with dotted slurs serving to indicate the disposition of neumatic groups.\footcite[35; Although this text was published in Mainz, a note in the backmatter confirms it was printed by Desclée in Belgium]{SchmetzDomPothierLiber1884}

\hlabel{hl:schmetz_1894}%
\hlabel{int:schmetz}%
Schmetz soon departed from Piel's chord-against-note style when he concluded that chords were only to be struck on certain notes, notably in melismatic chants:

\simplex{Bei syllabischen Gesängen bekommt im allgemeinen jede Melodienote ihren eigenen Akkord; bei reicher gestalteten Melodien dagegen, wo die Begleitung jeder Melodienote aus ästhetischen und technischen Rücksichten unzulässig ersceint, erhält in der Regel jede \emph{Notengruppe} eine Harmonie.

\parindent=10pt
Nur wenn die Begleitung zu dürftig erscheint, oder wenn Accentuierung, fliessendere Stimmführung, langsamer Rhythmus etc. es verlangen, gibt man den einzelnen Tonfiguren mehrere Harmonien.}
  {\cite[21--2]{SchmetzHarmonisierunggregorianischenChoralgesanges1894}}
{In the case of syllabic chants, each melody note generally gets its own chord; but in the case of melismatic chants where the accompaniment of each melodic note is inadmissible for aesthetic and technical reasons, each \emph{group of notes} usually receives one harmony.

\parindent=10pt
Only when the accompaniment seems too meager or when accentuation, more fluent part writing, or slower rhythm are required are melismata furnished with several harmonies.}
\hlabel{int:schmetz_END}%
\noindent
Schmetz was already aware of Van Damme's ideas concerning groups of notes, and reproduced some examples from the latter's accompanied \emph{Ordinarium Missæ}.
The array of dissonances accepted by Schmetz comprised passing notes, auxiliaries, suspensions, anticipations and the `Hilfston', an arcane term for a kind of unaccented appoggiatura.
His Düsseldorf-based publisher L.\ Schwann seems not to have had access to Desclée's type when demonstrating Schmetz's new harmonic ideas in 1894, when extracts from the \emph{Liber gradualis} (such as that in \cref{mus:pothier_christe_7star}) were typeset in quite a different style of quadratic notation (quoted in \cref{mus:schmetz_christe_59}).\footcites[p.~7*]{PothierLiberGradualis1883}[59]{SchmetzHarmonisierunggregorianischenChoralgesanges1894}
One may remark that the inclusion of dashed vertical lines clarified where chords were to be struck, and that the placement of the latter was dictated by the neumatic layout of the chant.
\noclub[2]


\section{Popularising free rhythm in France and beyond}
\label{sc:france}%
\subsection{Prosodic analysis and the Lhoumeau effect}
\label{sc:lhoumeau_effect}%
Contrary to those novel notational approaches followed by Van Damme and Schmetz, French theorists devised their own methods of representing free rhythm.
Among such theorists was Lhoumeau who exchanged correspondence with Pothier on the matter as early as 1882,\footnote{H. Clemens, `Le Très Révérend Père Lhoumeau' within a folder marked `Père Antonin L'houmeau à D. Mocquereau + biographie' in \so{}} and who sought to codify the principles underpinning free rhythm for the benefit of accompanists.
The fruits of Lhoumeau's endeavours bear some similarities to Van Damme's proposals on the same subject, though Lhoumeau was careful to note in 1884 that his own were the first to appear in print.
Lhoumeau deployed modern notation to represent the chant and accompanying chords which the accompanist was advised to read as though they were amensural; the singer, by contrast, was to read from the typographical neumes placed directly above the accompaniment.

\hlabel{hl:lhoumeau_1884}%
The example of Lhoumeau's practice shown in \cref{mus:lhoumeau_epiphany_53} is markedly distinct from Van Damme's.
The chant was not always placed in the top part of the keyboard texture but either flitted between inner parts or was omitted entirely.
Quasi-orchestral writing called on the player to switch between `Grand orgue' and `Récit' manuals, and the arpeggiated chords at `quia gloria' required the right hand to make a pianistic foray into the top half of the keyboard.
Although the \emph{pressus} neume at `super te' was considered inherently expressive and a natural cause of crescendo and diminuendo, the distinct lack of contrapuntal rigour in the accompaniment causes the suspended fourth to shun its conventional resolution downward by step, thereby inviting the suspicion that Lhoumeau's compositional process consisted of feeling his way around a keyboard prior to committing his thoughts to paper.
Should that have been the case, then the trill in the last system may be considered to be notional instead of neoclassical.
Accompaniments of this type might be most charitably explained as a first step for Lhoumeau who abandoned the style within several years.
As will be evident from examples to be cited shortly, his later work---despite other, significant eccentricities---was to be characterised by more consistent textures and more rigorous part-writing.\footcite[unpaginated preface, pp. 44, 48--9, 53--4, p.~87 n.~1]{Lhoumeauharmonisationmelodiesgregoriennes1884}

Lhoumeau's notion that the \emph{pressus} caused a crescendo--decrescendo effect was likely influenced by theorists of secular music, who were evaluating the topic of musical expression and the role for amensural nuances to supplement meter.
\hlabel{ln:riemann}%
In 1884, Hugo Riemann (1849--1919) proposed the idea that \emph{sforzandi} caused a similar crescendo--decrescendo effect,\footcite[48--9]{RiemannMusikalischeDynamikund1884} and it was not long before the notion of a metaphysical rise and fall was being applied by analysts to musical phrases and even to works as a whole.
The Latin terms describing rise and fall, \emph{arsis} and \emph{thesis}, first entered the \emph{lingua franca} of nineteenth-century plainchant theory with Edmond de Coussemaker (1805--76) in 1852, whose study of the origin of neumes attempted to prove how closely their graphical forms matched the rise and fall of the voice in prosodic expression.
But we might question whether Coussemaker's conclusions owed more to French pronunciation practice than to any theories of classical Latin prosody, because he held an \emph{arsis} to be equivalent to the grave accent and a \emph{thesis} to be equivalent to the acute.
Circumflexes were then described as a vague combination of both.\footcites[158]{CoussemakerHistoireharmonieau1852}[For an Anglophone description of Coussemaker's contributions, see][10]{RayburnGregorianChantHistory1964}

A quirk concerning the usage of the terms \emph{arsis} and \emph{thesis} by nineteenth-century theorists should also be acknowledged here, for it continues to influence understandings of chant rhythm today.
The prehistory of those terms stretches further back into antiquity than some nineteenth-century theorists realised, to Ancient Greece where \emph{arsis}~(\greekfont{ἄρσις}\rmfamily) and \emph{thesis}~(\greekfont{θέσις}\rmfamily) were taken to mean unaccented and accented beats respectively.
When the writings of classical authorities appeared to contradict that status quo, some later editors silently `corrected' what they believed to be mistakes, in vain attempts to eliminate confusion.
That practice has recently been criticised by Tosca~A.~C.\ Lynch, who makes a compelling case for a more complex reality in the music-making of Ancient Greece than such scholars were willing at first glance to admit.\footcite[492, 496]{LynchArsisThesisAncient2016}
It was Riemann who popularised the notion that the definitions of \emph{arsis} and \emph{thesis} had been inverted by Roman scholars, who reportedly confused them in a way reminiscent of the \emph{rhythmus}/\emph{numerus} debacle discussed above (\cpageref{ln:rhythmus_numerus,ln:rhythmus_numerus_END}).
Those scholars were reported to have defined \emph{arsis} as the accented beat and \emph{thesis} as the unaccented one, exchanging their definitions compared with accepted Ancient Greek usage.

As the terms came down to the nineteenth century, a further complication arose when two separate disciplines allied themselves with opposing definitions.
Metricians, who concerned themselves with prosody, preferred \emph{arsis} for the strong verbal accent, whereas musicians preferred \emph{thesis} for the strong musical accent.\footcites[44]{RiemannMusikLexikon1882}[34--5]{RiemannArsis1899}
What was presumably of no great disadvantage to either faction in isolation became problematic when nineteenth-century scholarship on chant rhythm caused textual and musical matters to come hurtling together.
Seeking to reconcile verbal and musical accents, chant theorists settled on the term \emph{arsis} to mean the strong accent, whether that happened to describe phenomena in words or in the chant.
This ran contrary to the musical convention that \emph{thesis} should be the strong accent, and was to have sometimes bizarre consequences when theories of chant rhythm were taken as the basis for chant harmonisation, as we shall see.

\label{hl:lhoumeau_1892}%
One early consequence of the tension between definitions arose in Lhoumeau's accompaniment manual of 1892.
He, like other chant theorists, adopted \emph{arsis} as the rise and \emph{thesis} as the fall, and proposed a graphical system of arcs which he claimed represented the rise and fall of one's hand, an anacrustic note or notes being superscribed with a horizontal line attached to the ensuing arc.
In spite of maintaining the metrical definition of \emph{arsis}, Lhoumeau applied the musical custom of placing chords ordinarily on the \emph{thesis}, producing a syncopated effect whereby textual accents are desynchronised from chord changes in his accompaniments.
What we choose to term the `Lhoumeau effect' may be observed in \cref{mus:lhoumeau_syllabic}, where the chords at `dexteram Patris' align not with accented syllables but with the unaccented syllables instead.\footcite[pp.~5, 13, 240, 245, 329]{LhoumeauRhythmeexecutionaccompagnement1892}
\hlabel{ln:lhoumeau_thesis}%
The `Lhoumeau effect' cropped up in a later journal article, where the final notes of neumes were made to coincide with the downbeat of a bar (\cref{mus:lhoumeau_torculus}).\footnote{\cite[140]{Lhoumeauaccordpourchaque1893}; The final note is evidently a mistranscription.}
The consistency with which Lhoumeau set chords shows incontrovertibly that his practice was no fluke, and embodied a determined yoking of the contradictory definitions of \emph{arsis} and \emph{thesis}.
\noclub[2]

\hlabel{ln:lussy}%
It seems likely that Lhoumeau's concept of `masculine' and `feminine' phrase endings was derived from the Swiss music theorist Mathis Lussy (1828--1910), whose writings are liberally quoted in Lhoumeau's accompaniment manual.
According to Lussy, a slow tempo could transform a weak, feminine ending into a strong, masculine one.\footcite[19--20, 22]{LussyTraiteexpressionmusicale1874}
Lhoumeau held that an ending was invariably masculine when the terminal syllable was set to a single note, and feminine when set to a neume.
The masculine ending was obviously easier to harmonise, because it simply involved placing a single chord on the terminal chant note.
\hlabel{ln:lhoumeau_feminine}%
The feminine ending was not so straightforward, because it required a bass note to be placed on the first note of the neume and an inner note to be suspended, its resolution not occurring until the terminal chant note.
The beginning of a terminal neume could also be demarcated by an appoggiatura.
The arcs in \cref{mus:lhoumeau_feminine} terminate in short horizontal strokes to indicate endings of the feminine type.
The direction `r.\,p.' stands for the `rallentando poco' Lhoumeau believed would transform feminine endings into masculine ones.
\hlabel{ln:lhoumeau_feminine_END}%
\nowidow[2]

The accompaniment quoted in \cref{mus:lhoumeau_alleluia} demonstrates not only Lhoumeau's practice when faced with such \emph{rallentandi} but also how he applied rhythmic arcs to melismatic chants.
In the absence of changing syllables, the disposition of arcs was entirely dependent on the musical matter.
Notches midway through these arcs indicate subdivisions within neumes of many notes.
The junction between arcs or notches was simultaneously the \emph{thesis} of one phrase and the \emph{arsis} of the next, and it appears as though a hierarchy of musical accents in a phrase determined which accent was more important than another.
Perhaps Coussemaker's notion of the circumflex contributed to Lhoumeau's understanding of those junctions; perhaps, also, Lhoumeau simply applied the concept of elision to the musical material.\footcite[33--4, 240--43]{LhoumeauRhythmeexecutionaccompagnement1892}

\subsection{The decline of the École Niedermeyer}
\label{cc:gigout_teppe}%
A group of Francophone Catholics became suspicious of Niedermeyer's principles (including, in particular, the diatonicism that they espoused) owing to that theorist's being a Protestant.
\hlabel{hl:felixclement}%
Félix Clément (1822--85) argued in 1872 that Catholic music could not be based on any such heresy, and considered Niedermeyer's diatonicism to be unsuited to the Catholic Church:

\simplex{La théorie de l'accompagnement \emph{unitonique}, qui a envahi un grand nombre de nos églises, est une hérésie musicale dont l'oreille et la raison finiront par triompher, mais qui fait en attendant, et de jour en jour, des ravages incessants, et cause un chant religieux qu'il défigure le plus grand dommage.}
  {\cite[360]{ClementMethodecompleteplainchant1872}}
{The theory of \emph{unitonic} accompaniment, which has invaded a great number of our churches, is a musical heresy over which the ear and reason will eventually triumph, but which in the meantime, and day by day, wreaks boundless havoc and causes the greatest harm to the sacred chant which it disfigures.}
\hlabel{int:clement_accomp}%
\noindent
Clément proposed a method of his own which he reportedly derived from classical French practice: chant was presented alternately in the top and bottom parts of a four-part texture while chromatic inner parts accompanied.\footcite[191, 194]{ClementAccompagnementplainchant1873}
Clément believed his own was a more appropriate method, a belief he reiterated in 1894 when proposing the system again.
But the contrapuntal writing in \cref{mus:clement_awkward_191} can hardly be considered to be the work of someone concerned with tradition, because apart from the harmonic tautology the alto part frequently incorporates the prohibited interval of a diminished fifth.\footcite[191]{ClementMethodeorgueaccompagnement1894}
Clément's system was therefore anachronistic and a far cry from the practice of composers such as Titelouze and Nivers.
\hlabel{hl:lhoumeau_alteration}%
Leßmann attributed Lhoumeau's preference for cadential sharping to a lack of widespread adoption of Niedermeyer's theory,\footcite[220]{LessmannRezeptiongregorianischenChorals2016} Lhoumeau having taken the approach from Morelot's descriptions of tetrachordal subsitution (see \cpageref{ln:morelot_hexachord,ln:morelot_hexachord_END} above).\footcite[4, 8, 14, 16--17]{Lhoumeaualterationoudemiton1879}

A group of influential Catholic composers nonetheless continued to advocate Niedermeyer's anti-sharping principles.
Niedermeyer was survived at the École bearing his name by his son-in-law Gigout, who propagated those principles to the next generation of \mbox{musicians}, including to Gabriel Fauré (1845--1924).
Gigout not only delivered lessons in chant, counterpoint, fugue and the organ from 1860 to 1885, but also contributed to the second edition of Niedermeyer's \emph{Traité} in 1876, for which he composed an appendix of music examples.\footcite[p.~134 n.~1; Leßmann notes Gigout contributed to the second edition of Niedermeyer's text, referring quite correctly to the `Nouvelle' edition of 1876 and not the `2è tirage' of 1859 which was little more than a re-print of the first edition]{LessmannAllTheseRhythms2008}
As a complementary exercise, Gigout composed three volumes of \emph{Chants du Gradual et du Vespéral Romains} which were sold separately by the publisher at a reduced price to encourage their adoption by French and Belgian \emph{maîtres de chapelle} and to discourage unauthorised copying.\footnote{\cite{GigoutPartiepratique1876}, unpaginated `Note des éditeurs'.}

Gigout also applied Niedermeyer's principles to a new kind of diatonic composition published in \emph{Cent pièces brèves dans la tonalité du plain-chant} towards the end of the 1880s.
Although the collection has today been hailed as one of the first attempts to adapt the modes of plainchant to modern harmony,\footcite[177]{Viretchantgregorientradition2001} it might be more realistic to consider it the first attempt at expanding Niedermeyer's principles beyond the limits of chant accompaniment.\footnote{\covid{}\cite{GigoutCentpiecesbreves1888}.}
Gigout programmed some of those modal pieces in a recital at the Trocadéro and reaffirmed his commitment to Niedermeyer's legacy as late as 1921 by including further modal pieces in the collection \emph{Cent pièces brèves nouvelles...}, published in London by J.~\&~W.~Chester.\footcite[138]{OchseOrganistsOrganPlaying2000}

Gigout disagreed with certain reforms imposed on the École Niedermeyer by the anticlerical campaign of the Third Republic, leading him to resign his post and to establish a school of church music on his own account in 1885.\footcite[The year 1895 is incorrectly provided in][252, probably the result of a typographical error]{HartoppParisConciseMusical2019}
Ironically, Gigout's Institut d'orgue actually benefited from the same campaign of anticlericalism when it came to locate a venue for the school's activities.
The Salle-Albert-le-Grand, a recently vacated Dominican convent, offered itself as the ideal location, not only due to its central location in the Parisian eighth arrondissement but also because it housed a Merklin organ.
When the Dominicans reclaimed the premises in 1887, however, Gigout was required to relocate the Institut to his residence at 63 bis rue Jouffroy, for which he acquired a three-manual Orgue de salon from Cavaillé-Coll.\footcite[83]{Shuster-FournierorguessalonAristide1997}

Classes were divided into two strands.
The lower made a special study of `des cadences grégoriennes' and also the realisation of figured bass, along with studies on keyboard and pedal technique.
The upper studied the interpretation of repertoire, organ registration and improvisation.
Both strands also studied the accompaniment of chant according to Niedermeyer's principles.
The Institut was principally aimed at amateur musicians who paid the hefty sum of 40~F.\ in monthly tuition fees.\footcites[515]{LuedersGigout2003}[Also referenced in][116]{EllisPoliticsPlainchantfindesiecle2013}
In 1900 it was relocated once again, this time to 113 avenue de Villiers, and closed entirely in 1911 when Gigout was appointed organ teacher at the Paris Conservatoire.\footnote{Syllabus printed in \emph{Le ménestrel}, 11 October 1885, p. 360; Cited with discussion in \cite[76--9]{BaileyEugeneGigouthis1994}.}

\hlabel{hl:teppe_premier}%
Gigout's authority on church music made him attractive to chant theoreticians who sought approval for and approbation of their rhythmic theories.
One such theory was proposed in 1889 by the abbé Auguste Teppe (1838--1906) and is described in rather understated terms by a biographer as being `a bit abstract' (`un peu abstrait').\footcite[8, 337, 351]{JolyabbeAugusteTeppe1906}
That abstraction did not preclude Gigout from composing two accompaniments to the Christmas introit `Puer natus est', rhythmed according to the system.
The first (\cref{mus:teppe_gigoutone}) comprises mainly sustained chords while the second (\cref{mus:teppe_gigouttwo}) comprises a more independent keyboard part---in both cases the harmony remains resolutely diatonic.\footnote{Teppe's use of ties appears to follow no logical framework and therefore differs slightly from that represented here which have been modified to follow the disposition of syllables.}
Teppe claimed that the chant could be delegated to a violin, cello or flute, and, in presenting Gigout's harmonisations, made the bold assertion that the rhythmic theory enjoyed `l'adhésion des maîtres'.
\label{int:gigout_teppe}%
Gigout's allegiance lay not with Teppe, however, but with Niedermeyer, a fact Teppe was not careful enough to suppress when he relayed part of Gigout's correspondence:
\noclub[2]

\simplex{Il ne m'a fallu abandonner en rien les principes d'harmonisation grégorienne posés par mon illustre maître Niedermeyer, principes qui se combinent parfaitement avec n'importe quelle donnée rythmique du chant grégorien.}
  {Eugène Gigout to Auguste Teppe, 5 October 1890, published in \cite[5, 255--31]{TeppePremierproblemegregorien1889}; Inconsistencies in Teppe's slurring and omissions of dots are corrected in the transcriptions which are the work of the present author}
{I did not have to abandon in any way the principles of Gregorian harmonisation laid down by my illustrious teacher Niedermeyer, principles that align perfectly with any rhythmic approach to Gregorian chant.}
\noindent
Teppe commissioned other composers to write more harmonisations based on his system into the early years of the twentieth century,\footnote{\covid{}\cite{TeppeLivreorgue}, a collection of harmonisations that reportedly obviated the need for Teppe to publish \emph{Le second problème grégorien}. See \cite[19]{TeppeParallelismeversstrophes1900}\label{hl:teppe_second}} but not everyone proved receptive to Teppe's requests.
Émile Bouichère (1861--95), then \emph{maître de chapelle} of La Trinité, Paris, declined an invitation to compose an accompaniment because he simply did not hold with Teppe's theory.\fnletter{Dom Antoine Delpech}{Mocquereau}{8 February 1894}{\so{}}
That stance was also taken by the Widor student and editor of the Parisian journal \emph{Le Ménestrel} Henry Eymieu (1860--1931), who, along with one F.~Emery-Desbrousses, nevertheless politely acknowledged Teppe's competence in musical matters.\footcites[167]{EymieuEtudesbiographiesmusicales1892}
\nowidow[2]

\subsection{Charles Mégret and a forum for investigative composition}
Teppe was not alone in commissioning accompaniments to promote a rhythmic theory. The Solesmes monk Dom Charles Mégret (1853--1933) used the same strategy to popularise Pothier's free oratorical rhythm.
Mégret had been tasked with photographing European manuscripts for the new Solesmian publication \emph{Paléographie musicale} which will be discussed in more detail below (\cref{sc:moc_paleo}),\footnote{Mégret's ordination date of 11 July 1884 as included in \cite[133]{BibliographieBenedictinscongregation1889} pre-dates his entry into Solesmes; The \emph{Paléographie}'s initial members are noted in \cite[161 n.~27]{ScarcezecritsplainchantGevaert2010}.} and when he left Solesmes for Saint-Martin de Ligugé in the early 1890s he established himself as something of an authority on chant performance practice.
Mégret published a textbook on the subject under the nom-de-plume `Gregorianus', which was nonetheless attributed to Mégret in 1893 by one sharp-eyed German cataloguer.\footcite[The textbook in question, \emph{Des conseils pratiques sur le chant liturgique dans les seminaires, les communautés et les paroisses}, is attributed to Mégret in][104]{PlaineBeitraegezurGeschichte1893}

Archival material on Mégret is either vague or has not yet come to light,\footnote{Some eighty six letters between Mégret and Pothier are preserved in the Saint-Wandrille archives, but at the time of writing are not yet digitised for consultation: Saint-Wandrille archivist Frère Thomas Zanetti to the author, 26 October 2020 and 24 February 2021 on the subject of \swf{}~105.} and not even his obituary discloses much, offering little more than a taciturn account of his support for Pothier's rhythm.\footcite[63]{DomCharlesMegret1933}
Although Mégret discarded his anonymity in 1892 when he vouchsafed that Pothier's was the only true method (`à notre humble avis, il n'y a qu'une bonne méthode~: c'est celle de D. Pothier'),\footcite[53]{Megretchantliturgiquedans1892} he evidently preferred keeping his contributions on the subject anonymous.
He edited a five-volume collection of accompaniments between 1892 and 1893 that do not bear his name, yet his identity was something of an open secret in Benedictine circles where the collection was known, at least to some, as `Mégret's volumes' (`les livraisons de Mégret').\footnote{\letter{Antoine Delpech}{André Mocquereau}{17 April 1903}{\so{}}; Mégret's involvement is made explicit in 1914 by the Librairie de l'art catholique which printed his name in square brackets. See the catalogue of works in \cite[6]{Parisotaccompagnementmodalchant1914}.}

Pothier outlined his preferences for accompaniments to Mégret directly in January of 1892:

\simplex{Deux principes sont admis : 1\textsuperscript{er} Il faut que les accords respectent la tonalité grégorienne, certaines altérations sont-elles, ou ne sont-elles pas contraires ? Ceci peut encore être discuté, bien que je regarde ces altérations comme très dangereuses et que je ne les aime pas. 2\textsuperscript{è} Le rythme également doit être respecté, et pour cela, il faut n’accompagner que les notes appelées réelles par les harmonistes : notes qui sont surtout celles qui tombent au point du départ (arsis) et au point d’arrivé (thesis) de chaque division ou petite partie du mouvement. Les notes non accompagnées ne font pas à proprement parlant dissonance, elles ne comptent pas dans l’harmonie.}
  {\letter{Pothier}{Mégret}{January 1892}{\so{}}}
{Two principles are accepted. First, chords must respect the Gregorian \emph{tonalité}. Do certain changes [sharps] contradict it or do they not? This can be discussed further, although I consider these changes [sharps] to be most dangerous and I dislike them. Second, the rhythm must also be respected, and to achieve that only the notes harmonisers call real are to be accompanied, especially those that occur at the up-beat (\emph{arsis}) and at the down-beat (\emph{thesis}) of each division or small part of the \emph{mouvement}. Unaccompanied notes are not strictly speaking dissonances, they do not count in the harmony.}
\noindent
Pothier would later voice his support for Lhoumeau's treatise, though he conceded that there existed many different ways of applying chant rhythm to the accompaniment.\footnote{Pothier to Lhoumeau, 10 May 1892, unpaginated `Approbations' in \cite{LhoumeauRhythmeexecutionaccompagnement1892}.}
Although not keen to embroil himself publicly in the debate, Pothier proposed to Mégret that harmonisers be tasked with composing accompaniments so that an optimum approach could be arrived at through practice.

Solesmian chant books furnished a kind of scientific control for each experiment, the reader being directed to find the original melody in either the \emph{Liber gradualis} or two other chant books co-produced by Pothier and Dom Raphaël Andoyer, \emph{Variæ preces ad Benedictionem SS. Sacramenti praesertim cantandae} and \emph{Processionale monasticum}.\footnote{\cites[142]{CombeHistoirerestaurationchant1969}[122]{CombeRestorationGregorianChant2003}; References to the \emph{Variæ} do not concord with page numbers in the third edition which appeared in 1892 and either refer to \covid{}\cite{Variaeprecesad1888} or to the second edition that appeared in 1889; References to the \emph{Processionale} concord with \cite{Processionalemonasticum1888}.}
The layout of Mégret's volumes was comparative in nature, usually presenting the chant in quadratic notation atop a transcription into modern notation, the lowest two staves of the system comprising the accompaniment itself.
Accompaniments were commissioned of the best known organist-composers of the age and were reproduced in the notational style used by each composer.
Thus, notational differences could be exemplified at the expense of a complicated \emph{mise-en-page} that seemingly exceeded the capabilities of the Ligugé printing press.
Instead of producing the volumes in-house, then, Ligugé delegated the process of their lithography to the Baudoux printing firm, whose premises at 13 rue Saint-François in the nearby city of Poitiers was probably convenient enough for Mégret to inspect the proofs.
Ligugé nevertheless advertised the collection as part of the Imprimerie de Saint-Martin catalogue, an example of which was included in the back-matter of Mégret's 1892 textbook.
That advertisement claims the publication incorporated notation from twelfth-century manuscripts (`écriture imitée des manuscrits du XII\textsuperscript{e} siècle'), a notion that had perhaps germinated while Mégret photographed similar such manuscripts for the \emph{Paléographie musicale}.
Each volume cost 6~F., the first four being available together for the discounted price of 20~F.\footnote{\cite{Megretchantliturgiquedans1892}, 19 and appendix entitled `Imprimerie Saint-Martin à Ligugé (Vienne)' p.~4.}
Though not mentioned in any of these first four volumes, a fifth was subsequently published.
Considering that the copy of the fifth volume deposited at the British Museum bears the stamped date of July 1893, it is likely that it was not produced until that year, and after Mégret's textbook had been sent to print.

Several contributions to Mégret's volumes will be discussed over the following paragraphs, but the reader should recognise the difficulty in deducing a composer's personal preferences from the guidelines which he may have been advised to follow.
Composers could have been coached in Pothier's dictum, and may have therefore contrived accompaniments that avoided chromatic pitches which they might otherwise have employed---a few accompaniments making use of sharps are included, though they are conspicuously not attributed to anybody.
Personal preference is particularly difficult to quantify in the case of musical chameleons such as Gigout, who seemed quite at ease composing in different idioms.
Nonetheless, the fact that many accompaniments omit the chant entirely should not be dismissed, since it evidences a growing trend among composers to fashion a less intrusive organ part.
\nowidow[2]

\subsection{The quest for an optimum accompaniment}
The accompaniments in Mégret's volumes will here be discussed according to their layouts which divide into at least four categories.
\label{int:gigout_megret}%
The first category concerns Gigout's adaptability and comprises four accompaniments.
In two, the chant is not always reproduced in the accompaniment, and some passages are left unaccompanied since they were presumably intended to be delegated to a cantor (\cref{mus:gigout_1_5}).
A footnote advises the reader that `notes retain their customary values' (`les notes conservent leur valeur habituelle'),\footnote{\cite[1:3, 5]{Melodieschantgregorien1892}.} opening something of a lexical gap in English where `habituelle' could mean either a mensural accompaniment (if the notation were considered divisible) or a freely rhythmed one (if modern notation were used purely for the sake of convenience).
In another case, Gigout maintains the chant in the top part throughout while other parts are laid out in a four-part chorale texture (\cref{mus:gigout_2_7}).\footnote{\cite[2:7]{Melodieschantgregorien1892}.}
A further case not only pits `tutti' and `solo' forces against each other but also marks certain passages `più lento' and others `mouvement plus vif'.
Some of the former seem to incorporate more rests, as if the slower tempo required more acoustic space (\cref{mus:gigout_4_20}), though it is difficult to tell whether that was determined by rule or by flight of fancy.
In any event, an editorial note describes one chant as being solely `of archaeological interest', but it is unclear whether the piece was proscribed by ecclesiastical authorities or whether it simply held little value in Mégret's eyes.\footnote{Note des Editeurs: `Cette pièce n'a qu'un intérêt purement archéologique. Nous l'avons cependant laissée à titre de curiosité, persuadés qu'elle sera étudiée avec plaisir. See \cite[4:20]{Melodieschantgregorien1892}.}

Contrasting sections are also a characteristic of an accompaniment by the Versailles cathedral organist Dominique-Charles Planchet (1857--1946) who permitted his accompanying parts to drop in and out without reproducing the melody.\footnote{Compare, for instance, the homophonic texture at `Precibus ergo tuis' to the refrain at `O Hilari' in \cite[4:6--9]{Melodieschantgregorien1892}.}
Léon Boëllmann (1862--97) also regulates the number of parts by omitting the chant, a procedure he could have derived from Gigout himself since the two shared close personal links (the Boëllmann--Gigout connection was strengthened by Boëllmann's marriage to Gigout's niece in 1885, after which the newly wedded couple joined Gigout at his residence and Léon began teaching at the Institut d'orgue).\footcite[216--17]{OchseOrganistsOrganPlaying2000}
Like Gigout, Boëllmann freely alternated between three and four parts; but unlike him, Boëllmann tried out a different way of indicating rests when the parts strayed above or below the staff.
Unlike conventional semibreve and minim rests which would affix themselves to upper and lower ledger lines (\wholeNoteRest{}~and~\halfNoteRest{}), Boëllmann rendered the horizontal rectangle without the ledger line (\cref{mus:boellmann_3_7}).
%\footnote{\cite[vol.~3 p.~7]{Melodieschantgregorien1892}.}
While this remains an innocuous detail, it was borne of Boëllmann's understanding of chant mensuration which also required rests of longer durations to be notated similarly to breve rests.

\label{ln:gounod_chordagainstnote}%
The second category concerns the chord-against-note procedure which, contrary to Kunc's best expectations, had not disappeared entirely by 1892.
\hlabel{int:gounod}%
Charles-François Gounod (1818--93) was among those composers who adopted the style, admitting 5/3 and 6/3 chords alone.
This approach was consistent with Niedermeyer's method but was avowedly old hat for the 1890s.
At first glance, Gounod's accompaniment (quoted in \cref{mus:gounod_3_18}) appears rather more turgid than accompaniments by other composers who employed fewer chords, though his advice that it should `flow with care' was probably intended as a warning against laborious performances (`Couler cet accompagnement avec soin').\footnote{\cite[3:7, 13]{Melodieschantgregorien1892}; The second part from the top in the third chord following the double barline is probably a misprint and should be a `G' since it is rendered as such when the same accompaniment is repeated later on.}
Performative challenges led Émile Brune to advise that proportional notation was only intended for inner parts and that the chant was intended to be performed in free rhythm (vol.~5 p.~25).\footnote{Brune advises `Les notes n'ont de valeur proportionnelle que dans les parties intermédiaires et inférieures'.}
%See \cite[vol.~3 pp.~7, 13; ]{Melodieschantgregorien1892} where

\pagebreak{}
Gounod's preference for a supple performance of chant was corroborated by several journalistic accounts that appeared after his death.
The following anecdote, for instance, notes Gounod's preference for Pothier's chant editions:

\simplex{Puis, montant à son orgue, il chanta en s'accompagnant l'\emph{Alleluia} du Commun des Martyrs, le \emph{Beatus vir} d'un Confesseur non Pontife, le \emph{Sicut lilium} de la messe de la \emph{Pureté de la Vierge}, des graduels pris au hasard dans ce merveilleux Common des Saints. `N'est-ce pas que c'est beau~?' me disit-il. `C'est une gerbe mélodique qui monte, comme un nuage d'encens, jusqu'au ciel.'}
  {\cite[pp.~xi--xii]{BoyerdAgenConsiderationsgeniechristianisme1894}. The \emph{Revue du chant grégorien} ascribes the anecdote to the Lyon organist Jules Ruest in 1893---see \cite[62]{RuestGounodchantgregorien1893}---though it retracts that attribution in a subsequent issue; The anecdote could either be set in the Parisian church of Saint-Cloud where Gounod was the organist since 1877 or at his private residence at 20 place Malesherbes, for which he acquired an Orgue de salon from Cavaillé coll in 1879. See \cite[73]{Shuster-FournierorguessalonAristide1997}}
{Then, going up to his organ, he accompanied himself singing the \emph{Alleluia} from the Common of Martyrs, the \emph{Beatus vir} of a non-Pontiff Confessor, the \emph{Sicut lilium} from the Mass of the Purity of the Virgin, and some graduals taken at random from the wonderful Common of Saints. `Isn't that beautiful?', he said to me. `It is a melodic wreath that rises like a cloud of incense to the heavens.'}
\noindent
Gounod had declined an invitation from Lhoumeau to compose an accompaniment in 1889, saying he awaited a certain forthcoming text by Pothier before committing harmonic thoughts to paper.
The sixteen-page publication entitled \covid{}\emph{Principes pour la bonne exécution du chant grégorien} of 1891 coincided with Gounod's completion of a new \emph{Requiem},\footnote{\cites[180]{CombeHistoirerestaurationchant1969}[156]{CombeRestorationGregorianChant2003}; It has not been possible to consult the \emph{Requiem}'s MS to determine the dates of composition since it is held in a private collection, but they range from 1889 to 1891. See \cite[p.~iii]{GounodRequiem2011}; Lhoumeau was put in contact with Gounod by a mutual acquaintance and former school friend of Gounod's Bishop Charles-Louis Gay; For Lhoumeau's report of Gounod's refusal, see \letter{Lhoumeau}{Pothier}{n.d.}{\swf{}~153~(a)~59}.} factors that likely made him more amenable to receiving Mégret's advances.
Among the other composers who preferred the chord-against-note style were Dubois and Hanon who offered no performance directions to refute the allegation that their methods remained largely unchanged from those we examined above in the previous chapter.
Widor also adopted the texture (\cref{mus:widor_2_1}), though we shall return to his contribution below (\cref{sc:widor_megret}).
%\footnote{\cite[Vol.~2 p.~1]{Melodieschantgregorien1892}.}

The third category concerns contributions largely by Belgian musicians who adopted the filled-and-void notational style, though not without certain novel elaborations on Van Damme's practice.
A contribution by Van Damme is joined, for instance, by two others by Tinel, whose preference for an unobtrusive registration (see \cpageref{ln:tinel_registration,ln:tinel_registration_END} above) found its match in an equally unobtrusive keyboard texture.
Only two or sometimes three parts are used in the accompaniment which does not comprehend the notes of the chant itself: the \mbox{accompanying} parts are placed in the lower registers of the keyboard, perhaps to match the pitch of sung voices (\cref{mus:tinel_4_2}).\footnote{\cite[4:1--2]{Melodieschantgregorien1892}.}
Another adherent to filled-and-void notation was one `abbé Busschaert', who was probably Tinel's acquaintance, the Belgian priest Pieter Lodewijk Busschaert (1840--92).\footnote{Tinel had previously dedicated his opus 39 composition \emph{Der XXIX. Psalm} to Busschaert with the dedication reading `seinem Freunde dem Hochw[ürden] Herrn P[ieter] Busschaert'. See \cite[1]{TinelXXIXPsalmVierstimmiger1890}.}
Although it is possible that filled-and-void notation passed from Tinel to Busschaert outside the normal regimen of classes at the Lemmens Institute, a closer scrutiny of Busschaert's accompaniments reveals an avoidance of Tinel's obliques (\cref{mus:busschaert_4_11}).\footnote{\cite[4:11]{Melodieschantgregorien1892}.}
Tinel notes in a seven-column necrology published in 1892 that Busschaert was a musical autodidact,\footnote{\cite[The necrology is printed in the Brussels-based \emph{Musica sacra}, spelled as \emph{Musis Sacrae} in][300]{AntcliffeMusicFlemishMovement1946}; Note that a \dagger{} symbol is placed adjacent to Busschaert's name, indicating that he predeceased the appearance of his accompaniments in print.} so he probably arrived at his own conclusions on the matter.
The inclusion of Busschaert's accompaniment begs the question, however: why was the work of a musical amateur published alongside compositions by professionals?
Few answers come readily from the accompaniments themselves, though maybe Pothier and Mégret deemed his being a priest sufficient, perhaps believing that musical inspiration could arrive by spiritual means.
Given the somewhat exploratory \emph{modus operandi} of Mégret's volumes, the idea does not seem far-fetched.

We have already evaluated how Aloys Desmet's use of rests differed from Tinel's (see \cpageref{ln:tinel_verticals,ln:tinel_verticals_END} above), but he too elaborated on Van Damme's advice that singers need not dwell on notes in parentheses.
Such notes were not to be played at all, in fact,\footnote{A footnote reads `Note de l'auteur: Les notes entre parenthèses ne se jouent pas.'} a recommendation that was also made by one Louis Vanhoutte.
%\footnote{\cite[Vol.~2 p.~19]{Melodieschantgregorien1892}.}
Meanwhile, Émile de Groote transcribed the chant by slurring each neume; chords were changed at the first notes of these neumes in a manner that was not dissimilar to Van Damme's procedure.
%\footnote{\cite[Vol.~2 pp.~16--17]{Melodieschantgregorien1892}.}
The Jesuit F.~J.~Brault, by contrast, transcribed neumes and certain individual notes as small notes which were not to be played by the organ at all (\cref{mus:brault_2_12}).
%\footnote{\cite[Vol.~2 p.~12]{Melodieschantgregorien1892}.}

The fourth category concerns mensural transcriptions without conventional barlines or indications of meter.
In one case, Gevaert transcribed neumes in shorter note values, with some even being notated as triplets (\cref{mus:gevaert_4_3}).
The dotted, single and double barlines which, along with fermata marks, delimit sections of the chant seem to embody Gevaert's view that one phrase was to be separated from the next.\footnote{\cite[2:12, 16--17, 19; 4:2--3]{Melodieschantgregorien1892}.}
In another case, the Swedish composer and plainchant historian Oscar Byström (1821--1909) anticipated a wider movement in the reform of Sweden's church music by contributing a kind of mensural accompaniment of his own.
He used triplets and dotted rhythms with barlines to cast the accompaniment quoted in \cref{mus:bystrom_5_27} in a metrical scheme that changed between 6/8 and 4/4 time signatures.
Moreover, Byström later included a mensurated example of the Christmas sequence `Lætabundus' in a revision of his chamber work \emph{Quartetto Svedese} in 1895,\footcite[210]{JullanderRetainingFineBouquet2008} introducing a new `Intermezzo' as the third movement.
In this, the chant was provided a section of its own.\footnote{A reproduction of what appears to be the fair copy of Byström's \emph{Quartetto svedese} is published by Hans Ahlborg Musik: \cite{BystromStringQuartetSwedish}.}
Byström's interest in chant eventually sparked what may be considered a kind of Lutheran Cecilian movement in Sweden.
In consequence, the Swedish king appointed the composer and former ecclesiastic Gunnar Wennerberg (1817--1901) to oversee a new publication fit for the nation's church services, and from 1897 \emph{Musika till Svenska Mässan} reinstated Renaissance music in the Swedish liturgy.\footcite[58]{JullanderGregoriansksangsvensk2012}

Lhoumeau's contribution to Mégret's volumes is worthy of note in itself since it proposes a metronome marking and dispenses with the quadratic notation above the transcription.
His \emph{arsis}-\emph{thesis}, masculine-feminine approach is evident in chord changes that coincide with the first and last notes of neumes (\cref{mus:processionale_1888_287}).\footcite[1:287]{Processionalemonasticum1888}
The \emph{quilisma} shown in \cref{mus:lhoumeau_benedicta_5_12} is also noteworthy for being transcribed as a diminutive note.
Moreover, Lhoumeau indicated section breaks by using barlines.\footnote{\cite[4:12]{Melodieschantgregorien1892}.}

\subsection{Accompaniment and the popularising of free rhythm}
Lhoumeau recognised the potential in Mégret's venture to benefit not just the practice of accompaniment but also the chant restoration movement more generally, vocalising to Pothier that a more consolidated effort could prove most beneficial.\footnote{\letter{Lhoumeau}{Pothier}{12 May [\emph{c}.1892]}{\swf{}~153~(a)~28}; In this letter Lhoumeau indicates he had just sent off the above accompaniment to Mégret.}
By the beginning of 1893, Lhoumeau had set his sights on a wider suite of measures to popularise Benedictine chant and proposed an ambitious plan to recruit compositional talent under the aegis of Pothier's authority:

\simplex{Il y a de Lyon, les Trillat, [Jules] Rüest, l'abbé [C.] Marcetteau, probablement [Émile] Brune et un autre. Je verrai à [Alexandre] Guilmant, [Charles-Marie] Widor de St Sulpice et [Edgar] Tinel avec [Henri] Eymieu, le rédacteur du \emph{Ménestrel}. Le Comité étudiera et travaillera avec moi, sous votre présidence effective. Donc on travaillera à la publication de pièces accompagnées, traduites pour cela en musique, arrangées p\textsuperscript{r} voix et orgue, ou p\textsuperscript{r} ch\oe{}urs à plusieurs parties, soit pour orgue seul, même pour orchestre. On les fera connaître partout, on intéressera le public musicien à cela, on fera du grégorien au lutrin, mais aussi dans les chorales, à l'orgue et aux concerts, mais du vrai, selon \emph{votre} école, et mes livraisons annoncées seront le 1\textsuperscript{er} champ de bataille de ces MM.\ Qu'en dites-vous~? Ce sera l'affaire de D. Mégret mais plus sérieux, dans le vrai cette fois.}
  {\letter{Lhoumeau}{Pothier}{25 January [1893]}{\swf{}~153~(a)~19}; 1893 seems the most plausible year since the letter anticipates reviews of \emph{Rhythme, exécution et accompagnement} in \emph{L'Univers} (published on 13 February) and in \emph{La Croix} (published on 18 February). Lhoumeau also anticipates the following text which does not come to light until later that year: \cite{Comirechantgregorienrythme1893}}
{From Lyon, we have the Trillats, [Jules] Ruest, abbé [C.] Marcetteau, probably [Émile] Brune and one other. I will see as to [Alexandre] Guilmant, [Charles-Marie] Widor of Saint-Sulpice and [Edgar] Tinel with [Henry] Eymieu, the editor of \emph{Ménestrel}.
The committee will study and work with me basically under your presidency. So, we will work on the publication of accompaniments transcribed for the purpose into music [modern notation] and arranged either for voices and organ or for choir in several parts or for organ solo, or even for orchestra.
We will publicise them everywhere, we will excite the interest of the musical public in them, we will have Gregorian chant performed at the lectern, but also by choirs, on the organ and at concerts, but authentically according to \emph{your} school, and my proposed volumes will be the first battleground of these gentlemen. What do you say? It will be Dom Mégret's business but more serious, for real this time.}
\hlabel{ln:lhoumeau_practical_supplement}%
\noindent
Lhoumeau had announced a practical supplement to his 1892 accompaniment manual that was to contain transcriptions of Office and other chants into modern notation (`des offices et des pièces diverses').\footcite[pp.~xviii--xix, 312--20]{LhoumeauRhythmeexecutionaccompagnement1892}
The first volume `Douze mélodies' was announced in \emph{Revue du chant grégorien} on 15 December 1893 and was promised to contain several compositions along the lines of those modal pieces with which Gigout had been experimenting since the 1880s.
The dotted barlines in \cref{mus:lhoumeau_justus,mus:lhoumeau_facnos} divided up pieces into binary and ternary groups, pieces that were intended for performance by solo organ.\footnote{Original copies of Lhoumeau's volumes were not available to consult, though extracts therefrom were reproduced some years later, in \cite{GastoueTraiteharmonisationchant1910}, pp~74--6; Lhoumeau probably transcribed the chants from \cite{PothierLiberGradualis1883}, pp.~465--6, {[}43--4{]}.}
\hlabel{ln:lhoumeau_practical_supplement_END}%

Lhoumeau's use of cadential sharping soon clashed with Pothier's preference for diatonicism.
It was a matter of some embarrassment for Lhoumeau when Pothier took to the \emph{Revue du chant grégorien} to dismiss harmonisers such as Lemmens and Lhoumeau for preferring sharped cadences over the so-called `tonalité régulière'.\footcite[133]{PothiertonaliteSanctusAgnus1894}
Lhoumeau admitted in the very next issue that his use of sharps had been a mistake,\footcite[152]{LhoumeauObservation1894} and signalled to Pothier in private correspondence that he would delay the third volume to `improve its content' (`de soigner le travail'),\fnletter{Lhoumeau}{Pothier}{6 March {[}\emph{c}.1894{]}}{\swf{}~153~(a)~6} probably to remove sharps.
Although Lhoumeau's collection eventually ran to five volumes (\cref{tab:lhoumeau-pieces}), he admitted that his `science' was amateurish (`je n'ai qu'une science d'amateur'), and vowed to approach other musicians for assistance.\fnletter{Lhoumeau}{Pothier}{29 February {[}\emph{c}.1894{]}}{\swf{}~153~(a)~4}

Lhoumeau consulted Tinel, Lussy and Guilmant for advice on the transcription of chant into modern notation.
He sent each a transcribed music example for their comments, and each confirmed that the chant was well laid out and no difficulties arose from singing it, admittedly quite innocuous statements which hardly justify Lhoumeau's claim that his approach to transcription was successful.\footnote{\letter{Lhoumeau}{Pothier}{[August 1893]}{\swf{}~153~(a)~12}; Lhoumeau refers to his recent attendance at the installation of Mgr Joulain as bishop of Jaffna, which took place on 24 August 1893. See \cite{ordinationVilatte1907}.\label{fn:joulain}}
\label{int:gigout_lhoumeau}%
Lhoumeau also sent a transcription to Gigout who in turn sent back an accompaniment, asking whether it demonstrated his understanding of the rhythmic scheme.
Lhoumeau recounted the event to Pothier:

\simplex{Gigout a fait mieux~; il m'a renvoyé mon spécimen harmonisé me demandant si son accomp\textsuperscript{t} prouvait qu'il avait compris~? Et de fait, lui qui pour D. Mégret et ailleurs avait tant gâché, m'a donné du premier coup un accomp\textsuperscript{t} fort correct~; preuve que le rythme était clair pour lui.}
  {\letter{Lhoumeau}{Pothier}{n.d.}{\swf{}~153~(a)~38}\label{n:gigout}}
{Gigout did better; he returned my piece harmonised asking if his accompaniment proved whether he had understood? And in fact, he, who according to Dom Mégret and others had bungled everything, gave me at a stroke a completely correct accompaniment, proof that the rhythm was clear to him.}
\typeout{check semicolon is not a colon}
\noindent
Lhoumeau's side of this correspondence with Gigout could be preserved among some letters presently held at the BNF.
One discusses how a transcription of chant into modern notation could lessen  difficulties encountered by a harmoniser:

\simplex{Quant à l'harmonisation du chant grégorien en ce qui concerne le rythme, cette mensuration lève toute difficulté pour le musicien.}
  {\letter{Antonin Lhoumeau}{Unaddressed}{13 or 19 August [\emph{c}.1893]}{\bnf{} Dm~78-199 MUS LA-LHOUMEAU ANTONIN-3 and digitised at \bnf{} IFN-53150297. Although the BNF catalogues the letter as bearing the date `1\textsuperscript{er} août', the glyph following `1' is more likely either a `3' or a `9', based on further evidence of Lhoumeau's handwriting viewed by the present author. Moreover, the BNF catalogues the letters in reverse chronological order, a fact made clear by comparing them with correspondence in \swf{}~153}}
{As for the harmonisation of Gregorian chant, as far as the rhythm is concerned, this mensuration lessens any difficulty for the musician.}
\noindent
Another thanks the correspondee for a `petit essai d'accomp\textsuperscript{t}' which according to Lhoumeau `proves that my musical transcription is a clear and steadfast guide' (`[L'accompagnement] me prove que ma traduction musicale est un guide clair et sûr').
The letter goes on to mention how, with the support of `your institute' (`votre Institut'), French-taught chant practice could be popularised among musicians, particularly in the wake of the École Niedermeyer's decline.\footnote{\letter{Antonin Lhoumeau}{Unaddressed}{12 September [\emph{c}.1893]}{\bnf{} Dm~78-199 MUS `LA-LHOUMEAU ANTONIN-2' and digitised at \bnf{} IFN-53150298}.}
The way was clear to Lhoumeau: where Regensburg had chant and liturgical compositions in the \emph{Palestrinastil}, France could have chant and liturgical compositions based on Pothier's rhythm.

First, though, Lhoumeau believed Gigout's allegiance still lay with Teppe (so effective had that individual's bluster been in 1889) and set about trying to win over Gigout to Pothier's side.
Lhoumeau acted on Gigout's request for Pothier to send a complimentary copy of `Mélodies' by conveying the sentiment to Pothier directly.
Considering the number of texts published in the era bearing that word in their titles, it is necessarily difficult to pick out which one was to be sent to Gigout, though we may speculate that the text in question must have been \emph{Mélodies grégoriennes d'après la tradition}.
In any case, Pothier's cooperation was deemed a prerequisite for gaining Gigout's support:

\simplex{J'ai reçu une charmante lettre de Gigout qui paraît bien disposé à nous seconder. Il m'a dit qu'il serait très-désireux et très-honoré d'avoir de vous l'hommage des \emph{Mélodies} \emph{avec une suscription}. Je crois que pour le gagner à la cause et nous en faire un solide appui, vous feriez bien de l'amorcer ainsi, puisqu'il y tient tant. Ce sera le moyen de le détacher de Teppe et autres.}
  {See \cref{fn:joulain}}
{I received a charming letter from Gigout who seems well inclined to endorse us.
He told me that he would be very eager and honoured to be sent a \emph{signed}, \emph{complimentary copy} of \emph{Mélodies}.
I believe that to win him over to our cause and to gain his full backing you would do well to entice him in this way, since it matters so much to him.
This will be the way to separate him from Teppe and others.}
\noindent
A copy had not yet been sent by the following December owing to Pothier's being abroad.
So, instead, Lhoumeau elected to send on `ma 1\textsuperscript{re} livraison', which was presumably the first of those volumes of composed pieces discussed above.
Evidently in an attempt to pique the interest of his correspondee, Lhoumeau indicated that the tenth piece in the volume had been performed at two organ inauguration concerts in an arrangement for cello solo, organ and violins.\fnletter{Antonin Lhoumeau}{Unaddressed}{18 December [\emph{c}.1893]}{\bnf{} Dm~78-199 MUS `LA-LHOUMEAU ANTONIN-1' and digitised at \bnf{} IFN-53150299}
In the \emph{Revue du chant grégorien}, Lhoumeau outlined one such inauguration that took place on 23 June 1894, at which chant harmonisations by Niedermeyer and others were performed alongside organ repertoire to demonstrate how effectively the organ could manage in a liturgical setting.
One attendee remarked of the `mystic effect' produced by the chant, a trait Lhoumeau put down to the chant's having gained for itself an artistic status completely separate from other repertoire.\footcite[196--7]{LhoumeauInaugurationorgue1894}
Gigout did not seem particularly receptive to Lhoumeau's advances, though the potential to gain ground on Regensburg with the backing of a French musical institution was probably responsible for Lhoumeau's seeking out other figures who could lend him their support.

\subsection{Guilmant, Early Music, and the Schola Cantorum} % Schola cantorum
Institutions like Gigout's Institut d'orgue benefited from a certain degree of independence that was not afforded to state-funded institutions such as the École Niedermeyer and the Paris Conservatoire.
Their church music curricula were susceptible to anticlerical misgivings, particularly in the case of the latter where the pedagogy became increasingly focused on secular music.
The Conservatoire will be discussed in more detail next, but for the moment we shall examine what came of a failed attempt to reform its curriculum in 1892; namely, that a faction splintered off to establish yet another school of church music, the Schola Cantorum.
Led by Charles Bordes (1863--1909), Guilmant and Vincent d'Indy (1851--1931), the Schola Cantorum was established between June and December of 1894 as an alternative to the Conservatoire.
Guilmant earned himself the title of president of a related society with far-reaching aims to promote Early Music and plainchant in the liturgy.
Bordes established the journal, \emph{La Tribune de Saint-Gervais} (\tsg{}), calling it after his own position as \emph{maître de chapelle} at the Parisian church of the same name.
The Schola Cantorum received its first enrollment of pupils in 1896 and embarked on several years of trial and error as its teachers refined the curriculum.\footcite[221]{OchseOrganistsOrganPlaying2000}

Since the 1870s, Guilmant had established himself as something of a musical \mbox{antiquarian} by programming Bach and Handel's music in organ recitals.\footcite[5, 20, 80--82]{FlintScholaCantorumEarly2006}
Moreover, his authority as \emph{titulaire} at La Trinité permitted him to replace the orchestral transcriptions parishioners had come to expect with more Early Music, particularly when the organ intervened in the course of the liturgy.
By 1898, secular music at such occasions had been replaced entirely.\footcite[364]{LuedersAlexandreGuilmant18372002}

\label{ln:guilmant_style}%
Prior to embarking for Brussels around 1860, Guilmant had gained some experience of accompanying chant when he succeeded Hanon as the organist of the église Saint-Joseph, Boulogne-sur-Mer, later becoming the \emph{maître de chapelle} and then the organist at the nearby église Saint-Nicolas (where his father had played).\footcite[69]{OchseOrganistsOrganPlaying2000}
His biographer Kurt Lueders noted that several MSS of chant harmonisations in Guilmant's hand date from this era,\footcite[pp.~693--4 \S{}28]{LuedersAlexandreGuilmant18372002} though Lueders omitted the holograph \bnf{}~MS~6979 which deserves some consideration here.

Although space was allotted beneath transcriptions of chants for harmonisations, only some received four-part accompaniments, others being notated in figured bass or not at all (\cref{tab:ms6979}).
Four elements of Guilmant's process are particularly noteworthy.
First, the mensural transcriptions of the chants probably matched the layout of neumes in the chant book Guilmant used.
Second, chords were mainly of the 5/3 variety but a smattering of 6/3 were included for good measure.
Third, Guilmant sometimes indicated organ registrations at particular moments, such as one instance of an enclosed clarinet stop accompanied by `Bourdon' and pedal (\cref{mus:guilmant_6979_1v}).
Foundation stops took over and were probably preferred from the outset of the harmonisation too.
Another instance required the player to draw a `Gambe' before the same phrase was repeated, this time transposed up an octave on an `Oboë' stop (\cref{mus:guilmant_6979}).
The `G\textsuperscript{d} Chœur' resumes the harmonisation with a \emph{forte} dynamic, producing what is arguably a romantic \emph{Affekt}.
Fourth, the pedal part is doubled at the octave above, where its span exceeds that of an octave from the next part up: perhaps the accompaniment was intended to be played on manuals alone or even on a harmonium.\footnote{\bnf{}~MS~6979 ff.~1v, 2r.}

To Mégret's volumes, Guilmant contributed one accompaniment in the chord-against-note style and another with chords coinciding with the first notes of neumes.
Lueders reports that the latter, a harmonisation of `Media vita', was composed at Guilmant's Meudon residence on 25 June 1891,\footcite[p.~695 \S{}32; Guilmant's `Media vita' is preserved in \covid{}\bnf{}~MS~17194, on which the date and place of composition are reported to have been preserved.]{LuedersAlexandreGuilmant18372002} making it predate Pothier's dictum quoted above.
Neumatic chord changes pervade the accompaniment (\cref{mus:guilmant_media_15}),\footnote{\cite[1:15]{Melodieschantgregorien1892}.} and that tacit practice of tying notes common to consecutive chords is made explicit in the notation.
The slurring of neumatic groups is also noteworthy.

It is unclear when Lhoumeau's accompaniments came to Guilmant's attention, though the organist received the Montfortian at La Trinité when Guilmant demonstrated several interludes in a new compositional style.\fnletter{Lhoumeau}{Pothier}{n.d.}{\swf{}~153~(a)~74}
During the 1890s, Guilmant published \emph{Soixante interludes dans la tonalité grégoriennne pour orgue ou harmonium} op. 68, which might have proceeded from his demonstrations to Lhoumeau.
Like Gigout, therefore, Guilmant turned his attention to composing pieces according to his conception of the Gregorian \emph{tonalité}, and agreed to work with Lhoumeau not only on so-called `pure' arrangements of chant for organ but on accompaniments too:
\noclub[2]

\simplex{Il est question de faire \emph{à nous deux}, mais avec beaucoup de temps et de soins, des travaux d'accompagnement sur le chant grégorien et aussi d'arrangements \emph{purs} des chants pour orgue. Il m'a dit que cependant il ne croyait pas pouvoir se passer de vos conseils. Vous avez là une belle conquète, plus docile que Lemmens. Qu'en pensez-vous~?}
  {\letter{Lhoumeau}{Pothier}{14 June 1894}{\swf{}~153~(b)~131}}
{It is a matter \emph{between us} of taking much time and care on accompaniments for Gregorian chant and also on \emph{pure} arrangements of the chants for organ. He told me, moreover, that he did not think it possible to do without your advice. You have there a fine devotee, more amenable than Lemmens. What do you think?}
\noindent
Guilmant was pleased to learn of Pothier's assent to the proposal and set out to produce accompaniments which, in Lhoumeau's words, would be as unobtrusive as possible.\fnletter{Lhoumeau}{Pothier}{19 November 1894}{\swf{}~153~(b)~137}
Faced with the perennial question of where chords should change, Guilmant was unsure how best to proceed.
The more Guilmant wished for unobtrusiveness, the more he perceived inadequacies in the way the repertory had been transcribed.
Further work was required in rhythmic analysis, a task which, although Lhoumeau accepted it, made him all the more conscious of his musical limitations:
\pagebreak{}

\simplex{Guilmant ne voit d'espérance que dans un accomp\textsuperscript{nt} sobre à l'excès, fort doux, et où le chant soit traité comme un récitatif. C'est tout à fait selon vos idées et les miennes aussi, mais la réalisation de ce plan exige une analyse minutieuse du rythme qu'il me faut faire, et une pratique du métier harmonique que je n'ai pas et qui me fait hésiter souvent.}
  {\letter{Lhoumeau}{Pothier}{24 April [n.y.]}{\swf{}~153~(b)~126}}
{Guilmant sees merit only in a very soft accompaniment that is excessively solemn, where the chant is treated as a recitative. That is entirely according to your ideas and mine too, but the realisation of this plan requires both a detailed analysis of the rhythm (which I have to do) and a practice of the harmonic craft which I do not have and which makes me hesitate often.}
\noindent
Results were delayed for quite some time as Lhoumeau undertook work on transcriptions into modern notation.\fnletter{Lhoumeau}{Pothier}{29 December [n.y.]}{\swf{}~153~(b)~143}
He even took to the \tsg{} to try out a quasi-mensural approach where time signatures continually changed to accommodate shifting accents.\footcite[6]{Lhoumeauplainchantorgue1895}

It was in the spirit of mimicking something along the lines of \emph{recitative secco} that Guilmant began introducing rests into the accompaniment, thereby leaving the chant unaccompanied for several notes or perhaps even for several bars.
In an uncharacteristic appeal to Regensburg's practice, Lhoumeau mused how Cecilian composers made use of similar devices,\footcite[44--5]{LhoumeauEtudeaccompagnement1898} though one notes that Mégret's volumes had contained examples by Belgian composers who had made use of a similar approach.

Guilmant's wish for discretion might explicate his preference for foundation stops.
Lhoumeau set out Guilmant's position on the matter in an 1897 issue of the \tsg{}:

\simplex{Notre vénéré président M. Guilmant, ne cesse d'enseigner aux organistes la prédominance des fonds sur les anches, et l'emploi plus que rare de ces derniers jeux pour accompagner les voix.}
  {\cite[103]{Lhoumeauaccompagnementchantgregorien1897}}
{Our revered president Mr Guilmant, continues to teach organists the predominance of foundations over reeds, and the rarer use of these latter stops for accompanying voices.}
\noindent
Lhoumeau's description was admittedly rather \emph{maladroit}, however, and could be taken to imply Guilmant advocated the use of reeds, though the practice was generally considered rare.
As we have seen, Guilmant was not averse to using clarinet and oboe stops for colouristic effect, though his preference appears to have lain with foundation stops for the \mbox{accompaniment} of voices.
The confusion explains why one Anglophone writer has claimed Guilmant preferred reeds to foundations,\footcite[29--30]{WagstaffGuilmantCredoCatholic2015} though since the musical evidence indicates the contrary the writer might also have mistranslated `la prédominance des fonds sur les anches'.

Incidentally, there is nothing contained in the Guilmant--Pothier correspondence viewed by the present author to suggest that either party consulted the other on rhythm or harmony.
If any such discussion did take place it was probably either conducted in person or via Lhoumeau, who functioned as a kind of translator between musician and chant expert.
Guilmant was responsible for inviting Pothier to review Gevaert's newly published \emph{La mélopée antique dans le chant de l'église} for the \emph{TSG} in March 1896,\footnote{\letter{Guilmant}{Pothier}{19 January 1896}{\swf{}~204 item 2}; \cite{PothierReviewmelopeeantique1896}.} though the relevant correspondence does not treat of any topics, musical or chant-based, in any amount of detail.
Lhoumeau's archives were held by the Company of Mary but their whereabouts, at the time of writing, are unknown.\footnote{Patrick Hala to the author, 24 June 2020.}

In 1895, another forum to discuss the appropriateness of church music was convened at Rodez, at which Guilmant and Bordes represented the Schola Cantorum.
Guilmant's address on the role of the liturgical organ proposed that any interventions made during the liturgy were to match the style of chant.
What that style might encapsulate when translated into solo repertoire was probably informed by his deliberations with Lhoumeau and his experiments with the Gregorian \emph{tonalité}.
Guilmant then laid down the gauntlet for composers to devise a new genre of liturgical composition along the lines of Lemmens's \emph{École d'orgue}, though the genre was not to be based on Lemmens's preferred `plain-chant parisien' which was considered ill equipped for the \emph{fin-de-siècle} liturgy.\footcite[The first volume of \emph{La Tribune de Saint-Gervais} restarted each issue's pagination. Only from the second volume did the pagination run contiguously][11]{Guilmantroleorguedans1895}

\subsection{A new method at the Paris Conservatoire}
While the Schola Cantorum and Gigout's Institut d'orgue were trailblazers, the Paris Conservatoire remained conservative.
Benoist's `choral' procedure remained the staple diet of the Conservatoire's organ students long after Benoist had retired.
Considering Franck's \mbox{experience} in a more sustained style of chant accompaniment, one might suppose the examination rubrics would be updated.
\label{sc:vierne_choral}%
But Vierne encountered the `choral' on his entry to the Conservatoire in 1890 and described it as a somewhat ossified procedure that was deemed a traditional component of organ studies:\footnote{Vierne had also studied at the Institution des jeunes aveugles since 1881, where Braille had invented his eponymous writing system some seven decades prior. See \cite[430]{AprahamianLouisVierne187019371970}.}

\duplex{Elle existait depuis la fondation de la classe d'orgue. Elle consistait en l'accompagnement note contre note d'un chant liturgique à la partie supérieure ; puis ce chant devenait basse en rondes, non transposée, accompagnée de trois parties supérieures d'une sorte de contrepoint fleuri d'école~; les rondes passaient ensuite à la partie supérieure ; une quarte plus haut et recevaient à leur tour l'accompagnement du «~fleuri~» d'école. Rien n'était plus formulaire que ce contrepoint, rigoureux sans l'être, bourré de quintes retardées, d'accords de septième prolongée avec quintes, de marches, en un mot de tout ce qu'on interdit en contrepoint écrit. C'était la «~tradition~», et Franck n'y pouvait rien changer.}
  {\cite[22]{VierneMessouvenirs1970} wherein `elle' refers to the content of Franck's plainchant class, on which was set one of four tests (`épreuves') faced by Conservatoire organ students in their examination}
{Existing since the inception of the organ class, it consisted of a note-for-note accompaniment of a liturgical chant in the upper voice; the chant then became the bass in whole notes, not transposed, accompanied by three upper parts in a sort of academic florid counterpoint. The semibreves then passed into the top voice, transposed a fourth higher, receiving in their turn a `florid' academic accompaniment. Nothing was more formalized than that counterpoint---strict without being exactly so, crammed with retarded fifths, seventh chords prolonged with fifths, and sequences---in a word, with all that is forbidden in written counterpoint. It was `traditional,' and Franck could not change anything about it.}
  {Adapted from \cite[pp.~41, 43]{SmithLouisVierneOrganist2009}}
\noindent
\hlabel{hl:tournemire_franck}%
Vierne was not the only organist to encounter the `choral': Tournemire also included an example of the procedure in a 1936 textbook where the chant was placed alternately in top and bottom parts.\footnote{\cite[105]{TournemirePrecisexecutionregistration1936}; Reproduced in \cite[40]{SmithLouisVierneOrganist2009}.}
Moreover, the `choral' idiom surely inspired the compositional process of \emph{L'Orgue mystique} where Tournemire annotated prominent occurrences of chant with the term `choral', just like the instance of \emph{Beata gens} quoted in \cref{mus:tournemire_alleluiatique_9}.\footcites[9]{TournemireDominicaXVIIpost1932}[265]{GotlundGuideChantCharles2015}
A section bearing the phrase `Fragments du choral' later in the same movement occurs when the chant was placed in the top voice of a four-part chorale-like texture, suggesting Benoist's influence might run deeper than the scholarship on the subject has yet acknowledged.\footcite[David Connolly, whose study relies chiefly on Anglophone literature, does not engage with the `choral' idiom in Tournemire. Nevertheless, see][194--215]{ConnollyInfluencePlainchantFrench2013}

Bordes mused some fifteen years after Franck's death that it was simply too early in the chant restoration movement for Franck to have become involved.\footcite[578]{Bordessentimentreligieuxdans1904}
That reason did not satisfy d'Indy, however, who couched Bordes's view in stronger language, claiming Franck ignored Benedictine research altogether.
D'Indy then went further by suggesting Franck's inaction cast doubt on whether his liturgical music was to be considered fit for liturgical use.\footcite[p.~107 with citations from Bordes spanning pp.~108 and 109]{DIndyCesarFranck1907}
It is difficult to agree with d'Indy's view since he himself seemingly did not encounter Pothier's \emph{Liber gradualis} until 1890.\footnote{D'Indy's copy is preserved at \covid{}\bnf{} RES VMC MS-223.}
D'Indy seemed keen to adopt the latest techniques of music theory to parse the chants in that book, borrowing formal, symbolic and technical methodologies from Riemann and Lussy.
Given hindsight, however, it is hard not to see exaggeration in Gilles Saint-Arroman's claim that, thereby, nothing escaped d'Indy's analysis (`aucun aspect n'échappe à son analyse').\footcite[82, 86]{Saint-Arromaninfluencelivressolesmiens2019}

The pedagogical tradition of the `choral' changed little when Widor succeeded Franck in 1891.
Records dating from 1892 show the class divided into two strands, each cohort learning (among other things) different aspects of the accompaniment of chant.
The lower, `premier degré' applied to the chant repertory the chord-against-note routine and also a method called `harmonie figurée' (likely a type of arpeggiated accompaniment rather than the traditional `basse chiffrée').
The upper, `supérieur', refined those studies with an in-depth examination of the modes and their origin (`la connaissance des modes du plain-chant et de leur formation historique').\footnote{\cite[375]{PierreConservatoirenationalmusique1900}; Translated in \cite[101]{SmithLouisVierneOrganist2009} where `harmonie figurée' is confusingly rendered as `unfigured harmony'. It should be noted that the style is unlikely to have been figured bass which was usually rendered as `la basse chiffrée'.}
If at the Brussels Conservatoire Lemmens had taught Widor how to accompany using Fétis's method, then it is evident that sometime after his return to Paris Widor was inclined to adopt Niedermeyer's principles.
\hlabel{int:widordubois}\hlabel{hl:widordubois}%
Emery-Desbrousses asserted in 1892 that Widor and Dubois both adhered to them,\footcite[160]{EymieuEtudesbiographiesmusicales1892} an assertion corroborated by each organist's contributions to Mégret's volumes.

\label{sc:widor_megret}%
Widor's contribution was one of a select few of his pieces intended for multi-author collections, another being a short polyphonic mass composed to mark, in 1888, the tenth anniversary of Leo XIII's election to the papacy.\footcite[`Avis des éditeurs' and pp.~15--16]{GrivetLitaniessaintevierge1888}
Michael Bundy has recently noted that it exists alongside `another collection of simple chant harmonisations' in the Widor canon entitled `Laetare puerpera -- Séquence pour le temps de Noël', reporting the latter to have been published in 1893.\footcite[86--7]{BundyVisionsEternityChoral2017}
But it is more likely that the `collection' in question was none other than Widor's contribution to Mégret's volumes, since it shares the same name and a similar date of publication.
Whether or not the harmonisation was ever published separately has been impossible to ascertain.

\hlabel{ln:widor_alternatim}%
Widor's duties at Saint-Sulpice probably afforded him few opportunities to accompany the chanting at Offices, this being the ordinary responsibility of the \emph{organiste de chœur}.
One reliable account of chant practice at Saint-Sulpice in 1896 tells of snail's-pace chanting accompanied on full organ---presumably on the \emph{orgue de chœur} (`avec plain-chant à l'escargot et accompagnement à organo pleno')---and waxes lyrical about how effectively Widor played Magnificat versets on the grand orgue, the chant set in the bass part on the pedal while contrapuntal material was played on the manuals.
By that account, alternatim practice was anything but a retired technique.\fnletter{Peter Wagner}{Dom Antoine Delpech}{27 August 1896}{\so{}}
\hlabel{ln:widor_alternatim_END}%

\pagebreak{}
When Guilmant inherited the Conservatoire's organ class in 1896 the Schola Cantorum's curriculum was just getting under way, and it is therefore unsurprising that one of Guilmant's first decisions at the Conservatoire was to replace the `choral' with a more supple accompaniment based on his own researches.\footcite[For the events leading to Widor's succession, see][116]{OchseOrganistsOrganPlaying2000}
He took up the matter with his \emph{assistant} Vierne, who recalled the exchange in his memoirs:\footnote{The circumstances surrounding the meeting are also discussed in \cite[362]{Lessmannanachronismemusicalaccompagnement2019}.}

\duplex{Nous convînmes de ce qui suit, après en avoir mûrement discuté. Le contrepoint hybride et formulaire, dont j'ai parlé au début de ce chapitre, disparaîtrait et serait remplaé par le commentaire du chant liturgique préalablement accompagné comme à l'église, non plus `note contre note', mais dans un style plus large, admettant les ornements mélodiques tels que \emph{broderies} et \emph{notes de passage}, les accords étant réservés aux notes principales.}
  {\cite[54]{VierneMessouvenirs1970}}
{After serious discussion, we agreed upon the following: the hybrid, stereotyped counterpoint should be dropped and replaced by a commentary on the liturgical chant, no longer accompanied `note for note,' as in church, but in a broader style, admitting melodic ornaments, such as embellishments and passing notes, chords being reserved for principal notes.}
  {\cite[121, 123]{SmithLouisVierneOrganist2009}}
\noindent
Guilmant's more sustained style may more advantageously be understood in the light of further developments at Solesmes where so-called `free musical rhythm' elaborated on the tenets of Pothier's `free oratorical rhythm'.
This led to a new procedure of accompaniment and the publication of new accompaniment books by the monks of Solesmes.
