\chapter*{Summary}
This dissertation establishes the techniques employed in the organ accompaniment of plainchant, determines whether consensus was reached on the adoption of such techniques, and illustrates the musical and commercial factors in the publishing of accompaniment books and theoretical manuals.
Publications in some twenty languages provide the basis for the discussion, which is illustrated by quoted music examples in the second volume.
These bear witness to a highly adaptable set of musical techniques that responded to changes in fashion and discoveries in music history.
The discussion is further illustrated by referencing a rich vein of letters and archival material---here being brought to light for the first time---that show how ideas passed between theorist and practitioner, and how methods, manuals, theory and practice transcended international and religious boundaries, leaving an enduring legacy that may still be felt in sacred music today.
A new story is told, largely in chronological order: one of musical idealism, political wrangling and the commercial shrewdness of musicians and publishers who responded to the demands of the market.
The theory and practice of chant accompaniment at Solesmes serves as a particular focus, since the importance of accompaniments in the chant restoration movement has not yet been considered.

In Chapter 1, a prehistory summarises the organ's involvement in the liturgy from the twelfth to the nineteenth centuries, describing the passing from alternatim practice to chorale singing in Lutheran churches and the gradual introduction of organ accompaniment of singing.
The rise of the continuo proved useful to Catholic and Protestant musicians alike, whose enthusiasm for novelty led to secular genres being included as part of the church service.
The Catholic Church in particular recognised this as a corrupting trend, and decreed that a more pious style of playing be adopted instead.
This was left to Cecilian musicians in Germanic countries to codify, and their reasons for adopting supposedly austere textures are examined in the second half of the chapter.
Owing to the Vatican's official adoption of Cecilian chant books, the resulting demand for relevant books of accompaniments led to the spread of Cecilian ideals across Europe, to the United States, and to South Africa.

In Chapter 2, the introduction of the accompanying organ into French churches is shown to have started a trend.
A lack of available trained organists drew reservations from some quarters, and the growing popularity of chanting established a demand for automated instruments that could take the place of a trained accompanist.
Commercially savvy but musically dubious pedagogical manuals claimed to simplify the practice of chant accompaniment for untrained parishoners, seminarians and pianists, but made such bogus claims that they cannot have been enlightening to the hapless amateurs at which they were aimed.
Meanwhile, more serious theorists were engaged in seeking authentically venerable methods of harmonisation in the musical practice of antiquity.
The diatonic method codified by Niedermeyer is shown to be based on specious claims to antiquity, though that did not preclude it from capturing the ninteenth-century imagination; thereafter, it was widely applied in chant accompaniments.

In Chapter 3, the matter of free rhythm and how it could be applied to the accompaniment is examined in detail, particularly with respect to the notation of such accompaniments.
What we term the `Lhoumeau effect' (the changing of chords on unstressed syllables) is also discussed, as is the use of accompaniments as plainchant propaganda to popularise free rhythm in France and further afield.

In Chapter 4, those Solesmian accompaniments written to demonstrate the controversial theory of the \emph{ictus} are considered, as is the training up of the Solesmes monk Antoine Delpech in harmony that preceded the production of Solesmes's first accompaniment books.
They proved to be highly controversial and Delpech's involvement in harmonising for Solesmes was discontinued following a dispute with his monastic superiors.
The baton passed to Giulio Bas, whose two-decade collaboration with André Mocquereau is discussed with reference to his letters and published accompaniments.
The publication of the Vatican Edition led to Bas's needing to revise accompaniments published previously, owing to the Vatican commission having revised some of the chants.
The unprecedented demand for accompaniments tailored to the Vatican Edition led multiple publishers to bring out their own editions: these are described in connection with the methods adopted by individual harmonisers.
The methods publishers used to encourage the adoption of their accompaniment books are also considered, for they reveal the accommodation of the repertory to specific geographical locations.

In Chapter 5, two approaches to chant harmony are compared with earlier traditions.
On the one hand, the unrestricted admittance of chromatic notes to chant harmonisations railed against the diatonicism preferred in France and Belgium; on the other, a more austere method developed at Solesmes restricted the notes in the harmony to those comprehended by the chant.
The influence of modernism is considered in both cases, and the blurred line between modality and major-minor harmony at Solesmes is contextualised with reference to correspondence between one of its monks and the composer André Caplet.
Finally, a postscript illustrates some developments in chant accompaniment that have taken place since the reforms of the Second Vatican Council.
